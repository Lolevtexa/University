% ==================
% answer.tex
% (Основной материал: "Формула Ньютона–Лейбница.")
% ==================

\begin{customtheorem}[Формула Ньютона–Лейбница]
	Пусть $f$ непрерывна на $[a,b]$ и $F$ — её первообразная на $[a,b]$, то есть $F'(x)=f(x)$ на $(a,b)$.
	Тогда
	\[
		\int_a^b f(x)\,dx \;=\; F(b) \;-\; F(a).
	\]
\end{customtheorem}

\begin{proofplan}
	\begin{enumerate}
		\item Рассмотреть разбиение отрезка $[a,b]$ и интегральную сумму $S$.
		\item Применить теорему о среднем значении к приращению $F(x_i)-F(x_{i-1})$,
		      показав, что $f(\eta_i)\,\Delta x_i$ совпадает с этим приростом.
		\item Просуммировать (телескопическая сумма) и получить
		      $\sum [F(x_i)-F(x_{i-1})] = F(b)-F(a)$.
		\item Переход к пределу при мелкости разбиения даёт равенство с $\int_a^b f(x)\,dx$.
	\end{enumerate}
\end{proofplan}

\begin{customproof}
	\textbf{1. Разбиение.} Пусть $a=x_0 < x_1 < \dots < x_n = b$ — произвольное разбиение отрезка $[a,b]$ с
	$\Delta x_i = x_i - x_{i-1}$. Возьмём точки $\xi_i \in [x_{i-1}, x_i]$.

	\smallskip

	\textbf{2. Интегральная сумма.} По определению,
	\[
		S = \sum_{i=1}^n f(\xi_i)\,\Delta x_i.
	\]
	Мы хотим связать это с приращением $F$.

	\smallskip

	\textbf{3. Прирост первообразной (теорема о среднем значении).}
	На каждом отрезке $[x_{i-1}, x_i]$ существует $\eta_i$ (похожа на $\xi_i$) такая, что
	\[
		F(x_i) - F(x_{i-1})
		\;=\; F'(\eta_i)\,(x_i - x_{i-1})
		\;=\; f(\eta_i)\,\Delta x_i.
	\]
	Таким образом,
	\[
		\sum_{i=1}^n \bigl[F(x_i)-F(x_{i-1})\bigr]
		= \sum_{i=1}^n f(\eta_i)\,\Delta x_i.
	\]
	Заметим, что $\sum_{i=1}^n [F(x_i)-F(x_{i-1})]$ — телескопическая сумма:
	\[
		F(x_n)-F(x_0) = F(b)-F(a).
	\]

	\smallskip

	\textbf{4. Предел.}
	Если $\|D\| = \max_i \Delta x_i \to 0$, то, поскольку $f$ непрерывна,
	\(\sum f(\eta_i)\,\Delta x_i\) стремится к \(\int_a^b f(x)\,dx\).
	Но мы выяснили, что это же $\sum f(\eta_i)\,\Delta x_i = F(b)-F(a)$.
	Следовательно,
	\[
		\int_a^b f(x)\,dx \;=\; F(b)-F(a).
	\]
\end{customproof}

\begin{customexample}
	\textbf{Применение к $f(x)=x^2$.}
	У функции $x^2$ есть примитив $F(x)=\tfrac{x^3}{3}$. По формуле Ньютона–Лейбница получаем
	\[
		\int_0^2 x^2\,dx
		=
		\left.\frac{x^3}{3}\right|_0^2
		=
		\frac{2^3}{3}\;-\;0
		=
		\frac{8}{3}.
	\]
\end{customexample}
