% ===== sub_14.tex =====
% Вспомогательные определения/теоремы, не являющиеся
% "центральными" в вопросе 14, но используемые в доказательстве
% (Определение интеграла Римана. Отличие от «обычного» предела).

\textbf{Верхние и нижние суммы (Дарбу).}
Для разбиения $D=\{x_0,\dots,x_n\}$:
\[
\overline{S}(f,D)=\sum_{i=1}^n \sup\limits_{x\in[x_{i-1},x_i]}f(x)\;\Delta x_i,
\quad
\underline{S}(f,D)=\sum_{i=1}^n \inf\limits_{x\in[x_{i-1},x_i]}f(x)\;\Delta x_i.
\]
Если при $\|D\|\to 0$, \(\overline{S}(f,D)\) и \(\underline{S}(f,D)\) сходятся к одной величине, эта величина называется \(\int_a^b f\).

\medskip

\textbf{Уточнение разбиения.}
Дано два разбиения $D$, $D'$. Их \emph{уточнением} называют разбиение, содержащее \textbf{все} точки $D$ и $D'$. При сопоставлении интегральных сумм на этом уточнённом разбиении можно показать, что они близки при мелком $\|D\|\to 0$.

\medskip

\textbf{Сущность «обычного» предела vs. интеграл.}
- Обычный предел $\lim_{x\to x_0}f(x)$ говорит о локальном поведении $f$ возле одной точки $x_0$.  
- Интеграл Римана — предел \emph{сумм} на всем отрезке $[a,b]$ с измельчающимся разбиением. Глобальное свойство.
7