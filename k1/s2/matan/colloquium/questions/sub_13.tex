% ===== sub_13.tex =====
% Вспомогательные определения/теоремы, не являющиеся
% "центральными" в вопросе 13, но используемые в доказательстве
% (Формула Стирлинга (с равенством)).

\textbf{Формула Эйлера–Маклорена (намёк).}
Для достаточно гладкой функции $f$, при суммировании:
\[
\sum_{k=a}^{b} f(k)
\;\approx\;
\int_{a}^{b}f(x)\,dx \;+\; \text{(пограничные и высшие члены)},
\]
используются числа Бернулли. В частности, для $f(k)=\ln k$ даёт точное выражение логарифма факториала:
\[
\ln(n!)=n\ln n -n + \tfrac12\ln(n) + \dots
\]

\medskip

\textbf{Числа Бернулли.}
Обозначаются $B_{m}$, входят в разложение. Не нужны формулы здесь, достаточно знать: они позволяют оценивать дополнительный член, который даёт диапазон $0<\theta_n<\frac{1}{12n}$.

\medskip

\textbf{Точный вид остатка.}
При экспоненцировании логарифмической оценки:
\[
\ln(n!)=n\ln n -n + \tfrac12\ln(2\pi n)+\theta_n
\]
возникает $n! = \sqrt{2\pi n}\,\bigl(\tfrac{n}{e}\bigr)^n e^{\theta_n}$. Ограничения на $\theta_n$ следуют из дополнительных членов Эйлера–Маклорены.
