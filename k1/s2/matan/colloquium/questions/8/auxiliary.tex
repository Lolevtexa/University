% ==================
% auxiliary.tex
% (Вспомогательные определения для темы:
%  "Достаточные условия существования экстремума (по второй производной).")
% ==================

\paragraph{Локальный минимум и максимум.}
Точка $x_0$ внутри промежутка $(a,b)$ называется \emph{локальным минимумом} функции $f$,
если существует $\delta>0$ такое, что при $|x - x_0| < \delta$
выполняется $f(x) \ge f(x_0)$.
Аналогично, $x_0$ называется \emph{локальным максимумом},
если в некоторой окрестности $x_0$ верно $f(x) \le f(x_0)$.

\bigskip

\paragraph{Вторая производная.}
Пусть $f$ дифференцируема на интервале $(a,b)$, и $f'(x)$ тоже дифференцируема на $(a,b)$.
Тогда в точках, где это возможно, определена \emph{вторая производная} $f''(x)=\bigl(f'(x)\bigr)'$.

\bigskip

\paragraph{Теорема Ферма (напоминание).}
Если функция $f$ дифференцируема в точке $x_0$ и имеет там локальный минимум или максимум,
то $f'(x_0)=0$.
Это — необходимое условие экстремума (без второй производной).
