% Вспомогательная информация по вопросу
% (понятия из 1.1 и 4.1 вашего конспекта, необходимые перед формулировкой теоремы)

\paragraph{Компакт в $\mathbb{R}$.}
Множество $K \subset \mathbb{R}$ называется \textit{компактным},
если оно замкнуто и ограничено.

\paragraph{Непрерывность (точечное определение).}
Функция $f$ называется непрерывной в точке $x_0$, если
\[
	\forall \varepsilon > 0\ \exists \delta > 0:\
	|x - x_0| < \delta \;\Longrightarrow\; |f(x) - f(x_0)| < \varepsilon.
\]
Если это выполняется для любой точки множества $X$, то $f$ непрерывна на $X$.

\paragraph{Теорема Больцано--Вейерштрасса.}
Любая ограниченная последовательность в $\mathbb{R}$ имеет сходящуюся подпоследовательность.
Если $(x_n)$ лежит в компактном $K$, то любая подпоследовательность $(x_{n_k})$
содержит сходящуюся подпоследовательность с пределом в $K$.

\paragraph{Равномерная непрерывность.}
Функция $f$, определённая на $X \subset \mathbb{R}$, называется \textit{равномерно непрерывной},
если
\[
	\forall \varepsilon > 0\ \exists \delta(\varepsilon) > 0:\
	\forall x_1, x_2 \in X,\ |x_1 - x_2| < \delta
	\;\Longrightarrow\; |f(x_1) - f(x_2)| < \varepsilon.
\]
Важное отличие: величина $\delta$ зависит только от $\varepsilon$, а не от точки $x_0 \in X$.
