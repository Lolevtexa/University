

\begin{customtheorem}[Теорема Кантора]
	Если функция $f$ непрерывна на компактном множестве $K \subset \mathbb{R}$,
	то $f$ равномерно непрерывна на $K$.
\end{customtheorem}

\begin{proofplan}
	\begin{enumerate}
		\item Допустим, что \(f\) непрерывна на \(K\), но не является равномерно непрерывной.
		\item Покажем, что существует \(\varepsilon_0 > 0\), при котором нельзя «раз и навсегда»
		      выбрать \(\delta\), подходящее всем точкам в \(K\).
		\item Используем компактность \(K\) и теорему Больцано–Вейерштрасса для
		      извлечения сходящейся подпоследовательности, приводящей к противоречию
		      с неравномерной непрерывностью.
		\item Следовательно, \(f\) равномерно непрерывна.
	\end{enumerate}
\end{proofplan}

\begin{customproof}
	Пусть $f$ непрерывна на $K$. Предположим, что $f$ не равномерно непрерывна. Тогда
	\[
		\exists \varepsilon_0 > 0\ \forall \delta > 0:\
		\exists x, y \in K:\ |x - y| < \delta,\quad |f(x) - f(y)| \ge \varepsilon_0.
	\]
	Устанавливаем $\delta = 1/n$ и строим пары $(x_n, y_n)$. Поскольку $K$ компактно,
	можно выделить сходящуюся подпоследовательность $(x_{n_k}, y_{n_k})$
	с $x_{n_k} \to c$ и $y_{n_k} \to c$.
	Из непрерывности $f$ следует
	\[
		|f(x_{n_k}) - f(y_{n_k})| \to 0,
	\]
	что противоречит условию $|f(x_{n_k}) - f(y_{n_k})| \ge \varepsilon_0$.
	Следовательно, наше предположение было неверным, и $f$ равномерно непрерывна на $K$.
\end{customproof}

\begin{customexample}
	\begin{itemize}
		\item Функция $f(x) = kx + b$ равномерно непрерывна на всей $\mathbb{R}$, так как
		      \[
			      |f(x_1) - f(x_2)| = |k|\cdot |x_1 - x_2|.
		      \]
		\item Функция $g(x) = x^2$ не является равномерно непрерывной на $\mathbb{R}$,
		      хотя непрерывна. Но на любом отрезке $[a,b]$ она будет равномерно непрерывна
		      (по Теореме Кантора).
	\end{itemize}
\end{customexample}

