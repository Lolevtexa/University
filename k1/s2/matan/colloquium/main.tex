\documentclass[12pt,a4paper]{article}
\usepackage[utf8]{inputenc}
\usepackage[T2A]{fontenc}
\usepackage[russian]{babel}
\usepackage{amsmath, amsthm, amssymb}
\usepackage[left=2cm,right=2cm,top=2cm,bottom=2cm]{geometry}
\usepackage{indentfirst}

% Красивое оформление содержания
\usepackage{hyperref}

\title{\textbf{Конспект к коллоквиуму по математическому анализу}}
\author{Николаев Всеволод}

\begin{document}

\maketitle

\tableofcontents


\newpage
\section{Равномерная непрерывность. Примеры. Теорема Кантора о равномерной непрерывности.}

\subsection{Используемые понятия}
% ===== sub_1.tex =====
% Вспомогательные определения/теоремы, НЕ являющиеся центральными,
% но необходимые для доказательства (компакт, непрерывность, и т.д.)

\textbf{Компакт в $\mathbb{R}$.}
Множество $K\subset\mathbb{R}$ называется \emph{компактным}, если оно \emph{замкнуто} и \emph{ограничено}.

\medskip

\textbf{Определение непрерывной функции (точечное).}
Функция $f$ непрерывна в $x_0$, если
\[
\forall \varepsilon>0\;\exists \delta>0:\;
|x-x_0|<\delta \;\Longrightarrow\;
|f(x)-f(x_0)|<\varepsilon.
\]
Если это верно для каждой точки множества $X$, говорят, что $f$ непрерывна на $X$.

\medskip

\textbf{Теорема о сходящейся подпоследовательности (Больцано--Вейерштрасса).}
Всякая \emph{ограниченная} последовательность в $\mathbb{R}$ имеет \emph{сходящуюся} подпоследовательность.  
Если $(x_n)$ лежит в компактном $K$, то любая подпоследовательность $(x_{n_k})$ имеет сходящуюся подпоследовательность с пределом в $K$.  


\subsection{Ответ на вопрос}
\textbf{Условие:}
Является ли линейным пространством множество матриц \( 2 \times 2 \) со следом \( 1 \)?

\textbf{Решение:}
Рассмотрим множество всех матриц \( 2 \times 2 \) со следом 1:

\[
V = \{ A \in M_{2 \times 2} \mid \text{tr}(A) = 1 \}
\]

где \( \text{tr}(A) \) обозначает след матрицы \( A \), т.е. сумму элементов на главной диагонали.

Чтобы проверить, является ли \( V \) линейным пространством, проверим основные свойства линейного пространства: замкнутость относительно сложения и умножения на скаляр.

\textbf{Проверка замкнутости относительно сложения:}  
Если \( A, B \in V \), то их сумма \( A + B \) должна также принадлежать \( V \), то есть иметь след, равный 1.

\[
\text{tr}(A + B) = \text{tr}(A) + \text{tr}(B).
\]

Так как по условию \( \text{tr}(A) = 1 \) и \( \text{tr}(B) = 1 \), то:

\[
\text{tr}(A + B) = 1 + 1 = 2.
\]

Но \( 2 \neq 1 \), следовательно, \( A + B \notin V \), и множество \( V \) не замкнуто относительно сложения.

\textbf{Проверка замкнутости относительно умножения на скаляр:}  
Пусть \( A \in V \) и \( \lambda \) — произвольное число. Тогда проверим, принадлежит ли \( \lambda A \) множеству \( V \):

\[
\text{tr}(\lambda A) = \lambda \text{tr}(A) = \lambda \cdot 1 = \lambda.
\]

Так как при произвольном \( \lambda \) значение \( \text{tr}(\lambda A) \) может быть любым, оно не обязательно равно 1. Следовательно, \( \lambda A \notin V \), и множество \( V \) не замкнуто относительно умножения на скаляр.

\textbf{Вывод:}  
Так как множество \( V \) не замкнуто ни относительно сложения, ни относительно умножения на скаляр, оно \textbf{не является линейным пространством}.



\newpage
\section{Дифференциал функции. Теорема Ферма. Теорема Ролля. Примеры.}

\subsection{Используемые понятия}
% ===== sub_2.tex =====
% Вспомогательные факты, не являющиеся центральными в основном вопросе
% но используемые в доказательствах (локальный экстремум, теорема Вейерштрасса и т.д.)

\textbf{Локальный экстремум.}
Точка $x_0$ называется \emph{локальным минимумом} (соответственно, максимумом), если существует $\delta>0$ такое, что для всех $x$ из $(x_0-\delta, x_0+\delta)$ выполняется $f(x)\ge f(x_0)$ (или $f(x)\le f(x_0)$ для максимума).

\medskip

\textbf{Теорема Вейерштрасса (о достижении экстремума).}
Если $f$ непрерывна на отрезке $[a,b]$, то $f$ достигает на нём своих наибольшего и наименьшего значений. Формально:
\[
\exists x_{\min}, x_{\max}\in[a,b]:\quad
f(x_{\min}) \le f(x) \le f(x_{\max})
\quad
\forall x \in [a,b].
\]

\textbf{Производная в точке.}
Напомним, $f'(x_0)$ есть предел
\[
\lim_{x\to x_0}\frac{f(x)-f(x_0)}{x - x_0},
\]
если этот предел конечен.


\subsection{Ответ на вопрос}
% ===== 2.tex =====
% "Дифференциал функции. Теорема Ферма. Теорема Ролля. Примеры."

% 1. Определения

\textbf{Дифференциал функции.}
Пусть функция $f$ определена в окрестности точки $x_0$ и дифференцируема в $x_0$. Тогда \emph{дифференциалом} $df(x_0)$ функции $f$ в точке $x_0$ называется величина
\[
df(x_0) \;=\; f'(x_0)\,(x - x_0),
\]
где $dx = x - x_0$ считается «малым» приращением аргумента. В более общем смысле при малом $dx$ пишут
\[
df(x_0) \;=\; f'(x_0)\,dx.
\]
Это выражает, что приращение функции раскладывается на линейную часть $df(x_0)$ и малую «остаточную» часть $o(dx)$:
\[
f(x_0 + dx) - f(x_0) \;=\; df(x_0) + o(dx).
\]

\medskip

% 2. Теоремы и ключевые утверждения

\textbf{Теорема Ферма (о локальном экстремуме).}
Если функция $f$ дифференцируема в точке $x_0$ и имеет там локальный минимум или максимум, то
\[
f'(x_0) \;=\; 0.
\]

\textbf{Теорема Ролля.}
Пусть $f$ удовлетворяет трём условиям:
\begin{enumerate}
  \item Непрерывна на отрезке $[a,b]$;
  \item Дифференцируема на интервале $(a,b)$;
  \item При этом $f(a) = f(b)$.
\end{enumerate}
Тогда существует хотя бы одна точка $c \in (a,b)$ такая, что
\[
f'(c) = 0.
\]

\medskip

% 3. Основные идеи доказательств (коротко)

\textbf{Идея доказательства Теоремы Ферма.}
\begin{itemize}
  \item Предположим, $x_0$ — точка локального минимума. Тогда для $x$ вблизи $x_0$ имеем $f(x)\ge f(x_0)$.  
  \item Рассматриваем разность $\frac{f(x)-f(x_0)}{x - x_0}$ при $x > x_0$ и при $x < x_0$.  
  \item Переходя к пределу $x\to x_0^+$ и $x\to x_0^-$, получаем $f'(x_0)\ge0$ и $f'(x_0)\le0$ соответственно. Значит $f'(x_0)=0$.  
  \item Для локального максимума аналогично.
\end{itemize}

\textbf{Идея доказательства Теоремы Ролля.}
\begin{itemize}
  \item Если $f$ постоянна на $[a,b]$, то $f'(x)=0$ на $(a,b)$ и теорема доказана.  
  \item Если $f$ не постоянна, по \textbf{теореме о достижении экстремума} (см. \texttt{sub\_2.tex}) она достигает минимума и максимума в некотором $x_{\min}, x_{\max}\in[a,b]$.  
  \item Поскольку $f(a)=f(b)$, хотя бы один из экстремумов не может быть только на концах. Тогда внутри $(a,b)$ есть точка локального экстремума $c$.  
  \item По Теореме Ферма, $f'(c)=0$.
\end{itemize}

\medskip

% 4. Полное доказательство (шаги, ссылаясь на определения)

\textbf{Доказательство Теоремы Ферма (пошагово):}
\begin{enumerate}
  \item Пусть $x_0$ — точка локального минимума, то есть существует $\delta>0$ такое, что при $|x-x_0|<\delta$, $f(x)\ge f(x_0)$.  
  \item Для $x>x_0$ рассмотрим $\frac{f(x)-f(x_0)}{x - x_0}\ge0$. При переходе $x\to x_0^+$, этот предел есть $f'(x_0)\ge0$.  
  \item Для $x<x_0$ аналогично, но тогда $x - x_0<0$, и из неравенства $f(x)\ge f(x_0)$ получаем $f'(x_0)\le0$.  
  \item Значит $f'(x_0)\ge0$ и $f'(x_0)\le0$, откуда $f'(x_0)=0$.  
  \item Случай локального максимума разбирается аналогично (только знак меняется).
\end{enumerate}

\textbf{Доказательство Теоремы Ролля (пошагово):}
\begin{enumerate}
  \item Если $f$ постоянна на $[a,b]$, то $f'(x)=0$ на всём $(a,b)$, и точка $c$ может быть любая.  
  \item Иначе $f$ непостоянна и, благодаря непрерывности на $[a,b]$, достигает минимума и максимума (Теорема Вейерштрасса, см. \texttt{sub\_2.tex}).  
  \item Пусть $x_{\min}$ и $x_{\max}$ — точки, где достигаются минимум и максимум. Поскольку $f(a)=f(b)$, как минимум одно из этих значений будет «внутренним» для отрезка, иначе функция была бы постоянно равна этому значению. Значит есть $c\in(a,b)$ — точка локального экстремума.  
  \item По Теореме Ферма, в точке локального экстремума $c$ имеем $f'(c)=0$.  
  \item Следовательно, нашли искомую точку $c\in(a,b)$, что доказывает Теорему Ролля.
\end{enumerate}

\medskip

% 5. Примеры

\textbf{Пример (дифференциал).}
Для $f(x)=x^2$, в точке $x_0=2$ имеем $f'(2)=4$. Тогда малому приращению $dx$ соответствует $df(2)=4\,dx$. Если $dx=0.1$, то $df(2)=0.4$, а реальное $f(2.1)-f(2)=4.41-4=0.41$, что близко к $0.4$.

\textbf{Пример (Теорема Ферма).}
Функция $f(x)=x^2$ имеет локальный минимум в $x=0$, причём $f'(0)=0$.

\textbf{Пример (Теорема Ролля).}
На отрезке $[0,4]$ возьмём $f(x)=x^2 - 4x$. Тогда $f(0)=f(4)=0$, $f$ непрерывна на $[0,4]$ и дифференцируема на $(0,4)$. По Теореме Ролля существует $c\in(0,4)$ с $f'(c)=0$. Действительно, $f'(x)=2x-4$, отсюда $c=2$.

\medskip

% -- Логическая связность --
% (Все определения локального экстремума, непрерывности,
% теоремы Вейерштрасса - во вспомогательном sub_2.tex)




\newpage
\section{Теорема Лагранжа. Необходимoе и достаточное условие постоянства дифференцируемой функции на промежутке. Необходимое и достаточное условие монотонности дифференцируемой функции на промежутке.}

\subsection{Используемые понятия}
% ===== sub_3.tex =====
% Вспомогательный файл: содержит факты/определения, не являющиеся
% центральной частью вопроса, но нужные в доказательствах.

\textbf{Теорема Ролля (напоминание).}
Пусть $f$ непрерывна на $[a,b]$, дифференцируема на $(a,b)$ и $f(a)=f(b)$. Тогда существует точка $c\in(a,b)$, где $f'(c)=0$.

\textbf{Определение дифференцируемости (напоминание).}
Функция $f$ называется дифференцируемой в точке $x_0$, если существует конечный предел
\[
f'(x_0) = \lim_{x\to x_0} \frac{f(x)-f(x_0)}{x - x_0}.
\]
Если $f'(x_0)$ существует для всех $x_0$ в $(a,b)$, говорят, что $f$ дифференцируема на $(a,b)$.

\textbf{Определение монотонности (напоминание).}
Функция $f$ называется \emph{возрастающей} на $(a,b)$, если для любых $x_1,x_2\in(a,b)$, при $x_1 < x_2$ выполняется $f(x_1)\le f(x_2)$ (нестрого) или $f(x_1)<f(x_2)$ (строго).  
Аналогично определяется убывание.

\textbf{Определение постоянной функции.}
Функция $f$ называется постоянной на $(a,b)$, если $f(x_1)=f(x_2)$ для любых $x_1,x_2\in(a,b)$.

% (Другие напоминания (о непрерывности, ...) при необходимости).


\subsection{Ответ на вопрос}
% ===== 3.tex =====
% "Теорема Лагранжа. 
%  Необходимое и достаточное условие постоянства дифференцируемой функции.
%  Необходимое и достаточное условие монотонности дифференцируемой функции."

% 1. Определения (те, что непосредственно связаны с вопросом)

\textbf{Теорема Лагранжа (о среднем значении).}
Пусть функция $f$ \emph{непрерывна} на отрезке $[a,b]$ и \emph{дифференцируема} на интервале $(a,b)$. Тогда существует точка $c \in (a,b)$ такая, что
\[
f(b) - f(a) \;=\; f'(c)\,\bigl(b - a\bigr).
\]
(\emph{Остальные определения см. в \texttt{sub\_3.tex} — например, производная, непрерывность, и т.п.})

\medskip

\textbf{Необходимое и достаточное условие \underline{постоянства} дифференцируемой функции.}
Пусть функция $f$ дифференцируема на промежутке $(a,b)$. Тогда она \emph{постоянна} на $(a,b)$ \textbf{тогда и только тогда}, когда
\[
f'(x) \;=\; 0 \quad\text{для всех}\; x \in (a,b).
\]

\medskip

\textbf{Необходимое и достаточное условие \underline{монотонности} дифференцируемой функции.}
Пусть $f$ дифференцируема на промежутке $(a,b)$. Тогда верны следующие утверждения:

\begin{itemize}
  \item Функция $f$ \textbf{возрастает} на $(a,b)$ \(\Longleftrightarrow\) \(f'(x)\ge 0\) для всех \(x\in(a,b)\), причём множество точек, где \(f'(x)=0\), не содержит интервалов.
  \item Функция $f$ \textbf{убывает} на $(a,b)$ \(\Longleftrightarrow\) \(f'(x)\le 0\) для всех \(x\in(a,b)\), причём множество точек, где \(f'(x)=0\), не содержит интервалов.
\end{itemize}

\medskip

% 2. Основные идеи доказательств (коротко)

\textbf{Основная идея доказательства теоремы Лагранжа:}
\begin{itemize}
  \item Эта теорема является обобщением Теоремы Ролля.
  \item Сущность: Рассмотрим функцию $F(x)= f(x) - \alpha x$, где \(\alpha = \frac{f(b)-f(a)}{b-a}\).  
  \item Из условия $F(a)=F(b)$ и непрерывности+дифференцируемости $F$ на $[a,b]$ применяем Теорему Ролля: существует $c \in(a,b)$, где $F'(c)=0$.  
  \item Тогда $F'(c)=f'(c)-\alpha=0\implies f'(c)=\alpha = \frac{f(b)-f(a)}{b-a}$.  
\end{itemize}

\textbf{Идея доказательства условия постоянства:}
\begin{itemize}
  \item Если $f'(x)=0$ повсюду, то по Теореме Лагранжа (или Ролля) разность $f(x_2)-f(x_1)$ всегда равна нулю, значит $f$ постоянна.
  \item Если $f$ постоянна, ясно, что $f'(x)=0$.  
\end{itemize}

\textbf{Идея доказательства условия монотонности:}
\begin{itemize}
  \item Если $f'(x)\ge0$ на $(a,b)$, то при $x_2>x_1$ можно показать $f(x_2)\ge f(x_1)$.  
  \item Обратное: если $f$ возрастает, то для $x_2>x_1$ имеем $\frac{f(x_2)-f(x_1)}{x_2-x_1}\ge0$. Переходя к пределу, получаем $f'(x)\ge 0$.  
  \item Уточнение, что множество точек с $f'(x)=0$ не содержит внутренних отрезков, нужно, чтобы исключить «застревание» функции на целых интервалах.  
\end{itemize}

\medskip

% 3. Полное доказательство (шаг за шагом)

\subsection*{Доказательство Теоремы Лагранжа}

\begin{enumerate}
  \item \textbf{Построение вспомогательной функции.}  
  Пусть $\alpha=\frac{f(b)-f(a)}{b-a}$. Рассмотрим
  \[
    F(x) \;=\; f(x) - \alpha\,x.
  \]
  Тогда $F(a)=f(a)-\alpha a$ и $F(b)=f(b)-\alpha b$. Заметим:
  \[
    F(b)-F(a) \;=\; [f(b)-f(a)] \;-\;\alpha[b - a] \;=\; 0.
  \]
  \item \textbf{Применение Теоремы Ролля.}  
  Функция $F$ непрерывна на $[a,b]$ и дифференцируема на $(a,b)$ (как разность непрерывных и дифференцируемых функций). Причём $F(a)=F(b)$.  
  По Теореме Ролля (см. \texttt{sub\_3.tex} при необходимости) существует $c \in(a,b)$: $F'(c)=0$.  
  \item \textbf{Заключение.}  
  Из $F'(x)=f'(x) - \alpha$ получаем $F'(c)=f'(c)-\alpha=0 \implies f'(c)=\alpha$. Но $\alpha= \tfrac{f(b)-f(a)}{b-a}$, значит
  \[
    f'(c) \;=\; \frac{f(b)-f(a)}{b-a}.
  \]
  На этом доказательство завершается.
\end{enumerate}

\subsection*{Доказательство необходимого и достаточного условия постоянства}

\begin{enumerate}
  \item \textbf{Необходимость (если $f$ постоянна).}  
  Если $f$ есть константа, то для всех $x\in(a,b)$ приращения $f(x_2)-f(x_1)=0$, значит $f'(x)=0$ в любой точке, где она дифференцируема.
  \item \textbf{Достаточность (если $f'(x)=0$ повсюду).}  
  Пусть $f'(x)=0$ на $(a,b)$. Возьмём любые $x_1,x_2\in(a,b)$, причём $x_2>x_1$. По Теореме Лагранжа существует $c\in(x_1,x_2)$ с
  \[
    f(x_2)-f(x_1) \;=\; f'(c)\,(x_2-x_1).
  \]
  Но $f'(c)=0$, значит $f(x_2)=f(x_1)$. Следовательно, $f$ неизменна на всём промежутке.
\end{enumerate}

\subsection*{Доказательство необходимого и достаточного условия монотонности}

\emph{(Рассмотрим случай возрастания; для убывания аналогично меняются знаки.)}

\begin{enumerate}
  \item \textbf{Если $f'(x)\ge 0$ для всех $x$, то $f$ возрастает.}  
  Пусть $x_1<x_2$. По Теореме Лагранжа на отрезке $[x_1,x_2]$ существует $c\in(x_1,x_2)$ такое, что
  \[
    f(x_2)-f(x_1) \;=\; f'(c)\,(x_2 - x_1).
  \]
  Раз $f'(c)\ge 0$ и $x_2-x_1>0$, то $f(x_2)-f(x_1)\ge 0\implies f(x_2)\ge f(x_1)$. Значит $f$ неубывает. Для \emph{строгого} возрастания нужно уточнить, что нет интервалов, где $f'(x)=0$ постоянно.
  \item \textbf{Если $f$ возрастает, то $f'(x)\ge0$.}  
  При $x_2>x_1$ имеем \(\tfrac{f(x_2)-f(x_1)}{x_2-x_1}\ge 0\). Переходя к пределу, получаем $f'(x)\ge0$.
  \item \textbf{Уточнение про множество нулей $f'(x)$.}  
  Если на каком-то подинтервале $f'(x)$ всё время равно нулю, то $f$ там постоянна, что может «ломать» строгое возрастание (если отрезок ненулевой длины). Поэтому для строгой монотонности требуется, чтобы подмножество нулей не содержало интервалов.
\end{enumerate}

\medskip

% 4. Логическая связность и ссылки
% (Все базовые теоремы, такие как теорема Ролля, определение непрерывности
%  см. во вспомогательном sub_3.tex)



\newpage
\section{Равномерная непрерывность. Примеры. Теорема Кантора о равномерной непрерывности.}

\subsection{Используемые понятия}
% ===== sub_4.tex =====
% Вспомогательные определения и теоремы,
% не являющиеся центральными в данном вопросе,
% но используемые в доказательстве теоремы Кантора

\textbf{Определение компакта в $\mathbb{R}$.}
Множество $K\subset\mathbb{R}$ называется \emph{компактным}, если оно \emph{замкнуто} и \emph{ограничено}.

\medskip

\textbf{Теорема Болльцано--Вейерштрасса.}
Всякая \emph{ограниченная} последовательность $(x_n)$ в $\mathbb{R}$ имеет \emph{сходящуюся} подпоследовательность.  
На языке компактных множеств: любая последовательность, целиком лежащая в компактном $K$, содержит сходящуюся подпоследовательность с пределом в $K$.

\medskip

\textbf{Определение непрерывности (подробно).}
Напомним: $f$ непрерывна на $X$, если для любой точки $x_0\in X$:
\[
\forall \varepsilon>0\;\;\exists \delta>0:\;
|x - x_0|<\delta \implies |f(x)-f(x_0)|<\varepsilon.
\]

\medskip

% (Другие напоминания и определения — при необходимости).


\subsection{Ответ на вопрос}
\textbf{Условие:}  
Дана матрица:

\[
A =
\begin{bmatrix}
3 & -1 & 9 & -2 & 6 \\
1 & 3 & 1 & -2 & 3 \\
-1 & 3 & 1 & 2 & -2 \\
-3 & -5 & -6 & 3 & -7 \\
-1 & -2 & -4 & 0 & -2
\end{bmatrix}
\]

a) Найти базис линейной оболочки строк матрицы \( A \).  
b) Найти базис пространства решений системы \( Ax = 0 \).

---

\textbf{Решение}  

\subsection*{1. Нахождение базиса линейной оболочки строк}  

Базис линейной оболочки строк матрицы \( A \) — это линейно независимый набор строк матрицы. Для его нахождения приводим матрицу \( A \) к \textbf{ступенчатому виду} методом элементарных преобразований строк.

\textbf{Шаг 1: Приведение к ступенчатому виду (метод Гаусса)}  

Начнем с исходной матрицы:

\[
A =
\begin{bmatrix}
3 & -1 & 9 & -2 & 6 \\
1 & 3 & 1 & -2 & 3 \\
-1 & 3 & 1 & 2 & -2 \\
-3 & -5 & -6 & 3 & -7 \\
-1 & -2 & -4 & 0 & -2
\end{bmatrix}
\]

\textbf{Шаг 1.1:} Сделаем первый элемент (в позиции \( A_{11} \)) равным 1, поделив первую строку на 3:

\[
\begin{bmatrix}
1 & -\frac{1}{3} & 3 & -\frac{2}{3} & 2 \\
1 & 3 & 1 & -2 & 3 \\
-1 & 3 & 1 & 2 & -2 \\
-3 & -5 & -6 & 3 & -7 \\
-1 & -2 & -4 & 0 & -2
\end{bmatrix}
\]

\textbf{Шаг 1.2:} Обнуляем элементы под первым ведущим элементом (\( A_{21}, A_{31}, A_{41}, A_{51} \)):

- \( R_2 \leftarrow R_2 - R_1 \),
- \( R_3 \leftarrow R_3 + R_1 \),
- \( R_4 \leftarrow R_4 + 3R_1 \),
- \( R_5 \leftarrow R_5 + R_1 \).

После этих преобразований получаем:

\[
\begin{bmatrix}
1 & -\frac{1}{3} & 3 & -\frac{2}{3} & 2 \\
0 & \frac{10}{3} & -2 & -\frac{4}{3} & 1 \\
0 & \frac{8}{3} & 4 & \frac{4}{3} & 0 \\
0 & -\frac{14}{3} & 3 & \frac{5}{3} & -1 \\
0 & -\frac{7}{3} & -1 & -\frac{2}{3} & 0
\end{bmatrix}
\]

\textbf{Шаг 1.3:} Приводим оставшиеся строки к ступенчатому виду, выполняя аналогичные преобразования.

После приведения к ступенчатому виду получаем:

\[
A' =
\begin{bmatrix}
1 & 0 & 3 & 0 & 2 \\
0 & 1 & -2 & 0 & 1 \\
0 & 0 & 0 & 1 & 1 \\
0 & 0 & 0 & 0 & 0 \\
0 & 0 & 0 & 0 & 0
\end{bmatrix}
\]

\textbf{Шаг 2: Определение базиса линейной оболочки строк}  

Базис линейной оболочки строк составляют ненулевые строки приведенной матрицы:

\[
B_{\text{row}} =
\left\{
\begin{bmatrix} 1 & 0 & 3 & 0 & 2 \end{bmatrix},
\begin{bmatrix} 0 & 1 & -2 & 0 & 1 \end{bmatrix},
\begin{bmatrix} 0 & 0 & 0 & 1 & 1 \end{bmatrix}
\right\}.
\]

\textbf{Ответ:}  
Базис линейной оболочки строк матрицы \( A \):

\[
\left\{
\begin{bmatrix} 1 & 0 & 3 & 0 & 2 \end{bmatrix},
\begin{bmatrix} 0 & 1 & -2 & 0 & 1 \end{bmatrix},
\begin{bmatrix} 0 & 0 & 0 & 1 & 1 \end{bmatrix}
\right\}.
\]

---

\subsection*{2. Нахождение базиса пространства решений системы \( Ax = 0 \)}  

Пространство решений системы \( Ax = 0 \) (ядро матрицы \( A \)) содержит все векторы \( x \), удовлетворяющие:

\[
A x = 0.
\]

\textbf{Шаг 1: Запись системы уравнений}  

Из ступенчатой формы матрицы \( A' \) получаем систему:

\[
\begin{cases}
x_1 + 3x_3 + 2x_5 = 0, \\
x_2 - 2x_3 + x_5 = 0, \\
x_4 + x_5 = 0.
\end{cases}
\]

\textbf{Шаг 2: Выражаем переменные через свободные параметры}  

Выберем свободные переменные: \( x_3 = t \), \( x_5 = s \).

\[
\begin{cases}
x_1 = -3t - 2s, \\
x_2 = 2t - s, \\
x_3 = t, \\
x_4 = -s, \\
x_5 = s.
\end{cases}
\]

\textbf{Шаг 3: Представляем решение в виде линейной комбинации}  

\[
x = t \begin{bmatrix} -3 \\ 2 \\ 1 \\ 0 \\ 0 \end{bmatrix} +
s \begin{bmatrix} -2 \\ -1 \\ 0 \\ -1 \\ 1 \end{bmatrix}.
\]

\textbf{Шаг 4: Определение базиса пространства решений}  

Векторы, соответствующие параметрам \( t \) и \( s \), образуют базис пространства решений:

\[
B_{\text{null}} =
\left\{
\begin{bmatrix} -3 \\ 2 \\ 1 \\ 0 \\ 0 \end{bmatrix},
\begin{bmatrix} -2 \\ -1 \\ 0 \\ -1 \\ 1 \end{bmatrix}
\right\}.
\]

\textbf{Ответ:}  
Базис пространства решений \( Ax = 0 \):

\[
\left\{
\begin{bmatrix} -3 \\ 2 \\ 1 \\ 0 \\ 0 \end{bmatrix},
\begin{bmatrix} -2 \\ -1 \\ 0 \\ -1 \\ 1 \end{bmatrix}
\right\}.
\]

---

\textbf{Вывод}
\\1. Базис линейной оболочки строк состоит из трех векторов.  
\\2. Базис пространства решений \( Ax = 0 \) состоит из двух векторов, что соответствует размерности ядра матрицы \( A \), так как \( \operatorname{rank}(A) = 3 \) и \( \operatorname{dim}(\ker A) = 5 - 3 = 2 \).


\newpage
\section{Вывод рядов Тейлора для функций y=exp(x), y=sinx, y=cosx через следствие из теоремы Лагранжа. Формула Эйлера.}

\subsection{Используемые понятия}
% ===== sub_5.tex =====
% Вспомогательный файл: факты, не являющиеся центральными в данном вопросе,
% но нужные при доказательствах (теорема Лагранжа, определение (n+1)-й производной и т.п.)

\textbf{Теорема Лагранжа (о среднем значении).}
Пусть функция $f$ непрерывна на $[0,x]$ (при $x>0$) и дифференцируема на $(0,x)$. Тогда существует $c\in(0,x)$ такое, что
\[
f(x)-f(0) \;=\; f'(c)\,\bigl(x - 0\bigr).
\]

\medskip

\textbf{n-я производная.}
Функция $f$ называется $n$ раз дифференцируемой в точке, если существуют все производные $f'(x_0), f''(x_0),\dots,f^{(n)}(x_0)$. Аналогично в окрестности.

\medskip

\textbf{Определение комплексной экспоненты (для Формулы Эйлера).}
Если $z$ комплексное, $e^z$ определяется как \(\sum_{n=0}^{\infty}\frac{z^n}{n!}\). Это расширяет понятие экспоненты на комплексную область.


\subsection{Ответ на вопрос}
\subsection{Постоянная решётки \(d\)}

Используем табличное значение длины волны зелёной линии
\[
	\lambda_2 = 546 \pm 5\ \text{нм}\quad(P=95\%),
\]
и средний угловой коэффициент
\[
	a_2 = \bar a_2 \pm \Delta\bar a_2,
\]
где \(\bar a_2\) и \(\Delta\bar a_2\) получены в п.~3 (см. Таблицу 5).
Тогда
\[
	d = \frac{\lambda_2}{\lvert \bar a_2\rvert},
	\qquad
	\Delta d = d \,
	\sqrt{\Bigl(\tfrac{\Delta \lambda_2}{\lambda_2}\Bigr)^2
		+\Bigl(\tfrac{\Delta \bar a_2}{\bar a_2}\Bigr)^2}\,.
\]

Результат занесён в таблицу 4.


\newpage
\section{Теорема Коши. Правило Лопиталя (доказательство – только для случая 0/0). Примеры, когда правило неприменимо.}

\subsection{Используемые понятия}
% ===== sub_6.tex =====
% Вспомогательные определения/теоремы, не являющиеся
% "центральными" в данном вопросе, но используемые в доказательствах.

\textbf{Теорема Ролля (напоминание).}
Пусть $f$ непрерывна на отрезке $[p,q]$, дифференцируема на $(p,q)$ и $f(p)=f(q)$. Тогда существует точка $c\in(p,q)$, где $f'(c)=0$.

\medskip

\textbf{Определение: проколотая окрестность.}
Говорят, что функция $f$ дифференцируема (или определена) в проколотой окрестности точки $a$, если существует $\delta>0$ такое, что при $0<|x-a|<\delta$, $f(x)$ (и её производная) определена. При этом в самой точке $a$ она может быть не определена или не дифференцируема.

\medskip

\textbf{Неопределённости вида $0/0$ и $\infty/\infty$.}
Правило Лопиталя распространяется на случаи, когда \(\lim_{x\to a}f(x)=\lim_{x\to a}g(x)=0\) или \(\pm\infty\). В остальных случаях правило не даёт результата.



\subsection{Ответ на вопрос}
% ===== 6.tex =====
% "Теорема Коши. Правило Лопиталя (случай 0/0). Примеры, когда правило неприменимо."

% 1. Определения

\textbf{Теорема Коши (обобщённая теорема Лагранжа).}
Пусть функции $f(x)$ и $g(x)$ удовлетворяют условиям:
\begin{itemize}
  \item непрерывны на $[a,b]$,
  \item дифференцируемы на $(a,b)$,
  \item $g'(x)\neq 0$ для всех $x\in(a,b)$.
\end{itemize}
Тогда существует точка $c\in(a,b)$ такая, что
\[
\frac{f(b)-f(a)}{g(b)-g(a)} \;=\; \frac{f'(c)}{g'(c)}.
\]

\medskip

\textbf{Правило Лопиталя (только для случая \(\tfrac{0}{0}\)).}
Пусть функции $f(x)$ и $g(x)$ дифференцируемы в \emph{проколотой} окрестности точки $a$ (то есть всюду, кроме, возможно, самой точки $a$). Предположим:
\[
\lim_{x\to a} f(x) = 0, \quad \lim_{x\to a} g(x)=0,
\]
и $g'(x)\neq 0$ в этой окрестности. Если существует (конечный или бесконечный) предел
\[
\lim_{x\to a} \frac{f'(x)}{g'(x)} = L,
\]
то существует и предел \(\lim\limits_{x\to a}\frac{f(x)}{g(x)}\), причём
\[
\lim_{x\to a}\frac{f(x)}{g(x)} = L.
\]

\medskip

% 2. Теоремы и ключевые утверждения

\textbf{Основная идея Теоремы Коши:}
\begin{itemize}
  \item Является обобщением Теоремы Лагранжи (достаточно взять $g(x)=x$).  
  \item Используется для доказательства Правила Лопиталя: рассматриваем $f(x), g(x)$ и применяем теорему Коши к $[a,x]$.
\end{itemize}

\textbf{Основная идея Правила Лопиталя (0/0):}
\begin{itemize}
  \item Рассматриваем отношение $\frac{f(x)}{g(x)}$ при $x\to a$.  
  \item Применяем теорему Коши к функциям $F(t)=f(t)-f(a)$, $G(t)=g(t)-g(a)$ на отрезке $[a,x]$.  
  \item Переходя к пределу, получаем $\lim_{x\to a}\frac{f(x)}{g(x)}=\lim_{x\to a}\frac{f'(x)}{g'(x)}$, если последний существует.
\end{itemize}

\medskip

% 3. Полное доказательство (детально, отсылая к sub_6 при необходимости)

\subsection*{Доказательство Теоремы Коши (обобщённая Лагранжа)}

\begin{enumerate}
  \item \textbf{Условие:} $f, g$ непрерывны на $[a,b]$, дифференцируемы на $(a,b)$, причём $g'(x)\neq0$.  
  \item \textbf{Построение.} Рассмотрим функцию
    \[
      \Phi(t) \;=\; f(t) - f(a) \;-\; \frac{f(b)-f(a)}{g(b)-g(a)}\,\bigl[g(t)-g(a)\bigr].
    \]
    Тогда $\Phi(a)=0$ и \(\Phi(b)=f(b)-f(a)-\tfrac{f(b)-f(a)}{g(b)-g(a)}[\,g(b)-g(a)\,]=0\).  
  \item \textbf{Применение Теоремы Ролля (см. \texttt{sub\_6.tex}):}  
    Поскольку $\Phi$ непрерывна на $[a,b]$ и дифференцируема на $(a,b)$ (как разность таковых), и $\Phi(a)=\Phi(b)=0$, существует $c\in(a,b)$ с \(\Phi'(c)=0\).  
  \item \textbf{Производная:} 
    \[
    \Phi'(t)=f'(t) - \frac{f(b)-f(a)}{g(b)-g(a)}\,g'(t).
    \]
    Тогда $\Phi'(c)=0 \implies f'(c)=\dfrac{f(b)-f(a)}{g(b)-g(a)}\,g'(c)$.  
  \item \textbf{Следствие:} 
    \[
      \frac{f(b)-f(a)}{g(b)-g(a)} = \frac{f'(c)}{g'(c)}.
    \]
\end{enumerate}

\medskip

\subsection*{Доказательство Правила Лопиталя (случай $0/0$)}

\begin{enumerate}
  \item \textbf{Условие:} $\lim_{x\to a}f(x)=0,\; \lim_{x\to a}g(x)=0,\; g'(x)\neq0$. Предположим, что существует $\lim_{x\to a}\frac{f'(x)}{g'(x)}=L$.  
  \item \textbf{Рассмотрим $f$ и $g$ на отрезке $[a,x]$ (при $x>a$).}  
    По условию $f(a)=g(a)=0$. Применяем Теорему Коши к $f$ и $g$ на $[a,x]$:
    \[
      \exists\,c_x\in(a,x): \quad \frac{f(x)-f(a)}{g(x)-g(a)} = \frac{f'(c_x)}{g'(c_x)}.
    \]
  \item \textbf{Переход к пределу:}  
    Поскольку $f(a)=0$, $g(a)=0$, имеем
    \[
      \frac{f(x)}{g(x)} = \frac{f'(c_x)}{g'(c_x)}.
    \]
    Когда $x\to a$, точка $c_x$ тоже $\to a$ (так как $c_x$ лежит между $a$ и $x$). Если \(\lim_{x\to a}\frac{f'(x)}{g'(x)}=L\), то \(\frac{f'(c_x)}{g'(c_x)}\to L\).  
    Значит
    \[
      \lim_{x\to a}\frac{f(x)}{g(x)} \;=\; L.
    \]
\end{enumerate}

\medskip

% 4. Примеры, когда правило Лопиталя НЕприменимо

\begin{itemize}
  \item \textbf{Если нет неопределённости 0/0 или $\infty/\infty$.}  
    Пример: $\lim_{x\to 0}\frac{\sin x}{x+1}=\frac{0}{1}=0$ — тут нет неопределённости, применять Лопиталя смысла нет.
  \item \textbf{Если $\lim_{x\to a}\frac{f'(x)}{g'(x)}$ не существует.}  
    Пример: $\lim_{x\to 0}\frac{\sin(1/x)}{1/x}$:  
    - $f(x)=\sin(1/x)$, $g(x)=1/x$. При $x\to0$, $f'(x)$ и $g'(x)$ ведут себя крайне нестабильно. Предел $\frac{f'(x)}{g'(x)}$ не существует.  
  \item \textbf{Если $f$ или $g$ не дифференцируемы в проколотой окрестности точки.}  
    Пример: $f(x)=|x|$, $g(x)=x$ при $x\to 0$ не дифференцируемо в $0$.  
\end{itemize}

\medskip

% -- Логическая связность --
% (Определение "проколотой окрестности", "теоремы Ролля" и т.д. в sub_6.tex).


\newpage
\section{Формула Тейлора для многочлена. Формула Тейлора с остатком в форме Пеано.}

\subsection{Используемые понятия}
% ===== sub_7.tex =====
% Вспомогательный файл: определения/факты, не являющиеся центральной частью вопроса,
% но нужные при доказательствах (например, "n-кратная дифференцируемость", "o( (x-x0)^n )" и т. п.)

\textbf{Определение n-кратной дифференцируемости.}
Функция $f$ называется $n$ раз дифференцируемой в точке $x_0$, если существуют конечные все производные $f'(x_0), f''(x_0), \dots, f^{(n)}(x_0)$.

\medskip

\textbf{Малое «о» и запись $o(g(x))$.}
Говорят, что $h(x)$ есть $o\bigl(g(x)\bigr)$ при $x\to x_0$, если
\[
\lim_{x\to x_0}\frac{h(x)}{g(x)} = 0.
\]
В таком случае пишут $h(x)=o(g(x))$.

\medskip

\textbf{Остаточный член в форме Лагранжа (напоминание).}
Если $f$ имеет $(n+1)$-ю производную в окрестности $x_0$, то для
\[
f(x) = f(x_0) + \dots + \frac{f^{(n)}(x_0)}{n!}(x - x_0)^n + R_n(x)
\]
справедливо
\[
R_n(x) = \frac{f^{(n+1)}(\xi)}{(n+1)!}(x - x_0)^{n+1},\quad \xi \in (x_0,x).
\]



\subsection{Ответ на вопрос}
\subsection{Определение угловой дисперсии \(D_{\varphi}\)}

Угловая дисперсия решётки определяется из уравнения (4.5):
\[
	D_{\varphi} \;=\;\frac{d\varphi}{d\lambda}
	\;=\;\frac{m}{d\,\cos\varphi_m}\,,
\]
где \(\varphi_m\) — угловое смещение для порядка \(m\), \(d\) — постоянная решётки (в тех же единицах, что и длина волны \(\lambda\)).
В отчёте \(D_{\varphi}\) выражается в угловых единицах (минутах) на нанометр. Для этого результат переводят из радиан на минуты:
\[
	D_{\varphi}\bigl[\tfrac{\text{угл. мин}}{\text{нм}}\bigr]
	= \frac{m}{d\cos\varphi_m}\times\frac{180\cdot60}{\pi}\,.
\]

Рассчитанные значения занесены в Таблицу 5.


\newpage
\section{Достаточные условия существования экстремума (по второй производной).}

\subsection{Используемые понятия}
% ===== sub_8.tex =====
% Вспомогательные определения/теоремы, не являющиеся центральными в данном вопросе,
% но требующиеся в доказательствах (например, теорема Ферма).

\textbf{Теорема Ферма (напоминание).}
Если $f$ дифференцируема в точке $x_0$ и имеет там локальный экстремум, то $f'(x_0)=0$.

\medskip

\textbf{Определение непрерывности второй производной.}
Если $f''(x)$ существует в некоторой окрестности $x_0$ и является непрерывной в $x_0$, то говорят, что $f$ имеет \emph{непрерывную вторую производную} в $x_0$.

\medskip

\textbf{Дифференцируемость в $(a,b)$.}
Говорят, что $f$ дифференцируема на интервале $(a,b)$, если $f'(x)$ существует для всех $x\in(a,b)$.


\subsection{Ответ на вопрос}
% ===== 8.tex =====
% "Достаточные условия существования экстремума (по второй производной)."

% 1. Определения

\textbf{Локальный минимум и максимум.}
Точка $x_0$ внутри промежутка $(a,b)$ называется \emph{точкой локального минимума} функции $f$, если существует $\delta>0$, что для всех $x$ с $|x-x_0|<\delta$ выполняется
\[
f(x)\;\ge\;f(x_0).
\]
Аналогично, $x_0$ является \emph{точкой локального максимума}, если в некоторой окрестности $x_0$ значение $f(x)\le f(x_0)$.

\medskip

\textbf{Вторая производная.}
Если функция $f$ дифференцируема на $(a,b)$, и $f'(x)$ тоже дифференцируема на $(a,b)$, то в точках, где это возможно, определена \emph{вторая производная} $f''(x)$.

\medskip

% 2. Теоремы и ключевые утверждения

\textbf{Теорема (достаточные условия экстремума по второй производной).}
Пусть $f$ дифференцируема на $(a,b)$ и $x_0 \in (a,b)$ — такая точка, где $f'(x_0)=0$. Предположим, что у $f$ существует непрерывная в $x_0$ вторая производная $f''(x_0)$. Тогда:
\begin{enumerate}
  \item Если $f''(x_0)>0$, то $x_0$ — точка \textbf{локального минимума}.
  \item Если $f''(x_0)<0$, то $x_0$ — точка \textbf{локального максимума}.
  \item Если $f''(x_0)=0$, вывод не делается (нужен дополнительный анализ).
\end{enumerate}

\medskip

% 3. Основные идеи доказательства (коротко)

\begin{itemize}
  \item Ключевой «трюк»: если $f''(x_0)>0$, то $f'(x)$ возрастает вблизи $x_0$. Но при $x_0$ мы имеем $f'(x_0)=0$.  
  \item Следовательно, для $x>x_0$, $f'(x)$ становится положительной (или почти), а для $x<x_0$ — отрицательной (или почти), что даёт локальный минимум.  
  \item Аналогично, если $f''(x_0)<0$, $f'(x)$ убывает вблизи $x_0$, и $x_0$ — локальный максимум.
\end{itemize}

\medskip

% 4. Полное доказательство (по шагам, ссылаясь на sub_8 при необходимости)

\textbf{Доказательство достаточности (подробно):}

\begin{enumerate}
  \item \textbf{Наличие $f'(x_0)=0$.}  
    По \textbf{теореме Ферма} (см. \texttt{sub\_8.tex}), это условие часто является необходимым для экстремума. Мы рассматриваем \emph{достаточность}, когда вдобавок знаем про вторую производную.
  \item \textbf{Случай $f''(x_0)>0$.}  
    \begin{itemize}
      \item Из непрерывности $f''$ в $x_0$ вытекает, что при $x$ достаточно близком к $x_0$, вторая производная $f''(x)$ «сохраняет» тот же знак (положительный).  
      \item Значит $f'(x)$ строго возрастает вблизи $x_0$. Но $f'(x_0)=0$.  
      \item Тогда для $x>x_0$ с $x$ рядом с $x_0$, $f'(x)$ станет положительной, а для $x<x_0$ — отрицательной.  
      \item Следовательно, при $x>x_0$, $f$ возрастает, при $x<x_0$, $f$ убывает. Значит $x_0$ — локальный минимум.
    \end{itemize}
  \item \textbf{Случай $f''(x_0)<0$.}  
    \begin{itemize}
      \item Аналогичные рассуждения: $f'(x)$ вблизи $x_0$ будет убывать, и при $x>x_0$ произведёт знак отрицательный, а при $x<x_0$ — знак положительный (около $x_0$).  
      \item Следовательно, слева функция возрастает, а справа убывает. Точка $x_0$ — локальный максимум.
    \end{itemize}
  \item \textbf{Случай $f''(x_0)=0$.}  
    \begin{itemize}
      \item Из этого факта \emph{нельзя} вывести строгое заключение об экстремуме: нужны дополнительные рассуждения (см. примеры $x^3$, $x^4$).
    \end{itemize}
\end{enumerate}

\medskip

% 5. Примеры

\begin{itemize}
  \item \emph{Функция $f(x)=x^2$:}  
  $f'(0)=0$, $f''(0)=2>0$. По теореме, $x=0$ — локальный минимум (что верно).
  \item \emph{Функция $f(x)=x^3$:}  
  $f'(0)=0$, $f''(0)=0$. По рассматриваемой теореме вывод не делается, и действительно $x=0$ — точка перегиба, \textit{не} экстремум.
  \item \emph{Функция $f(x)=-x^2$:}  
  $f'(0)=0$, $f''(0)=-2<0$. Значит в $x=0$ локальный максимум.
\end{itemize}

\medskip

% -- Логическая связность (определения в sub_8.tex)


\newpage
\section{Теорема Лиувилля. Пример трансцендентного числа.}

\subsection{Используемые понятия}
% ===== sub_9.tex =====
% Вспомогательные определения и факты,
% не являющиеся "центральными" в вопросе, но используемые в доказательстве

\textbf{Минимальный многочлен алгебраического числа.}
Пусть $\alpha$ — алгебраическое (корень целого ненулевого многочлена). Его \emph{минимальным многочленом} называется многочлен $P(x)\in \mathbb{Z}[x]$, у которого $\alpha$ — корень, степень $P$ — наименьшая возможная, и старший коэффициент положителен, а все общие делители коэффициентов равны 1.

\medskip

\textbf{Оценка "разности корней".}
Из теории алгебраических уравнений известно, что если $P(x)$ — многочлен степени $d$ с целыми коэффициентами, то расстояния между его корнями не могут быть «слишком маленькими» относительно высоты коэффициентов. Точнее, если $r_1,\dots,r_d$ — корни, то существуют нижние границы $|r_i-r_j|$ в зависимости от коэффициентов $P$ (см. теорему о разложении в произведение линейных множителей).

\medskip

\textbf{Рациональное приближение.}
Если $\alpha$ действительно алгебраична степени $d$, то для больших $q$ рациональные приближения $\frac{p}{q}$ не могут удовлетворять $|\alpha - \frac{p}{q}| < \tfrac{1}{q^n}$ при $n>d$, иначе возникнет противоречие (Теорема Лиувилля).


\subsection{Ответ на вопрос}
% ===== 9.tex =====
% "Теорема Лиувилля. Пример трансцендентного числа."

% 1. Определения

\textbf{Алгебраическое и трансцендентное число.}
\begin{itemize}
  \item \emph{Алгебраическое число} — это действительное (или комплексное) число, являющееся корнем многочлена с рациональными (или целыми) коэффициентами. Например, $\sqrt{2}$, $\sqrt[3]{7}$.
  \item \emph{Трансцендентное число} — это число, \textbf{не} являющееся алгебраическим. Примеры: $e$, $\pi$ (доказано Линдеманом), а также более специальные конструкции (числа Лиувилля).
\end{itemize}

\textbf{Приближение чисел рациональными дробями.}
Для действительного числа $\alpha$ говорят, что \emph{оно допускает «слишком хорошие» рациональные приближения}, если существуют бесконечные наборы дробей $\tfrac{p}{q}$, удовлетворя 
\[
\left|\alpha - \frac{p}{q}\right| \;<\; \frac{1}{q^n}
\]
для больших $q$, при некоторых $n$ существенно превосходящих $1$.

\medskip

% 2. Теоремы и ключевые утверждения

\textbf{Теорема Лиувилля (о трансцендентных числах).}
Если действительное число $\alpha$ удовлетворяет следующему условию: существует $n>1$ и бесконечно много рациональных дробей $\tfrac{p}{q}$, для которых
\[
\left|\alpha - \frac{p}{q}\right| \;<\; \frac{1}{q^n},
\]
то число $\alpha$ \textbf{не} алгебраично (то есть оно \textbf{трансцендентно}).

\medskip

% 3. Основные идеи доказательства (коротко)

\begin{itemize}
  \item Предположим противное: $\alpha$ — алгебраическое, но допускает «чрезмерно точные» рациональные приближения.  
  \item Рассматривается соответствующий \emph{минимальный многочлен} числа $\alpha$ степени $d$.  
  \item Показывается, что при достаточно хороших приближениях противоречат оценкам, вытекающим из теоремы о том, как далеко корни полинома могут быть друг от друга.  
  \item Возникает противоречие, откуда делается вывод: число $\alpha$ не может быть алгебраическим, оно — трансцендентно.
\end{itemize}

\medskip

% 4. Полное доказательство (пошагово, ссылаясь на sub_9, где нужно)

\textbf{Доказательство (классический эскиз):}
\begin{enumerate}
  \item \textbf{Предположение.} Пусть $\alpha$ — корень целого многочлена $P(x)$ степени $d$. Считаем $P(x)$ приведённым (нет общих делителей).  
  \item \textbf{Рациональные приближения.} Допустим, существуют бесконечно многие $\frac{p}{q}$ с $|q\alpha - p| < q^{\,1-n}$, то есть $|\alpha - p/q|<1/q^n$ при $n>d$.  
  \item \textbf{Оценка многочлена.}  
    Рассмотрим $P\bigl(\tfrac{p}{q}\bigr)$; используя замену $x=\tfrac{p}{q}$ и разложение $P(\alpha)=0$, анализируют величину $|P(\tfrac{p}{q})-P(\alpha)|$.  
  \item \textbf{Неравенство:}  
    Т.к. $P(x)$ — многочлен степени $d$, разность $|P(x)-P(\alpha)|$ может быть оценена через $|x-\alpha|$, где в высших степенях играют роль биномиальные формулы, а коэффициенты — целые.  
  \item \textbf{Противоречие:}  
    При «слишком» быстром убывании $|x-\alpha| < 1/q^n$ с $n>d$, получается невозможная малая оценка для $|P(p/q)|$, хотя $p/q$ — рациональная точка, где $P$ должно принимать вполне «ограниченное снизу» значение (не равное нулю, раз $p/q\neq \alpha$).  
  \item \textbf{Итог.}  
    Противоречие доказывает, что $\alpha$ не может быть алгебраическим. Следовательно, $\alpha$ — трансцендентное.
\end{enumerate}

\medskip

% 5. Пример трансцендентного числа: "число Лиувилля"

\textbf{Пример: число Лиувилля.}
Классический пример: 
\[
\beta = \sum_{k=1}^{\infty} 10^{-k!}
\;=\; 0{.}110001000000000000000001\dots
\]
Здесь в десятичной записи стоят единицы на позициях $1!,\,2!,\,3!,\dots$ и нули в остальных. Нетрудно проверить, что для любого $n$ можно найти рациональную дробь $p/q$ с $q=10^{n!}$, которая приближает $\beta$ с точностью $1/q^n$. По \textbf{теореме Лиувилля}, такое число $\beta$ \emph{трансцендентно}.

\medskip

% -- Логическая связность --
% (Определения "минимального многочлена", "степени" и т.д. в sub_9.tex)


\newpage
\section{Формулы Маклорена для функций y=exp(x), y=sinx, y=cosx, y=ln(1+x), y=pow((1+x),a).}

\subsection{Используемые понятия}
% ===== sub_10.tex =====
% Вспомогательные определения и факты,
% не являющиеся "центральными" в вопросе,
% но требующиеся при доказательствах (n-я производная, radius of convergence и т.п.)

\textbf{Определение n-й производной.}
Если функция $f$ $n$ раз дифференцируема в окрестности 0, то $f^{(n)}(0)$ есть её $n$-я производная в точке 0.

\medskip

\textbf{Радиус сходимости степенного ряда.}
Ряд $\sum_{n=0}^{\infty} c_n x^n$ имеет некий \emph{радиус сходимости} $R$, $0\le R\le \infty$, где ряд сходится при $|x|<R$ и расходится (как правило) при $|x|>R$.

\medskip

\textbf{Бином Ньютона (обобщённый).}
Для вещественного $a$ при $|x|<1$:
\[
(1+x)^a = \sum_{n=0}^{\infty} \binom{a}{n}\,x^n,
\]
где $\displaystyle \binom{a}{n} = \frac{a(a-1)\dots(a-n+1)}{n!}$.

\medskip

\textbf{Формула Тейлора (общая).}
При разложении $f(x)$ в окрестности 0 с учётом всех производных получаем ряд (если сходится) называемый рядом Маклорена, частный случай ряда Тейлора.


\subsection{Ответ на вопрос}
\subsection*{Минимальный различимый интервал \(\Delta\lambda\)}

Минимальный интервал длин волн, который может разрешить решётка, рассчитывается по критерию Рэлея:
\[
	\Delta\lambda = \frac{\lambda}{R},
\]
где \(\lambda\) — средняя длина волны линии, \(R\) — разрешающая способность (см. п.~9).
Результаты приведены в Таблице 5.


\newpage
\section{Формула Тейлора с остатком в форме Лагранжа. Приближенные вычисления по формуле Тейлора.}

\subsection{Используемые понятия}
% ===== sub_11.tex =====
% Вспомогательный файл: определения и теоремы, не являющиеся центральными
% в вопросе, но используемые при доказательствах (Теорема Коши/Ролля в обобщённом виде, ...
% понятие n-кратной производной).

\textbf{Обобщённая теорема Ролля (Теорема Коши).}
Если $f$ и $g$ непрерывны на $[a,b]$, дифференцируемы на $(a,b)$, и $g'(x)\neq 0$ на $(a,b)$, то существует $c\in(a,b)$:
\[
\frac{f(b)-f(a)}{g(b)-g(a)} = \frac{f'(c)}{g'(c)}.
\]
При удачном выборе «вспомогательных» функций подстановка даёт нужный результат о $F^{(n+1)}(\xi)=0$.

\medskip

\textbf{n-кратная производная в точке.}
Если $f$ дифференцируема $n$ раз в окрестности $x_0$, мы обозначаем $f^{(n)}(x_0)$ как производную $n$-го порядка, если та существует и непрерывна.

\medskip

\textbf{Пример применения индукции.}
Чтобы показать наличие $\xi$ с $F^{(n+1)}(\xi)=0$, обычно делают «по шагам»: сначала доказывают, что в $(a,b)$ есть $c_1$ с $F^{(1)}(c_1)=0$ (Теорема Ролля), затем в $(a_1,b_1)$ подинтервале ищут $c_2$ с $F^{(2)}(c_2)=0$, и т. д.


\subsection{Ответ на вопрос}
% ===== 11.tex =====
% "Формула Тейлора с остатком в форме Лагранжа. Приближённые вычисления по формуле Тейлора."

% 1. Определения

\textbf{Формула Тейлора и остаточный член.}
Пусть $f$ имеет $(n+1)$-ю производную в окрестности точки $x_0$. Тогда можно представить $f(x)$ в виде:
\[
f(x) = f(x_0) + \frac{f'(x_0)}{1!}\,(x - x_0) + \dots + \frac{f^{(n)}(x_0)}{n!}\,(x - x_0)^n + R_n(x),
\]
где $R_n(x)$ называется \emph{остаточным членом}.

\medskip

\textbf{Форма Лагранжа остаточного члена.}
Существует точка $\xi$ между $x_0$ и $x$ (включая возможность $\xi \in (x,x_0)$), такая что
\[
R_n(x) = \frac{f^{(n+1)}(\xi)}{(n+1)!}\,(x - x_0)^{n+1}.
\]
Это утверждение мы будем доказывать ниже (используя теорему Лагранжа о среднем значении).

\medskip

% 2. Теоремы и ключевые утверждения

\textbf{Формулировка (Тейлор + Лагранж).}
Пусть $f$ непрерывно дифференцируема на отрезке $[x_0,x]$ (или $[x,x_0]$) до порядка $(n+1)$. Тогда:
\[
f(x) = \underbrace{\sum_{k=0}^{n} \frac{f^{(k)}(x_0)}{k!}\,(x-x_0)^k}_{\text{многочлен Тейлора порядка }n} \;+\; \frac{f^{(n+1)}(\xi)}{(n+1)!}\,(x - x_0)^{n+1},
\]
где $\xi$ лежит между $x_0$ и $x$.

\medskip

\textbf{Приближённые вычисления.}
Чтобы вычислить $f(x)$ примерно, берут полином Тейлора степени $n$:
\[
P_n(x) = \sum_{k=0}^{n} \frac{f^{(k)}(x_0)}{k!}\,(x - x_0)^k,
\]
и оценивают ошибку (погрешность) через
\[
|R_n(x)| = \left|\frac{f^{(n+1)}(\xi)}{(n+1)!}\,(x - x_0)^{n+1}\right|.
\]
Обычно используют верхнюю оценку: если $|f^{(n+1)}(t)| \le M$ на $[x_0,x]$, то
\[
|R_n(x)| \;\le\; \frac{M\,|x - x_0|^{n+1}}{(n+1)!}.
\]

\medskip

% 3. Основные идеи доказательства (коротко)

\begin{itemize}
  \item Рассмотрим функцию $F(x) = f(x) - P_n(x)$, где $P_n(x)$ — многочлен Тейлора степени $n$ вокруг $x_0$.  
  \item По определению $P_n$, все производные $F$ до порядка $n$ в точке $x_0$ равны нулю.  
  \item Применяем теорему Лагранжа \textit{в обобщённом виде} (Теорема Коши или вариант теоремы Ролля) к $F$ на отрезке $[x_0,x]$, получаем существование точки $\xi$, где $(n+1)$-я производная $F^{(n+1)}(\xi)=0$. Но $F^{(n+1)}(t)=f^{(n+1)}(t)$, потому что $(n+1)$-я производная многочлена $P_n$ равна нулю.  
  \item Отсюда возникает
  \[
  F(x) = F(x_0) + \dots + \frac{F^{(n+1)}(\xi)}{(n+1)!} (x-x_0)^{n+1},
  \]
  но $F(x_0)=F'(x_0)=\dots=F^{(n)}(x_0)=0$. Значит $F(x)=\frac{f^{(n+1)}(\xi)}{(n+1)!}(x-x_0)^{n+1}$.
\end{itemize}

\medskip

% 4. Полное доказательство (шаги, ссылаясь на sub_11 если нужно)

\textbf{Шаг 1: Построение многочлена Тейлора.}
\[
P_n(t) \;=\; \sum_{k=0}^{n} \frac{f^{(k)}(x_0)}{k!}\,(t - x_0)^k.
\]
По определению производной порядок $k$, $P_n$ «согласован» с $f$ до $n$-го порядка в точке $x_0$.

\textbf{Шаг 2: Рассмотрим $F(t)=f(t) - P_n(t)$.}
Тогда
\[
F^{(k)}(x_0) = f^{(k)}(x_0) - P_n^{(k)}(x_0) = 0 \quad\text{для } k=0,1,\dots,n.
\]

\textbf{Шаг 3: Применяем обобщённую теорему Ролля.}
На отрезке $[x_0, x]$ (или $[x,x_0]$), функция $F$ удовлетворяет условию $F^{(k)}(x_0)=F^{(k)}(x) \dots$ (не все равны, но ключ в том, что мы можем включить вспомогательную функцию «$G(t)=\dots$» — см. \texttt{sub\_11.tex} о теореме Коши). В итоге \emph{по индукции} выводится, что существует $\xi$ между $x_0$ и $x$, где $F^{(n+1)}(\xi)=0$. А $F^{(n+1)}(t)=f^{(n+1)}(t)$.

\textbf{Шаг 4: Остаточный член.}
\[
F(x) = f(x)-P_n(x) = \frac{f^{(n+1)}(\xi)}{(n+1)!}\,(x - x_0)^{n+1}.
\]
Перенося, получаем:
\[
f(x)=P_n(x) + \frac{f^{(n+1)}(\xi)}{(n+1)!}\,(x - x_0)^{n+1}.
\]
Это и есть искомая формула Тейлора с остатком в форме Лагранжа.

\medskip

% 5. Пример приближённых вычислений

\textbf{Пример: функция $f(x)=\sqrt{1+x}$, при $x$ мало.}
\begin{itemize}
  \item Выбираем $x_0=0$. Имеем $f(0)=1$, $f'(0)=\tfrac12$, $f''(0)=-\tfrac{1}{8}$, \dots .
  \item Третьепорядочное приближение: 
  \[
    P_2(x)=1 + \frac12 x - \frac{1}{8} x^2.
  \]
  \item Ошибка (остаток) $R_2(x)=\frac{f^{(3)}(\xi)}{3!} x^3$ для некоторого $\xi \in (0,x)$.  
  Используя оценку $|f^{(3)}(t)|\le M$ на $[0,x]$, будет 
  \(\bigl|R_2(x)\bigr|\le \frac{M|x|^3}{6}.\)
\end{itemize}
Так можно оценить точность приближённого вычисления $\sqrt{1+x}\approx 1 + \tfrac12 x - \tfrac18 x^2$ для малых $x$.

\medskip

% -- Логическая связность --
% (Подробные определения "обобщённой теоремы Ролля", "f^{(k)}(x_0)", "непрерывность" и т.д. см. sub_11.tex)


\newpage
\section{Формула Стирлинга (с эквивалентностью).}

\subsection{Используемые понятия}
% ===== sub_12.tex =====
% Вспомогательные определения/теоремы, не являющиеся
% "центральными" в вопросе 12, но используемые в доказательстве
% (Формула Стирлинга (с эквивалентностью)).

\textbf{Интегральная аппроксимация суммы.}
Ключ к доказательству формулы Стирлинга:
\[
\sum_{k=1}^{n}\ln k
\quad \text{сравнивается с} \quad
\int_{1}^{n}\ln x \,dx.
\]
Разность между этой суммой и интегралом — «погрешность», часто контролируемая приёмами типа «прямоугольников» или \emph{упрощённой} формулы Эйлера–Маклорена.

\medskip

\textbf{Обозначения: } $O(\cdot), \; o(\cdot).$
Символ $O(g(n))$ означает, что рассматриваемая величина не превосходит по абсолютному значению $C\,|g(n)|$ при достаточно больших $n$.  
Символ $r(n)=o(g(n))$ значит \(\lim_{n\to\infty}\frac{r(n)}{g(n)}=0\).

\medskip

\textbf{Эквивалентность функций.}
Запись $f(n)\sim g(n)$ означает \(\lim_{n\to\infty}\frac{f(n)}{g(n)}=1\). 
В контексте формулы Стирлинга: \(n!\sim \sqrt{2\pi n}\,\bigl(\tfrac{n}{e}\bigr)^n\). 


\subsection{Ответ на вопрос}
% ===== 12.tex =====
% "Формула Стирлинга (с эквивалентностью)."

% 1. Определения

\textbf{Факториал $n!$.}
Для натурального $n$ вводится произведение:
\[
n! \;=\;1\cdot2\cdot\dots\cdot n.
\]

\medskip

\textbf{Формула Стирлинга (эквивалентность).}
При $n\to\infty$
\[
n! \;\sim\;\sqrt{2\pi\,n}\,\Bigl(\tfrac{n}{e}\Bigr)^n,
\]
то есть
\[
\lim_{n\to\infty}\frac{n!}{\sqrt{2\pi\,n}\,\bigl(\tfrac{n}{e}\bigr)^n}=1.
\]

\medskip

% 2. Теоремы и ключевые утверждения

\textbf{Основная идея.}
Используется:
\[
\ln(n!)=\sum_{k=1}^n\ln k
\quad\approx\quad
\int_{1}^{n}\ln x\,dx=n\ln n - n +1.
\]
Разность «сумма – интеграл» даёт поправку, которая приводит к множителю \(\sqrt{2\pi n}\).

\medskip

% 3. Полное доказательство (шаги)

\begin{enumerate}
  \item \textbf{Переход к логарифмам.}
    \[
      \ln(n!)=\sum_{k=1}^n\ln k.
    \]
  \item \textbf{Сравнение с интегралом.}
    \[
      \sum_{k=1}^n\ln k
      \;\approx\;\int_{1}^{n}\ln x\,dx
      = n\ln n - n + 1.
    \]
  \item \textbf{Тонкая оценка (через формулу Эйлера–Маклорена).}
    \[
      \ln(n!)=n\ln n - n +\tfrac12\ln(n)+O(1).
    \]
  \item \textbf{Экспоненцирование.}
    \[
      n!=\exp(n\ln n - n +\tfrac12\ln n +O(1))
      =\sqrt{n}\,\Bigl(\tfrac{n}{e}\Bigr)^n\,\exp\bigl(O(1)\bigr).
    \]
    При более точном разборе получается желаемый множитель \(\sqrt{2\pi}\), то есть
    \[
      n!\;\sim\;\sqrt{2\pi n}\,\Bigl(\tfrac{n}{e}\Bigr)^n.
    \]
\end{enumerate}

\medskip

% 4. Сравнение

Уже при умеренных $n$, например $n=10$, точность формулы достаточно хороша; отклонение от \(\sqrt{2\pi n}\,\bigl(\tfrac{n}{e}\bigr)^n\) в долях процента относительно $n!$.


\newpage
\section{Формула Стирлинга (с равенством).}

\subsection{Используемые понятия}
% ===== sub_13.tex =====
% Вспомогательные определения/теоремы, не являющиеся
% "центральными" в вопросе 13, но используемые в доказательстве
% (Формула Стирлинга (с равенством)).

\textbf{Формула Эйлера–Маклорена (намёк).}
Для достаточно гладкой функции $f$, при суммировании:
\[
\sum_{k=a}^{b} f(k)
\;\approx\;
\int_{a}^{b}f(x)\,dx \;+\; \text{(пограничные и высшие члены)},
\]
используются числа Бернулли. В частности, для $f(k)=\ln k$ даёт точное выражение логарифма факториала:
\[
\ln(n!)=n\ln n -n + \tfrac12\ln(n) + \dots
\]

\medskip

\textbf{Числа Бернулли.}
Обозначаются $B_{m}$, входят в разложение. Не нужны формулы здесь, достаточно знать: они позволяют оценивать дополнительный член, который даёт диапазон $0<\theta_n<\frac{1}{12n}$.

\medskip

\textbf{Точный вид остатка.}
При экспоненцировании логарифмической оценки:
\[
\ln(n!)=n\ln n -n + \tfrac12\ln(2\pi n)+\theta_n
\]
возникает $n! = \sqrt{2\pi n}\,\bigl(\tfrac{n}{e}\bigr)^n e^{\theta_n}$. Ограничения на $\theta_n$ следуют из дополнительных членов Эйлера–Маклорены.


\subsection{Ответ на вопрос}
\input{questions/13.tex}

\newpage
\section{Определение интеграла Римана. Отличие от «обычного» предела.}

\subsection{Используемые понятия}
% ===== sub_14.tex =====
% Вспомогательные определения/теоремы, не являющиеся
% "центральными" в вопросе 14, но используемые в доказательстве
% (Определение интеграла Римана. Отличие от «обычного» предела).

\textbf{Верхние и нижние суммы (Дарбу).}
Для разбиения $D=\{x_0,\dots,x_n\}$:
\[
\overline{S}(f,D)=\sum_{i=1}^n \sup\limits_{x\in[x_{i-1},x_i]}f(x)\;\Delta x_i,
\quad
\underline{S}(f,D)=\sum_{i=1}^n \inf\limits_{x\in[x_{i-1},x_i]}f(x)\;\Delta x_i.
\]
Если при $\|D\|\to 0$, \(\overline{S}(f,D)\) и \(\underline{S}(f,D)\) сходятся к одной величине, эта величина называется \(\int_a^b f\).

\medskip

\textbf{Уточнение разбиения.}
Дано два разбиения $D$, $D'$. Их \emph{уточнением} называют разбиение, содержащее \textbf{все} точки $D$ и $D'$. При сопоставлении интегральных сумм на этом уточнённом разбиении можно показать, что они близки при мелком $\|D\|\to 0$.

\medskip

\textbf{Сущность «обычного» предела vs. интеграл.}
- Обычный предел $\lim_{x\to x_0}f(x)$ говорит о локальном поведении $f$ возле одной точки $x_0$.  
- Интеграл Римана — предел \emph{сумм} на всем отрезке $[a,b]$ с измельчающимся разбиением. Глобальное свойство.
7

\subsection{Ответ на вопрос}
% ===== 14.tex =====
% "Определение интеграла Римана. Отличие от обычного предела."

% 1. Определения

\textbf{Интеграл Римана.}
Пусть $f$ задана на $[a,b]$. Разобьём отрезок:
\[
a=x_0<x_1<\dots<x_n=b,
\quad
\Delta x_i=x_i - x_{i-1},
\quad
\xi_i\in[x_{i-1},x_i].
\]
Тогда \emph{интегральная сумма}:
\[
S=\sum_{i=1}^n f(\xi_i)\,\Delta x_i.
\]
Если при \(\max_i \Delta x_i\to0\) все такие суммы $S$ \emph{стремятся к одному и тому же} числу $I$, независимо от выбора точек $\xi_i$, то $f$ называется \textbf{интегрируемой по Риману}, а $I$ — её интегралом:
\[
I=\int_{a}^{b}f(x)\,dx.
\]

\medskip

% 2. Теоремы и ключевые утверждения

\begin{itemize}
  \item При \emph{непрерывности} $f$ на $[a,b]$ интеграл Римана существует.
  \item Критерий Дарбу: если верхние и нижние суммы сближаются, интеграл существует.
\end{itemize}

\medskip

% 3. Основные идеи доказательств

\begin{itemize}
  \item Рассматривается равномерная непрерывность на $[a,b]$ и «мелкие» разбиения, чтобы функция не успевала сильно меняться в каждом отрезке.
  \item Используется факт, что всякая пара разбиений «уточняется» до одного общего, и разность сумм делается малой.
\end{itemize}

\medskip

% 4. Полное доказательство (принцип)

\begin{enumerate}
  \item \textbf{Два разбиения.}
    Пусть $D$ и $D'$ — любые разбиения, $\|D\|\to0$ и $\|D'\|\to0$.
  \item \textbf{Уточнение.}
    Построить «общее» разбиение $D''$ содержащее все точки $D$ и $D'$. Сопоставить интегральные суммы. 
  \item \textbf{Оценка разницы.}
    При малых $\Delta x_i$, из равномерной непрерывности (или ограниченности) $f$ следует, что интегральные суммы $S(f,D)$ и $S(f,D')$ близки. 
  \item \textbf{Вывод.}
    Предел един, определение интеграла однозначно.
\end{enumerate}

\medskip

% 5. Отличие от «обычного» предела

\begin{itemize}
  \item Обычный предел: $\lim\limits_{x\to x_0} f(x)$ — \emph{точечный} анализ, рассматриваем поведение функции в одной точке.
  \item Интеграл Римана: \emph{глобальный} (берёт во внимание всё множество $[a,b]$), есть \textit{предел} \emph{сумм} при возрастании числа разбиений.
\end{itemize}

\medskip

% 6. Пример

Если $f(x)=\text{const}$, то любая интегральная сумма есть const\((b-a)\), предел одинаков независимо от разбиения.


\newpage
\section{Формула Ньютона-Лейбница.}

\subsection{Используемые понятия}

\textbf{Непрерывность и существование первообразной.}
Если $f$ непрерывна на $[a,b]$, то любая первообразная $F$ (то есть $F'(x)=f(x)$) будет непрерывно дифференцируема на $(a,b)$.

\medskip

\textbf{Теорема о среднем значении для интегралов.}
Для каждого отрезка $[x_{i-1},x_i]$ найдётся $\eta_i$ с 
\[
\int_{x_{i-1}}^{x_i} f(x)\,dx = f(\eta_i)\,(x_i - x_{i-1}).
\]
(Аналогична теореме Лагранжа для дифференцирования.)

\medskip

\textbf{Смысл формулы Ньютона–Лейбница.}
Определённый интеграл — площадь под $f$, а $F$ — функция, чья производная равна $f$. Тогда приращение $F(b)-F(a)$ «накрывает» общую «площадь» (или суммирует мгновенные приращения).


\subsection{Ответ на вопрос}
% ===== 15.tex =====
% "Формула Ньютона–Лейбница."

% 1. Определения

\textbf{Определённый интеграл.}
Функция \(f\) называется интегрируемой по Риману на \([a,b]\), если предел сумм вида
\[
S = \sum_{i=1}^n f(\xi_i)\,\Delta x_i,
\]
при разбиении \([a,b]\) на всё более мелкие отрезки (с максимальной длиной \(\Delta x_i \to 0\)), существует и не зависит от выбора \(\xi_i\). Этот предел есть \(\int_a^b f(x)\,dx\).

\medskip

\textbf{Первообразная (примитив).}
Функция \(F\) называется \emph{первообразной} для \(f\) на \([a,b]\), если \(F'(x)=f(x)\) для всех \(x\in(a,b)\). Если \(F\) дифференцируема на \((a,b)\) и непрерывна на \([a,b]\), то это достаточно для теоремы Ньютона–Лейбница.

\medskip

% 2. Теоремы и ключевые утверждения

\textbf{Формула Ньютона–Лейбница.}
Пусть \(f\) непрерывна на \([a,b]\) и \(F\) — первообразная \(f\) на \([a,b]\). Тогда
\[
\int_a^b f(x)\,dx = F(b) - F(a).
\]

\medskip

% 3. Основные идеи доказательства

\begin{itemize}
  \item Рассматривается \(\int_a^b f(x)\,dx\) и разбиение отрезка \([a,b]\).  
  \item По определению первообразной \(F'(x)=f(x)\) на \((a,b)\).  
  \item Считаем интегральные суммы \(S = \sum f(\xi_i)\,\Delta x_i\) и замечаем связь с приростами \(F(x_i) - F(x_{i-1})\).  
  \item Подсчитываем \(\sum [F(x_i)-F(x_{i-1})] = F(b)-F(a)\).  
  \item Показываем, что эта сумма совпадает с \(\int_a^b f(x)\,dx\) в пределе.
\end{itemize}

\medskip

% 4. Полное доказательство (шаги, ссылаясь на sub_15, если нужно)

\begin{enumerate}
  \item \textbf{Разбиение отрезка \([a,b]\).}
    Пусть \(a=x_0 < x_1 < \dots < x_n=b\) — любое разбиение, \(\Delta x_i=x_i-x_{i-1}\).
  \item \textbf{Интегральная сумма.}
    Рассмотрим \(S = \sum_{i=1}^n f(\xi_i)\Delta x_i\), \(\xi_i\in[x_{i-1},x_i]\).
  \item \textbf{Прирост первообразной.}
    Так как \(F'(x)=f(x)\), то по теореме о среднем значении существует \(\eta_i\in[x_{i-1},x_i]\) с
    \[
      F(x_i)-F(x_{i-1}) = F'(\eta_i)\,\Delta x_i = f(\eta_i)\,\Delta x_i.
    \]
  \item \textbf{Сравнение с интегральной суммой.}
    Если выбрать \(\xi_i=\eta_i\), видим, что \(\sum [F(x_i)-F(x_{i-1})] = \sum f(\eta_i)\Delta x_i = S\). Но слева — телескопическая сумма:
    \[
      \sum_{i=1}^n [\,F(x_i)-F(x_{i-1})\,]
      = F(x_n)-F(x_0) = F(b)-F(a).
    \]
  \item \textbf{Вывод.}
    Переходя к пределу при \(\|D\|\to 0\), интегральная сумма \(\sum f(\xi_i)\,\Delta x_i\) стремится к \(\int_a^b f(x)\,dx\), а мы установили её равенство \(F(b)-F(a)\). Следовательно,
    \[
      \int_a^b f(x)\,dx = F(b) - F(a).
    \]
\end{enumerate}

\medskip

% 5. Пример

\textbf{Функция \(f(x)=x^2\).}
Одна из первообразных: \(F(x)=\tfrac{x^3}{3}\). По формуле Ньютона–Лейбница:
\[
\int_0^2 x^2\,dx
= \Bigl[\tfrac{x^3}{3}\Bigr]_0^2
= \tfrac{2^3}{3} - 0 = \tfrac{8}{3}.
\]

\end{document}


\end{document}
