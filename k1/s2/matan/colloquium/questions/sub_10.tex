% ===== sub_10.tex =====
% Вспомогательные определения и факты,
% не являющиеся "центральными" в вопросе,
% но требующиеся при доказательствах (n-я производная, radius of convergence и т.п.)

\textbf{Определение n-й производной.}
Если функция $f$ $n$ раз дифференцируема в окрестности 0, то $f^{(n)}(0)$ есть её $n$-я производная в точке 0.

\medskip

\textbf{Радиус сходимости степенного ряда.}
Ряд $\sum_{n=0}^{\infty} c_n x^n$ имеет некий \emph{радиус сходимости} $R$, $0\le R\le \infty$, где ряд сходится при $|x|<R$ и расходится (как правило) при $|x|>R$.

\medskip

\textbf{Бином Ньютона (обобщённый).}
Для вещественного $a$ при $|x|<1$:
\[
(1+x)^a = \sum_{n=0}^{\infty} \binom{a}{n}\,x^n,
\]
где $\displaystyle \binom{a}{n} = \frac{a(a-1)\dots(a-n+1)}{n!}$.

\medskip

\textbf{Формула Тейлора (общая).}
При разложении $f(x)$ в окрестности 0 с учётом всех производных получаем ряд (если сходится) называемый рядом Маклорена, частный случай ряда Тейлора.
