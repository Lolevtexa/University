% ===== sub_11.tex =====
% Вспомогательный файл: определения и теоремы, не являющиеся центральными
% в вопросе, но используемые при доказательствах (Теорема Коши/Ролля в обобщённом виде, ...
% понятие n-кратной производной).

\textbf{Обобщённая теорема Ролля (Теорема Коши).}
Если $f$ и $g$ непрерывны на $[a,b]$, дифференцируемы на $(a,b)$, и $g'(x)\neq 0$ на $(a,b)$, то существует $c\in(a,b)$:
\[
\frac{f(b)-f(a)}{g(b)-g(a)} = \frac{f'(c)}{g'(c)}.
\]
При удачном выборе «вспомогательных» функций подстановка даёт нужный результат о $F^{(n+1)}(\xi)=0$.

\medskip

\textbf{n-кратная производная в точке.}
Если $f$ дифференцируема $n$ раз в окрестности $x_0$, мы обозначаем $f^{(n)}(x_0)$ как производную $n$-го порядка, если та существует и непрерывна.

\medskip

\textbf{Пример применения индукции.}
Чтобы показать наличие $\xi$ с $F^{(n+1)}(\xi)=0$, обычно делают «по шагам»: сначала доказывают, что в $(a,b)$ есть $c_1$ с $F^{(1)}(c_1)=0$ (Теорема Ролля), затем в $(a_1,b_1)$ подинтервале ищут $c_2$ с $F^{(2)}(c_2)=0$, и т. д.
