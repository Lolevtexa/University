% ===== sub_2.tex =====
% Вспомогательные факты, не являющиеся центральными в основном вопросе
% но используемые в доказательствах (локальный экстремум, теорема Вейерштрасса и т.д.)

\textbf{Локальный экстремум.}
Точка $x_0$ называется \emph{локальным минимумом} (соответственно, максимумом), если существует $\delta>0$ такое, что для всех $x$ из $(x_0-\delta, x_0+\delta)$ выполняется $f(x)\ge f(x_0)$ (или $f(x)\le f(x_0)$ для максимума).

\medskip

\textbf{Теорема Вейерштрасса (о достижении экстремума).}
Если $f$ непрерывна на отрезке $[a,b]$, то $f$ достигает на нём своих наибольшего и наименьшего значений. Формально:
\[
\exists x_{\min}, x_{\max}\in[a,b]:\quad
f(x_{\min}) \le f(x) \le f(x_{\max})
\quad
\forall x \in [a,b].
\]

\textbf{Производная в точке.}
Напомним, $f'(x_0)$ есть предел
\[
\lim_{x\to x_0}\frac{f(x)-f(x_0)}{x - x_0},
\]
если этот предел конечен.
