% ===== 8.tex =====
% "Достаточные условия существования экстремума (по второй производной)."

% 1. Определения

\textbf{Локальный минимум и максимум.}
Точка $x_0$ внутри промежутка $(a,b)$ называется \emph{точкой локального минимума} функции $f$, если существует $\delta>0$, что для всех $x$ с $|x-x_0|<\delta$ выполняется
\[
f(x)\;\ge\;f(x_0).
\]
Аналогично, $x_0$ является \emph{точкой локального максимума}, если в некоторой окрестности $x_0$ значение $f(x)\le f(x_0)$.

\medskip

\textbf{Вторая производная.}
Если функция $f$ дифференцируема на $(a,b)$, и $f'(x)$ тоже дифференцируема на $(a,b)$, то в точках, где это возможно, определена \emph{вторая производная} $f''(x)$.

\medskip

% 2. Теоремы и ключевые утверждения

\textbf{Теорема (достаточные условия экстремума по второй производной).}
Пусть $f$ дифференцируема на $(a,b)$ и $x_0 \in (a,b)$ — такая точка, где $f'(x_0)=0$. Предположим, что у $f$ существует непрерывная в $x_0$ вторая производная $f''(x_0)$. Тогда:
\begin{enumerate}
  \item Если $f''(x_0)>0$, то $x_0$ — точка \textbf{локального минимума}.
  \item Если $f''(x_0)<0$, то $x_0$ — точка \textbf{локального максимума}.
  \item Если $f''(x_0)=0$, вывод не делается (нужен дополнительный анализ).
\end{enumerate}

\medskip

% 3. Основные идеи доказательства (коротко)

\begin{itemize}
  \item Ключевой «трюк»: если $f''(x_0)>0$, то $f'(x)$ возрастает вблизи $x_0$. Но при $x_0$ мы имеем $f'(x_0)=0$.  
  \item Следовательно, для $x>x_0$, $f'(x)$ становится положительной (или почти), а для $x<x_0$ — отрицательной (или почти), что даёт локальный минимум.  
  \item Аналогично, если $f''(x_0)<0$, $f'(x)$ убывает вблизи $x_0$, и $x_0$ — локальный максимум.
\end{itemize}

\medskip

% 4. Полное доказательство (по шагам, ссылаясь на sub_8 при необходимости)

\textbf{Доказательство достаточности (подробно):}

\begin{enumerate}
  \item \textbf{Наличие $f'(x_0)=0$.}  
    По \textbf{теореме Ферма} (см. \texttt{sub\_8.tex}), это условие часто является необходимым для экстремума. Мы рассматриваем \emph{достаточность}, когда вдобавок знаем про вторую производную.
  \item \textbf{Случай $f''(x_0)>0$.}  
    \begin{itemize}
      \item Из непрерывности $f''$ в $x_0$ вытекает, что при $x$ достаточно близком к $x_0$, вторая производная $f''(x)$ «сохраняет» тот же знак (положительный).  
      \item Значит $f'(x)$ строго возрастает вблизи $x_0$. Но $f'(x_0)=0$.  
      \item Тогда для $x>x_0$ с $x$ рядом с $x_0$, $f'(x)$ станет положительной, а для $x<x_0$ — отрицательной.  
      \item Следовательно, при $x>x_0$, $f$ возрастает, при $x<x_0$, $f$ убывает. Значит $x_0$ — локальный минимум.
    \end{itemize}
  \item \textbf{Случай $f''(x_0)<0$.}  
    \begin{itemize}
      \item Аналогичные рассуждения: $f'(x)$ вблизи $x_0$ будет убывать, и при $x>x_0$ произведёт знак отрицательный, а при $x<x_0$ — знак положительный (около $x_0$).  
      \item Следовательно, слева функция возрастает, а справа убывает. Точка $x_0$ — локальный максимум.
    \end{itemize}
  \item \textbf{Случай $f''(x_0)=0$.}  
    \begin{itemize}
      \item Из этого факта \emph{нельзя} вывести строгое заключение об экстремуме: нужны дополнительные рассуждения (см. примеры $x^3$, $x^4$).
    \end{itemize}
\end{enumerate}

\medskip

% 5. Примеры

\begin{itemize}
  \item \emph{Функция $f(x)=x^2$:}  
  $f'(0)=0$, $f''(0)=2>0$. По теореме, $x=0$ — локальный минимум (что верно).
  \item \emph{Функция $f(x)=x^3$:}  
  $f'(0)=0$, $f''(0)=0$. По рассматриваемой теореме вывод не делается, и действительно $x=0$ — точка перегиба, \textit{не} экстремум.
  \item \emph{Функция $f(x)=-x^2$:}  
  $f'(0)=0$, $f''(0)=-2<0$. Значит в $x=0$ локальный максимум.
\end{itemize}

\medskip

% -- Логическая связность (определения в sub_8.tex)
