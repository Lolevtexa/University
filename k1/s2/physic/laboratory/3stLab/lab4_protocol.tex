\documentclass[a4paper,12pt]{article}

\usepackage[utf8]{inputenc}
\usepackage[T2A]{fontenc}
\usepackage[russian]{babel}
\usepackage{amsmath,amssymb}
\usepackage{geometry}
\usepackage{graphicx}
\usepackage{array}
\usepackage{tabularray}
\geometry{top=5mm,bottom=15mm,left=15mm,right=15mm}

\usepackage{caption}
\captionsetup[table]{skip=0pt}    % Убрать отступ между caption и таблицей

\setlength{\textfloatsep}{0pt} 
\setlength{\intextsep}{6pt}
\setlength{\floatsep}{0pt}

\begin{document}

\begin{center}
	{\Large \textbf{Протокол наблюдений}}\\
	\textbf{Лабораторная работа №\,4 «Дифракционная решётка»}\\
	\textbf{Николаев Всеволод Юрьевич\, 4395}
\end{center}

\begin{table}[htb]
	\centering
	\caption{Константы эксперимента}
	\begin{tblr}{
		colspec = {Q[wd=3cm]Q[wd=4.2cm]Q[wd=4cm]Q[wd=3cm]},
		vlines = {},
		hlines = {},
		cells = {valign = m, halign = c},
		}
		Длина волны залёного цвета, $\lambda = \overline{\lambda} \pm \Delta \overline{\lambda} \text{, нм}$ &
		Постоянная решётки, $d = \overline{d} \pm \Delta \overline{d} \text{, мкм}$                          &
		Длина решётки, см                                                                                    &
		Число штрихов $N=\dfrac{L}{\overline{d}}$                                                                  \\
		$\lambda = 546 \pm 5 \text{, нм} $ $ \text{с } P = 95 \%$                                            &   &
		1,5 см                                                                                               &     \\
	\end{tblr}
\end{table}

\begin{table}[htb]
	\centering
	\caption{Определение длины волны и характеристик дифракционной решётки}
	\label{tab:lambda-params}
	\begin{tblr}{
		colspec = {Q[wd=2.1cm]Q[wd=2.4cm]Q[wd=2.6cm]Q[wd=1.6cm]Q[wd=2.8cm]Q[wd=1.7cm]Q[wd=1.6cm]},
		vlines = {},
		hlines = {},
		row{1} = {valign = m, halign = c},
		cell{2,4,6}{1,2,3} = {r = 2}{},
			}
		Цвет спектральной линии                                            &
		Угловой коэффициент
		$a = \overline{a} \pm \Delta \overline{a}$                         &
		Длина волны
		$\lambda = \overline{\lambda} \pm \Delta \overline{\lambda},\,$ нм &
		Порядок
		спектра, $m$                                                       &
		$D_{\varphi} = \dfrac{m}{\,\overline{d} \cos \varphi_m}$ мин/нм    &
		$R = mN$                                                           &
		$\Delta \lambda = \dfrac{\overline{\lambda}}{R}$ нм                                   \\
		Жёлтая                                                             &   &  & 1 &  &  & \\
		                                                                   &   &  & 3 &  &  & \\

		Зелёная                                                            &   &  & 1 &  &  & \\
		                                                                   &   &  & 3 &  &  & \\

		Синяя                                                              &   &  & 1 &  &  & \\
		                                                                   &   &  & 3 &  &  & \\
	\end{tblr}
\end{table}

\begin{table}[htb]
	\centering
	\caption{Измерение углов дифракции для линий жёлтого цвета}
	\begin{tblr}{
			vlines = {},
			hlines = {},
			cells = {c, mode = dmath},
			cell{2}{1} = {r = 3}{},
			cell{12}{1} = {r = 3}{},
		}
		\left\lvert m \right\rvert                        &
		\qquad 0 \qquad                                   &
		\qquad 1 \qquad                                   &
		\qquad 2 \qquad                                   &
		\qquad 3 \qquad                                               \\
		\bar{\alpha}_{+m}                                 &   &  &  & \\
		                                                  &   &  &  & \\
		                                                  &   &  &  & \\
		\varphi_{+m} = \bar{\alpha}_{+m} - \bar{\alpha}_0 &   &  &  & \\
		a = \frac{\sin(\varphi_{+m})}{m}                  &   &  &  & \\
		\theta_a = \frac{\cos(\varphi_{+m})}{m}           &   &  &  & \\
		\theta_a = \frac{\cos(\varphi_{-m})}{m}           &   &  &  & \\
		a = \frac{\sin(\varphi_{-m})}{m}                  &   &  &  & \\
		\varphi_{-m} = \bar{\alpha}_{-m} - \bar{\alpha}_0 &   &  &  & \\
		\bar{\alpha}_{-m}                                 &   &  &  & \\
		\alpha_{-m}                                       &   &  &  & \\
		                                                  &   &  &  & \\
		                                                  &   &  &  & \\
	\end{tblr}
\end{table}

\begin{table}[htb]
	\centering
	\caption{Измерение углов дифракции для линий зелёного цвета}
	\begin{tblr}{
			vlines = {},
			hlines = {},
			cells = {c, mode = dmath},
			cell{2}{1} = {r = 3}{},
			cell{12}{1} = {r = 3}{},
		}
		\left\lvert m \right\rvert                        &
		\qquad 0 \qquad                                   &
		\qquad 1 \qquad                                   &
		\qquad 2 \qquad                                   &
		\qquad 3 \qquad                                               \\
		\bar{\alpha}_{+m}                                 &   &  &  & \\
		                                                  &   &  &  & \\
		                                                  &   &  &  & \\
		\varphi_{+m} = \bar{\alpha}_{+m} - \bar{\alpha}_0 &   &  &  & \\
		a = \frac{\sin(\varphi_{+m})}{m}                  &   &  &  & \\
		\theta_a = \frac{\cos(\varphi_{+m})}{m}           &   &  &  & \\
		\theta_a = \frac{\cos(\varphi_{-m})}{m}           &   &  &  & \\
		a = \frac{\sin(\varphi_{-m})}{m}                  &   &  &  & \\
		\varphi_{-m} = \bar{\alpha}_{-m} - \bar{\alpha}_0 &   &  &  & \\
		\bar{\alpha}_{-m}                                 &   &  &  & \\
		\alpha_{-m}                                       &   &  &  & \\
		                                                  &   &  &  & \\
		                                                  &   &  &  & \\
	\end{tblr}
\end{table}

\begin{table}[htb]
	\centering
	\caption{Измерение углов дифракции для линий синего цвета}
	\begin{tblr}{
			vlines = {},
			hlines = {},
			cells = {c, mode = dmath},
			cell{2}{1} = {r = 3}{},
			cell{12}{1} = {r = 3}{},
		}
		\left\lvert m \right\rvert                        &
		\qquad 0 \qquad                                   &
		\qquad 1 \qquad                                   &
		\qquad 2 \qquad                                   &
		\qquad 3 \qquad                                               \\
		\bar{\alpha}_{+m}                                 &   &  &  & \\
		                                                  &   &  &  & \\
		                                                  &   &  &  & \\
		\varphi_{+m} = \bar{\alpha}_{+m} - \bar{\alpha}_0 &   &  &  & \\
		a = \frac{\sin(\varphi_{+m})}{m}                  &   &  &  & \\
		\theta_a = \frac{\cos(\varphi_{+m})}{m}           &   &  &  & \\
		\theta_a = \frac{\cos(\varphi_{-m})}{m}           &   &  &  & \\
		a = \frac{\sin(\varphi_{-m})}{m}                  &   &  &  & \\
		\varphi_{-m} = \bar{\alpha}_{-m} - \bar{\alpha}_0 &   &  &  & \\
		\bar{\alpha}_{-m}                                 &   &  &  & \\
		\alpha_{-m}                                       &   &  &  & \\
		                                                  &   &  &  & \\
		                                                  &   &  &  & \\
	\end{tblr}
\end{table}

\end{document}
