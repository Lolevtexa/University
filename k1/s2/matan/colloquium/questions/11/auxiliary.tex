% ==================
% auxiliary.tex
% (Вспомогательные определения для темы:
%  "Формула Тейлора с остатком в форме Лагранжа.
%   Приближённые вычисления по формуле Тейлора.")
% ==================

\paragraph{Формула Тейлора и остаточный член.}
Пусть $f$ имеет $(n+1)$-ю производную в окрестности точки $x_0$.
Тогда можно представить $f(x)$ в виде:
\[
	f(x)
	\;=\;
	f(x_0)
	\;+\;
	\frac{f'(x_0)}{1!}\,(x - x_0)
	\;+\;\dots\;+\;
	\frac{f^{(n)}(x_0)}{n!}\,(x - x_0)^n
	\;+\;
	R_n(x),
\]
где $R_n(x)$ — \emph{остаточный член}.

\bigskip

\paragraph{Форма Лагранжа для остаточного члена.}
Существует точка $\xi$ между $x_0$ и $x$ такая, что
\[
	R_n(x)
	\;=\;
	\frac{f^{(n+1)}(\xi)}{(n+1)!}\,(x - x_0)^{n+1}.
\]
Это часто доказывают, используя обобщённую \emph{Теорему Лагранжа} (или Коши) и идеи, связанные с «нулевыми» значениям производных при замене на многочлен Тейлора до порядка $n$.

\bigskip

\paragraph{Приближённые вычисления.}
Чтобы приблизительно вычислить $f(x)$, берут полином Тейлора
\[
	P_n(x)
	\;=\;
	\sum_{k=0}^n
	\frac{f^{(k)}(x_0)}{k!}\,(x - x_0)^k,
\]
оценивая погрешность через
\[
	|R_n(x)|
	= \left|
	\frac{f^{(n+1)}(\xi)}{(n+1)!}\,(x - x_0)^{n+1}
	\right|.
\]
Если $|f^{(n+1)}(t)| \le M$ на $[x_0,x]$, то
\[
	|R_n(x)|
	\;\le\;
	\frac{M\,|x-x_0|^{n+1}}{(n+1)!},
\]
что даёт верхнюю границу ошибки.
