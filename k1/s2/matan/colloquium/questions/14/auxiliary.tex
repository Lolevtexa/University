% ==================
% auxiliary.tex
% (Вспомогательные определения для темы:
%  "Определение интеграла Римана. Отличие от обычного предела.")
% ==================

\paragraph{Интеграл Римана.}
Пусть $f$ задана на отрезке $[a,b]$.
Разобьём $[a,b]$ на промежутки:
\[
	a = x_0 < x_1 < \dots < x_n = b,
	\quad \Delta x_i = x_i - x_{i-1},
	\quad \xi_i \in [x_{i-1}, x_i].
\]
Определяется \emph{интегральная сумма}:
\[
	S = \sum_{i=1}^n f(\xi_i)\,\Delta x_i.
\]
Если при $\max_i \Delta x_i \to 0$ все такие суммы $S$ стремятся к одному и тому же числу $I$, независимо от выбора точек $\xi_i$ внутри отрезков, то говорят, что $f$ \emph{интегрируема по Риману}, а $I$ есть её \emph{интеграл}:
\[
	I = \int_a^b f(x)\,dx.
\]

\bigskip

\paragraph{Непрерывность и критерий Дарбу.}
Если $f$ непрерывна на $[a,b]$, то по теореме о непрерывных функциях и равномерной непрерывности на компактном промежутке интеграл Римана существует.
Критерий Дарбу: интеграл существует тогда и только тогда, когда верхние и нижние суммы (Darbo sums) сближаются при мелкости разбиения $\to 0$.

\bigskip

\paragraph{Отличие от «обычного» предела.}
Обычный предел \(\lim\limits_{x\to x_0} f(x)\) — локальный (точечный) анализ поведения функции в одной точке.
Интеграл Римана рассматривает «глобальное» поведение $f$ на всём отрезке $[a,b]$ и определяется как предел интегральных сумм, когда число разбиений возрастает (шаги уменьшаются).
