% ===== 1.tex =====
% Равномерная непрерывность. Теорема Кантора о равномерной непрерывности.

% -- 1. Определения --

\textbf{Определение (Обычная непрерывность).}
Функция $f$ называется \emph{непрерывной} в точке $x_0$, если
\[
\forall \varepsilon>0\;\;\exists \delta(\varepsilon,x_0)>0:
\quad |x - x_0| < \delta
\;\;\Longrightarrow\;\;
|f(x) - f(x_0)| < \varepsilon.
\]

\textbf{Определение (Равномерная непрерывность).}
Пусть $f$ задана на $X \subset \mathbb{R}$. Тогда $f$ называется \emph{равномерно непрерывной} на $X$, если
\[
\forall \varepsilon>0\;\;\exists \delta(\varepsilon)>0:\quad
\forall x_1,x_2 \in X,\; |x_1 - x_2|<\delta
\;\;\Longrightarrow\;\;
|f(x_1)-f(x_2)| < \varepsilon.
\]
Ключевая особенность: \(\delta\) \textbf{не} зависит от конкретной точки в $X$, а лишь от \(\varepsilon\).

\medskip

% -- 2. Теоремы и ключевые утверждения --

\textbf{Теорема Кантора (формулировка).}
\emph{Если функция $f$ непрерывна на компактном множестве $X\subset\mathbb{R}$, то $f$ равномерно непрерывна на $X$.}

\medskip

% -- 3. Основные идеи доказательства (коротко) --

\begin{enumerate}
  \item Доказываем \emph{от противного}: предположим, что $f$ непрерывна на компакте $X$, но \textit{не} является равномерно непрерывной.
  \item Это означает, что существует \(\varepsilon_0>0\), для которого \emph{никогда} нельзя выбрать \(\delta\) «раз и навсегда». Формально:
  \[
    \forall \delta>0,\; \exists\,x,y\in X:\; |x-y|<\delta,\; |f(x)-f(y)|\ge\varepsilon_0.
  \]
  \item Выбираем \(\delta=1/n\), строим пары \((x_n,y_n)\). При компактности $X$ можно выделить подпоследовательность $(x_{n_k}) \to c$. Тогда и $y_{n_k}\to c$.
  \item По непрерывности $f$: $f(x_{n_k})\to f(c)$ и $f(y_{n_k})\to f(c)$, значит \(|f(x_{n_k})-f(y_{n_k})|\to0\). Это \emph{противоречит} условию \(\ge \varepsilon_0\).
\end{enumerate}

\medskip

% -- 4. Полное доказательство (пошаговое) --

\textbf{Доказательство теоремы Кантора.} 
\begin{enumerate}
  \item \textbf{Шаг 1. Предположение противного.}  
  Пусть $f$ непрерывна на компактном $X$, но \textbf{не} равномерно непрерывна. Тогда существует \(\varepsilon_0>0\) такое, что для всякого \(\delta>0\) можно найти $x,y\in X$ с
  \[
  |x-y|<\delta,\quad |f(x)-f(y)| \ge \varepsilon_0.
  \]

  \item \textbf{Шаг 2. Построение последовательностей.}  
  Для $n\in\mathbb{N}$ возьмём \(\delta=1/n\). Найдём $x_n,y_n \in X$:
  \[
    |x_n-y_n| < \frac{1}{n},\quad |f(x_n)-f(y_n)| \ge \varepsilon_0.
  \]

  \item \textbf{Шаг 3. Извлечение сходящейся подпоследовательности.}  
  Поскольку $X$ \emph{компактно} (см. \texttt{sub\_1.tex} о необходимости замкнутости и ограниченности), существует подпоследовательность $(x_{n_k})\to c\in X$. Из условия $|x_{n_k}-y_{n_k}|<1/{n_k}\to0$ следует $y_{n_k}\to c$ тоже.

  \item \textbf{Шаг 4. Применение непрерывности $f$.}  
  По непрерывности $f$:
  \[
    f(x_{n_k})\to f(c),\quad f(y_{n_k})\to f(c).
  \]
  Тогда
  \[
    |f(x_{n_k}) - f(y_{n_k})|\;\to\; 0.
  \]

  \item \textbf{Шаг 5. Противоречие.}  
  Но по выбору $(x_{n_k}, y_{n_k})$ мы всегда имели $|f(x_{n_k}) - f(y_{n_k})|\ge\varepsilon_0>0$. Получено противоречие. Значит исходное предположение неверно.

  \item \textbf{Шаг 6. Вывод.}  
  Следовательно, $f$ равномерно непрерывна на $X$.
\end{enumerate}

\medskip

% -- Логическая связность --
% (Все ссылки на определения "компакта" и непрерывности
% предполагаются в sub_1.tex).

