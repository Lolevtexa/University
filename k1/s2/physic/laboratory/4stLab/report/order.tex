\section*{Порядок выполнения работы}

\begin{enumerate}
    \item Вращением рукоятки потенциометра \( R_i \) установить силу тока \( I = 2\,\text{А} \) в пластине. Установить диапазон измерения температуры до \( 1400\,^\circ\text{C} \) и включить соответствующий ему жёлтый светофильтр.
    
    \item Включить накал нити с помощью кнопки \( K \), которая во время проведения измерений должна оставаться во включённом (замкнутом) положении. Регулируя силу тока в нити пирометра, добиться исчезновения видимого изображения нити на фоне светящейся поверхности пластины.
    
    \item Зафиксировать следующие величины:
    \begin{itemize}
        \item температуру нити \( t_H \) (в градусах Цельсия), 
        \item силу тока \( I \) в цепи нагрева пластины,
        \item напряжение \( U \) на пластине.
    \end{itemize}
    Записать полученные данные в таблицу ~\ref{tab:temp-emission}.
    
    \item Повторить цикл измерений (шаги 1–3) при других значениях силы тока \( I \) в пластине от \( 2{,}5\,\text{А} \) до \( 5{,}5\,\text{А} \) с шагом \( 0{,}5\,\text{А} \). Всего выполнить 8 измерений. 
    
    \item При превышении температуры \( 1400\,^\circ\text{C} \) переключить диапазон измерения пирометра на \( 2000\,^\circ\text{C} \) и заменить жёлтый светофильтр на красный. Провести измерения аналогично, с шагом по току \( 0{,}5\,\text{А} \), записывая значения \( t_H \), \( I \), \( U \), а также используемый светофильтр.
    
    \item Для повышения точности измерений провести каждое измерение температуры нити дважды:
    \begin{enumerate}
        \item при увеличении тока в нити (нагрев),
        \item при уменьшении тока (охлаждение).
    \end{enumerate}
    Результаты внести в таблицу ~\ref{tab:temp-emission}.
\end{enumerate}
