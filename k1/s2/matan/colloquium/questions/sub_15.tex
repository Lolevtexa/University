
\textbf{Непрерывность и существование первообразной.}
Если $f$ непрерывна на $[a,b]$, то любая первообразная $F$ (то есть $F'(x)=f(x)$) будет непрерывно дифференцируема на $(a,b)$.

\medskip

\textbf{Теорема о среднем значении для интегралов.}
Для каждого отрезка $[x_{i-1},x_i]$ найдётся $\eta_i$ с 
\[
\int_{x_{i-1}}^{x_i} f(x)\,dx = f(\eta_i)\,(x_i - x_{i-1}).
\]
(Аналогична теореме Лагранжа для дифференцирования.)

\medskip

\textbf{Смысл формулы Ньютона–Лейбница.}
Определённый интеграл — площадь под $f$, а $F$ — функция, чья производная равна $f$. Тогда приращение $F(b)-F(a)$ «накрывает» общую «площадь» (или суммирует мгновенные приращения).
