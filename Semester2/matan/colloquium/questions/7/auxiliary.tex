% ==================
% auxiliary.tex
% (Вспомогательные определения для темы
% "Формула Тейлора для многочлена.
%  Формула Тейлора с остатком в форме Пеано.")
% ==================

\paragraph{Многочлен и его производные.}
Пусть $P(x)$ — многочлен степени $n$, то есть
\[
	P(x) \;=\; a_0 + a_1 x + a_2 x^2 + \dots + a_n x^n.
\]
Все производные $P^{(k)}(x)$ существуют на $\mathbb{R}$, причём для $k>n$ эти производные тождественно равны нулю.

\bigskip

\paragraph{Малое «o» и запись $o\bigl((x-x_0)^n\bigr)$.}
Говорят, что $r_n(x) = o\bigl((x - x_0)^n\bigr)$ при $x \to x_0$,
если
\[
	\lim_{x\to x_0}
	\frac{r_n(x)}{(x - x_0)^n}
	\;=\; 0.
\]
Иными словами, функция $r_n(x)$ «уходит в ноль» быстрее, чем $(x - x_0)^n$, когда $x$ приближается к $x_0$.

\bigskip

\paragraph{Общее представление о разложении Тейлора.}
Формула Тейлора обычно записывается в виде
\[
	f(x) \;=\;
	\sum_{k=0}^n \frac{f^{(k)}(x_0)}{k!}(x - x_0)^k \;+\; R_n(x),
\]
где $R_n(x)$ — остаточный член, который может принимать разные формы (например, форма Лагранжа или форма Пеано).

