% ==================
% answer.tex
% (Основной материал:
%  "Формула Тейлора с остатком в форме Лагранжа.
%   Приближённые вычисления по формуле Тейлора.")
% ==================

\begin{customtheorem}[Формула Тейлора с остатком в форме Лагранжа]
	Пусть $f$ непрерывно дифференцируема на отрезке $[x_0,x]$ (или $[x,x_0]$) до порядка $(n+1)$. Тогда
	\[
		f(x)
		\;=\;
		\underbrace{
			\sum_{k=0}^n \frac{f^{(k)}(x_0)}{k!}\,(x-x_0)^k
		}_{\text{многочлен Тейлора порядка }n}
		\;+\;
		\frac{f^{(n+1)}(\xi)}{(n+1)!}\,(x - x_0)^{n+1},
	\]
	где $\xi$ лежит между $x_0$ и $x$.
\end{customtheorem}

\begin{proofplan}
	\begin{enumerate}
		\item Построить многочлен Тейлора $P_n(t)$ степени $n$ вокруг $x_0$.
		\item Рассмотреть функцию $F(t)=f(t)-P_n(t)$ и показать, что все её производные до $n$-го порядка в $x_0$ равны 0.
		\item Применить обобщённую теорему Ролля (либо Коши) на отрезке $[x_0,x]$, чтобы найти точку $\xi$, где $(n+1)$-я производная $F$ равна 0.
		\item Учитывая, что $F^{(n+1)}(t)=f^{(n+1)}(t)$, получаем остаток в форме Лагранжа.
	\end{enumerate}
\end{proofplan}

\begin{customproof}
	\textbf{Шаг 1.}
	Определим
	\[
		P_n(t)
		\;=\;
		\sum_{k=0}^n
		\frac{f^{(k)}(x_0)}{k!}\,(t - x_0)^k.
	\]
	Тогда $P_n$ — многочлен, согласующийся с $f$ до порядка $n$ в точке $x_0$.

	\smallskip

	\textbf{Шаг 2.}
	Рассмотрим
	\[
		F(t)
		\;=\;
		f(t)
		\;-\;
		P_n(t).
	\]
	Проверяем, что для $k=0,1,\dots,n$ имеем
	\[
		F^{(k)}(x_0)
		=
		f^{(k)}(x_0)
		\;-\;
		P_n^{(k)}(x_0)
		=
		0,
	\]
	поскольку $P_n$ «копирует» $f$ в производных до порядка $n$.

	\smallskip

	\textbf{Шаг 3.}
	Применим теорему Ролля (или Коши) в подходящей форме:
	поскольку $F^{(k)}(x_0)=0$ для $k\le n$, по индукции доказывают, что найдётся $\xi$ между $x_0$ и $x$, где
	\[
		F^{(n+1)}(\xi) \;=\; 0.
	\]
	Но $F^{(n+1)}(t) = f^{(n+1)}(t) - P_n^{(n+1)}(t)$, а старшие производные $P_n$ равны нулю, значит $F^{(n+1)}(t)=f^{(n+1)}(t)$.

	\smallskip

	\textbf{Шаг 4.}
	Следовательно,
	\[
		F(x)
		=
		f(x)-P_n(x)
		=
		\frac{f^{(n+1)}(\xi)}{(n+1)!}\,(x - x_0)^{n+1}.
	\]
	Перенося $P_n(x)$, получаем:
	\[
		f(x)
		=
		P_n(x)
		+
		\frac{f^{(n+1)}(\xi)}{(n+1)!}\,(x - x_0)^{n+1}.
	\]
	Это и есть остаточный член в форме Лагранжа.
\end{customproof}

\begin{customexample}
	\textbf{Приближённые вычисления.}
	Предположим, надо вычислить $f(x)$ при $x$ близком к $x_0$, зная производные $f^{(k)}(x_0)$.
	\begin{itemize}
		\item Строим полином $P_n(x)$ и считаем $P_n(x)$ за «главный вклад».
		\item Ошибка $|R_n(x)|$ можно оценить сверху, если есть ограничение
		      $|f^{(n+1)}(t)| \le M$ для $t$ между $x_0$ и $x$.
		\item Тогда
		      \(\displaystyle |R_n(x)| \le \frac{M\,|x-x_0|^{n+1}}{(n+1)!}.\)
	\end{itemize}
	Таким образом, зная $M$ и нужный порядок $n$, можно оценить, сколько членов нужно взять, чтобы добиться требуемой точности вычислений.
\end{customexample}
