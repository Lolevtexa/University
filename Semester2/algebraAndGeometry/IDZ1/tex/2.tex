\textbf{Условие:}  
Найти координаты столбца \( x = (-2, -1, 3)^T \) в базисе  

\[
f_1 = \begin{bmatrix} 1 \\ 1 \\ -1 \end{bmatrix}, \quad
f_2 = \begin{bmatrix} 2 \\ 3 \\ -4 \end{bmatrix}, \quad
f_3 = \begin{bmatrix} 0 \\ 1 \\ -1 \end{bmatrix}.
\]

\textbf{Решение:}  
Координаты вектора \( x \) в базисе \( \{ f_1, f_2, f_3 \} \) — это такие коэффициенты \( c_1, c_2, c_3 \), при которых выполняется разложение:

\[
x = c_1 f_1 + c_2 f_2 + c_3 f_3.
\]

\textbf{Шаг 1: Запись системы уравнений}  

Подставляем векторы:

\[
\begin{bmatrix} -2 \\ -1 \\ 3 \end{bmatrix}
=
c_1 \begin{bmatrix} 1 \\ 1 \\ -1 \end{bmatrix}
+ c_2 \begin{bmatrix} 2 \\ 3 \\ -4 \end{bmatrix}
+ c_3 \begin{bmatrix} 0 \\ 1 \\ -1 \end{bmatrix}.
\]

Распишем по координатам:

\[
\begin{cases}
c_1 + 2c_2 = -2, \\
c_1 + 3c_2 + c_3 = -1, \\
- c_1 - 4c_2 - c_3 = 3.
\end{cases}
\]

\textbf{Шаг 2: Решение системы уравнений методом Гаусса}  

Запишем расширенную матрицу системы:

\[
\begin{bmatrix}
1 & 2 & 0 & | -2 \\
1 & 3 & 1 & | -1 \\
-1 & -4 & -1 & | 3
\end{bmatrix}.
\]

Вычтем первую строку из второй:

\[
\begin{bmatrix}
1 & 2 & 0 & | -2 \\
0 & 1 & 1 & | 1 \\
-1 & -4 & -1 & | 3
\end{bmatrix}.
\]

Прибавим первую строку к третьей:

\[
\begin{bmatrix}
1 & 2 & 0 & | -2 \\
0 & 1 & 1 & | 1 \\
0 & -2 & -1 & | 1
\end{bmatrix}.
\]

Умножим вторую строку на 2 и сложим с третьей:

\[
\begin{bmatrix}
1 & 2 & 0 & | -2 \\
0 & 1 & 1 & | 1 \\
0 & 0 & 1 & | 3
\end{bmatrix}.
\]

\textbf{Шаг 3: Вычисление неизвестных}  

Из третьего уравнения:

\[
c_3 = 3.
\]

Из второго уравнения:

\[
c_2 + c_3 = 1 \quad \Rightarrow \quad c_2 + 3 = 1 \quad \Rightarrow \quad c_2 = -2.
\]

Из первого уравнения:

\[
c_1 + 2c_2 = -2 \quad \Rightarrow \quad c_1 + 2(-2) = -2 \quad \Rightarrow \quad c_1 - 4 = -2 \quad \Rightarrow \quad c_1 = 2.
\]

\textbf{Ответ:}  
Координаты вектора \( x \) в данном базисе:

\[
[x]_f = \begin{bmatrix} 2 \\ -2 \\ 3 \end{bmatrix}.
\]
