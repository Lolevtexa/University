\textbf{Условие:}  
Даны столбцы:

\[
e_1 = \begin{bmatrix} 1 \\ -3 \\ -1 \end{bmatrix}, \quad
e_2 = \begin{bmatrix} -1 \\ 4 \\ 1 \end{bmatrix}, \quad
e_3 = \begin{bmatrix} 0 \\ -2 \\ 1 \end{bmatrix}.
\]

\[
f_1 = \begin{bmatrix} 1 \\ -3 \\ -2 \end{bmatrix}, \quad
f_2 = \begin{bmatrix} -1 \\ 3 \\ 3 \end{bmatrix}, \quad
f_3 = \begin{bmatrix} 1 \\ -2 \\ -1 \end{bmatrix}.
\]

\[
x = \begin{bmatrix} 3 \\ -3 \\ -2 \end{bmatrix}.
\]

a) Найти матрицы перехода \( C_{e \to f} \) и \( C_{f \to e} \).  
b) Найти координаты \( x \) в базисе \( e \).

---

\textbf{Решение}  

\textbf{1. Нахождение матрицы перехода \( C_{e \to f} \)}

Матрица перехода \( C_{e \to f} \) выражает базисные векторы \( e \) через базис \( f \):

\[
E = F C_{e \to f}.
\]

\textbf{Шаг 1: Запись матриц базисов}

\[
E = \begin{bmatrix} 
1 & -1 & 0 \\ 
-3 & 4 & -2 \\ 
-1 & 1 & 1
\end{bmatrix}, 
\quad
F = \begin{bmatrix} 
1 & -1 & 1 \\ 
-3 & 3 & -2 \\ 
-2 & 3 & -1
\end{bmatrix}.
\]

\textbf{Шаг 2: Нахождение обратной матрицы \( F^{-1} \)}  

Для нахождения \( C_{e \to f} \) воспользуемся формулой:

\[
C_{e \to f} = F^{-1} E.
\]

Вычисляя \( F^{-1} \), получаем:

\[
F^{-1} =
\begin{bmatrix} 
-1 & -1 & -1 \\ 
1 & -2 & 3 \\ 
3 & -3 & 3
\end{bmatrix}.
\]

\textbf{Шаг 3: Вычисление \( C_{e \to f} \)}  

Умножаем:

\[
C_{e \to f} = F^{-1} E.
\]

После вычислений:

\[
C_{e \to f} = 
\begin{bmatrix} 
3 & -4 & 2 \\ 
-2 & 3 & -1 \\ 
-3 & 4 & -2
\end{bmatrix}.
\]

\textbf{2. Нахождение матрицы перехода \( C_{f \to e} \)}

\[
C_{f \to e} = (C_{e \to f})^{-1}.
\]

Так как \( C_{e \to f} \) аналогична \( F^{-1} \), то:

\[
C_{f \to e} = 
\begin{bmatrix} 
-1 & -1 & -1 \\ 
1 & -2 & 3 \\ 
3 & -3 & 3
\end{bmatrix}.
\]

---

\textbf{3. Нахождение координат \( x \) в базисе \( e \)}

Ищем коэффициенты \( \alpha_1, \alpha_2, \alpha_3 \), такие что:

\[
x = \alpha_1 e_1 + \alpha_2 e_2 + \alpha_3 e_3.
\]

Решаем систему:

\[
E \alpha = x.
\]

\[
\begin{bmatrix} 
1 & -1 & 0 \\ 
-3 & 4 & -2 \\ 
-1 & 1 & 1
\end{bmatrix}
\begin{bmatrix} \alpha_1 \\ \alpha_2 \\ \alpha_3 \end{bmatrix}
=
\begin{bmatrix} 3 \\ -3 \\ -2 \end{bmatrix}.
\]

\textbf{Шаг 1: Приведение к ступенчатому виду}

\[
\begin{bmatrix} 
1 & -1 & 0 & | 3 \\ 
-3 & 4 & -2 & | -3 \\ 
-1 & 1 & 1 & | -2
\end{bmatrix}.
\]

Прибавим первую строку ко второй:

\[
\begin{bmatrix} 
1 & -1 & 0 & | 3 \\ 
0 & 1 & -2 & | 0 \\ 
-1 & 1 & 1 & | -2
\end{bmatrix}.
\]

Прибавим первую строку к третьей:

\[
\begin{bmatrix} 
1 & -1 & 0 & | 3 \\ 
0 & 1 & -2 & | 0 \\ 
0 & 0 & 1 & | 1
\end{bmatrix}.
\]

\textbf{Шаг 2: Вычисление неизвестных}

Из третьего уравнения:

\[
\alpha_3 = 1.
\]

Из второго уравнения:

\[
\alpha_2 - 2\alpha_3 = 0 \quad \Rightarrow \quad \alpha_2 - 2 = 0 \quad \Rightarrow \quad \alpha_2 = 2.
\]

Из первого уравнения:

\[
\alpha_1 - \alpha_2 = 3 \quad \Rightarrow \quad \alpha_1 - 2 = 3 \quad \Rightarrow \quad \alpha_1 = 5.
\]

\textbf{Ответ:}  

1. Матрица перехода \( C_{e \to f} \):

\[
\begin{bmatrix} 
3 & -4 & 2 \\ 
-2 & 3 & -1 \\ 
-3 & 4 & -2
\end{bmatrix}
\]

2. Матрица перехода \( C_{f \to e} \):

\[
\begin{bmatrix} 
-1 & -1 & -1 \\ 
1 & -2 & 3 \\ 
3 & -3 & 3
\end{bmatrix}
\]

3. Координаты \( x \) в базисе \( e \):

\[
[x]_e = \begin{bmatrix} 5 \\ 2 \\ 1 \end{bmatrix}.
\]
