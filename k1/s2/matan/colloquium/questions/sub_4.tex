% ===== sub_4.tex =====
% Вспомогательные определения и теоремы,
% не являющиеся центральными в данном вопросе,
% но используемые в доказательстве теоремы Кантора

\textbf{Определение компакта в $\mathbb{R}$.}
Множество $K\subset\mathbb{R}$ называется \emph{компактным}, если оно \emph{замкнуто} и \emph{ограничено}.

\medskip

\textbf{Теорема Болльцано--Вейерштрасса.}
Всякая \emph{ограниченная} последовательность $(x_n)$ в $\mathbb{R}$ имеет \emph{сходящуюся} подпоследовательность.  
На языке компактных множеств: любая последовательность, целиком лежащая в компактном $K$, содержит сходящуюся подпоследовательность с пределом в $K$.

\medskip

\textbf{Определение непрерывности (подробно).}
Напомним: $f$ непрерывна на $X$, если для любой точки $x_0\in X$:
\[
\forall \varepsilon>0\;\;\exists \delta>0:\;
|x - x_0|<\delta \implies |f(x)-f(x_0)|<\varepsilon.
\]

\medskip

% (Другие напоминания и определения — при необходимости).
