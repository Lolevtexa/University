% ===== 9.tex =====
% "Теорема Лиувилля. Пример трансцендентного числа."

% 1. Определения

\textbf{Алгебраическое и трансцендентное число.}
\begin{itemize}
  \item \emph{Алгебраическое число} — это действительное (или комплексное) число, являющееся корнем многочлена с рациональными (или целыми) коэффициентами. Например, $\sqrt{2}$, $\sqrt[3]{7}$.
  \item \emph{Трансцендентное число} — это число, \textbf{не} являющееся алгебраическим. Примеры: $e$, $\pi$ (доказано Линдеманом), а также более специальные конструкции (числа Лиувилля).
\end{itemize}

\textbf{Приближение чисел рациональными дробями.}
Для действительного числа $\alpha$ говорят, что \emph{оно допускает «слишком хорошие» рациональные приближения}, если существуют бесконечные наборы дробей $\tfrac{p}{q}$, удовлетворя 
\[
\left|\alpha - \frac{p}{q}\right| \;<\; \frac{1}{q^n}
\]
для больших $q$, при некоторых $n$ существенно превосходящих $1$.

\medskip

% 2. Теоремы и ключевые утверждения

\textbf{Теорема Лиувилля (о трансцендентных числах).}
Если действительное число $\alpha$ удовлетворяет следующему условию: существует $n>1$ и бесконечно много рациональных дробей $\tfrac{p}{q}$, для которых
\[
\left|\alpha - \frac{p}{q}\right| \;<\; \frac{1}{q^n},
\]
то число $\alpha$ \textbf{не} алгебраично (то есть оно \textbf{трансцендентно}).

\medskip

% 3. Основные идеи доказательства (коротко)

\begin{itemize}
  \item Предположим противное: $\alpha$ — алгебраическое, но допускает «чрезмерно точные» рациональные приближения.  
  \item Рассматривается соответствующий \emph{минимальный многочлен} числа $\alpha$ степени $d$.  
  \item Показывается, что при достаточно хороших приближениях противоречат оценкам, вытекающим из теоремы о том, как далеко корни полинома могут быть друг от друга.  
  \item Возникает противоречие, откуда делается вывод: число $\alpha$ не может быть алгебраическим, оно — трансцендентно.
\end{itemize}

\medskip

% 4. Полное доказательство (пошагово, ссылаясь на sub_9, где нужно)

\textbf{Доказательство (классический эскиз):}
\begin{enumerate}
  \item \textbf{Предположение.} Пусть $\alpha$ — корень целого многочлена $P(x)$ степени $d$. Считаем $P(x)$ приведённым (нет общих делителей).  
  \item \textbf{Рациональные приближения.} Допустим, существуют бесконечно многие $\frac{p}{q}$ с $|q\alpha - p| < q^{\,1-n}$, то есть $|\alpha - p/q|<1/q^n$ при $n>d$.  
  \item \textbf{Оценка многочлена.}  
    Рассмотрим $P\bigl(\tfrac{p}{q}\bigr)$; используя замену $x=\tfrac{p}{q}$ и разложение $P(\alpha)=0$, анализируют величину $|P(\tfrac{p}{q})-P(\alpha)|$.  
  \item \textbf{Неравенство:}  
    Т.к. $P(x)$ — многочлен степени $d$, разность $|P(x)-P(\alpha)|$ может быть оценена через $|x-\alpha|$, где в высших степенях играют роль биномиальные формулы, а коэффициенты — целые.  
  \item \textbf{Противоречие:}  
    При «слишком» быстром убывании $|x-\alpha| < 1/q^n$ с $n>d$, получается невозможная малая оценка для $|P(p/q)|$, хотя $p/q$ — рациональная точка, где $P$ должно принимать вполне «ограниченное снизу» значение (не равное нулю, раз $p/q\neq \alpha$).  
  \item \textbf{Итог.}  
    Противоречие доказывает, что $\alpha$ не может быть алгебраическим. Следовательно, $\alpha$ — трансцендентное.
\end{enumerate}

\medskip

% 5. Пример трансцендентного числа: "число Лиувилля"

\textbf{Пример: число Лиувилля.}
Классический пример: 
\[
\beta = \sum_{k=1}^{\infty} 10^{-k!}
\;=\; 0{.}110001000000000000000001\dots
\]
Здесь в десятичной записи стоят единицы на позициях $1!,\,2!,\,3!,\dots$ и нули в остальных. Нетрудно проверить, что для любого $n$ можно найти рациональную дробь $p/q$ с $q=10^{n!}$, которая приближает $\beta$ с точностью $1/q^n$. По \textbf{теореме Лиувилля}, такое число $\beta$ \emph{трансцендентно}.

\medskip

% -- Логическая связность --
% (Определения "минимального многочлена", "степени" и т.д. в sub_9.tex)
