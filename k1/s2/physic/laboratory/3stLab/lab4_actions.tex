\documentclass[a4paper,12pt]{article}

\usepackage[utf8]{inputenc}   
\usepackage[T2A]{fontenc}     
\usepackage[russian]{babel}   
\usepackage{amsmath,amssymb}  
\usepackage{geometry}
\geometry{top=2cm,bottom=2cm,left=2cm,right=2cm}
\usepackage{booktabs}

\begin{document}

\begin{center}
  \Large \textbf{Подробный план действий \\
  по выполнению лабораторной работы №4}
\end{center}

\vspace{1em}

\section*{Общие сведения}
В данной работе изучается дифракция Фраунгофера на прозрачной дифракционной решётке, определяется положение главных дифракционных максимумов (углы), а далее, по формулам дифракции, вычисляются:
\begin{itemize}
  \item длины волн световых линий (если решётка известна),
  \item или постоянная решётки $d$ (если одна из линий светового спектра известна).
\end{itemize}
Также определяется угловая дисперсия $D_\varphi$ и разрешающая способность $R$ решётки.

\section*{Порядок выполнения эксперимента}
\begin{enumerate}
  \item \textbf{Подготовка лампы.} 
  \begin{itemize}
    \item Включите ртутную лампу и дайте ей прогреться достаточное время, чтобы линии спектра стали стабильными.
  \end{itemize}
  
  \item \textbf{Установка гониометра.}
  \begin{itemize}
    \item Убедитесь, что коллиматор (с щелью) направлен на ртутную лампу.
    \item Отрегулируйте положение щели и коллиматора, чтобы на выходе коллиматора сформировался параллельный пучок.
  \end{itemize}
  
  \item \textbf{Установка дифракционной решётки.}
  \begin{itemize}
    \item Закрепите решётку на держателе гониометра так, чтобы она была перпендикулярна лучам, выходящим из коллиматора.
    \item Зрительную трубу гониометра установите примерно в направлении нормали к решётке.
  \end{itemize}
  
  \item \textbf{Наблюдение спектра.}
  \begin{itemize}
    \item Через окуляр зрительной трубы найдите \textbf{центральный максимум} ($m=0$). 
    \item Совместите перекрестие нити окуляра с яркой линией в центре (нулевой порядок) и снимите отсчёт $\alpha_0$ с лимба гониометра.
    \item Поверните зрительную трубу \textbf{в одну сторону (например, вправо)}, чтобы поймать последовательно максимумы $m=+1, +2, +3$ и для каждого зафиксируйте угол $\alpha_m^+$.
    \item Поверните зрительную трубу \textbf{в другую сторону (влево)}, снимите углы $\alpha_m^-$ для $m=-1, -2, -3$.
    \item Повторите это \textbf{для каждой видимой спектральной линии} (желтая, зелёная, синяя). 
      \\
      Если в методичке требуется по три повтора — сделайте три серии измерений, чтобы усреднить результат.
  \end{itemize}
  
  \item \textbf{Заполнение протокола.}
  \begin{itemize}
    \item Все углы ($\alpha_0, \alpha_m^\pm$) заносите в Таблицу измерений (см. \texttt{lab4\_protocol.tex}).
    \item Для каждой линии сформируйте отдельную таблицу.
  \end{itemize}
  
  \item \textbf{Расчёт углов $\varphi_m$.}
  \begin{itemize}
    \item По формуле $\varphi_m = \alpha_m - \alpha_0$ (или $\varphi_m^+ = \alpha_m^+ - \alpha_0$, $\varphi_m^- = \alpha_m^- - \alpha_0$) найдите углы дифракции.
    \item При необходимости возьмите \textbf{среднее} значение между $\varphi_m^+$ и $\varphi_m^-$ (если методика это оговаривает).
  \end{itemize}
  
  \item \textbf{Вычисление параметров.}
  \begin{itemize}
    \item Если в работе требуется найти \textbf{длину волны} $\lambda$:
    \[
      \lambda = \frac{d \,\sin \varphi_m}{m}.
    \]
    \item Если наоборот, требуется найти \textbf{постоянную решётки} $d$ (при известной $\lambda$):
    \[
      d = \frac{m\, \lambda}{\sin \varphi_m}.
    \]
    \item По полученным величинам найдите:
      \[
        D_{\varphi} = \frac{m}{d\, \cos(\varphi_m)}, 
        \qquad
        R = m \, N.
      \]
  \end{itemize}
  
  \item \textbf{Обработка погрешностей.}
  \begin{itemize}
    \item Используйте методику обработки погрешностей, изложенную в пособии (метод наименьших квадратов, выборочный метод или метод переноса погрешностей).
    \item Оцените $\Delta \lambda$, $\Delta d$ и т.п. с заданной доверительной вероятностью (обычно $P=0.95$).
  \end{itemize}
  
  \item \textbf{Анализ результатов.}
  \begin{itemize}
    \item Сравните результаты с табличными значениями для линий ртутной лампы.
    \item Обсудите расхождения и возможные причины (погрешность в установке решётки, неточная юстировка, ширина щели и т.д.).
  \end{itemize}
  
  \item \textbf{Вывод.} 
  \begin{itemize}
    \item Сделайте вывод, совпадают ли измеренные $\lambda$ (или $d$) с известными значениями в пределах ошибок.
    \item Упомяните, чему равны полученные $D_{\varphi}$ и $R$ и каков порядок спектра $m$, при котором наблюдается наилучшее разложение в спектр.
  \end{itemize}
\end{enumerate}

\section*{Что сдавать преподавателю}
\begin{itemize}
  \item Заполненный \textbf{протокол} (таблицы с углами, расчётами).
  \item Графики (если требуются): зависят от методики, иногда строят зависимость $\sin \varphi_m$ от $m$.
  \item Подробные вычисления (формулы, числовые результаты, погрешности).
  \item Формальные выводы (пункт 9).
\end{itemize}

\vfill
\textit{Подпись преподавателя:} \hrulefill

\end{document}
