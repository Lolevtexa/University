% ===== 11.tex =====
% "Формула Тейлора с остатком в форме Лагранжа. Приближённые вычисления по формуле Тейлора."

% 1. Определения

\textbf{Формула Тейлора и остаточный член.}
Пусть $f$ имеет $(n+1)$-ю производную в окрестности точки $x_0$. Тогда можно представить $f(x)$ в виде:
\[
f(x) = f(x_0) + \frac{f'(x_0)}{1!}\,(x - x_0) + \dots + \frac{f^{(n)}(x_0)}{n!}\,(x - x_0)^n + R_n(x),
\]
где $R_n(x)$ называется \emph{остаточным членом}.

\medskip

\textbf{Форма Лагранжа остаточного члена.}
Существует точка $\xi$ между $x_0$ и $x$ (включая возможность $\xi \in (x,x_0)$), такая что
\[
R_n(x) = \frac{f^{(n+1)}(\xi)}{(n+1)!}\,(x - x_0)^{n+1}.
\]
Это утверждение мы будем доказывать ниже (используя теорему Лагранжа о среднем значении).

\medskip

% 2. Теоремы и ключевые утверждения

\textbf{Формулировка (Тейлор + Лагранж).}
Пусть $f$ непрерывно дифференцируема на отрезке $[x_0,x]$ (или $[x,x_0]$) до порядка $(n+1)$. Тогда:
\[
f(x) = \underbrace{\sum_{k=0}^{n} \frac{f^{(k)}(x_0)}{k!}\,(x-x_0)^k}_{\text{многочлен Тейлора порядка }n} \;+\; \frac{f^{(n+1)}(\xi)}{(n+1)!}\,(x - x_0)^{n+1},
\]
где $\xi$ лежит между $x_0$ и $x$.

\medskip

\textbf{Приближённые вычисления.}
Чтобы вычислить $f(x)$ примерно, берут полином Тейлора степени $n$:
\[
P_n(x) = \sum_{k=0}^{n} \frac{f^{(k)}(x_0)}{k!}\,(x - x_0)^k,
\]
и оценивают ошибку (погрешность) через
\[
|R_n(x)| = \left|\frac{f^{(n+1)}(\xi)}{(n+1)!}\,(x - x_0)^{n+1}\right|.
\]
Обычно используют верхнюю оценку: если $|f^{(n+1)}(t)| \le M$ на $[x_0,x]$, то
\[
|R_n(x)| \;\le\; \frac{M\,|x - x_0|^{n+1}}{(n+1)!}.
\]

\medskip

% 3. Основные идеи доказательства (коротко)

\begin{itemize}
  \item Рассмотрим функцию $F(x) = f(x) - P_n(x)$, где $P_n(x)$ — многочлен Тейлора степени $n$ вокруг $x_0$.  
  \item По определению $P_n$, все производные $F$ до порядка $n$ в точке $x_0$ равны нулю.  
  \item Применяем теорему Лагранжа \textit{в обобщённом виде} (Теорема Коши или вариант теоремы Ролля) к $F$ на отрезке $[x_0,x]$, получаем существование точки $\xi$, где $(n+1)$-я производная $F^{(n+1)}(\xi)=0$. Но $F^{(n+1)}(t)=f^{(n+1)}(t)$, потому что $(n+1)$-я производная многочлена $P_n$ равна нулю.  
  \item Отсюда возникает
  \[
  F(x) = F(x_0) + \dots + \frac{F^{(n+1)}(\xi)}{(n+1)!} (x-x_0)^{n+1},
  \]
  но $F(x_0)=F'(x_0)=\dots=F^{(n)}(x_0)=0$. Значит $F(x)=\frac{f^{(n+1)}(\xi)}{(n+1)!}(x-x_0)^{n+1}$.
\end{itemize}

\medskip

% 4. Полное доказательство (шаги, ссылаясь на sub_11 если нужно)

\textbf{Шаг 1: Построение многочлена Тейлора.}
\[
P_n(t) \;=\; \sum_{k=0}^{n} \frac{f^{(k)}(x_0)}{k!}\,(t - x_0)^k.
\]
По определению производной порядок $k$, $P_n$ «согласован» с $f$ до $n$-го порядка в точке $x_0$.

\textbf{Шаг 2: Рассмотрим $F(t)=f(t) - P_n(t)$.}
Тогда
\[
F^{(k)}(x_0) = f^{(k)}(x_0) - P_n^{(k)}(x_0) = 0 \quad\text{для } k=0,1,\dots,n.
\]

\textbf{Шаг 3: Применяем обобщённую теорему Ролля.}
На отрезке $[x_0, x]$ (или $[x,x_0]$), функция $F$ удовлетворяет условию $F^{(k)}(x_0)=F^{(k)}(x) \dots$ (не все равны, но ключ в том, что мы можем включить вспомогательную функцию «$G(t)=\dots$» — см. \texttt{sub\_11.tex} о теореме Коши). В итоге \emph{по индукции} выводится, что существует $\xi$ между $x_0$ и $x$, где $F^{(n+1)}(\xi)=0$. А $F^{(n+1)}(t)=f^{(n+1)}(t)$.

\textbf{Шаг 4: Остаточный член.}
\[
F(x) = f(x)-P_n(x) = \frac{f^{(n+1)}(\xi)}{(n+1)!}\,(x - x_0)^{n+1}.
\]
Перенося, получаем:
\[
f(x)=P_n(x) + \frac{f^{(n+1)}(\xi)}{(n+1)!}\,(x - x_0)^{n+1}.
\]
Это и есть искомая формула Тейлора с остатком в форме Лагранжа.

\medskip

% 5. Пример приближённых вычислений

\textbf{Пример: функция $f(x)=\sqrt{1+x}$, при $x$ мало.}
\begin{itemize}
  \item Выбираем $x_0=0$. Имеем $f(0)=1$, $f'(0)=\tfrac12$, $f''(0)=-\tfrac{1}{8}$, \dots .
  \item Третьепорядочное приближение: 
  \[
    P_2(x)=1 + \frac12 x - \frac{1}{8} x^2.
  \]
  \item Ошибка (остаток) $R_2(x)=\frac{f^{(3)}(\xi)}{3!} x^3$ для некоторого $\xi \in (0,x)$.  
  Используя оценку $|f^{(3)}(t)|\le M$ на $[0,x]$, будет 
  \(\bigl|R_2(x)\bigr|\le \frac{M|x|^3}{6}.\)
\end{itemize}
Так можно оценить точность приближённого вычисления $\sqrt{1+x}\approx 1 + \tfrac12 x - \tfrac18 x^2$ для малых $x$.

\medskip

% -- Логическая связность --
% (Подробные определения "обобщённой теоремы Ролля", "f^{(k)}(x_0)", "непрерывность" и т.д. см. sub_11.tex)
