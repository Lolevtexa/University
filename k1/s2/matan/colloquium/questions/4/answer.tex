% ==================
% answer.tex
% "Равномерная непрерывность. Примеры. Теорема Кантора о равномерной непрерывности."
% ==================

\begin{customtheorem}[Теорема Кантора]
	Если $f$ непрерывна на \emph{компактном} множестве $X \subset \mathbb{R}$,
	то $f$ равномерно непрерывна на $X$.
\end{customtheorem}

\begin{proofplan}
	\begin{enumerate}
		\item Доказывать будем \textit{от противного}: считаем, что $f$ непрерывна на компакте,
		      но \textbf{не} равномерно непрерывна.
		\item Из этого следует существование $\varepsilon_0>0$, при котором нельзя подобрать
		      «глобальное» $\delta$, годящееся для всех точек в $X$.
		\item Для каждой $n$, пусть $\delta = \frac{1}{n}$. Тогда находятся точки $(x_n,y_n)$
		      с $|x_n-y_n|<\tfrac{1}{n}$, но $|f(x_n)-f(y_n)|\ge\varepsilon_0$.
		\item Используем компактность $X$: извлекаем сходящуюся подпоследовательность $(x_{n_k})\to c$.
		      Поскольку $|x_{n_k}-y_{n_k}|<1/{n_k}\to 0$, получаем $y_{n_k}\to c$ тоже.
		\item По непрерывности $f$ имеем $f(x_{n_k})\to f(c)$ и $f(y_{n_k})\to f(c)$,
		      значит $|f(x_{n_k}) - f(y_{n_k})|\to 0$, что противоречит условию $\ge\varepsilon_0$.
	\end{enumerate}
\end{proofplan}

\begin{customproof}
	Пусть $f$ непрерывна на компактном $X$, но, вопреки теореме, \emph{не} равномерно непрерывна. Тогда
	\[
		\exists\,\varepsilon_0>0
		\quad\forall\,\delta>0,\;\exists x,y\in X :
		\quad |x-y|<\delta,\;\;\;|f(x)-f(y)| \ge \varepsilon_0.
	\]
	Выберем $\delta=1/n$ и построим пары $(x_n,y_n)$ c
	\[
		|x_n - y_n| < \tfrac{1}{n}, \quad |f(x_n)-f(y_n)| \ge \varepsilon_0.
	\]
	Так как $X$ — компакт, последовательность $(x_n)$ имеет сходящуюся подпоследовательность
	$(x_{n_k}) \to c\in X$. По условию $|x_{n_k}-y_{n_k}| < 1/{n_k}\to0$, значит $y_{n_k}\to c$ тоже.
	Из непрерывности $f$ в точке $c$ следует
	\[
		f(x_{n_k}) \to f(c),\quad f(y_{n_k}) \to f(c),
	\]
	так что $|f(x_{n_k}) - f(y_{n_k})|\to 0$. Но по построению
	$|f(x_{n_k}) - f(y_{n_k})|\ge \varepsilon_0 > 0$. Это даёт противоречие.

	Следовательно, наша гипотеза о «неравномерной непрерывности» была ошибочной,
	и $f$ действительно равномерно непрерывна на $X$.
\end{customproof}

\begin{customexample}
	\begin{itemize}
		\item \textbf{Пример (равномерная непрерывность на $\mathbb{R}$):}
		      Линейная функция $f(x)=kx + b$. Имеем
		      $|f(x_1)-f(x_2)|=|k|\cdot|x_1-x_2|$, что легко «контролируется»
		      выбором $\delta=\tfrac{\varepsilon}{|k|}$.
		\item \textbf{Пример (не равномерная на $\mathbb{R}$):}
		      $f(x)=x^2$. Несмотря на непрерывность на $\mathbb{R}$, не получается
		      «глобально» связать $|x_1-x_2|$ с $|f(x_1)-f(x_2)|$ единым $\delta(\varepsilon)$,
		      поскольку при больших $|x|$ влияние приращения аргумента сильно возрастает.
		\item Аналогично, $e^x$ неравномерно непрерывна на всей оси: чем больше $x$,
		      тем чувствительнее функция к малым изменениям $x$.
	\end{itemize}
\end{customexample}
