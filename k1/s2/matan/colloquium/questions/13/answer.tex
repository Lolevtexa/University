% ==================
% answer.tex
% (Основной материал:
%  "Формула Стирлинга (с равенством).")
% ==================

\begin{customtheorem}[Формула Стирлинга с остаточным множителем]
	Для любого натурального $n$ справедливо:
	\[
		n!
		\;=\;
		\sqrt{2\pi\,n}
		\,\Bigl(\tfrac{n}{e}\Bigr)^n
		\,\exp\bigl(\theta_n\bigr),
	\]
	где поправка $\theta_n$ удовлетворяет некоторому неравенству вида
	\[
		\frac{1}{12n+1}
		\;<\;
		\theta_n
		\;<\;
		\frac{1}{12n}.
	\]
	В частности,
	\[
		\theta_n \;\to\;0,
		\quad \text{при }n\to\infty,
	\]
	и мы получаем строгую «формулу Стирлинга с равенством» и контролируем остаток.
\end{customtheorem}

\begin{proofplan}
	\begin{enumerate}
		\item Переход к логарифмам: $\ln(n!)= \sum_{k=1}^n\ln k$.
		\item Сравнение суммы $\sum \ln k$ с интегралом $\int \ln x\,dx$ и далее точная оценка
		      (формула Эйлера–Маклорена), дающая экспоненциальный вид остатка.
		\item Получение не только эквивалентности, но и точных границ для «ошибки»
		      $\theta_n$ (или эквивалентно $e^{\theta_n}$).
		\item Экспоненцирование результата и проверка, что
		      $n! = \sqrt{2\pi\,n}\,\bigl(\tfrac{n}{e}\bigr)^n \exp(\theta_n)$,
		      при $\theta_n$ в указанных границах.
	\end{enumerate}
\end{proofplan}

\begin{customproof}
	\textbf{1. Переход к логарифмам.}
	Аналогично стандартному выводу формулы Стирлинга (см. «эквивалентность»),
	\[
		\ln(n!)
		\;=\;
		\sum_{k=1}^n \ln k.
	\]
	Мы хотим получить \emph{не} просто асимптотику, а точное равенство вида
	\(\ln(n!) = \ln\bigl(\sqrt{2\pi n}(\tfrac{n}{e})^n\bigr) + \theta_n\).

	\smallskip

	\textbf{2. Применение формулы Эйлера–Маклорена (укороченный вид).}
	Более полная форма Эйлера–Маклорена гласит, что
	\[
		\sum_{k=1}^n \ln k
		\;=\;
		\int_1^n \ln x\,dx
		\;+\; \frac12[\ln 1 + \ln n]
		\;+\; R(n),
	\]
	где $R(n)$ оценивается с помощью ряда Бернулли, давая интервальные границы, например
	\[
		\frac{1}{12(n+1)} < R(n) < \frac{1}{12n}.
	\]
	При более аккуратной записи получается
	\[
		\ln(n!)
		\;=\;
		n\ln n - n + \tfrac12\ln n
		\;+\; \theta_n,
	\]
	c оценкой \(\theta_n\) между \(\tfrac{1}{12(n+1)}\) и \(\tfrac{1}{12n}\) (с разными знаками, в зависимости от формы записи).

	\smallskip

	\textbf{3. Экспоненцирование.}
	Пусть
	\[
		L_n
		= n\ln n - n + \tfrac12\ln(2\pi n)
		\quad (\text{здесь } \sqrt{2\pi n} \text{ уже включено}).
	\]
	Тогда
	\[
		\ln(n!)
		\;=\;
		L_n
		\;+\;
		\bigl(\theta_n - \tfrac12\ln(2\pi)\bigr),
	\]
	или, эквивалентно,
	\[
		n!
		\;=\;
		\exp(L_n)
		\cdot \exp(\theta'_n),
	\]
	где $\theta'_n$ есть скорректированная «ошибка». Развитие показывает, что
	\[
		\exp(L_n)
		=
		\sqrt{2\pi\,n}\,\Bigl(\tfrac{n}{e}\Bigr)^n.
	\]
	В итоге
	\[
		n!
		\;=\;
		\sqrt{2\pi\,n}\,\Bigl(\tfrac{n}{e}\Bigr)^n
		\exp(\theta'_n),
	\]
	и формула даёт более тонкий контроль над $\theta'_n$, в частности
	\[
		0 < \theta'_n < \frac{1}{12n},\quad \text{или другие варианты}.
	\]

	\smallskip

	\textbf{4. Итог.}
	Таким образом, получается «формула Стирлинга с равенством»:
	\[
		n!
		=
		\sqrt{2\pi\,n}\,\Bigl(\tfrac{n}{e}\Bigr)^n \exp(\theta_n),
	\]
	с конкретными узкими границами для \(\theta_n\). Например, одно из классических утверждений —
	\[
		\frac{1}{12n+1}
		\;<\;
		\theta_n
		\;<\;
		\frac{1}{12n}.
	\]
	Это завершает доказательство.
\end{customproof}

\begin{customexample}
	\textbf{Сравнение при $n=5$ или $n=10$.}
	\begin{itemize}
		\item $5! = 120$.
		      По формуле Стирлинга (с равенством),
		      \[
			      5!
			      = \sqrt{2\pi\cdot 5}\,\Bigl(\tfrac{5}{e}\Bigr)^5 \,\exp(\theta_5).
		      \]
		      Можно вычислить левую часть (120) и сравнить с $\sqrt{10\pi}\,(\frac{5}{e})^5$,
		      оценив $\exp(\theta_5)$.
		\item При больших $n$ (например, $n=10$) точность заметно выше, и можно видеть,
		      что $\theta_n$ становится ближе к 0 (около $0.03$...$0.02$, в зависимости от
		      формы оценки).
	\end{itemize}
\end{customexample}
