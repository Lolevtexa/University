% ==================
% answer.tex
% (Основной материал:
%  "Теорема Лиувилля. Пример трансцендентного числа")
% ==================

\begin{customtheorem}[Теорема Лиувилля]
	Пусть действительное число $\alpha$ удовлетворяет следующему условию:
	существует $n>1$ и бесконечно много рациональных дробей $\tfrac{p}{q}$,
	для которых
	\[
		\left|\alpha - \frac{p}{q}\right|
		\;<\;
		\frac{1}{\,q^n\,}.
	\]
	Тогда $\alpha$ \textbf{не} алгебраично (то есть оно \textbf{трансцендентно}).
\end{customtheorem}

\begin{proofplan}
	\begin{enumerate}
		\item Предположим противное: $\alpha$ — алгебраическое, корень некоторого
		      целочисленного многочлена $P(x)$ степени $d$.
		\item Допустим, есть бесконечно много $\frac{p}{q}$,
		      дающих $|\alpha - \frac{p}{q}| < \frac{1}{q^n}$, причём $n>d$.
		\item Рассмотреть $|P(\tfrac{p}{q}) - P(\alpha)|$ и использовать свойства
		      многочлена $P$ (его степень, целые коэффициенты).
		\item Получить противоречие из «слишком хорошего» приближения, показывая,
		      что $P(\tfrac{p}{q})$ оказывается «слишком близко» к нулю, но не равно нулю.
		\item Заключить, что $\alpha$ не может быть алгебраическим, значит трансцендентно.
	\end{enumerate}
\end{proofplan}

\begin{customproof}
	Пусть, ради противного, $\alpha$ — корень целого многочлена $P(x)$ степени $d$
	(приведённого, без общих делителей).
	Предположим, что существуют бесконечно многие дроби $\tfrac{p}{q}$ с
	\[
		\left|\alpha - \frac{p}{q}\right|
		\;<\;
		\frac{1}{\,q^n\,}
	\]
	для некоторого $n > d$. Тогда $|q\alpha - p| < q^{\,1-n}$.

	Рассмотрим $P\!\bigl(\tfrac{p}{q}\bigr) - P(\alpha)=P\!\bigl(\tfrac{p}{q}\bigr)$ (ведь $P(\alpha)=0$).
	С помощью разложения многочлена (по формуле Тейлора или биному),
	учитывая целые коэффициенты и тот факт, что $|\tfrac{p}{q}-\alpha|<\tfrac{1}{q^n}$,
	при достаточно больших $q$ получаем оценку $|P(\tfrac{p}{q})|$ слишком малой для
	ненулевого целочисленного $P$.
	Например, высшие степени $(\tfrac{p}{q}-\alpha)^d$ дают вклад порядка $\frac{1}{q^{dn}}$, что при $n>d$ «умирает» так быстро, что невозможно без $P(\tfrac{p}{q})$ быть нулём.
	Иными словами, получаем противоречие с тем, что $P(\tfrac{p}{q})$ — целая комбинация,
	не может быть «чересчур» малой, если $\tfrac{p}{q}\neq \alpha$.

	Следовательно, наше допущение о «слишком хороших» приближениях алгебраического $\alpha$ ложно. Значит $\alpha$ — трансцендентное.
\end{customproof}

\begin{customexample}
	\textbf{Число Лиувилля.}
	Рассмотрим
	\[
		\beta
		\;=\;
		\sum_{k=1}^{\infty} 10^{-k!}
		\;=\;
		0.110001000000000000000001\dots
	\]
	(десятичная запись имеет единицы в позициях $1!,\,2!,\,3!,\dots$).
	Нетрудно проверить, что для любого $n>1$ существует рациональная дробь
	$\tfrac{p}{q}$ (с $q=10^{\,n!}$) приближающая $\beta$ с точностью
	\(\frac{1}{\,q^n\,}\). По Теореме Лиувилля, такое $\beta$ не алгебраично,
	значит \textbf{трансцендентно}.
\end{customexample}
