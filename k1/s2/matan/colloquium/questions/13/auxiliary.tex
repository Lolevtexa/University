% ==================
% auxiliary.tex
% (Вспомогательные определения для темы:
%  "Формула Стирлинга (с равенством).")
% ==================

\paragraph{Факториал $n!$.}
Для натурального числа $n$ вводится произведение
\[
	n!
	\;=\;
	1 \cdot 2 \cdot 3 \cdot \dots \cdot n.
\]
Известно, что при больших $n$ факториал растёт очень быстро.

\bigskip

\paragraph{Формула Стирлинга (классическое приближение).}
Ранее было рассмотрено \textit{эквивалентность}:
\[
	n!
	\;\sim\;
	\sqrt{2\pi\,n}\,\Bigl(\tfrac{n}{e}\Bigr)^n,
	\quad
	\text{как }n\to\infty.
\]
Однако можно записать и более точную форму с \emph{остаточным} (корректирующим) множителем, чтобы иметь «равенство» с некоторым уточнением:
\[
	n!
	\;=\;
	\sqrt{2\pi\,n}\,\Bigl(\tfrac{n}{e}\Bigr)^n
	\cdot \exp(\,\epsilon_n\,),
\]
где $\epsilon_n$ — небольшая поправка, про которую известны конкретные оценки.
