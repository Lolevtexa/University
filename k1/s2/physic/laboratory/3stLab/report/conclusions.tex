\section*{Выводы}

\begin{itemize}
	\item В ходе работы измерены углы дифракции для красной, зелёной и фиолетовой линий ртутной лампы в порядках $m=\pm1,\pm2,\pm3$. На их основе рассчитаны средние углы $\bar\alpha_{m}$, угловые смещения $\varphi_{m}=\bar\alpha_{m}-\bar\alpha_{0}$ и параметр $a_m=\sin\varphi_m/m$.

	\item По методу переноса погрешностей для зелёной линии ($\lambda_2=546\pm5\,$нм) определена постоянная решётки
	      \[
		      d = 447.8\,\mu\mathrm{m}.
	      \]

	\item Расчётными формулами найдены длины волн красной и фиолетовой линий:
	      \[
		      \lambda_\mathrm{красн} = 669\,\mathrm{нм},
		      \quad
		      \lambda_\mathrm{фиол} = 437\,\mathrm{нм},
	      \]
	      которые согласуются с известными табличными значениями в пределах эксперимента.

	\item Зная длину активной части решётки $L=1.5\,$см и $d$, получено число штрихов $N\approx34$; на его основе рассчитана разрешающая способность
	      \[
		      R = mN,\quad R_1\approx34,\ R_3\approx102.
	      \]

	\item Проведённая обработка показала, что метод дифракционной решётки позволяет с достаточной точностью определить длины волн и характеристики спектрального прибора без углублённого анализа систематических погрешностей.
\end{itemize}
