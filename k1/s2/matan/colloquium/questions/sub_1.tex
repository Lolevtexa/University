% ===== sub_1.tex =====
% Вспомогательные определения/теоремы, НЕ являющиеся центральными,
% но необходимые для доказательства (компакт, непрерывность, и т.д.)

\textbf{Компакт в $\mathbb{R}$.}
Множество $K\subset\mathbb{R}$ называется \emph{компактным}, если оно \emph{замкнуто} и \emph{ограничено}.

\medskip

\textbf{Определение непрерывной функции (точечное).}
Функция $f$ непрерывна в $x_0$, если
\[
\forall \varepsilon>0\;\exists \delta>0:\;
|x-x_0|<\delta \;\Longrightarrow\;
|f(x)-f(x_0)|<\varepsilon.
\]
Если это верно для каждой точки множества $X$, говорят, что $f$ непрерывна на $X$.

\medskip

\textbf{Теорема о сходящейся подпоследовательности (Больцано--Вейерштрасса).}
Всякая \emph{ограниченная} последовательность в $\mathbb{R}$ имеет \emph{сходящуюся} подпоследовательность.  
Если $(x_n)$ лежит в компактном $K$, то любая подпоследовательность $(x_{n_k})$ имеет сходящуюся подпоследовательность с пределом в $K$.  
