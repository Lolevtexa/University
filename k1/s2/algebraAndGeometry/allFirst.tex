
\documentclass[a4paper,12pt]{article}
\usepackage[utf8]{inputenc}
\usepackage[russian]{babel}
\usepackage{amsmath,amssymb}
\begin{document}

\section*{Разбиение задач на типы и решения}

% Тип 1: Верхнетреугольные матрицы
\textbf{Тип 1: Верхнетреугольные матрицы.} \\
\textbf{Аналогичные задачи:} 35, 40, 45, 48, 49, 51, 52\\
\textbf{Пример (вариант 35):} Является ли линейным пространством множество верхнетреугольных матриц размера \(4 \times 4\)?

\textbf{Решение:} \\
Сумма и произведение на число верхнетреугольных матриц также являются верхнетреугольными, так как элементы ниже главной диагонали остаются нулями после выполнения операций.

\textbf{Ответ:} Да, это линейное пространство.

% Тип 2: Симметричные матрицы
\bigskip
\textbf{Тип 2: Симметричные матрицы.}\\
\textbf{Аналогичные задачи:} 31, 34, 39, 60, 62\\
\textbf{Пример (вариант 31):} Является ли линейным пространством множество симметричных матриц размера \(4\times4\)?

\textbf{Решение:} \\
Пусть \( A=A^T, B=B^T \). Тогда
\[
(A+B)^T = A^T+B^T = A+B, \quad (\lambda A)^T = \lambda A^T = \lambda A,
\]
то есть сумма и произведение на скаляр также симметричны.

\textbf{Ответ:} Да, является линейным пространством.

% Тип 3: Матрицы с фиксированным следом
\bigskip
\textbf{Тип 3: Матрицы с фиксированным следом.}\\
\textbf{Аналогичные задачи:} 33, 36, 44, 47, 53, 54, 61\\
\textbf{Пример (вариант 33):} Является ли линейным пространством множество матриц \(5\times 5\) со следом \(1\)?

\textbf{Решение:}\\
Проверим замкнутость относительно сложения:
\[
\text{tr}(A+B) = \text{tr}(A)+\text{tr}(B) = 1+1=2\neq 1.
\]
Таким образом, множество не замкнуто.

\textbf{Ответ:} Нет, не является.

% Тип 4: Многочлены с фиксированным корнем
\bigskip
\textbf{Тип 4: Многочлены с фиксированным корнем.}\\
\textbf{Аналогичные задачи:} 38, 43, 50, 55, 57\\
\textbf{Пример (вариант 38):} Является ли линейным пространством множество многочленов, имеющих корень \(2\)?

\textbf{Решение:}\\
Пусть \(f(2)=0\), \(g(2)=0\). Тогда
\[
(f+g)(2)=f(2)+g(2)=0+0=0,\quad (\lambda f)(2)=\lambda f(2)=\lambda\cdot 0=0,
\]
значит множество замкнуто.

\textbf{Ответ:} Да, является.

% Тип 5: Экспоненциальные функции
\bigskip
\textbf{Тип 5: Экспоненциальные функции.}\\
\textbf{Аналогичные задачи:} 32, 42\\
\textbf{Пример (вариант 32):} Является ли линейным пространством множество функций вида \( e^{\lambda x} \)?

\textbf{Решение:}\\
Сумма двух функций:
\[
e^{\lambda x}+e^{\mu x},
\]
не представляется в виде одной функции \( e^{k x} \), значит множество не замкнуто.

\textbf{Ответ:} Нет, не является.

% Тип 6: Многочлены ограниченной степени
\bigskip
\textbf{Тип 6: Многочлены ограниченной степени.}\\
\textbf{Аналогичные задачи:} 41\\
\textbf{Пример (вариант 41):} Является ли линейным пространством множество многочленов степени не выше \(3\)?

\textbf{Решение:}\\
Сумма и умножение на число многочленов степени не выше 3 сохраняет степень многочлена, значит множество замкнуто.

\textbf{Ответ:} Да, является.

% Тип 7: Матрицы с нулевым определителем
\bigskip
\textbf{Тип 7: Матрицы с нулевым определителем.}\\
\textbf{Аналогичные задачи:} 46, 58, 59\\
\textbf{Пример (вариант 46):} Является ли линейным пространством множество матриц \(3\times3\) с определителем \(0\)?

\textbf{Решение:}\\
Сумма двух вырожденных матриц не всегда является вырожденной, например, если одна матрица противоположна другой. Значит множество не замкнуто.

\textbf{Ответ:} Нет, не является.

\end{document}
