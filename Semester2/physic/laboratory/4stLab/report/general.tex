\section{Общие сведения}

Тепловое излучение (ТИ) представляет собой явление генерации электромагнитных волн нагретым телом. Основу эффекта составляют процессы преобразования тепловой энергии макроскопической системы (нагретого тела) в энергию электромагнитного поля. Все остальные виды излучения, возбуждаемые за счёт видов энергии, отличных от тепловой, называют люминесценцией.

ТИ является изотропным, то есть вероятности испускания излучения разных длин волн или частот и поляризаций в разных направлениях равновероятны (одинаковы).

ТИ имеет сплошной спектр, т.е. его спектральные энергетические (характеристики) светимости \( r_{\omega,T} \) и \( r_{\lambda,T} \) (см. далее), зависящие от частот \( \omega \) или длин волн \( \lambda \) излучения, изменяются непрерывно, без скачков.

ТИ — это единственный вид излучения в природе, которое является равновесным, то есть находится в термодинамическом или тепловом равновесии с излучающим его телом. Тепловое равновесие означает, что излучающее тело и поле излучения имеют одинаковую температуру.

В качестве меры преобразования энергии используется мощность:
\[
P = \frac{dW}{dt},
\]
где \( dW \) — количество энергии, которое за время \( dt \) преобразуется из одного вида в другой.

В связи с тем, что излучение электромагнитных волн происходит с поверхности тела, а мощность теплового излучения \( P \) пропорциональна площади поверхности \( S \), в качестве характеристики используют интегральную энергетическую светимость тела:
\[
R_T = \frac{1}{S} \frac{dW}{dt} \quad \text{(Вт/м}^2\text{)} \tag{11.1}
\]

Правая часть равенства (11.1) задаёт суммарную плотность потока энергии электромагнитных волн всех частот, испускаемой поверхностью нагретого тела.

Для характеристики зависимости светимости нагретого тела от частоты вводятся спектральные энергетические светимости \( r_{\omega,T} \) и \( r_{\lambda,T} \):
\[
r_{\omega,T} = \frac{dR_T}{d\omega}, \quad
r_{\lambda,T} = \frac{dR_T}{d\lambda} = \frac{dR_T}{d\omega} \cdot \frac{d\omega}{d\lambda} = r_{\omega,T} \cdot \frac{2\pi c}{\lambda^2} \tag{11.2}
\]

где \( dR_T \) — суммарная плотность потока энергии, переносимой волнами, частоты которых находятся в узком интервале \( (\omega, \omega + \Delta\omega) \) или \( (\lambda, \lambda + \Delta\lambda) \).

Наряду с излучением может происходить и обратное преобразование энергии: энергия электромагнитного излучения поглощается веществом, т.е. трансформируется в тепловую энергию макроскопической системы. Мерой обратного преобразования энергии служит спектральная поглощательная способность \( \alpha_{\omega,T} \), определяемая как:
\[
\alpha(\omega, T) = \frac{d\Phi_{\text{погл}}(\omega, T)}{d\Phi_{\text{пад}}(\omega, T)} \tag{11.3}
\]

где \( d\Phi_{\text{погл}}(\omega, T) \) — поток энергии, который поглощается телом, \( d\Phi_{\text{пад}}(\omega, T) \) — величина падающего потока в интервале частот \( (\omega, \omega + d\omega) \).

Тело, которое полностью поглощает энергию электромагнитных волн (\( \alpha_{\omega,T} = 1 \)), называют абсолютно чёрным телом (АЧТ). Если поглощательная способность в некоторой области частот меньше единицы и не зависит от частоты, то в этой области спектра тело считается серым.

Излучение и поглощение веществом электромагнитных волн представляют собой формы проявления способности частиц вещества (атомов, молекул) к взаимодействию с электромагнитным полем. Оба эффекта существуют неразрывно (\( \alpha_{\omega,T} < 1 \)). Это утверждение составляет основу закона Кирхгофа: для любого тела отношение спектральной энергетической светимости \( r_{\omega,T} \) к его поглощательной способности \( \alpha_{\omega,T} \) есть величина постоянная, равная спектральной энергетической светимости АЧТ \( r^{(0)}_{\omega,T} \), для которого \( \alpha_{\omega,T} = 1 \):
\[
\frac{r_{\omega,T}}{\alpha_{\omega,T}} = r^{(0)}_{\omega,T} \tag{11.4}
\]

Вином на основе законов термодинамики была доказана следующая теорема: спектральная энергетическая светимость АЧТ пропорциональна частоте \( \omega \) излучения и обратно пропорциональна его температуре \( T \), совпадающей с температурой излучающего тела:
\[
r^{(0)}_{\omega,T} = f\left(\frac{\omega}{T}\right)
\]

До построения Планком теории теплового излучения были также экспериментально открыты следующие законы теплового излучения:
\[
R^{(0)}_T = \int_0^\infty r^{(0)}_{\omega,T} d\omega = \sigma T^4, \quad \lambda_m = \frac{b}{T}, \quad r^{(0)}_{\lambda_m,T} = cT^5 \tag{11.5}
\]

Выражение для интегральной энергетической светимости АЧТ — это закон Стефана–Больцмана. Постоянная Стефана–Больцмана \( \sigma = 5{,}67 \cdot 10^{-8}~\text{Вт}/\text{м}^2\text{К}^4 \). Выражение для длины волны \( \lambda_m \), соответствующей максимуму спектральной энергетической светимости АЧТ, называется первым законом Вина (законом смещения Вина), а выражение для \( r^{(0)}_{\lambda_m,T} \) — вторым законом Вина. Постоянные:
\[
b = 2{,}91 \cdot 10^{-3}~\text{м}\cdot\text{К}, \quad c = 1{,}31 \cdot 10^5~\text{Вт}/(\text{м}^3\cdot\text{К}^5)
\]

В рамках своей теории теплового излучения Планк получил следующее выражение для спектральной энергетической светимости АЧТ, которое при использовании теоремы Вина принимает вид:
\[
r^{(0)}_{\omega,T} = \frac{\hbar \omega^3}{4\pi^3 c^2} \cdot \frac{1}{\exp(\hbar\omega/kT) - 1}, \quad
r^{(0)}_{\lambda,T} = \frac{2\pi hc^2}{\lambda^5} \cdot \frac{1}{\exp(hc/\lambda kT) - 1} \tag{11.6}
\]

В функции Планка используются:
\[
c = 3 \cdot 10^8~\text{м/с} \quad
h = 6{,}63 \cdot 10^{-34}~\text{Дж}\cdot\text{с}, \quad
\hbar = \frac{h}{2\pi} = 1{,}05 \cdot 10^{-34}~\text{Дж}\cdot\text{с}, \quad
k = 1{,}38 \cdot 10^{-23}~\text{Дж}/\text{К}
\]

\[
a = \frac{8\pi^5 k^4}{15c^2h^3} = 3{,}74 \cdot 10^{-16}~\text{Вт}/\text{м}^2\text{К}^4, \quad
a_2 = \frac{hc}{k} = 1{,}44 \cdot 10^{-2}~\text{м}\cdot\text{К}
\]
