% ===== sub_7.tex =====
% Вспомогательный файл: определения/факты, не являющиеся центральной частью вопроса,
% но нужные при доказательствах (например, "n-кратная дифференцируемость", "o( (x-x0)^n )" и т. п.)

\textbf{Определение n-кратной дифференцируемости.}
Функция $f$ называется $n$ раз дифференцируемой в точке $x_0$, если существуют конечные все производные $f'(x_0), f''(x_0), \dots, f^{(n)}(x_0)$.

\medskip

\textbf{Малое «о» и запись $o(g(x))$.}
Говорят, что $h(x)$ есть $o\bigl(g(x)\bigr)$ при $x\to x_0$, если
\[
\lim_{x\to x_0}\frac{h(x)}{g(x)} = 0.
\]
В таком случае пишут $h(x)=o(g(x))$.

\medskip

\textbf{Остаточный член в форме Лагранжа (напоминание).}
Если $f$ имеет $(n+1)$-ю производную в окрестности $x_0$, то для
\[
f(x) = f(x_0) + \dots + \frac{f^{(n)}(x_0)}{n!}(x - x_0)^n + R_n(x)
\]
справедливо
\[
R_n(x) = \frac{f^{(n+1)}(\xi)}{(n+1)!}(x - x_0)^{n+1},\quad \xi \in (x_0,x).
\]

