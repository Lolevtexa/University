% ===== sub_3.tex =====
% Вспомогательный файл: содержит факты/определения, не являющиеся
% центральной частью вопроса, но нужные в доказательствах.

\textbf{Теорема Ролля (напоминание).}
Пусть $f$ непрерывна на $[a,b]$, дифференцируема на $(a,b)$ и $f(a)=f(b)$. Тогда существует точка $c\in(a,b)$, где $f'(c)=0$.

\textbf{Определение дифференцируемости (напоминание).}
Функция $f$ называется дифференцируемой в точке $x_0$, если существует конечный предел
\[
f'(x_0) = \lim_{x\to x_0} \frac{f(x)-f(x_0)}{x - x_0}.
\]
Если $f'(x_0)$ существует для всех $x_0$ в $(a,b)$, говорят, что $f$ дифференцируема на $(a,b)$.

\textbf{Определение монотонности (напоминание).}
Функция $f$ называется \emph{возрастающей} на $(a,b)$, если для любых $x_1,x_2\in(a,b)$, при $x_1 < x_2$ выполняется $f(x_1)\le f(x_2)$ (нестрого) или $f(x_1)<f(x_2)$ (строго).  
Аналогично определяется убывание.

\textbf{Определение постоянной функции.}
Функция $f$ называется постоянной на $(a,b)$, если $f(x_1)=f(x_2)$ для любых $x_1,x_2\in(a,b)$.

% (Другие напоминания (о непрерывности, ...) при необходимости).
