% ===== sub_6.tex =====
% Вспомогательные определения/теоремы, не являющиеся
% "центральными" в данном вопросе, но используемые в доказательствах.

\textbf{Теорема Ролля (напоминание).}
Пусть $f$ непрерывна на отрезке $[p,q]$, дифференцируема на $(p,q)$ и $f(p)=f(q)$. Тогда существует точка $c\in(p,q)$, где $f'(c)=0$.

\medskip

\textbf{Определение: проколотая окрестность.}
Говорят, что функция $f$ дифференцируема (или определена) в проколотой окрестности точки $a$, если существует $\delta>0$ такое, что при $0<|x-a|<\delta$, $f(x)$ (и её производная) определена. При этом в самой точке $a$ она может быть не определена или не дифференцируема.

\medskip

\textbf{Неопределённости вида $0/0$ и $\infty/\infty$.}
Правило Лопиталя распространяется на случаи, когда \(\lim_{x\to a}f(x)=\lim_{x\to a}g(x)=0\) или \(\pm\infty\). В остальных случаях правило не даёт результата.

