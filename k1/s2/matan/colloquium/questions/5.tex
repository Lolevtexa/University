% ===== 5.tex =====
% "Вывод рядов Тейлора для y=e^x, y=sin x, y=cos x через следствие из теоремы Лагранжа. Формула Эйлера."

% 1. Определения

\textbf{Ряд Тейлора (Маклорена) и остаточный член.}
Пусть функция $f$ имеет все производные в некоторой окрестности точки $0$. Тогда \emph{ряд Маклорена} для $f$ есть
\[
f(0) \;+\; \frac{f'(0)}{1!}\,x \;+\; \frac{f''(0)}{2!}\,x^2 \;+\; \dots \;+\; \frac{f^{(n)}(0)}{n!}\,x^n + \dots
\]
\emph{Остаточный член} $R_n(x)$ (в форме Лагранжа) можно записать как
\[
R_n(x) \;=\; \frac{f^{(n+1)}(\xi)}{(n+1)!}\,x^{n+1},
\]
где $\xi$ лежит между $0$ и $x$ (см. \texttt{sub\_5.tex}, Теорема о среднем значении в форме Лагранжа).

\medskip

% 2. Теоремы и ключевые утверждения

\textbf{Формула Тейлора (с остатком в форме Лагранжа).}
Пусть $f$ имеет $(n+1)$-ю производную в окрестности $0$. Тогда для $x$ из этой окрестности выполняется:
\[
f(x) \;=\;
f(0) + \frac{f'(0)}{1!}\,x + \frac{f''(0)}{2!}\,x^2 + \dots + \frac{f^{(n)}(0)}{n!}\,x^n + \underbrace{\frac{f^{(n+1)}(\xi)}{(n+1)!}\,x^{n+1}}_{R_{n}(x)},
\]
где $\xi$ — некоторая точка между $0$ и $x$. Эта запись называется \emph{рядом Тейлора (Маклорена)} с остаточным членом в \emph{форме Лагранжа}.

\medskip

% 3. Основные идеи доказательства (коротко)

\textbf{Идея вывода:}
\begin{itemize}
  \item Используем \emph{теорему Лагранжа} о среднем значении производной. Рассматриваем многочлен $P_n(x)$, равный сумме до $n$-го члена ряда, и функцию $f(x)-P_n(x)$.  
  \item Применяем теорему Лагранжа к $f(x)-P_n(x)$ на отрезке $[0,x]$, получаем «остаток» $R_n(x)$ в виде $f^{(n+1)}(\xi)\, x^{n+1}/(n+1)!$.  
  \item Далее, зная все производные $f$, вычисляем $f^{(k)}(0)$ и подставляем в общую формулу.
\end{itemize}

\medskip

% 4. Полное доказательство (ссылаясь на теорему Лагранжа из sub_5)

\subsection*{Вывод ряда Тейлора для $e^x$}

\begin{enumerate}
  \item \textbf{Производные в окрестности $0$.}  
  Пусть $f(x)=e^x$. Тогда $f^{(k)}(x)=e^x$ для любого $k$. Следовательно, $f^{(k)}(0)=1$.
  \item \textbf{Запись общего члена.}  
  По Формуле Тейлора (Маклорена),
  \[
    e^x \;=\; \sum_{n=0}^{\infty}\frac{f^{(n)}(0)}{n!}\,x^n + R_n(x).
  \]
  Но $f^{(n)}(0)=1$, значит
  \[
    e^x = \sum_{n=0}^{\infty}\frac{x^n}{n!} + R_n(x).
  \]
  \item \textbf{Вид остатка в форме Лагранжа.}  
  \[
    R_n(x) \;=\; \frac{f^{(n+1)}(\xi)}{(n+1)!}\,x^{n+1} \;=\;\frac{e^\xi}{(n+1)!}\,x^{n+1},\quad \xi \in (0,x).
  \]
  \item \textbf{При $n\to \infty$} и фиксированном $x$, $e^\xi$ остаётся конечным. Получаем
  \(\lim_{n\to\infty} R_n(x)=0\), значит
  \[
    e^x \;=\; 1 + x + \frac{x^2}{2!} + \frac{x^3}{3!} + \dots .
  \]
\end{enumerate}

\medskip

\subsection*{Вывод ряда Тейлора для $\sin x$ и $\cos x$}

\begin{enumerate}
  \item \textbf{Пусть $f(x)=\sin x$.}  
  Производные: $f'(x)=\cos x$, $f''(x)=-\sin x$, $f^{(3)}(x)=-\cos x$, $f^{(4)}(x)=\sin x$ и т.д.  
  На $x=0$: $f(0)=0$, $f'(0)=1$, $f''(0)=0$, $f^{(3)}(0)=-1$, $f^{(4)}(0)=0$, и цикл повторяется.  
  \[
    \sin x = \sum_{n=0}^{\infty}\frac{f^{(n)}(0)}{n!}\,x^n + R_n(x).
  \]
  Подставляя значения, получаем чётные производные $=0$ в $0$, нечётные чередуются $1,-1,\dots$.
  \[
    \sin x = x - \frac{x^3}{3!} + \frac{x^5}{5!} - \dots + R_n(x).
  \]
  Остаток $R_n(x)$ (по Лагранжу) будет
  \(\frac{f^{(n+1)}(\xi)}{(n+1)!}\,x^{n+1}\) для некоторой \(\xi\in(0,x)\).

  \item \textbf{Аналогично для $g(x)=\cos x$.}  
  $g(0)=1$, $g'(0)=0$, $g''(0)=-1$, $g^{(3)}(0)=0$, $g^{(4)}(0)=1,\dots$  
  \[
    \cos x = 1 - \frac{x^2}{2!} + \frac{x^4}{4!} - \dots + R_n(x).
  \]
\end{enumerate}

\medskip

% 5. Формула Эйлера

\textbf{Формула Эйлера:}  
\[
e^{i x} = \cos x + i\,\sin x.
\]
\textbf{Краткое объяснение.}  
Если подставить в разложение $e^x$ вместо $x$ — комплексное $i x$, то
\[
e^{ix} \;=\; 1 + (ix) + \frac{(ix)^2}{2!} + \frac{(ix)^3}{3!} + \dots .
\]
Распределяя степени $i^n$ на действительную и мнимую часть, получаем чётные степени $i^{2k}=(-1)^k$ и нечётные $i^{2k+1}=i\,(-1)^k$. В итоге исходные суммы группируются exactly как в рядах для $\cos x$ и $\sin x$.

\[
e^{ix} = \left(1 - \frac{x^2}{2!} + \frac{x^4}{4!} - \dots\right)
\;+\; i\left(x - \frac{x^3}{3!} + \frac{x^5}{5!} - \dots\right)
= \cos x + i\,\sin x.
\]

\medskip

% -- Логическая связность --
% (Все базовые сведения о теореме Лагранжа, определение (n+1)-й производной, 
% непрерывности и дифференцируемости см. в sub_5.tex).

