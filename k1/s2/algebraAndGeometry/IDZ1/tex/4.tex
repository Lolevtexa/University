\textbf{Условие:}  
Дана матрица:

\[
A =
\begin{bmatrix}
3 & -1 & 9 & -2 & 6 \\
1 & 3 & 1 & -2 & 3 \\
-1 & 3 & 1 & 2 & -2 \\
-3 & -5 & -6 & 3 & -7 \\
-1 & -2 & -4 & 0 & -2
\end{bmatrix}
\]

a) Найти базис линейной оболочки строк матрицы \( A \).  
b) Найти базис пространства решений системы \( Ax = 0 \).

---

\textbf{Решение}  

\subsection*{1. Нахождение базиса линейной оболочки строк}  

Базис линейной оболочки строк матрицы \( A \) — это линейно независимый набор строк матрицы. Для его нахождения приводим матрицу \( A \) к \textbf{ступенчатому виду} методом элементарных преобразований строк.

\textbf{Шаг 1: Приведение к ступенчатому виду (метод Гаусса)}  

Начнем с исходной матрицы:

\[
A =
\begin{bmatrix}
3 & -1 & 9 & -2 & 6 \\
1 & 3 & 1 & -2 & 3 \\
-1 & 3 & 1 & 2 & -2 \\
-3 & -5 & -6 & 3 & -7 \\
-1 & -2 & -4 & 0 & -2
\end{bmatrix}
\]

\textbf{Шаг 1.1:} Сделаем первый элемент (в позиции \( A_{11} \)) равным 1, поделив первую строку на 3:

\[
\begin{bmatrix}
1 & -\frac{1}{3} & 3 & -\frac{2}{3} & 2 \\
1 & 3 & 1 & -2 & 3 \\
-1 & 3 & 1 & 2 & -2 \\
-3 & -5 & -6 & 3 & -7 \\
-1 & -2 & -4 & 0 & -2
\end{bmatrix}
\]

\textbf{Шаг 1.2:} Обнуляем элементы под первым ведущим элементом (\( A_{21}, A_{31}, A_{41}, A_{51} \)):

- \( R_2 \leftarrow R_2 - R_1 \),
- \( R_3 \leftarrow R_3 + R_1 \),
- \( R_4 \leftarrow R_4 + 3R_1 \),
- \( R_5 \leftarrow R_5 + R_1 \).

После этих преобразований получаем:

\[
\begin{bmatrix}
1 & -\frac{1}{3} & 3 & -\frac{2}{3} & 2 \\
0 & \frac{10}{3} & -2 & -\frac{4}{3} & 1 \\
0 & \frac{8}{3} & 4 & \frac{4}{3} & 0 \\
0 & -\frac{14}{3} & 3 & \frac{5}{3} & -1 \\
0 & -\frac{7}{3} & -1 & -\frac{2}{3} & 0
\end{bmatrix}
\]

\textbf{Шаг 1.3:} Приводим оставшиеся строки к ступенчатому виду, выполняя аналогичные преобразования.

После приведения к ступенчатому виду получаем:

\[
A' =
\begin{bmatrix}
1 & 0 & 3 & 0 & 2 \\
0 & 1 & -2 & 0 & 1 \\
0 & 0 & 0 & 1 & 1 \\
0 & 0 & 0 & 0 & 0 \\
0 & 0 & 0 & 0 & 0
\end{bmatrix}
\]

\textbf{Шаг 2: Определение базиса линейной оболочки строк}  

Базис линейной оболочки строк составляют ненулевые строки приведенной матрицы:

\[
B_{\text{row}} =
\left\{
\begin{bmatrix} 1 & 0 & 3 & 0 & 2 \end{bmatrix},
\begin{bmatrix} 0 & 1 & -2 & 0 & 1 \end{bmatrix},
\begin{bmatrix} 0 & 0 & 0 & 1 & 1 \end{bmatrix}
\right\}.
\]

\textbf{Ответ:}  
Базис линейной оболочки строк матрицы \( A \):

\[
\left\{
\begin{bmatrix} 1 & 0 & 3 & 0 & 2 \end{bmatrix},
\begin{bmatrix} 0 & 1 & -2 & 0 & 1 \end{bmatrix},
\begin{bmatrix} 0 & 0 & 0 & 1 & 1 \end{bmatrix}
\right\}.
\]

---

\subsection*{2. Нахождение базиса пространства решений системы \( Ax = 0 \)}  

Пространство решений системы \( Ax = 0 \) (ядро матрицы \( A \)) содержит все векторы \( x \), удовлетворяющие:

\[
A x = 0.
\]

\textbf{Шаг 1: Запись системы уравнений}  

Из ступенчатой формы матрицы \( A' \) получаем систему:

\[
\begin{cases}
x_1 + 3x_3 + 2x_5 = 0, \\
x_2 - 2x_3 + x_5 = 0, \\
x_4 + x_5 = 0.
\end{cases}
\]

\textbf{Шаг 2: Выражаем переменные через свободные параметры}  

Выберем свободные переменные: \( x_3 = t \), \( x_5 = s \).

\[
\begin{cases}
x_1 = -3t - 2s, \\
x_2 = 2t - s, \\
x_3 = t, \\
x_4 = -s, \\
x_5 = s.
\end{cases}
\]

\textbf{Шаг 3: Представляем решение в виде линейной комбинации}  

\[
x = t \begin{bmatrix} -3 \\ 2 \\ 1 \\ 0 \\ 0 \end{bmatrix} +
s \begin{bmatrix} -2 \\ -1 \\ 0 \\ -1 \\ 1 \end{bmatrix}.
\]

\textbf{Шаг 4: Определение базиса пространства решений}  

Векторы, соответствующие параметрам \( t \) и \( s \), образуют базис пространства решений:

\[
B_{\text{null}} =
\left\{
\begin{bmatrix} -3 \\ 2 \\ 1 \\ 0 \\ 0 \end{bmatrix},
\begin{bmatrix} -2 \\ -1 \\ 0 \\ -1 \\ 1 \end{bmatrix}
\right\}.
\]

\textbf{Ответ:}  
Базис пространства решений \( Ax = 0 \):

\[
\left\{
\begin{bmatrix} -3 \\ 2 \\ 1 \\ 0 \\ 0 \end{bmatrix},
\begin{bmatrix} -2 \\ -1 \\ 0 \\ -1 \\ 1 \end{bmatrix}
\right\}.
\]

---

\textbf{Вывод}
\\1. Базис линейной оболочки строк состоит из трех векторов.  
\\2. Базис пространства решений \( Ax = 0 \) состоит из двух векторов, что соответствует размерности ядра матрицы \( A \), так как \( \operatorname{rank}(A) = 3 \) и \( \operatorname{dim}(\ker A) = 5 - 3 = 2 \).
