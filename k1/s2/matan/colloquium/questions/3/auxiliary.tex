% =========================
% auxiliary.tex
% (Вспомогательная информация)
% =========================

\paragraph{Непрерывность на отрезке.}
Функция $f$ называется \emph{непрерывной} на отрезке $[a,b]$,
если для любой точки $x_0 \in [a,b]$ и любого $\varepsilon>0$ существует
$\delta>0$ такое, что для всех $x \in [a,b]$ с $|x - x_0| < \delta$ выполняется
$|f(x) - f(x_0)| < \varepsilon$.

\paragraph{Дифференцируемость.}
Функция $f$ называется \emph{дифференцируемой} на интервале $(a,b)$,
если в каждой точке $x_0 \in (a,b)$ существует конечный предел
\[
	f'(x_0) \;=\; \lim_{x \to x_0}\,\frac{f(x) - f(x_0)}{\,x - x_0\,}.
\]

\paragraph{Монотонность (возрастание, убывание).}
Говорят, что функция $f$ \emph{возрастает} на промежутке $(a,b)$,
если для любых $x_1, x_2 \in (a,b)$ при $x_1 < x_2$ выполняется
$f(x_1) \le f(x_2)$ (или $<$ для строго возрастающей).
Аналогично, $f$ \emph{убывает}, если $x_1 < x_2 \implies f(x_1) \ge f(x_2)$.

\paragraph{Теорема Ролля (напоминание).}
Пусть $f$ \emph{непрерывна} на $[a,b]$, \emph{дифференцируема} на $(a,b)$ и
$f(a)=f(b)$. Тогда существует точка $c\in(a,b)$, где $f'(c)=0$.

\bigskip

% Другие основные определения (локальный экстремум, ...),
% если нужно, можно добавить сюда.

