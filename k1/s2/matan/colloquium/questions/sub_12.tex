% ===== sub_12.tex =====
% Вспомогательные определения/теоремы, не являющиеся
% "центральными" в вопросе 12, но используемые в доказательстве
% (Формула Стирлинга (с эквивалентностью)).

\textbf{Интегральная аппроксимация суммы.}
Ключ к доказательству формулы Стирлинга:
\[
\sum_{k=1}^{n}\ln k
\quad \text{сравнивается с} \quad
\int_{1}^{n}\ln x \,dx.
\]
Разность между этой суммой и интегралом — «погрешность», часто контролируемая приёмами типа «прямоугольников» или \emph{упрощённой} формулы Эйлера–Маклорена.

\medskip

\textbf{Обозначения: } $O(\cdot), \; o(\cdot).$
Символ $O(g(n))$ означает, что рассматриваемая величина не превосходит по абсолютному значению $C\,|g(n)|$ при достаточно больших $n$.  
Символ $r(n)=o(g(n))$ значит \(\lim_{n\to\infty}\frac{r(n)}{g(n)}=0\).

\medskip

\textbf{Эквивалентность функций.}
Запись $f(n)\sim g(n)$ означает \(\lim_{n\to\infty}\frac{f(n)}{g(n)}=1\). 
В контексте формулы Стирлинга: \(n!\sim \sqrt{2\pi n}\,\bigl(\tfrac{n}{e}\bigr)^n\). 
