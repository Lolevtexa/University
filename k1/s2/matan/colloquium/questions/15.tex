% ===== 15.tex =====
% "Формула Ньютона–Лейбница."

% 1. Определения

\textbf{Определённый интеграл.}
Функция \(f\) называется интегрируемой по Риману на \([a,b]\), если предел сумм вида
\[
S = \sum_{i=1}^n f(\xi_i)\,\Delta x_i,
\]
при разбиении \([a,b]\) на всё более мелкие отрезки (с максимальной длиной \(\Delta x_i \to 0\)), существует и не зависит от выбора \(\xi_i\). Этот предел есть \(\int_a^b f(x)\,dx\).

\medskip

\textbf{Первообразная (примитив).}
Функция \(F\) называется \emph{первообразной} для \(f\) на \([a,b]\), если \(F'(x)=f(x)\) для всех \(x\in(a,b)\). Если \(F\) дифференцируема на \((a,b)\) и непрерывна на \([a,b]\), то это достаточно для теоремы Ньютона–Лейбница.

\medskip

% 2. Теоремы и ключевые утверждения

\textbf{Формула Ньютона–Лейбница.}
Пусть \(f\) непрерывна на \([a,b]\) и \(F\) — первообразная \(f\) на \([a,b]\). Тогда
\[
\int_a^b f(x)\,dx = F(b) - F(a).
\]

\medskip

% 3. Основные идеи доказательства

\begin{itemize}
  \item Рассматривается \(\int_a^b f(x)\,dx\) и разбиение отрезка \([a,b]\).  
  \item По определению первообразной \(F'(x)=f(x)\) на \((a,b)\).  
  \item Считаем интегральные суммы \(S = \sum f(\xi_i)\,\Delta x_i\) и замечаем связь с приростами \(F(x_i) - F(x_{i-1})\).  
  \item Подсчитываем \(\sum [F(x_i)-F(x_{i-1})] = F(b)-F(a)\).  
  \item Показываем, что эта сумма совпадает с \(\int_a^b f(x)\,dx\) в пределе.
\end{itemize}

\medskip

% 4. Полное доказательство (шаги, ссылаясь на sub_15, если нужно)

\begin{enumerate}
  \item \textbf{Разбиение отрезка \([a,b]\).}
    Пусть \(a=x_0 < x_1 < \dots < x_n=b\) — любое разбиение, \(\Delta x_i=x_i-x_{i-1}\).
  \item \textbf{Интегральная сумма.}
    Рассмотрим \(S = \sum_{i=1}^n f(\xi_i)\Delta x_i\), \(\xi_i\in[x_{i-1},x_i]\).
  \item \textbf{Прирост первообразной.}
    Так как \(F'(x)=f(x)\), то по теореме о среднем значении существует \(\eta_i\in[x_{i-1},x_i]\) с
    \[
      F(x_i)-F(x_{i-1}) = F'(\eta_i)\,\Delta x_i = f(\eta_i)\,\Delta x_i.
    \]
  \item \textbf{Сравнение с интегральной суммой.}
    Если выбрать \(\xi_i=\eta_i\), видим, что \(\sum [F(x_i)-F(x_{i-1})] = \sum f(\eta_i)\Delta x_i = S\). Но слева — телескопическая сумма:
    \[
      \sum_{i=1}^n [\,F(x_i)-F(x_{i-1})\,]
      = F(x_n)-F(x_0) = F(b)-F(a).
    \]
  \item \textbf{Вывод.}
    Переходя к пределу при \(\|D\|\to 0\), интегральная сумма \(\sum f(\xi_i)\,\Delta x_i\) стремится к \(\int_a^b f(x)\,dx\), а мы установили её равенство \(F(b)-F(a)\). Следовательно,
    \[
      \int_a^b f(x)\,dx = F(b) - F(a).
    \]
\end{enumerate}

\medskip

% 5. Пример

\textbf{Функция \(f(x)=x^2\).}
Одна из первообразных: \(F(x)=\tfrac{x^3}{3}\). По формуле Ньютона–Лейбница:
\[
\int_0^2 x^2\,dx
= \Bigl[\tfrac{x^3}{3}\Bigr]_0^2
= \tfrac{2^3}{3} - 0 = \tfrac{8}{3}.
\]

\end{document}
