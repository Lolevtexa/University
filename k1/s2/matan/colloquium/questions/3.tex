% ===== 3.tex =====
% "Теорема Лагранжа. 
%  Необходимое и достаточное условие постоянства дифференцируемой функции.
%  Необходимое и достаточное условие монотонности дифференцируемой функции."

% 1. Определения (те, что непосредственно связаны с вопросом)

\textbf{Теорема Лагранжа (о среднем значении).}
Пусть функция $f$ \emph{непрерывна} на отрезке $[a,b]$ и \emph{дифференцируема} на интервале $(a,b)$. Тогда существует точка $c \in (a,b)$ такая, что
\[
f(b) - f(a) \;=\; f'(c)\,\bigl(b - a\bigr).
\]
(\emph{Остальные определения см. в \texttt{sub\_3.tex} — например, производная, непрерывность, и т.п.})

\medskip

\textbf{Необходимое и достаточное условие \underline{постоянства} дифференцируемой функции.}
Пусть функция $f$ дифференцируема на промежутке $(a,b)$. Тогда она \emph{постоянна} на $(a,b)$ \textbf{тогда и только тогда}, когда
\[
f'(x) \;=\; 0 \quad\text{для всех}\; x \in (a,b).
\]

\medskip

\textbf{Необходимое и достаточное условие \underline{монотонности} дифференцируемой функции.}
Пусть $f$ дифференцируема на промежутке $(a,b)$. Тогда верны следующие утверждения:

\begin{itemize}
  \item Функция $f$ \textbf{возрастает} на $(a,b)$ \(\Longleftrightarrow\) \(f'(x)\ge 0\) для всех \(x\in(a,b)\), причём множество точек, где \(f'(x)=0\), не содержит интервалов.
  \item Функция $f$ \textbf{убывает} на $(a,b)$ \(\Longleftrightarrow\) \(f'(x)\le 0\) для всех \(x\in(a,b)\), причём множество точек, где \(f'(x)=0\), не содержит интервалов.
\end{itemize}

\medskip

% 2. Основные идеи доказательств (коротко)

\textbf{Основная идея доказательства теоремы Лагранжа:}
\begin{itemize}
  \item Эта теорема является обобщением Теоремы Ролля.
  \item Сущность: Рассмотрим функцию $F(x)= f(x) - \alpha x$, где \(\alpha = \frac{f(b)-f(a)}{b-a}\).  
  \item Из условия $F(a)=F(b)$ и непрерывности+дифференцируемости $F$ на $[a,b]$ применяем Теорему Ролля: существует $c \in(a,b)$, где $F'(c)=0$.  
  \item Тогда $F'(c)=f'(c)-\alpha=0\implies f'(c)=\alpha = \frac{f(b)-f(a)}{b-a}$.  
\end{itemize}

\textbf{Идея доказательства условия постоянства:}
\begin{itemize}
  \item Если $f'(x)=0$ повсюду, то по Теореме Лагранжа (или Ролля) разность $f(x_2)-f(x_1)$ всегда равна нулю, значит $f$ постоянна.
  \item Если $f$ постоянна, ясно, что $f'(x)=0$.  
\end{itemize}

\textbf{Идея доказательства условия монотонности:}
\begin{itemize}
  \item Если $f'(x)\ge0$ на $(a,b)$, то при $x_2>x_1$ можно показать $f(x_2)\ge f(x_1)$.  
  \item Обратное: если $f$ возрастает, то для $x_2>x_1$ имеем $\frac{f(x_2)-f(x_1)}{x_2-x_1}\ge0$. Переходя к пределу, получаем $f'(x)\ge 0$.  
  \item Уточнение, что множество точек с $f'(x)=0$ не содержит внутренних отрезков, нужно, чтобы исключить «застревание» функции на целых интервалах.  
\end{itemize}

\medskip

% 3. Полное доказательство (шаг за шагом)

\subsection*{Доказательство Теоремы Лагранжа}

\begin{enumerate}
  \item \textbf{Построение вспомогательной функции.}  
  Пусть $\alpha=\frac{f(b)-f(a)}{b-a}$. Рассмотрим
  \[
    F(x) \;=\; f(x) - \alpha\,x.
  \]
  Тогда $F(a)=f(a)-\alpha a$ и $F(b)=f(b)-\alpha b$. Заметим:
  \[
    F(b)-F(a) \;=\; [f(b)-f(a)] \;-\;\alpha[b - a] \;=\; 0.
  \]
  \item \textbf{Применение Теоремы Ролля.}  
  Функция $F$ непрерывна на $[a,b]$ и дифференцируема на $(a,b)$ (как разность непрерывных и дифференцируемых функций). Причём $F(a)=F(b)$.  
  По Теореме Ролля (см. \texttt{sub\_3.tex} при необходимости) существует $c \in(a,b)$: $F'(c)=0$.  
  \item \textbf{Заключение.}  
  Из $F'(x)=f'(x) - \alpha$ получаем $F'(c)=f'(c)-\alpha=0 \implies f'(c)=\alpha$. Но $\alpha= \tfrac{f(b)-f(a)}{b-a}$, значит
  \[
    f'(c) \;=\; \frac{f(b)-f(a)}{b-a}.
  \]
  На этом доказательство завершается.
\end{enumerate}

\subsection*{Доказательство необходимого и достаточного условия постоянства}

\begin{enumerate}
  \item \textbf{Необходимость (если $f$ постоянна).}  
  Если $f$ есть константа, то для всех $x\in(a,b)$ приращения $f(x_2)-f(x_1)=0$, значит $f'(x)=0$ в любой точке, где она дифференцируема.
  \item \textbf{Достаточность (если $f'(x)=0$ повсюду).}  
  Пусть $f'(x)=0$ на $(a,b)$. Возьмём любые $x_1,x_2\in(a,b)$, причём $x_2>x_1$. По Теореме Лагранжа существует $c\in(x_1,x_2)$ с
  \[
    f(x_2)-f(x_1) \;=\; f'(c)\,(x_2-x_1).
  \]
  Но $f'(c)=0$, значит $f(x_2)=f(x_1)$. Следовательно, $f$ неизменна на всём промежутке.
\end{enumerate}

\subsection*{Доказательство необходимого и достаточного условия монотонности}

\emph{(Рассмотрим случай возрастания; для убывания аналогично меняются знаки.)}

\begin{enumerate}
  \item \textbf{Если $f'(x)\ge 0$ для всех $x$, то $f$ возрастает.}  
  Пусть $x_1<x_2$. По Теореме Лагранжа на отрезке $[x_1,x_2]$ существует $c\in(x_1,x_2)$ такое, что
  \[
    f(x_2)-f(x_1) \;=\; f'(c)\,(x_2 - x_1).
  \]
  Раз $f'(c)\ge 0$ и $x_2-x_1>0$, то $f(x_2)-f(x_1)\ge 0\implies f(x_2)\ge f(x_1)$. Значит $f$ неубывает. Для \emph{строгого} возрастания нужно уточнить, что нет интервалов, где $f'(x)=0$ постоянно.
  \item \textbf{Если $f$ возрастает, то $f'(x)\ge0$.}  
  При $x_2>x_1$ имеем \(\tfrac{f(x_2)-f(x_1)}{x_2-x_1}\ge 0\). Переходя к пределу, получаем $f'(x)\ge0$.
  \item \textbf{Уточнение про множество нулей $f'(x)$.}  
  Если на каком-то подинтервале $f'(x)$ всё время равно нулю, то $f$ там постоянна, что может «ломать» строгое возрастание (если отрезок ненулевой длины). Поэтому для строгой монотонности требуется, чтобы подмножество нулей не содержало интервалов.
\end{enumerate}

\medskip

% 4. Логическая связность и ссылки
% (Все базовые теоремы, такие как теорема Ролля, определение непрерывности
%  см. во вспомогательном sub_3.tex)

