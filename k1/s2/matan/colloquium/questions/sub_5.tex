% ===== sub_5.tex =====
% Вспомогательный файл: факты, не являющиеся центральными в данном вопросе,
% но нужные при доказательствах (теорема Лагранжа, определение (n+1)-й производной и т.п.)

\textbf{Теорема Лагранжа (о среднем значении).}
Пусть функция $f$ непрерывна на $[0,x]$ (при $x>0$) и дифференцируема на $(0,x)$. Тогда существует $c\in(0,x)$ такое, что
\[
f(x)-f(0) \;=\; f'(c)\,\bigl(x - 0\bigr).
\]

\medskip

\textbf{n-я производная.}
Функция $f$ называется $n$ раз дифференцируемой в точке, если существуют все производные $f'(x_0), f''(x_0),\dots,f^{(n)}(x_0)$. Аналогично в окрестности.

\medskip

\textbf{Определение комплексной экспоненты (для Формулы Эйлера).}
Если $z$ комплексное, $e^z$ определяется как \(\sum_{n=0}^{\infty}\frac{z^n}{n!}\). Это расширяет понятие экспоненты на комплексную область.
