\documentclass[12pt,a4paper]{article}
\usepackage[utf8]{inputenc}
\usepackage[T2A]{fontenc}
\usepackage[russian]{babel}
\usepackage{amsmath, amsthm, amssymb}
\usepackage[left=2cm,right=2cm,top=2cm,bottom=2cm]{geometry}
\usepackage{indentfirst}
\usepackage{hyperref}

% --- Здесь объявляем все окружения ---
\newtheorem{definition}{Определение}[section]

\makeatletter
\newenvironment{customtheorem}[1][]{%
  \par\noindent\textbf{Теорема}%
  % Если пользователь написал \begin{customtheorem}[НАЗВАНИЕ],
  % то #1 НЕ пустая; если пустая -- не выводим ничего.
  \@ifempty{#1}{}{\,(\textit{#1})}%
  .\par % точка, перенос строки + пол-эм отступ
}{%
  \par\bigskip % небольшой вертикальный отступ в конце
}
\makeatother

\newenvironment{customexample}{%
  \par\noindent\textbf{Пример.}\par%
}{%
  \par\bigskip
}

\newenvironment{proofplan}{%
  \par\noindent\textbf{План доказательства.}\par%
}{%
  \hfill$\square$\par\bigskip
}

\newenvironment{customproof}{%
  \par\noindent\textbf{Доказательство.}\par%
}{%
  \hfill$\blacksquare$\par\bigskip
}
% -------------------------------------

\begin{document}

\tableofcontents
\newpage

\section{Равномерная непрерывность. Примеры. Теорема Кантора}

\subsection*{Вспомогательные понятия}
% ==================
% auxiliary.tex
% (Вспомогательные определения для темы
% "Вывод рядов Тейлора для e^x, sin x, cos x и Формула Эйлера")
% ==================

\paragraph{Ряд Маклорена.}
Пусть функция $f$ имеет все производные в некоторой окрестности точки $0$. Тогда
\emph{рядом Маклорена} для $f$ называют
\[
	f(0) \;+\; \frac{f'(0)}{1!}\,x \;+\; \frac{f''(0)}{2!}\,x^2 \;+\; \dots \;+\;
	\frac{f^{(n)}(0)}{n!}\,x^n \;+\; \dots
\]
Если этот ряд сходится к $f(x)$ при соответствующих значениях $x$, то мы получаем разложение $f(x)$ в степенной ряд около $0$.

\bigskip

\paragraph{Остаточный член в форме Лагранжа.}
В случае, когда у функции $f$ есть $(n+1)$-я производная в окрестности $0$, можно записать
\[
	R_n(x)
	\;=\;
	\frac{f^{(n+1)}(\xi)}{(n+1)!}\,x^{n+1},
\]
где $\xi$ — некоторая точка между $0$ и $x$. Это называют \emph{остаточным членом} (или недостающим звеном) в формуле Тейлора (Маклорена).

\bigskip

\paragraph{Идея применения Теоремы Лагранжа.}
Для доказательства формулы Тейлора с остатком в форме Лагранжа часто используют теорему Лагранжа о среднем значении для производных:
если $P_n(x)$ — многочлен Тейлора степени $n$, то на отрезке $[0,x]$ применяют теорему Лагранжа к функции $f(x)-P_n(x)$, чтобы получить вид «\,$f^{(n+1)}(\xi)\,x^{n+1}/(n+1)!$».

\bigskip

\paragraph{Комплексная экспонента.}
Функция $e^{ix}$, где $i$ — мнимая единица, можно рассматривать как обобщение экспоненты на комплексную плоскость:
\[
	e^{ix} \;=\; \sum_{n=0}^{\infty} \frac{(i x)^n}{n!}.
\]
Из этого получается \emph{Формула Эйлера} при разбиении на действительную и мнимую часть.


\subsection*{Ответ на вопрос}
% ==================
% answer.tex
% (Основной материал:
%  "Определение интеграла Римана. Отличие от обычного предела.")
% ==================

\begin{customtheorem}[Интеграл Римана]
	Пусть функция $f$ задана на $[a,b]$. Если при всех возможных способах разбиения $[a,b]$ на малые отрезки,
	и выборе точек $\xi_i$ внутри этих отрезков, интегральные суммы
	\[
		S = \sum_{i=1}^n f(\xi_i)\,\Delta x_i
	\]
	стремятся к одному и тому же числу $I$ по мере $\max_i \Delta x_i \to 0$,
	то $f$ \textbf{интегрируема по Риману}, а $I$ называется \emph{интегралом Римана}:
	\[
		I = \int_{a}^{b} f(x)\,dx.
	\]
\end{customtheorem}

\begin{proofplan}
	\begin{enumerate}
		\item Рассмотреть любые два разбиения $D$ и $D'$ на отрезке $[a,b]$ с мелкостью $\|D\|\to0$, $\|D'\|\to0$.
		\item Построить общее уточнённое разбиение $D''$, включающее все точки из $D$ и $D'$.
		\item Оценить разницу сумм $S(f,D)$ и $S(f,D')$ через равномерную непрерывность (или ограниченность) $f$ на $[a,b]$.
		\item Показать, что эта разница становится сколь угодно малой при $\|D\|\to0$ и $\|D'\|\to0$.
		\item Вывод: предел един, определение интеграла однозначно.
	\end{enumerate}
\end{proofplan}

\begin{customproof}
	\textbf{Шаг 1: Два разбиения.}
	Пусть $D=\{a=x_0<x_1<\dots<x_n=b\}$ и $D'=\{a=y_0<y_1<\dots<y_m=b\}$ — любые разбиения отрезка $[a,b]$.
	Предположим, что $\|D\|=\max_i(x_i-x_{i-1})$ и $\|D'\|=\max_j(y_j-y_{j-1})$ оба стремятся к нулю.

	\smallskip

	\textbf{Шаг 2: Уточнение.}
	Построим «общее» разбиение $D''$, содержащее все точки из $D$ и $D'$.
	То есть объединим набор $\{x_i\}$ с $\{y_j\}$ в одну возрастающую последовательность.
	Теперь можно рассмотреть интегральные суммы относительно $D''$.

	\smallskip

	\textbf{Шаг 3: Оценка разницы.}
	На каждом элементе разбиения $[z_{k-1}, z_k]$ из $D''$ значения $f(\xi_i)$ меняются незначительно, если $f$ равномерно непрерывна (или ограничена).
	Тогда можно показать, что разность сумм $S(f,D)$ и $S(f,D')$ не превосходит
	некоторой малой величины, зависящей от $\|D''\|$, которая стремится к нулю,
	когда и $\|D\|\to0$, $\|D'\|\to0$.

	\smallskip

	\textbf{Шаг 4: Вывод.}
	Таким образом, любая интегральная сумма при мелкости разбиения стремится к одной и той же границе.
	Значит определение интеграла Римана корректно, и этот предел называется $\int_a^b f(x)\,dx$.
\end{customproof}

\begin{customexample}
	\begin{itemize}
		\item Если $f(x) \equiv C$ — константа, любая интегральная сумма $=C\cdot(b-a)$.
		      При любом разбиении ответ один: $\int_a^b C\,dx = C\,(b-a)$.
		\item Сравнение с обычным пределом:
		      \(\lim_{x\to x_0}f(x)\) — локальный анализ окрестности $x_0$.
		      \(\int_a^b f(x)\,dx\) — «суммарный» (глобальный) взгляд на отрезок $[a,b]$.
	\end{itemize}
\end{customexample}



\newpage
\section{Дифференциал функции. Теорема Ферма. Теорема Ролля. Примеры.}

\subsection*{Вспомогательные понятия}
% ==================
% auxiliary.tex
% (Вспомогательные определения для темы
% "Вывод рядов Тейлора для e^x, sin x, cos x и Формула Эйлера")
% ==================

\paragraph{Ряд Маклорена.}
Пусть функция $f$ имеет все производные в некоторой окрестности точки $0$. Тогда
\emph{рядом Маклорена} для $f$ называют
\[
	f(0) \;+\; \frac{f'(0)}{1!}\,x \;+\; \frac{f''(0)}{2!}\,x^2 \;+\; \dots \;+\;
	\frac{f^{(n)}(0)}{n!}\,x^n \;+\; \dots
\]
Если этот ряд сходится к $f(x)$ при соответствующих значениях $x$, то мы получаем разложение $f(x)$ в степенной ряд около $0$.

\bigskip

\paragraph{Остаточный член в форме Лагранжа.}
В случае, когда у функции $f$ есть $(n+1)$-я производная в окрестности $0$, можно записать
\[
	R_n(x)
	\;=\;
	\frac{f^{(n+1)}(\xi)}{(n+1)!}\,x^{n+1},
\]
где $\xi$ — некоторая точка между $0$ и $x$. Это называют \emph{остаточным членом} (или недостающим звеном) в формуле Тейлора (Маклорена).

\bigskip

\paragraph{Идея применения Теоремы Лагранжа.}
Для доказательства формулы Тейлора с остатком в форме Лагранжа часто используют теорему Лагранжа о среднем значении для производных:
если $P_n(x)$ — многочлен Тейлора степени $n$, то на отрезке $[0,x]$ применяют теорему Лагранжа к функции $f(x)-P_n(x)$, чтобы получить вид «\,$f^{(n+1)}(\xi)\,x^{n+1}/(n+1)!$».

\bigskip

\paragraph{Комплексная экспонента.}
Функция $e^{ix}$, где $i$ — мнимая единица, можно рассматривать как обобщение экспоненты на комплексную плоскость:
\[
	e^{ix} \;=\; \sum_{n=0}^{\infty} \frac{(i x)^n}{n!}.
\]
Из этого получается \emph{Формула Эйлера} при разбиении на действительную и мнимую часть.


\subsection*{Ответ на вопрос}
% ==================
% answer.tex
% (Основной материал:
%  "Определение интеграла Римана. Отличие от обычного предела.")
% ==================

\begin{customtheorem}[Интеграл Римана]
	Пусть функция $f$ задана на $[a,b]$. Если при всех возможных способах разбиения $[a,b]$ на малые отрезки,
	и выборе точек $\xi_i$ внутри этих отрезков, интегральные суммы
	\[
		S = \sum_{i=1}^n f(\xi_i)\,\Delta x_i
	\]
	стремятся к одному и тому же числу $I$ по мере $\max_i \Delta x_i \to 0$,
	то $f$ \textbf{интегрируема по Риману}, а $I$ называется \emph{интегралом Римана}:
	\[
		I = \int_{a}^{b} f(x)\,dx.
	\]
\end{customtheorem}

\begin{proofplan}
	\begin{enumerate}
		\item Рассмотреть любые два разбиения $D$ и $D'$ на отрезке $[a,b]$ с мелкостью $\|D\|\to0$, $\|D'\|\to0$.
		\item Построить общее уточнённое разбиение $D''$, включающее все точки из $D$ и $D'$.
		\item Оценить разницу сумм $S(f,D)$ и $S(f,D')$ через равномерную непрерывность (или ограниченность) $f$ на $[a,b]$.
		\item Показать, что эта разница становится сколь угодно малой при $\|D\|\to0$ и $\|D'\|\to0$.
		\item Вывод: предел един, определение интеграла однозначно.
	\end{enumerate}
\end{proofplan}

\begin{customproof}
	\textbf{Шаг 1: Два разбиения.}
	Пусть $D=\{a=x_0<x_1<\dots<x_n=b\}$ и $D'=\{a=y_0<y_1<\dots<y_m=b\}$ — любые разбиения отрезка $[a,b]$.
	Предположим, что $\|D\|=\max_i(x_i-x_{i-1})$ и $\|D'\|=\max_j(y_j-y_{j-1})$ оба стремятся к нулю.

	\smallskip

	\textbf{Шаг 2: Уточнение.}
	Построим «общее» разбиение $D''$, содержащее все точки из $D$ и $D'$.
	То есть объединим набор $\{x_i\}$ с $\{y_j\}$ в одну возрастающую последовательность.
	Теперь можно рассмотреть интегральные суммы относительно $D''$.

	\smallskip

	\textbf{Шаг 3: Оценка разницы.}
	На каждом элементе разбиения $[z_{k-1}, z_k]$ из $D''$ значения $f(\xi_i)$ меняются незначительно, если $f$ равномерно непрерывна (или ограничена).
	Тогда можно показать, что разность сумм $S(f,D)$ и $S(f,D')$ не превосходит
	некоторой малой величины, зависящей от $\|D''\|$, которая стремится к нулю,
	когда и $\|D\|\to0$, $\|D'\|\to0$.

	\smallskip

	\textbf{Шаг 4: Вывод.}
	Таким образом, любая интегральная сумма при мелкости разбиения стремится к одной и той же границе.
	Значит определение интеграла Римана корректно, и этот предел называется $\int_a^b f(x)\,dx$.
\end{customproof}

\begin{customexample}
	\begin{itemize}
		\item Если $f(x) \equiv C$ — константа, любая интегральная сумма $=C\cdot(b-a)$.
		      При любом разбиении ответ один: $\int_a^b C\,dx = C\,(b-a)$.
		\item Сравнение с обычным пределом:
		      \(\lim_{x\to x_0}f(x)\) — локальный анализ окрестности $x_0$.
		      \(\int_a^b f(x)\,dx\) — «суммарный» (глобальный) взгляд на отрезок $[a,b]$.
	\end{itemize}
\end{customexample}



\newpage
\section{Теорема Лагранжа. Необходимoе и достаточное условие постоянства дифференцируемой функции на промежутке. Необходимое и достаточное условие монотонности дифференцируемой функции на промежутке.}

\subsection*{Вспомогательные понятия}
% ==================
% auxiliary.tex
% (Вспомогательные определения для темы
% "Вывод рядов Тейлора для e^x, sin x, cos x и Формула Эйлера")
% ==================

\paragraph{Ряд Маклорена.}
Пусть функция $f$ имеет все производные в некоторой окрестности точки $0$. Тогда
\emph{рядом Маклорена} для $f$ называют
\[
	f(0) \;+\; \frac{f'(0)}{1!}\,x \;+\; \frac{f''(0)}{2!}\,x^2 \;+\; \dots \;+\;
	\frac{f^{(n)}(0)}{n!}\,x^n \;+\; \dots
\]
Если этот ряд сходится к $f(x)$ при соответствующих значениях $x$, то мы получаем разложение $f(x)$ в степенной ряд около $0$.

\bigskip

\paragraph{Остаточный член в форме Лагранжа.}
В случае, когда у функции $f$ есть $(n+1)$-я производная в окрестности $0$, можно записать
\[
	R_n(x)
	\;=\;
	\frac{f^{(n+1)}(\xi)}{(n+1)!}\,x^{n+1},
\]
где $\xi$ — некоторая точка между $0$ и $x$. Это называют \emph{остаточным членом} (или недостающим звеном) в формуле Тейлора (Маклорена).

\bigskip

\paragraph{Идея применения Теоремы Лагранжа.}
Для доказательства формулы Тейлора с остатком в форме Лагранжа часто используют теорему Лагранжа о среднем значении для производных:
если $P_n(x)$ — многочлен Тейлора степени $n$, то на отрезке $[0,x]$ применяют теорему Лагранжа к функции $f(x)-P_n(x)$, чтобы получить вид «\,$f^{(n+1)}(\xi)\,x^{n+1}/(n+1)!$».

\bigskip

\paragraph{Комплексная экспонента.}
Функция $e^{ix}$, где $i$ — мнимая единица, можно рассматривать как обобщение экспоненты на комплексную плоскость:
\[
	e^{ix} \;=\; \sum_{n=0}^{\infty} \frac{(i x)^n}{n!}.
\]
Из этого получается \emph{Формула Эйлера} при разбиении на действительную и мнимую часть.


\subsection*{Ответ на вопрос}
% ==================
% answer.tex
% (Основной материал:
%  "Определение интеграла Римана. Отличие от обычного предела.")
% ==================

\begin{customtheorem}[Интеграл Римана]
	Пусть функция $f$ задана на $[a,b]$. Если при всех возможных способах разбиения $[a,b]$ на малые отрезки,
	и выборе точек $\xi_i$ внутри этих отрезков, интегральные суммы
	\[
		S = \sum_{i=1}^n f(\xi_i)\,\Delta x_i
	\]
	стремятся к одному и тому же числу $I$ по мере $\max_i \Delta x_i \to 0$,
	то $f$ \textbf{интегрируема по Риману}, а $I$ называется \emph{интегралом Римана}:
	\[
		I = \int_{a}^{b} f(x)\,dx.
	\]
\end{customtheorem}

\begin{proofplan}
	\begin{enumerate}
		\item Рассмотреть любые два разбиения $D$ и $D'$ на отрезке $[a,b]$ с мелкостью $\|D\|\to0$, $\|D'\|\to0$.
		\item Построить общее уточнённое разбиение $D''$, включающее все точки из $D$ и $D'$.
		\item Оценить разницу сумм $S(f,D)$ и $S(f,D')$ через равномерную непрерывность (или ограниченность) $f$ на $[a,b]$.
		\item Показать, что эта разница становится сколь угодно малой при $\|D\|\to0$ и $\|D'\|\to0$.
		\item Вывод: предел един, определение интеграла однозначно.
	\end{enumerate}
\end{proofplan}

\begin{customproof}
	\textbf{Шаг 1: Два разбиения.}
	Пусть $D=\{a=x_0<x_1<\dots<x_n=b\}$ и $D'=\{a=y_0<y_1<\dots<y_m=b\}$ — любые разбиения отрезка $[a,b]$.
	Предположим, что $\|D\|=\max_i(x_i-x_{i-1})$ и $\|D'\|=\max_j(y_j-y_{j-1})$ оба стремятся к нулю.

	\smallskip

	\textbf{Шаг 2: Уточнение.}
	Построим «общее» разбиение $D''$, содержащее все точки из $D$ и $D'$.
	То есть объединим набор $\{x_i\}$ с $\{y_j\}$ в одну возрастающую последовательность.
	Теперь можно рассмотреть интегральные суммы относительно $D''$.

	\smallskip

	\textbf{Шаг 3: Оценка разницы.}
	На каждом элементе разбиения $[z_{k-1}, z_k]$ из $D''$ значения $f(\xi_i)$ меняются незначительно, если $f$ равномерно непрерывна (или ограничена).
	Тогда можно показать, что разность сумм $S(f,D)$ и $S(f,D')$ не превосходит
	некоторой малой величины, зависящей от $\|D''\|$, которая стремится к нулю,
	когда и $\|D\|\to0$, $\|D'\|\to0$.

	\smallskip

	\textbf{Шаг 4: Вывод.}
	Таким образом, любая интегральная сумма при мелкости разбиения стремится к одной и той же границе.
	Значит определение интеграла Римана корректно, и этот предел называется $\int_a^b f(x)\,dx$.
\end{customproof}

\begin{customexample}
	\begin{itemize}
		\item Если $f(x) \equiv C$ — константа, любая интегральная сумма $=C\cdot(b-a)$.
		      При любом разбиении ответ один: $\int_a^b C\,dx = C\,(b-a)$.
		\item Сравнение с обычным пределом:
		      \(\lim_{x\to x_0}f(x)\) — локальный анализ окрестности $x_0$.
		      \(\int_a^b f(x)\,dx\) — «суммарный» (глобальный) взгляд на отрезок $[a,b]$.
	\end{itemize}
\end{customexample}


\newpage
\section{Равномерная непрерывность. Примеры. Теорема Кантора о равномерной непрерывности.}

\subsection*{Вспомогательные понятия}
% ==================
% auxiliary.tex
% (Вспомогательные определения для темы
% "Вывод рядов Тейлора для e^x, sin x, cos x и Формула Эйлера")
% ==================

\paragraph{Ряд Маклорена.}
Пусть функция $f$ имеет все производные в некоторой окрестности точки $0$. Тогда
\emph{рядом Маклорена} для $f$ называют
\[
	f(0) \;+\; \frac{f'(0)}{1!}\,x \;+\; \frac{f''(0)}{2!}\,x^2 \;+\; \dots \;+\;
	\frac{f^{(n)}(0)}{n!}\,x^n \;+\; \dots
\]
Если этот ряд сходится к $f(x)$ при соответствующих значениях $x$, то мы получаем разложение $f(x)$ в степенной ряд около $0$.

\bigskip

\paragraph{Остаточный член в форме Лагранжа.}
В случае, когда у функции $f$ есть $(n+1)$-я производная в окрестности $0$, можно записать
\[
	R_n(x)
	\;=\;
	\frac{f^{(n+1)}(\xi)}{(n+1)!}\,x^{n+1},
\]
где $\xi$ — некоторая точка между $0$ и $x$. Это называют \emph{остаточным членом} (или недостающим звеном) в формуле Тейлора (Маклорена).

\bigskip

\paragraph{Идея применения Теоремы Лагранжа.}
Для доказательства формулы Тейлора с остатком в форме Лагранжа часто используют теорему Лагранжа о среднем значении для производных:
если $P_n(x)$ — многочлен Тейлора степени $n$, то на отрезке $[0,x]$ применяют теорему Лагранжа к функции $f(x)-P_n(x)$, чтобы получить вид «\,$f^{(n+1)}(\xi)\,x^{n+1}/(n+1)!$».

\bigskip

\paragraph{Комплексная экспонента.}
Функция $e^{ix}$, где $i$ — мнимая единица, можно рассматривать как обобщение экспоненты на комплексную плоскость:
\[
	e^{ix} \;=\; \sum_{n=0}^{\infty} \frac{(i x)^n}{n!}.
\]
Из этого получается \emph{Формула Эйлера} при разбиении на действительную и мнимую часть.


\subsection*{Ответ на вопрос}
% ==================
% answer.tex
% (Основной материал:
%  "Определение интеграла Римана. Отличие от обычного предела.")
% ==================

\begin{customtheorem}[Интеграл Римана]
	Пусть функция $f$ задана на $[a,b]$. Если при всех возможных способах разбиения $[a,b]$ на малые отрезки,
	и выборе точек $\xi_i$ внутри этих отрезков, интегральные суммы
	\[
		S = \sum_{i=1}^n f(\xi_i)\,\Delta x_i
	\]
	стремятся к одному и тому же числу $I$ по мере $\max_i \Delta x_i \to 0$,
	то $f$ \textbf{интегрируема по Риману}, а $I$ называется \emph{интегралом Римана}:
	\[
		I = \int_{a}^{b} f(x)\,dx.
	\]
\end{customtheorem}

\begin{proofplan}
	\begin{enumerate}
		\item Рассмотреть любые два разбиения $D$ и $D'$ на отрезке $[a,b]$ с мелкостью $\|D\|\to0$, $\|D'\|\to0$.
		\item Построить общее уточнённое разбиение $D''$, включающее все точки из $D$ и $D'$.
		\item Оценить разницу сумм $S(f,D)$ и $S(f,D')$ через равномерную непрерывность (или ограниченность) $f$ на $[a,b]$.
		\item Показать, что эта разница становится сколь угодно малой при $\|D\|\to0$ и $\|D'\|\to0$.
		\item Вывод: предел един, определение интеграла однозначно.
	\end{enumerate}
\end{proofplan}

\begin{customproof}
	\textbf{Шаг 1: Два разбиения.}
	Пусть $D=\{a=x_0<x_1<\dots<x_n=b\}$ и $D'=\{a=y_0<y_1<\dots<y_m=b\}$ — любые разбиения отрезка $[a,b]$.
	Предположим, что $\|D\|=\max_i(x_i-x_{i-1})$ и $\|D'\|=\max_j(y_j-y_{j-1})$ оба стремятся к нулю.

	\smallskip

	\textbf{Шаг 2: Уточнение.}
	Построим «общее» разбиение $D''$, содержащее все точки из $D$ и $D'$.
	То есть объединим набор $\{x_i\}$ с $\{y_j\}$ в одну возрастающую последовательность.
	Теперь можно рассмотреть интегральные суммы относительно $D''$.

	\smallskip

	\textbf{Шаг 3: Оценка разницы.}
	На каждом элементе разбиения $[z_{k-1}, z_k]$ из $D''$ значения $f(\xi_i)$ меняются незначительно, если $f$ равномерно непрерывна (или ограничена).
	Тогда можно показать, что разность сумм $S(f,D)$ и $S(f,D')$ не превосходит
	некоторой малой величины, зависящей от $\|D''\|$, которая стремится к нулю,
	когда и $\|D\|\to0$, $\|D'\|\to0$.

	\smallskip

	\textbf{Шаг 4: Вывод.}
	Таким образом, любая интегральная сумма при мелкости разбиения стремится к одной и той же границе.
	Значит определение интеграла Римана корректно, и этот предел называется $\int_a^b f(x)\,dx$.
\end{customproof}

\begin{customexample}
	\begin{itemize}
		\item Если $f(x) \equiv C$ — константа, любая интегральная сумма $=C\cdot(b-a)$.
		      При любом разбиении ответ один: $\int_a^b C\,dx = C\,(b-a)$.
		\item Сравнение с обычным пределом:
		      \(\lim_{x\to x_0}f(x)\) — локальный анализ окрестности $x_0$.
		      \(\int_a^b f(x)\,dx\) — «суммарный» (глобальный) взгляд на отрезок $[a,b]$.
	\end{itemize}
\end{customexample}


\newpage
\section{Вывод рядов Тейлора для функций y=exp(x), y=sinx, y=cosx через следствие из теоремы Лагранжа. Формула Эйлера.}

\subsection*{Вспомогательные понятия}
% ==================
% auxiliary.tex
% (Вспомогательные определения для темы
% "Вывод рядов Тейлора для e^x, sin x, cos x и Формула Эйлера")
% ==================

\paragraph{Ряд Маклорена.}
Пусть функция $f$ имеет все производные в некоторой окрестности точки $0$. Тогда
\emph{рядом Маклорена} для $f$ называют
\[
	f(0) \;+\; \frac{f'(0)}{1!}\,x \;+\; \frac{f''(0)}{2!}\,x^2 \;+\; \dots \;+\;
	\frac{f^{(n)}(0)}{n!}\,x^n \;+\; \dots
\]
Если этот ряд сходится к $f(x)$ при соответствующих значениях $x$, то мы получаем разложение $f(x)$ в степенной ряд около $0$.

\bigskip

\paragraph{Остаточный член в форме Лагранжа.}
В случае, когда у функции $f$ есть $(n+1)$-я производная в окрестности $0$, можно записать
\[
	R_n(x)
	\;=\;
	\frac{f^{(n+1)}(\xi)}{(n+1)!}\,x^{n+1},
\]
где $\xi$ — некоторая точка между $0$ и $x$. Это называют \emph{остаточным членом} (или недостающим звеном) в формуле Тейлора (Маклорена).

\bigskip

\paragraph{Идея применения Теоремы Лагранжа.}
Для доказательства формулы Тейлора с остатком в форме Лагранжа часто используют теорему Лагранжа о среднем значении для производных:
если $P_n(x)$ — многочлен Тейлора степени $n$, то на отрезке $[0,x]$ применяют теорему Лагранжа к функции $f(x)-P_n(x)$, чтобы получить вид «\,$f^{(n+1)}(\xi)\,x^{n+1}/(n+1)!$».

\bigskip

\paragraph{Комплексная экспонента.}
Функция $e^{ix}$, где $i$ — мнимая единица, можно рассматривать как обобщение экспоненты на комплексную плоскость:
\[
	e^{ix} \;=\; \sum_{n=0}^{\infty} \frac{(i x)^n}{n!}.
\]
Из этого получается \emph{Формула Эйлера} при разбиении на действительную и мнимую часть.


\subsection*{Ответ на вопрос}
% ==================
% answer.tex
% (Основной материал:
%  "Определение интеграла Римана. Отличие от обычного предела.")
% ==================

\begin{customtheorem}[Интеграл Римана]
	Пусть функция $f$ задана на $[a,b]$. Если при всех возможных способах разбиения $[a,b]$ на малые отрезки,
	и выборе точек $\xi_i$ внутри этих отрезков, интегральные суммы
	\[
		S = \sum_{i=1}^n f(\xi_i)\,\Delta x_i
	\]
	стремятся к одному и тому же числу $I$ по мере $\max_i \Delta x_i \to 0$,
	то $f$ \textbf{интегрируема по Риману}, а $I$ называется \emph{интегралом Римана}:
	\[
		I = \int_{a}^{b} f(x)\,dx.
	\]
\end{customtheorem}

\begin{proofplan}
	\begin{enumerate}
		\item Рассмотреть любые два разбиения $D$ и $D'$ на отрезке $[a,b]$ с мелкостью $\|D\|\to0$, $\|D'\|\to0$.
		\item Построить общее уточнённое разбиение $D''$, включающее все точки из $D$ и $D'$.
		\item Оценить разницу сумм $S(f,D)$ и $S(f,D')$ через равномерную непрерывность (или ограниченность) $f$ на $[a,b]$.
		\item Показать, что эта разница становится сколь угодно малой при $\|D\|\to0$ и $\|D'\|\to0$.
		\item Вывод: предел един, определение интеграла однозначно.
	\end{enumerate}
\end{proofplan}

\begin{customproof}
	\textbf{Шаг 1: Два разбиения.}
	Пусть $D=\{a=x_0<x_1<\dots<x_n=b\}$ и $D'=\{a=y_0<y_1<\dots<y_m=b\}$ — любые разбиения отрезка $[a,b]$.
	Предположим, что $\|D\|=\max_i(x_i-x_{i-1})$ и $\|D'\|=\max_j(y_j-y_{j-1})$ оба стремятся к нулю.

	\smallskip

	\textbf{Шаг 2: Уточнение.}
	Построим «общее» разбиение $D''$, содержащее все точки из $D$ и $D'$.
	То есть объединим набор $\{x_i\}$ с $\{y_j\}$ в одну возрастающую последовательность.
	Теперь можно рассмотреть интегральные суммы относительно $D''$.

	\smallskip

	\textbf{Шаг 3: Оценка разницы.}
	На каждом элементе разбиения $[z_{k-1}, z_k]$ из $D''$ значения $f(\xi_i)$ меняются незначительно, если $f$ равномерно непрерывна (или ограничена).
	Тогда можно показать, что разность сумм $S(f,D)$ и $S(f,D')$ не превосходит
	некоторой малой величины, зависящей от $\|D''\|$, которая стремится к нулю,
	когда и $\|D\|\to0$, $\|D'\|\to0$.

	\smallskip

	\textbf{Шаг 4: Вывод.}
	Таким образом, любая интегральная сумма при мелкости разбиения стремится к одной и той же границе.
	Значит определение интеграла Римана корректно, и этот предел называется $\int_a^b f(x)\,dx$.
\end{customproof}

\begin{customexample}
	\begin{itemize}
		\item Если $f(x) \equiv C$ — константа, любая интегральная сумма $=C\cdot(b-a)$.
		      При любом разбиении ответ один: $\int_a^b C\,dx = C\,(b-a)$.
		\item Сравнение с обычным пределом:
		      \(\lim_{x\to x_0}f(x)\) — локальный анализ окрестности $x_0$.
		      \(\int_a^b f(x)\,dx\) — «суммарный» (глобальный) взгляд на отрезок $[a,b]$.
	\end{itemize}
\end{customexample}


\newpage
\section{Теорема Коши. Правило Лопиталя (доказательство – только для случая 0/0). Примеры, когда правило неприменимо.}

\subsection*{Вспомогательные понятия}
% ==================
% auxiliary.tex
% (Вспомогательные определения для темы
% "Вывод рядов Тейлора для e^x, sin x, cos x и Формула Эйлера")
% ==================

\paragraph{Ряд Маклорена.}
Пусть функция $f$ имеет все производные в некоторой окрестности точки $0$. Тогда
\emph{рядом Маклорена} для $f$ называют
\[
	f(0) \;+\; \frac{f'(0)}{1!}\,x \;+\; \frac{f''(0)}{2!}\,x^2 \;+\; \dots \;+\;
	\frac{f^{(n)}(0)}{n!}\,x^n \;+\; \dots
\]
Если этот ряд сходится к $f(x)$ при соответствующих значениях $x$, то мы получаем разложение $f(x)$ в степенной ряд около $0$.

\bigskip

\paragraph{Остаточный член в форме Лагранжа.}
В случае, когда у функции $f$ есть $(n+1)$-я производная в окрестности $0$, можно записать
\[
	R_n(x)
	\;=\;
	\frac{f^{(n+1)}(\xi)}{(n+1)!}\,x^{n+1},
\]
где $\xi$ — некоторая точка между $0$ и $x$. Это называют \emph{остаточным членом} (или недостающим звеном) в формуле Тейлора (Маклорена).

\bigskip

\paragraph{Идея применения Теоремы Лагранжа.}
Для доказательства формулы Тейлора с остатком в форме Лагранжа часто используют теорему Лагранжа о среднем значении для производных:
если $P_n(x)$ — многочлен Тейлора степени $n$, то на отрезке $[0,x]$ применяют теорему Лагранжа к функции $f(x)-P_n(x)$, чтобы получить вид «\,$f^{(n+1)}(\xi)\,x^{n+1}/(n+1)!$».

\bigskip

\paragraph{Комплексная экспонента.}
Функция $e^{ix}$, где $i$ — мнимая единица, можно рассматривать как обобщение экспоненты на комплексную плоскость:
\[
	e^{ix} \;=\; \sum_{n=0}^{\infty} \frac{(i x)^n}{n!}.
\]
Из этого получается \emph{Формула Эйлера} при разбиении на действительную и мнимую часть.


\subsection*{Ответ на вопрос}
% ==================
% answer.tex
% (Основной материал:
%  "Определение интеграла Римана. Отличие от обычного предела.")
% ==================

\begin{customtheorem}[Интеграл Римана]
	Пусть функция $f$ задана на $[a,b]$. Если при всех возможных способах разбиения $[a,b]$ на малые отрезки,
	и выборе точек $\xi_i$ внутри этих отрезков, интегральные суммы
	\[
		S = \sum_{i=1}^n f(\xi_i)\,\Delta x_i
	\]
	стремятся к одному и тому же числу $I$ по мере $\max_i \Delta x_i \to 0$,
	то $f$ \textbf{интегрируема по Риману}, а $I$ называется \emph{интегралом Римана}:
	\[
		I = \int_{a}^{b} f(x)\,dx.
	\]
\end{customtheorem}

\begin{proofplan}
	\begin{enumerate}
		\item Рассмотреть любые два разбиения $D$ и $D'$ на отрезке $[a,b]$ с мелкостью $\|D\|\to0$, $\|D'\|\to0$.
		\item Построить общее уточнённое разбиение $D''$, включающее все точки из $D$ и $D'$.
		\item Оценить разницу сумм $S(f,D)$ и $S(f,D')$ через равномерную непрерывность (или ограниченность) $f$ на $[a,b]$.
		\item Показать, что эта разница становится сколь угодно малой при $\|D\|\to0$ и $\|D'\|\to0$.
		\item Вывод: предел един, определение интеграла однозначно.
	\end{enumerate}
\end{proofplan}

\begin{customproof}
	\textbf{Шаг 1: Два разбиения.}
	Пусть $D=\{a=x_0<x_1<\dots<x_n=b\}$ и $D'=\{a=y_0<y_1<\dots<y_m=b\}$ — любые разбиения отрезка $[a,b]$.
	Предположим, что $\|D\|=\max_i(x_i-x_{i-1})$ и $\|D'\|=\max_j(y_j-y_{j-1})$ оба стремятся к нулю.

	\smallskip

	\textbf{Шаг 2: Уточнение.}
	Построим «общее» разбиение $D''$, содержащее все точки из $D$ и $D'$.
	То есть объединим набор $\{x_i\}$ с $\{y_j\}$ в одну возрастающую последовательность.
	Теперь можно рассмотреть интегральные суммы относительно $D''$.

	\smallskip

	\textbf{Шаг 3: Оценка разницы.}
	На каждом элементе разбиения $[z_{k-1}, z_k]$ из $D''$ значения $f(\xi_i)$ меняются незначительно, если $f$ равномерно непрерывна (или ограничена).
	Тогда можно показать, что разность сумм $S(f,D)$ и $S(f,D')$ не превосходит
	некоторой малой величины, зависящей от $\|D''\|$, которая стремится к нулю,
	когда и $\|D\|\to0$, $\|D'\|\to0$.

	\smallskip

	\textbf{Шаг 4: Вывод.}
	Таким образом, любая интегральная сумма при мелкости разбиения стремится к одной и той же границе.
	Значит определение интеграла Римана корректно, и этот предел называется $\int_a^b f(x)\,dx$.
\end{customproof}

\begin{customexample}
	\begin{itemize}
		\item Если $f(x) \equiv C$ — константа, любая интегральная сумма $=C\cdot(b-a)$.
		      При любом разбиении ответ один: $\int_a^b C\,dx = C\,(b-a)$.
		\item Сравнение с обычным пределом:
		      \(\lim_{x\to x_0}f(x)\) — локальный анализ окрестности $x_0$.
		      \(\int_a^b f(x)\,dx\) — «суммарный» (глобальный) взгляд на отрезок $[a,b]$.
	\end{itemize}
\end{customexample}


\newpage
\section{Формула Тейлора для многочлена. Формула Тейлора с остатком в форме Пеано.}

\subsection*{Вспомогательные понятия}
% ==================
% auxiliary.tex
% (Вспомогательные определения для темы
% "Вывод рядов Тейлора для e^x, sin x, cos x и Формула Эйлера")
% ==================

\paragraph{Ряд Маклорена.}
Пусть функция $f$ имеет все производные в некоторой окрестности точки $0$. Тогда
\emph{рядом Маклорена} для $f$ называют
\[
	f(0) \;+\; \frac{f'(0)}{1!}\,x \;+\; \frac{f''(0)}{2!}\,x^2 \;+\; \dots \;+\;
	\frac{f^{(n)}(0)}{n!}\,x^n \;+\; \dots
\]
Если этот ряд сходится к $f(x)$ при соответствующих значениях $x$, то мы получаем разложение $f(x)$ в степенной ряд около $0$.

\bigskip

\paragraph{Остаточный член в форме Лагранжа.}
В случае, когда у функции $f$ есть $(n+1)$-я производная в окрестности $0$, можно записать
\[
	R_n(x)
	\;=\;
	\frac{f^{(n+1)}(\xi)}{(n+1)!}\,x^{n+1},
\]
где $\xi$ — некоторая точка между $0$ и $x$. Это называют \emph{остаточным членом} (или недостающим звеном) в формуле Тейлора (Маклорена).

\bigskip

\paragraph{Идея применения Теоремы Лагранжа.}
Для доказательства формулы Тейлора с остатком в форме Лагранжа часто используют теорему Лагранжа о среднем значении для производных:
если $P_n(x)$ — многочлен Тейлора степени $n$, то на отрезке $[0,x]$ применяют теорему Лагранжа к функции $f(x)-P_n(x)$, чтобы получить вид «\,$f^{(n+1)}(\xi)\,x^{n+1}/(n+1)!$».

\bigskip

\paragraph{Комплексная экспонента.}
Функция $e^{ix}$, где $i$ — мнимая единица, можно рассматривать как обобщение экспоненты на комплексную плоскость:
\[
	e^{ix} \;=\; \sum_{n=0}^{\infty} \frac{(i x)^n}{n!}.
\]
Из этого получается \emph{Формула Эйлера} при разбиении на действительную и мнимую часть.


\subsection*{Ответ на вопрос}
% ==================
% answer.tex
% (Основной материал:
%  "Определение интеграла Римана. Отличие от обычного предела.")
% ==================

\begin{customtheorem}[Интеграл Римана]
	Пусть функция $f$ задана на $[a,b]$. Если при всех возможных способах разбиения $[a,b]$ на малые отрезки,
	и выборе точек $\xi_i$ внутри этих отрезков, интегральные суммы
	\[
		S = \sum_{i=1}^n f(\xi_i)\,\Delta x_i
	\]
	стремятся к одному и тому же числу $I$ по мере $\max_i \Delta x_i \to 0$,
	то $f$ \textbf{интегрируема по Риману}, а $I$ называется \emph{интегралом Римана}:
	\[
		I = \int_{a}^{b} f(x)\,dx.
	\]
\end{customtheorem}

\begin{proofplan}
	\begin{enumerate}
		\item Рассмотреть любые два разбиения $D$ и $D'$ на отрезке $[a,b]$ с мелкостью $\|D\|\to0$, $\|D'\|\to0$.
		\item Построить общее уточнённое разбиение $D''$, включающее все точки из $D$ и $D'$.
		\item Оценить разницу сумм $S(f,D)$ и $S(f,D')$ через равномерную непрерывность (или ограниченность) $f$ на $[a,b]$.
		\item Показать, что эта разница становится сколь угодно малой при $\|D\|\to0$ и $\|D'\|\to0$.
		\item Вывод: предел един, определение интеграла однозначно.
	\end{enumerate}
\end{proofplan}

\begin{customproof}
	\textbf{Шаг 1: Два разбиения.}
	Пусть $D=\{a=x_0<x_1<\dots<x_n=b\}$ и $D'=\{a=y_0<y_1<\dots<y_m=b\}$ — любые разбиения отрезка $[a,b]$.
	Предположим, что $\|D\|=\max_i(x_i-x_{i-1})$ и $\|D'\|=\max_j(y_j-y_{j-1})$ оба стремятся к нулю.

	\smallskip

	\textbf{Шаг 2: Уточнение.}
	Построим «общее» разбиение $D''$, содержащее все точки из $D$ и $D'$.
	То есть объединим набор $\{x_i\}$ с $\{y_j\}$ в одну возрастающую последовательность.
	Теперь можно рассмотреть интегральные суммы относительно $D''$.

	\smallskip

	\textbf{Шаг 3: Оценка разницы.}
	На каждом элементе разбиения $[z_{k-1}, z_k]$ из $D''$ значения $f(\xi_i)$ меняются незначительно, если $f$ равномерно непрерывна (или ограничена).
	Тогда можно показать, что разность сумм $S(f,D)$ и $S(f,D')$ не превосходит
	некоторой малой величины, зависящей от $\|D''\|$, которая стремится к нулю,
	когда и $\|D\|\to0$, $\|D'\|\to0$.

	\smallskip

	\textbf{Шаг 4: Вывод.}
	Таким образом, любая интегральная сумма при мелкости разбиения стремится к одной и той же границе.
	Значит определение интеграла Римана корректно, и этот предел называется $\int_a^b f(x)\,dx$.
\end{customproof}

\begin{customexample}
	\begin{itemize}
		\item Если $f(x) \equiv C$ — константа, любая интегральная сумма $=C\cdot(b-a)$.
		      При любом разбиении ответ один: $\int_a^b C\,dx = C\,(b-a)$.
		\item Сравнение с обычным пределом:
		      \(\lim_{x\to x_0}f(x)\) — локальный анализ окрестности $x_0$.
		      \(\int_a^b f(x)\,dx\) — «суммарный» (глобальный) взгляд на отрезок $[a,b]$.
	\end{itemize}
\end{customexample}


\newpage
\section{Достаточные условия существования экстремума (по второй производной).}

\subsection*{Вспомогательные понятия}
% ==================
% auxiliary.tex
% (Вспомогательные определения для темы
% "Вывод рядов Тейлора для e^x, sin x, cos x и Формула Эйлера")
% ==================

\paragraph{Ряд Маклорена.}
Пусть функция $f$ имеет все производные в некоторой окрестности точки $0$. Тогда
\emph{рядом Маклорена} для $f$ называют
\[
	f(0) \;+\; \frac{f'(0)}{1!}\,x \;+\; \frac{f''(0)}{2!}\,x^2 \;+\; \dots \;+\;
	\frac{f^{(n)}(0)}{n!}\,x^n \;+\; \dots
\]
Если этот ряд сходится к $f(x)$ при соответствующих значениях $x$, то мы получаем разложение $f(x)$ в степенной ряд около $0$.

\bigskip

\paragraph{Остаточный член в форме Лагранжа.}
В случае, когда у функции $f$ есть $(n+1)$-я производная в окрестности $0$, можно записать
\[
	R_n(x)
	\;=\;
	\frac{f^{(n+1)}(\xi)}{(n+1)!}\,x^{n+1},
\]
где $\xi$ — некоторая точка между $0$ и $x$. Это называют \emph{остаточным членом} (или недостающим звеном) в формуле Тейлора (Маклорена).

\bigskip

\paragraph{Идея применения Теоремы Лагранжа.}
Для доказательства формулы Тейлора с остатком в форме Лагранжа часто используют теорему Лагранжа о среднем значении для производных:
если $P_n(x)$ — многочлен Тейлора степени $n$, то на отрезке $[0,x]$ применяют теорему Лагранжа к функции $f(x)-P_n(x)$, чтобы получить вид «\,$f^{(n+1)}(\xi)\,x^{n+1}/(n+1)!$».

\bigskip

\paragraph{Комплексная экспонента.}
Функция $e^{ix}$, где $i$ — мнимая единица, можно рассматривать как обобщение экспоненты на комплексную плоскость:
\[
	e^{ix} \;=\; \sum_{n=0}^{\infty} \frac{(i x)^n}{n!}.
\]
Из этого получается \emph{Формула Эйлера} при разбиении на действительную и мнимую часть.


\subsection*{Ответ на вопрос}
% ==================
% answer.tex
% (Основной материал:
%  "Определение интеграла Римана. Отличие от обычного предела.")
% ==================

\begin{customtheorem}[Интеграл Римана]
	Пусть функция $f$ задана на $[a,b]$. Если при всех возможных способах разбиения $[a,b]$ на малые отрезки,
	и выборе точек $\xi_i$ внутри этих отрезков, интегральные суммы
	\[
		S = \sum_{i=1}^n f(\xi_i)\,\Delta x_i
	\]
	стремятся к одному и тому же числу $I$ по мере $\max_i \Delta x_i \to 0$,
	то $f$ \textbf{интегрируема по Риману}, а $I$ называется \emph{интегралом Римана}:
	\[
		I = \int_{a}^{b} f(x)\,dx.
	\]
\end{customtheorem}

\begin{proofplan}
	\begin{enumerate}
		\item Рассмотреть любые два разбиения $D$ и $D'$ на отрезке $[a,b]$ с мелкостью $\|D\|\to0$, $\|D'\|\to0$.
		\item Построить общее уточнённое разбиение $D''$, включающее все точки из $D$ и $D'$.
		\item Оценить разницу сумм $S(f,D)$ и $S(f,D')$ через равномерную непрерывность (или ограниченность) $f$ на $[a,b]$.
		\item Показать, что эта разница становится сколь угодно малой при $\|D\|\to0$ и $\|D'\|\to0$.
		\item Вывод: предел един, определение интеграла однозначно.
	\end{enumerate}
\end{proofplan}

\begin{customproof}
	\textbf{Шаг 1: Два разбиения.}
	Пусть $D=\{a=x_0<x_1<\dots<x_n=b\}$ и $D'=\{a=y_0<y_1<\dots<y_m=b\}$ — любые разбиения отрезка $[a,b]$.
	Предположим, что $\|D\|=\max_i(x_i-x_{i-1})$ и $\|D'\|=\max_j(y_j-y_{j-1})$ оба стремятся к нулю.

	\smallskip

	\textbf{Шаг 2: Уточнение.}
	Построим «общее» разбиение $D''$, содержащее все точки из $D$ и $D'$.
	То есть объединим набор $\{x_i\}$ с $\{y_j\}$ в одну возрастающую последовательность.
	Теперь можно рассмотреть интегральные суммы относительно $D''$.

	\smallskip

	\textbf{Шаг 3: Оценка разницы.}
	На каждом элементе разбиения $[z_{k-1}, z_k]$ из $D''$ значения $f(\xi_i)$ меняются незначительно, если $f$ равномерно непрерывна (или ограничена).
	Тогда можно показать, что разность сумм $S(f,D)$ и $S(f,D')$ не превосходит
	некоторой малой величины, зависящей от $\|D''\|$, которая стремится к нулю,
	когда и $\|D\|\to0$, $\|D'\|\to0$.

	\smallskip

	\textbf{Шаг 4: Вывод.}
	Таким образом, любая интегральная сумма при мелкости разбиения стремится к одной и той же границе.
	Значит определение интеграла Римана корректно, и этот предел называется $\int_a^b f(x)\,dx$.
\end{customproof}

\begin{customexample}
	\begin{itemize}
		\item Если $f(x) \equiv C$ — константа, любая интегральная сумма $=C\cdot(b-a)$.
		      При любом разбиении ответ один: $\int_a^b C\,dx = C\,(b-a)$.
		\item Сравнение с обычным пределом:
		      \(\lim_{x\to x_0}f(x)\) — локальный анализ окрестности $x_0$.
		      \(\int_a^b f(x)\,dx\) — «суммарный» (глобальный) взгляд на отрезок $[a,b]$.
	\end{itemize}
\end{customexample}


\newpage
\section{Теорема Лиувилля. Пример трансцендентного числа.}

\subsection*{Вспомогательные понятия}
% ==================
% auxiliary.tex
% (Вспомогательные определения для темы
% "Вывод рядов Тейлора для e^x, sin x, cos x и Формула Эйлера")
% ==================

\paragraph{Ряд Маклорена.}
Пусть функция $f$ имеет все производные в некоторой окрестности точки $0$. Тогда
\emph{рядом Маклорена} для $f$ называют
\[
	f(0) \;+\; \frac{f'(0)}{1!}\,x \;+\; \frac{f''(0)}{2!}\,x^2 \;+\; \dots \;+\;
	\frac{f^{(n)}(0)}{n!}\,x^n \;+\; \dots
\]
Если этот ряд сходится к $f(x)$ при соответствующих значениях $x$, то мы получаем разложение $f(x)$ в степенной ряд около $0$.

\bigskip

\paragraph{Остаточный член в форме Лагранжа.}
В случае, когда у функции $f$ есть $(n+1)$-я производная в окрестности $0$, можно записать
\[
	R_n(x)
	\;=\;
	\frac{f^{(n+1)}(\xi)}{(n+1)!}\,x^{n+1},
\]
где $\xi$ — некоторая точка между $0$ и $x$. Это называют \emph{остаточным членом} (или недостающим звеном) в формуле Тейлора (Маклорена).

\bigskip

\paragraph{Идея применения Теоремы Лагранжа.}
Для доказательства формулы Тейлора с остатком в форме Лагранжа часто используют теорему Лагранжа о среднем значении для производных:
если $P_n(x)$ — многочлен Тейлора степени $n$, то на отрезке $[0,x]$ применяют теорему Лагранжа к функции $f(x)-P_n(x)$, чтобы получить вид «\,$f^{(n+1)}(\xi)\,x^{n+1}/(n+1)!$».

\bigskip

\paragraph{Комплексная экспонента.}
Функция $e^{ix}$, где $i$ — мнимая единица, можно рассматривать как обобщение экспоненты на комплексную плоскость:
\[
	e^{ix} \;=\; \sum_{n=0}^{\infty} \frac{(i x)^n}{n!}.
\]
Из этого получается \emph{Формула Эйлера} при разбиении на действительную и мнимую часть.


\subsection*{Ответ на вопрос}
% ==================
% answer.tex
% (Основной материал:
%  "Определение интеграла Римана. Отличие от обычного предела.")
% ==================

\begin{customtheorem}[Интеграл Римана]
	Пусть функция $f$ задана на $[a,b]$. Если при всех возможных способах разбиения $[a,b]$ на малые отрезки,
	и выборе точек $\xi_i$ внутри этих отрезков, интегральные суммы
	\[
		S = \sum_{i=1}^n f(\xi_i)\,\Delta x_i
	\]
	стремятся к одному и тому же числу $I$ по мере $\max_i \Delta x_i \to 0$,
	то $f$ \textbf{интегрируема по Риману}, а $I$ называется \emph{интегралом Римана}:
	\[
		I = \int_{a}^{b} f(x)\,dx.
	\]
\end{customtheorem}

\begin{proofplan}
	\begin{enumerate}
		\item Рассмотреть любые два разбиения $D$ и $D'$ на отрезке $[a,b]$ с мелкостью $\|D\|\to0$, $\|D'\|\to0$.
		\item Построить общее уточнённое разбиение $D''$, включающее все точки из $D$ и $D'$.
		\item Оценить разницу сумм $S(f,D)$ и $S(f,D')$ через равномерную непрерывность (или ограниченность) $f$ на $[a,b]$.
		\item Показать, что эта разница становится сколь угодно малой при $\|D\|\to0$ и $\|D'\|\to0$.
		\item Вывод: предел един, определение интеграла однозначно.
	\end{enumerate}
\end{proofplan}

\begin{customproof}
	\textbf{Шаг 1: Два разбиения.}
	Пусть $D=\{a=x_0<x_1<\dots<x_n=b\}$ и $D'=\{a=y_0<y_1<\dots<y_m=b\}$ — любые разбиения отрезка $[a,b]$.
	Предположим, что $\|D\|=\max_i(x_i-x_{i-1})$ и $\|D'\|=\max_j(y_j-y_{j-1})$ оба стремятся к нулю.

	\smallskip

	\textbf{Шаг 2: Уточнение.}
	Построим «общее» разбиение $D''$, содержащее все точки из $D$ и $D'$.
	То есть объединим набор $\{x_i\}$ с $\{y_j\}$ в одну возрастающую последовательность.
	Теперь можно рассмотреть интегральные суммы относительно $D''$.

	\smallskip

	\textbf{Шаг 3: Оценка разницы.}
	На каждом элементе разбиения $[z_{k-1}, z_k]$ из $D''$ значения $f(\xi_i)$ меняются незначительно, если $f$ равномерно непрерывна (или ограничена).
	Тогда можно показать, что разность сумм $S(f,D)$ и $S(f,D')$ не превосходит
	некоторой малой величины, зависящей от $\|D''\|$, которая стремится к нулю,
	когда и $\|D\|\to0$, $\|D'\|\to0$.

	\smallskip

	\textbf{Шаг 4: Вывод.}
	Таким образом, любая интегральная сумма при мелкости разбиения стремится к одной и той же границе.
	Значит определение интеграла Римана корректно, и этот предел называется $\int_a^b f(x)\,dx$.
\end{customproof}

\begin{customexample}
	\begin{itemize}
		\item Если $f(x) \equiv C$ — константа, любая интегральная сумма $=C\cdot(b-a)$.
		      При любом разбиении ответ один: $\int_a^b C\,dx = C\,(b-a)$.
		\item Сравнение с обычным пределом:
		      \(\lim_{x\to x_0}f(x)\) — локальный анализ окрестности $x_0$.
		      \(\int_a^b f(x)\,dx\) — «суммарный» (глобальный) взгляд на отрезок $[a,b]$.
	\end{itemize}
\end{customexample}


\newpage
\section{Формулы Маклорена для функций y=exp(x), y=sinx, y=cosx, y=ln(1+x), y=pow((1+x),a).}

\subsection*{Вспомогательные понятия}
% ==================
% auxiliary.tex
% (Вспомогательные определения для темы
% "Вывод рядов Тейлора для e^x, sin x, cos x и Формула Эйлера")
% ==================

\paragraph{Ряд Маклорена.}
Пусть функция $f$ имеет все производные в некоторой окрестности точки $0$. Тогда
\emph{рядом Маклорена} для $f$ называют
\[
	f(0) \;+\; \frac{f'(0)}{1!}\,x \;+\; \frac{f''(0)}{2!}\,x^2 \;+\; \dots \;+\;
	\frac{f^{(n)}(0)}{n!}\,x^n \;+\; \dots
\]
Если этот ряд сходится к $f(x)$ при соответствующих значениях $x$, то мы получаем разложение $f(x)$ в степенной ряд около $0$.

\bigskip

\paragraph{Остаточный член в форме Лагранжа.}
В случае, когда у функции $f$ есть $(n+1)$-я производная в окрестности $0$, можно записать
\[
	R_n(x)
	\;=\;
	\frac{f^{(n+1)}(\xi)}{(n+1)!}\,x^{n+1},
\]
где $\xi$ — некоторая точка между $0$ и $x$. Это называют \emph{остаточным членом} (или недостающим звеном) в формуле Тейлора (Маклорена).

\bigskip

\paragraph{Идея применения Теоремы Лагранжа.}
Для доказательства формулы Тейлора с остатком в форме Лагранжа часто используют теорему Лагранжа о среднем значении для производных:
если $P_n(x)$ — многочлен Тейлора степени $n$, то на отрезке $[0,x]$ применяют теорему Лагранжа к функции $f(x)-P_n(x)$, чтобы получить вид «\,$f^{(n+1)}(\xi)\,x^{n+1}/(n+1)!$».

\bigskip

\paragraph{Комплексная экспонента.}
Функция $e^{ix}$, где $i$ — мнимая единица, можно рассматривать как обобщение экспоненты на комплексную плоскость:
\[
	e^{ix} \;=\; \sum_{n=0}^{\infty} \frac{(i x)^n}{n!}.
\]
Из этого получается \emph{Формула Эйлера} при разбиении на действительную и мнимую часть.


\subsection*{Ответ на вопрос}
% ==================
% answer.tex
% (Основной материал:
%  "Определение интеграла Римана. Отличие от обычного предела.")
% ==================

\begin{customtheorem}[Интеграл Римана]
	Пусть функция $f$ задана на $[a,b]$. Если при всех возможных способах разбиения $[a,b]$ на малые отрезки,
	и выборе точек $\xi_i$ внутри этих отрезков, интегральные суммы
	\[
		S = \sum_{i=1}^n f(\xi_i)\,\Delta x_i
	\]
	стремятся к одному и тому же числу $I$ по мере $\max_i \Delta x_i \to 0$,
	то $f$ \textbf{интегрируема по Риману}, а $I$ называется \emph{интегралом Римана}:
	\[
		I = \int_{a}^{b} f(x)\,dx.
	\]
\end{customtheorem}

\begin{proofplan}
	\begin{enumerate}
		\item Рассмотреть любые два разбиения $D$ и $D'$ на отрезке $[a,b]$ с мелкостью $\|D\|\to0$, $\|D'\|\to0$.
		\item Построить общее уточнённое разбиение $D''$, включающее все точки из $D$ и $D'$.
		\item Оценить разницу сумм $S(f,D)$ и $S(f,D')$ через равномерную непрерывность (или ограниченность) $f$ на $[a,b]$.
		\item Показать, что эта разница становится сколь угодно малой при $\|D\|\to0$ и $\|D'\|\to0$.
		\item Вывод: предел един, определение интеграла однозначно.
	\end{enumerate}
\end{proofplan}

\begin{customproof}
	\textbf{Шаг 1: Два разбиения.}
	Пусть $D=\{a=x_0<x_1<\dots<x_n=b\}$ и $D'=\{a=y_0<y_1<\dots<y_m=b\}$ — любые разбиения отрезка $[a,b]$.
	Предположим, что $\|D\|=\max_i(x_i-x_{i-1})$ и $\|D'\|=\max_j(y_j-y_{j-1})$ оба стремятся к нулю.

	\smallskip

	\textbf{Шаг 2: Уточнение.}
	Построим «общее» разбиение $D''$, содержащее все точки из $D$ и $D'$.
	То есть объединим набор $\{x_i\}$ с $\{y_j\}$ в одну возрастающую последовательность.
	Теперь можно рассмотреть интегральные суммы относительно $D''$.

	\smallskip

	\textbf{Шаг 3: Оценка разницы.}
	На каждом элементе разбиения $[z_{k-1}, z_k]$ из $D''$ значения $f(\xi_i)$ меняются незначительно, если $f$ равномерно непрерывна (или ограничена).
	Тогда можно показать, что разность сумм $S(f,D)$ и $S(f,D')$ не превосходит
	некоторой малой величины, зависящей от $\|D''\|$, которая стремится к нулю,
	когда и $\|D\|\to0$, $\|D'\|\to0$.

	\smallskip

	\textbf{Шаг 4: Вывод.}
	Таким образом, любая интегральная сумма при мелкости разбиения стремится к одной и той же границе.
	Значит определение интеграла Римана корректно, и этот предел называется $\int_a^b f(x)\,dx$.
\end{customproof}

\begin{customexample}
	\begin{itemize}
		\item Если $f(x) \equiv C$ — константа, любая интегральная сумма $=C\cdot(b-a)$.
		      При любом разбиении ответ один: $\int_a^b C\,dx = C\,(b-a)$.
		\item Сравнение с обычным пределом:
		      \(\lim_{x\to x_0}f(x)\) — локальный анализ окрестности $x_0$.
		      \(\int_a^b f(x)\,dx\) — «суммарный» (глобальный) взгляд на отрезок $[a,b]$.
	\end{itemize}
\end{customexample}


\newpage
\section{Формула Тейлора с остатком в форме Лагранжа. Приближенные вычисления по формуле Тейлора.}

\subsection*{Вспомогательные понятия}
% ==================
% auxiliary.tex
% (Вспомогательные определения для темы
% "Вывод рядов Тейлора для e^x, sin x, cos x и Формула Эйлера")
% ==================

\paragraph{Ряд Маклорена.}
Пусть функция $f$ имеет все производные в некоторой окрестности точки $0$. Тогда
\emph{рядом Маклорена} для $f$ называют
\[
	f(0) \;+\; \frac{f'(0)}{1!}\,x \;+\; \frac{f''(0)}{2!}\,x^2 \;+\; \dots \;+\;
	\frac{f^{(n)}(0)}{n!}\,x^n \;+\; \dots
\]
Если этот ряд сходится к $f(x)$ при соответствующих значениях $x$, то мы получаем разложение $f(x)$ в степенной ряд около $0$.

\bigskip

\paragraph{Остаточный член в форме Лагранжа.}
В случае, когда у функции $f$ есть $(n+1)$-я производная в окрестности $0$, можно записать
\[
	R_n(x)
	\;=\;
	\frac{f^{(n+1)}(\xi)}{(n+1)!}\,x^{n+1},
\]
где $\xi$ — некоторая точка между $0$ и $x$. Это называют \emph{остаточным членом} (или недостающим звеном) в формуле Тейлора (Маклорена).

\bigskip

\paragraph{Идея применения Теоремы Лагранжа.}
Для доказательства формулы Тейлора с остатком в форме Лагранжа часто используют теорему Лагранжа о среднем значении для производных:
если $P_n(x)$ — многочлен Тейлора степени $n$, то на отрезке $[0,x]$ применяют теорему Лагранжа к функции $f(x)-P_n(x)$, чтобы получить вид «\,$f^{(n+1)}(\xi)\,x^{n+1}/(n+1)!$».

\bigskip

\paragraph{Комплексная экспонента.}
Функция $e^{ix}$, где $i$ — мнимая единица, можно рассматривать как обобщение экспоненты на комплексную плоскость:
\[
	e^{ix} \;=\; \sum_{n=0}^{\infty} \frac{(i x)^n}{n!}.
\]
Из этого получается \emph{Формула Эйлера} при разбиении на действительную и мнимую часть.


\subsection*{Ответ на вопрос}
% ==================
% answer.tex
% (Основной материал:
%  "Определение интеграла Римана. Отличие от обычного предела.")
% ==================

\begin{customtheorem}[Интеграл Римана]
	Пусть функция $f$ задана на $[a,b]$. Если при всех возможных способах разбиения $[a,b]$ на малые отрезки,
	и выборе точек $\xi_i$ внутри этих отрезков, интегральные суммы
	\[
		S = \sum_{i=1}^n f(\xi_i)\,\Delta x_i
	\]
	стремятся к одному и тому же числу $I$ по мере $\max_i \Delta x_i \to 0$,
	то $f$ \textbf{интегрируема по Риману}, а $I$ называется \emph{интегралом Римана}:
	\[
		I = \int_{a}^{b} f(x)\,dx.
	\]
\end{customtheorem}

\begin{proofplan}
	\begin{enumerate}
		\item Рассмотреть любые два разбиения $D$ и $D'$ на отрезке $[a,b]$ с мелкостью $\|D\|\to0$, $\|D'\|\to0$.
		\item Построить общее уточнённое разбиение $D''$, включающее все точки из $D$ и $D'$.
		\item Оценить разницу сумм $S(f,D)$ и $S(f,D')$ через равномерную непрерывность (или ограниченность) $f$ на $[a,b]$.
		\item Показать, что эта разница становится сколь угодно малой при $\|D\|\to0$ и $\|D'\|\to0$.
		\item Вывод: предел един, определение интеграла однозначно.
	\end{enumerate}
\end{proofplan}

\begin{customproof}
	\textbf{Шаг 1: Два разбиения.}
	Пусть $D=\{a=x_0<x_1<\dots<x_n=b\}$ и $D'=\{a=y_0<y_1<\dots<y_m=b\}$ — любые разбиения отрезка $[a,b]$.
	Предположим, что $\|D\|=\max_i(x_i-x_{i-1})$ и $\|D'\|=\max_j(y_j-y_{j-1})$ оба стремятся к нулю.

	\smallskip

	\textbf{Шаг 2: Уточнение.}
	Построим «общее» разбиение $D''$, содержащее все точки из $D$ и $D'$.
	То есть объединим набор $\{x_i\}$ с $\{y_j\}$ в одну возрастающую последовательность.
	Теперь можно рассмотреть интегральные суммы относительно $D''$.

	\smallskip

	\textbf{Шаг 3: Оценка разницы.}
	На каждом элементе разбиения $[z_{k-1}, z_k]$ из $D''$ значения $f(\xi_i)$ меняются незначительно, если $f$ равномерно непрерывна (или ограничена).
	Тогда можно показать, что разность сумм $S(f,D)$ и $S(f,D')$ не превосходит
	некоторой малой величины, зависящей от $\|D''\|$, которая стремится к нулю,
	когда и $\|D\|\to0$, $\|D'\|\to0$.

	\smallskip

	\textbf{Шаг 4: Вывод.}
	Таким образом, любая интегральная сумма при мелкости разбиения стремится к одной и той же границе.
	Значит определение интеграла Римана корректно, и этот предел называется $\int_a^b f(x)\,dx$.
\end{customproof}

\begin{customexample}
	\begin{itemize}
		\item Если $f(x) \equiv C$ — константа, любая интегральная сумма $=C\cdot(b-a)$.
		      При любом разбиении ответ один: $\int_a^b C\,dx = C\,(b-a)$.
		\item Сравнение с обычным пределом:
		      \(\lim_{x\to x_0}f(x)\) — локальный анализ окрестности $x_0$.
		      \(\int_a^b f(x)\,dx\) — «суммарный» (глобальный) взгляд на отрезок $[a,b]$.
	\end{itemize}
\end{customexample}


\newpage
\section{Формула Стирлинга (с эквивалентностью).}

\subsection*{Вспомогательные понятия}
% ==================
% auxiliary.tex
% (Вспомогательные определения для темы
% "Вывод рядов Тейлора для e^x, sin x, cos x и Формула Эйлера")
% ==================

\paragraph{Ряд Маклорена.}
Пусть функция $f$ имеет все производные в некоторой окрестности точки $0$. Тогда
\emph{рядом Маклорена} для $f$ называют
\[
	f(0) \;+\; \frac{f'(0)}{1!}\,x \;+\; \frac{f''(0)}{2!}\,x^2 \;+\; \dots \;+\;
	\frac{f^{(n)}(0)}{n!}\,x^n \;+\; \dots
\]
Если этот ряд сходится к $f(x)$ при соответствующих значениях $x$, то мы получаем разложение $f(x)$ в степенной ряд около $0$.

\bigskip

\paragraph{Остаточный член в форме Лагранжа.}
В случае, когда у функции $f$ есть $(n+1)$-я производная в окрестности $0$, можно записать
\[
	R_n(x)
	\;=\;
	\frac{f^{(n+1)}(\xi)}{(n+1)!}\,x^{n+1},
\]
где $\xi$ — некоторая точка между $0$ и $x$. Это называют \emph{остаточным членом} (или недостающим звеном) в формуле Тейлора (Маклорена).

\bigskip

\paragraph{Идея применения Теоремы Лагранжа.}
Для доказательства формулы Тейлора с остатком в форме Лагранжа часто используют теорему Лагранжа о среднем значении для производных:
если $P_n(x)$ — многочлен Тейлора степени $n$, то на отрезке $[0,x]$ применяют теорему Лагранжа к функции $f(x)-P_n(x)$, чтобы получить вид «\,$f^{(n+1)}(\xi)\,x^{n+1}/(n+1)!$».

\bigskip

\paragraph{Комплексная экспонента.}
Функция $e^{ix}$, где $i$ — мнимая единица, можно рассматривать как обобщение экспоненты на комплексную плоскость:
\[
	e^{ix} \;=\; \sum_{n=0}^{\infty} \frac{(i x)^n}{n!}.
\]
Из этого получается \emph{Формула Эйлера} при разбиении на действительную и мнимую часть.


\subsection*{Ответ на вопрос}
% ==================
% answer.tex
% (Основной материал:
%  "Определение интеграла Римана. Отличие от обычного предела.")
% ==================

\begin{customtheorem}[Интеграл Римана]
	Пусть функция $f$ задана на $[a,b]$. Если при всех возможных способах разбиения $[a,b]$ на малые отрезки,
	и выборе точек $\xi_i$ внутри этих отрезков, интегральные суммы
	\[
		S = \sum_{i=1}^n f(\xi_i)\,\Delta x_i
	\]
	стремятся к одному и тому же числу $I$ по мере $\max_i \Delta x_i \to 0$,
	то $f$ \textbf{интегрируема по Риману}, а $I$ называется \emph{интегралом Римана}:
	\[
		I = \int_{a}^{b} f(x)\,dx.
	\]
\end{customtheorem}

\begin{proofplan}
	\begin{enumerate}
		\item Рассмотреть любые два разбиения $D$ и $D'$ на отрезке $[a,b]$ с мелкостью $\|D\|\to0$, $\|D'\|\to0$.
		\item Построить общее уточнённое разбиение $D''$, включающее все точки из $D$ и $D'$.
		\item Оценить разницу сумм $S(f,D)$ и $S(f,D')$ через равномерную непрерывность (или ограниченность) $f$ на $[a,b]$.
		\item Показать, что эта разница становится сколь угодно малой при $\|D\|\to0$ и $\|D'\|\to0$.
		\item Вывод: предел един, определение интеграла однозначно.
	\end{enumerate}
\end{proofplan}

\begin{customproof}
	\textbf{Шаг 1: Два разбиения.}
	Пусть $D=\{a=x_0<x_1<\dots<x_n=b\}$ и $D'=\{a=y_0<y_1<\dots<y_m=b\}$ — любые разбиения отрезка $[a,b]$.
	Предположим, что $\|D\|=\max_i(x_i-x_{i-1})$ и $\|D'\|=\max_j(y_j-y_{j-1})$ оба стремятся к нулю.

	\smallskip

	\textbf{Шаг 2: Уточнение.}
	Построим «общее» разбиение $D''$, содержащее все точки из $D$ и $D'$.
	То есть объединим набор $\{x_i\}$ с $\{y_j\}$ в одну возрастающую последовательность.
	Теперь можно рассмотреть интегральные суммы относительно $D''$.

	\smallskip

	\textbf{Шаг 3: Оценка разницы.}
	На каждом элементе разбиения $[z_{k-1}, z_k]$ из $D''$ значения $f(\xi_i)$ меняются незначительно, если $f$ равномерно непрерывна (или ограничена).
	Тогда можно показать, что разность сумм $S(f,D)$ и $S(f,D')$ не превосходит
	некоторой малой величины, зависящей от $\|D''\|$, которая стремится к нулю,
	когда и $\|D\|\to0$, $\|D'\|\to0$.

	\smallskip

	\textbf{Шаг 4: Вывод.}
	Таким образом, любая интегральная сумма при мелкости разбиения стремится к одной и той же границе.
	Значит определение интеграла Римана корректно, и этот предел называется $\int_a^b f(x)\,dx$.
\end{customproof}

\begin{customexample}
	\begin{itemize}
		\item Если $f(x) \equiv C$ — константа, любая интегральная сумма $=C\cdot(b-a)$.
		      При любом разбиении ответ один: $\int_a^b C\,dx = C\,(b-a)$.
		\item Сравнение с обычным пределом:
		      \(\lim_{x\to x_0}f(x)\) — локальный анализ окрестности $x_0$.
		      \(\int_a^b f(x)\,dx\) — «суммарный» (глобальный) взгляд на отрезок $[a,b]$.
	\end{itemize}
\end{customexample}


\newpage
\section{Формула Стирлинга (с равенством).}

\subsection*{Вспомогательные понятия}
% ==================
% auxiliary.tex
% (Вспомогательные определения для темы
% "Вывод рядов Тейлора для e^x, sin x, cos x и Формула Эйлера")
% ==================

\paragraph{Ряд Маклорена.}
Пусть функция $f$ имеет все производные в некоторой окрестности точки $0$. Тогда
\emph{рядом Маклорена} для $f$ называют
\[
	f(0) \;+\; \frac{f'(0)}{1!}\,x \;+\; \frac{f''(0)}{2!}\,x^2 \;+\; \dots \;+\;
	\frac{f^{(n)}(0)}{n!}\,x^n \;+\; \dots
\]
Если этот ряд сходится к $f(x)$ при соответствующих значениях $x$, то мы получаем разложение $f(x)$ в степенной ряд около $0$.

\bigskip

\paragraph{Остаточный член в форме Лагранжа.}
В случае, когда у функции $f$ есть $(n+1)$-я производная в окрестности $0$, можно записать
\[
	R_n(x)
	\;=\;
	\frac{f^{(n+1)}(\xi)}{(n+1)!}\,x^{n+1},
\]
где $\xi$ — некоторая точка между $0$ и $x$. Это называют \emph{остаточным членом} (или недостающим звеном) в формуле Тейлора (Маклорена).

\bigskip

\paragraph{Идея применения Теоремы Лагранжа.}
Для доказательства формулы Тейлора с остатком в форме Лагранжа часто используют теорему Лагранжа о среднем значении для производных:
если $P_n(x)$ — многочлен Тейлора степени $n$, то на отрезке $[0,x]$ применяют теорему Лагранжа к функции $f(x)-P_n(x)$, чтобы получить вид «\,$f^{(n+1)}(\xi)\,x^{n+1}/(n+1)!$».

\bigskip

\paragraph{Комплексная экспонента.}
Функция $e^{ix}$, где $i$ — мнимая единица, можно рассматривать как обобщение экспоненты на комплексную плоскость:
\[
	e^{ix} \;=\; \sum_{n=0}^{\infty} \frac{(i x)^n}{n!}.
\]
Из этого получается \emph{Формула Эйлера} при разбиении на действительную и мнимую часть.


\subsection*{Ответ на вопрос}
% ==================
% answer.tex
% (Основной материал:
%  "Определение интеграла Римана. Отличие от обычного предела.")
% ==================

\begin{customtheorem}[Интеграл Римана]
	Пусть функция $f$ задана на $[a,b]$. Если при всех возможных способах разбиения $[a,b]$ на малые отрезки,
	и выборе точек $\xi_i$ внутри этих отрезков, интегральные суммы
	\[
		S = \sum_{i=1}^n f(\xi_i)\,\Delta x_i
	\]
	стремятся к одному и тому же числу $I$ по мере $\max_i \Delta x_i \to 0$,
	то $f$ \textbf{интегрируема по Риману}, а $I$ называется \emph{интегралом Римана}:
	\[
		I = \int_{a}^{b} f(x)\,dx.
	\]
\end{customtheorem}

\begin{proofplan}
	\begin{enumerate}
		\item Рассмотреть любые два разбиения $D$ и $D'$ на отрезке $[a,b]$ с мелкостью $\|D\|\to0$, $\|D'\|\to0$.
		\item Построить общее уточнённое разбиение $D''$, включающее все точки из $D$ и $D'$.
		\item Оценить разницу сумм $S(f,D)$ и $S(f,D')$ через равномерную непрерывность (или ограниченность) $f$ на $[a,b]$.
		\item Показать, что эта разница становится сколь угодно малой при $\|D\|\to0$ и $\|D'\|\to0$.
		\item Вывод: предел един, определение интеграла однозначно.
	\end{enumerate}
\end{proofplan}

\begin{customproof}
	\textbf{Шаг 1: Два разбиения.}
	Пусть $D=\{a=x_0<x_1<\dots<x_n=b\}$ и $D'=\{a=y_0<y_1<\dots<y_m=b\}$ — любые разбиения отрезка $[a,b]$.
	Предположим, что $\|D\|=\max_i(x_i-x_{i-1})$ и $\|D'\|=\max_j(y_j-y_{j-1})$ оба стремятся к нулю.

	\smallskip

	\textbf{Шаг 2: Уточнение.}
	Построим «общее» разбиение $D''$, содержащее все точки из $D$ и $D'$.
	То есть объединим набор $\{x_i\}$ с $\{y_j\}$ в одну возрастающую последовательность.
	Теперь можно рассмотреть интегральные суммы относительно $D''$.

	\smallskip

	\textbf{Шаг 3: Оценка разницы.}
	На каждом элементе разбиения $[z_{k-1}, z_k]$ из $D''$ значения $f(\xi_i)$ меняются незначительно, если $f$ равномерно непрерывна (или ограничена).
	Тогда можно показать, что разность сумм $S(f,D)$ и $S(f,D')$ не превосходит
	некоторой малой величины, зависящей от $\|D''\|$, которая стремится к нулю,
	когда и $\|D\|\to0$, $\|D'\|\to0$.

	\smallskip

	\textbf{Шаг 4: Вывод.}
	Таким образом, любая интегральная сумма при мелкости разбиения стремится к одной и той же границе.
	Значит определение интеграла Римана корректно, и этот предел называется $\int_a^b f(x)\,dx$.
\end{customproof}

\begin{customexample}
	\begin{itemize}
		\item Если $f(x) \equiv C$ — константа, любая интегральная сумма $=C\cdot(b-a)$.
		      При любом разбиении ответ один: $\int_a^b C\,dx = C\,(b-a)$.
		\item Сравнение с обычным пределом:
		      \(\lim_{x\to x_0}f(x)\) — локальный анализ окрестности $x_0$.
		      \(\int_a^b f(x)\,dx\) — «суммарный» (глобальный) взгляд на отрезок $[a,b]$.
	\end{itemize}
\end{customexample}


\newpage
\section{Определение интеграла Римана. Отличие от «обычного» предела.}

\subsection*{Вспомогательные понятия}
% ==================
% auxiliary.tex
% (Вспомогательные определения для темы
% "Вывод рядов Тейлора для e^x, sin x, cos x и Формула Эйлера")
% ==================

\paragraph{Ряд Маклорена.}
Пусть функция $f$ имеет все производные в некоторой окрестности точки $0$. Тогда
\emph{рядом Маклорена} для $f$ называют
\[
	f(0) \;+\; \frac{f'(0)}{1!}\,x \;+\; \frac{f''(0)}{2!}\,x^2 \;+\; \dots \;+\;
	\frac{f^{(n)}(0)}{n!}\,x^n \;+\; \dots
\]
Если этот ряд сходится к $f(x)$ при соответствующих значениях $x$, то мы получаем разложение $f(x)$ в степенной ряд около $0$.

\bigskip

\paragraph{Остаточный член в форме Лагранжа.}
В случае, когда у функции $f$ есть $(n+1)$-я производная в окрестности $0$, можно записать
\[
	R_n(x)
	\;=\;
	\frac{f^{(n+1)}(\xi)}{(n+1)!}\,x^{n+1},
\]
где $\xi$ — некоторая точка между $0$ и $x$. Это называют \emph{остаточным членом} (или недостающим звеном) в формуле Тейлора (Маклорена).

\bigskip

\paragraph{Идея применения Теоремы Лагранжа.}
Для доказательства формулы Тейлора с остатком в форме Лагранжа часто используют теорему Лагранжа о среднем значении для производных:
если $P_n(x)$ — многочлен Тейлора степени $n$, то на отрезке $[0,x]$ применяют теорему Лагранжа к функции $f(x)-P_n(x)$, чтобы получить вид «\,$f^{(n+1)}(\xi)\,x^{n+1}/(n+1)!$».

\bigskip

\paragraph{Комплексная экспонента.}
Функция $e^{ix}$, где $i$ — мнимая единица, можно рассматривать как обобщение экспоненты на комплексную плоскость:
\[
	e^{ix} \;=\; \sum_{n=0}^{\infty} \frac{(i x)^n}{n!}.
\]
Из этого получается \emph{Формула Эйлера} при разбиении на действительную и мнимую часть.


\subsection*{Ответ на вопрос}
% ==================
% answer.tex
% (Основной материал:
%  "Определение интеграла Римана. Отличие от обычного предела.")
% ==================

\begin{customtheorem}[Интеграл Римана]
	Пусть функция $f$ задана на $[a,b]$. Если при всех возможных способах разбиения $[a,b]$ на малые отрезки,
	и выборе точек $\xi_i$ внутри этих отрезков, интегральные суммы
	\[
		S = \sum_{i=1}^n f(\xi_i)\,\Delta x_i
	\]
	стремятся к одному и тому же числу $I$ по мере $\max_i \Delta x_i \to 0$,
	то $f$ \textbf{интегрируема по Риману}, а $I$ называется \emph{интегралом Римана}:
	\[
		I = \int_{a}^{b} f(x)\,dx.
	\]
\end{customtheorem}

\begin{proofplan}
	\begin{enumerate}
		\item Рассмотреть любые два разбиения $D$ и $D'$ на отрезке $[a,b]$ с мелкостью $\|D\|\to0$, $\|D'\|\to0$.
		\item Построить общее уточнённое разбиение $D''$, включающее все точки из $D$ и $D'$.
		\item Оценить разницу сумм $S(f,D)$ и $S(f,D')$ через равномерную непрерывность (или ограниченность) $f$ на $[a,b]$.
		\item Показать, что эта разница становится сколь угодно малой при $\|D\|\to0$ и $\|D'\|\to0$.
		\item Вывод: предел един, определение интеграла однозначно.
	\end{enumerate}
\end{proofplan}

\begin{customproof}
	\textbf{Шаг 1: Два разбиения.}
	Пусть $D=\{a=x_0<x_1<\dots<x_n=b\}$ и $D'=\{a=y_0<y_1<\dots<y_m=b\}$ — любые разбиения отрезка $[a,b]$.
	Предположим, что $\|D\|=\max_i(x_i-x_{i-1})$ и $\|D'\|=\max_j(y_j-y_{j-1})$ оба стремятся к нулю.

	\smallskip

	\textbf{Шаг 2: Уточнение.}
	Построим «общее» разбиение $D''$, содержащее все точки из $D$ и $D'$.
	То есть объединим набор $\{x_i\}$ с $\{y_j\}$ в одну возрастающую последовательность.
	Теперь можно рассмотреть интегральные суммы относительно $D''$.

	\smallskip

	\textbf{Шаг 3: Оценка разницы.}
	На каждом элементе разбиения $[z_{k-1}, z_k]$ из $D''$ значения $f(\xi_i)$ меняются незначительно, если $f$ равномерно непрерывна (или ограничена).
	Тогда можно показать, что разность сумм $S(f,D)$ и $S(f,D')$ не превосходит
	некоторой малой величины, зависящей от $\|D''\|$, которая стремится к нулю,
	когда и $\|D\|\to0$, $\|D'\|\to0$.

	\smallskip

	\textbf{Шаг 4: Вывод.}
	Таким образом, любая интегральная сумма при мелкости разбиения стремится к одной и той же границе.
	Значит определение интеграла Римана корректно, и этот предел называется $\int_a^b f(x)\,dx$.
\end{customproof}

\begin{customexample}
	\begin{itemize}
		\item Если $f(x) \equiv C$ — константа, любая интегральная сумма $=C\cdot(b-a)$.
		      При любом разбиении ответ один: $\int_a^b C\,dx = C\,(b-a)$.
		\item Сравнение с обычным пределом:
		      \(\lim_{x\to x_0}f(x)\) — локальный анализ окрестности $x_0$.
		      \(\int_a^b f(x)\,dx\) — «суммарный» (глобальный) взгляд на отрезок $[a,b]$.
	\end{itemize}
\end{customexample}


\newpage
\section{Формула Ньютона-Лейбница.}

\subsection*{Вспомогательные понятия}
% ==================
% auxiliary.tex
% (Вспомогательные определения для темы
% "Вывод рядов Тейлора для e^x, sin x, cos x и Формула Эйлера")
% ==================

\paragraph{Ряд Маклорена.}
Пусть функция $f$ имеет все производные в некоторой окрестности точки $0$. Тогда
\emph{рядом Маклорена} для $f$ называют
\[
	f(0) \;+\; \frac{f'(0)}{1!}\,x \;+\; \frac{f''(0)}{2!}\,x^2 \;+\; \dots \;+\;
	\frac{f^{(n)}(0)}{n!}\,x^n \;+\; \dots
\]
Если этот ряд сходится к $f(x)$ при соответствующих значениях $x$, то мы получаем разложение $f(x)$ в степенной ряд около $0$.

\bigskip

\paragraph{Остаточный член в форме Лагранжа.}
В случае, когда у функции $f$ есть $(n+1)$-я производная в окрестности $0$, можно записать
\[
	R_n(x)
	\;=\;
	\frac{f^{(n+1)}(\xi)}{(n+1)!}\,x^{n+1},
\]
где $\xi$ — некоторая точка между $0$ и $x$. Это называют \emph{остаточным членом} (или недостающим звеном) в формуле Тейлора (Маклорена).

\bigskip

\paragraph{Идея применения Теоремы Лагранжа.}
Для доказательства формулы Тейлора с остатком в форме Лагранжа часто используют теорему Лагранжа о среднем значении для производных:
если $P_n(x)$ — многочлен Тейлора степени $n$, то на отрезке $[0,x]$ применяют теорему Лагранжа к функции $f(x)-P_n(x)$, чтобы получить вид «\,$f^{(n+1)}(\xi)\,x^{n+1}/(n+1)!$».

\bigskip

\paragraph{Комплексная экспонента.}
Функция $e^{ix}$, где $i$ — мнимая единица, можно рассматривать как обобщение экспоненты на комплексную плоскость:
\[
	e^{ix} \;=\; \sum_{n=0}^{\infty} \frac{(i x)^n}{n!}.
\]
Из этого получается \emph{Формула Эйлера} при разбиении на действительную и мнимую часть.


\subsection*{Ответ на вопрос}
% ==================
% answer.tex
% (Основной материал:
%  "Определение интеграла Римана. Отличие от обычного предела.")
% ==================

\begin{customtheorem}[Интеграл Римана]
	Пусть функция $f$ задана на $[a,b]$. Если при всех возможных способах разбиения $[a,b]$ на малые отрезки,
	и выборе точек $\xi_i$ внутри этих отрезков, интегральные суммы
	\[
		S = \sum_{i=1}^n f(\xi_i)\,\Delta x_i
	\]
	стремятся к одному и тому же числу $I$ по мере $\max_i \Delta x_i \to 0$,
	то $f$ \textbf{интегрируема по Риману}, а $I$ называется \emph{интегралом Римана}:
	\[
		I = \int_{a}^{b} f(x)\,dx.
	\]
\end{customtheorem}

\begin{proofplan}
	\begin{enumerate}
		\item Рассмотреть любые два разбиения $D$ и $D'$ на отрезке $[a,b]$ с мелкостью $\|D\|\to0$, $\|D'\|\to0$.
		\item Построить общее уточнённое разбиение $D''$, включающее все точки из $D$ и $D'$.
		\item Оценить разницу сумм $S(f,D)$ и $S(f,D')$ через равномерную непрерывность (или ограниченность) $f$ на $[a,b]$.
		\item Показать, что эта разница становится сколь угодно малой при $\|D\|\to0$ и $\|D'\|\to0$.
		\item Вывод: предел един, определение интеграла однозначно.
	\end{enumerate}
\end{proofplan}

\begin{customproof}
	\textbf{Шаг 1: Два разбиения.}
	Пусть $D=\{a=x_0<x_1<\dots<x_n=b\}$ и $D'=\{a=y_0<y_1<\dots<y_m=b\}$ — любые разбиения отрезка $[a,b]$.
	Предположим, что $\|D\|=\max_i(x_i-x_{i-1})$ и $\|D'\|=\max_j(y_j-y_{j-1})$ оба стремятся к нулю.

	\smallskip

	\textbf{Шаг 2: Уточнение.}
	Построим «общее» разбиение $D''$, содержащее все точки из $D$ и $D'$.
	То есть объединим набор $\{x_i\}$ с $\{y_j\}$ в одну возрастающую последовательность.
	Теперь можно рассмотреть интегральные суммы относительно $D''$.

	\smallskip

	\textbf{Шаг 3: Оценка разницы.}
	На каждом элементе разбиения $[z_{k-1}, z_k]$ из $D''$ значения $f(\xi_i)$ меняются незначительно, если $f$ равномерно непрерывна (или ограничена).
	Тогда можно показать, что разность сумм $S(f,D)$ и $S(f,D')$ не превосходит
	некоторой малой величины, зависящей от $\|D''\|$, которая стремится к нулю,
	когда и $\|D\|\to0$, $\|D'\|\to0$.

	\smallskip

	\textbf{Шаг 4: Вывод.}
	Таким образом, любая интегральная сумма при мелкости разбиения стремится к одной и той же границе.
	Значит определение интеграла Римана корректно, и этот предел называется $\int_a^b f(x)\,dx$.
\end{customproof}

\begin{customexample}
	\begin{itemize}
		\item Если $f(x) \equiv C$ — константа, любая интегральная сумма $=C\cdot(b-a)$.
		      При любом разбиении ответ один: $\int_a^b C\,dx = C\,(b-a)$.
		\item Сравнение с обычным пределом:
		      \(\lim_{x\to x_0}f(x)\) — локальный анализ окрестности $x_0$.
		      \(\int_a^b f(x)\,dx\) — «суммарный» (глобальный) взгляд на отрезок $[a,b]$.
	\end{itemize}
\end{customexample}


\end{document}
