\documentclass[12pt]{article}
\usepackage[T2A]{fontenc}        % поддержка кириллицы
\usepackage[utf8]{inputenc}      % кодировка utf-8
\usepackage[russian]{babel}      % включаем русский язык
\usepackage{amsmath,amssymb}
\usepackage{geometry}
\geometry{left=2cm,right=2cm,top=2cm,bottom=2cm}
\usepackage{graphicx}            % вставка изображений
\usepackage{hyperref}
\usepackage{indentfirst}         % отступ первого абзаца в разделе 

\title{Решение индивидуального домашнего задания №2}
\author{Студент: Николаев Всеволод Юрьевич \\ 
Группа: 4395\\
Вариант: 14
}
\date{}

\begin{document}

\maketitle

\subsection*{Условие}
Пусть $e$ - стандартный базис в $\mathbb{R}^3$, $f_1 = (0, 1, -3)^T$, $f_2 = (1, 4, -6)^T$, $f_3 = (-1, -3, 4)^T$. Даны линейные операторы $A$ и $B$, имеющие в базисе $e$ следующие матрицы: $A_e = \begin{pmatrix} 8 & -9 & 2 \\ 10 & -11 & 2 \\ 18 & -21 & 4 \end{pmatrix}$ $B_e = \begin{pmatrix} 2 & 1 & -1 \\ 2 & 1 & -1 \\ -10 & 10 & 2 \end{pmatrix}$; и вектор $x$ с координатами $x_e = (-4, -4, -7)^T$.

\subsection*{Задача 1. Проверка базиса и матрицы перехода}
Проверьте, что $ f = (f_1, f_2, f_3) $ — базис в $ \mathbb{R}^3 $ и найдите матрицы перехода $ C_{e \rightarrow f} $ и $ C_{f \rightarrow e} $.

$$
	f =
	\begin{pmatrix}
		0  & 1  & -1 \\
		1  & 4  & -3 \\
		-3 & -6 & 4  \\
	\end{pmatrix}
	\begin{matrix} \\  \\ +3II \\ \end{matrix} \sim
	\begin{pmatrix}
		0 & 1 & -1 \\
		1 & 4 & -3 \\
		0 & 6 & -5 \\
	\end{pmatrix}
	\begin{matrix} \\  \\ -6I \\ \end{matrix} \sim
	\begin{pmatrix}
		0 & 1 & -1 \\
		1 & 4 & -3 \\
		0 & 0 & 1  \\
	\end{pmatrix} \sim
	\begin{pmatrix}
		0 & 1 & 0 \\
		1 & 0 & 0 \\
		0 & 0 & 1 \\
	\end{pmatrix}
$$

Вектора $(0, 1, 0)^T$, $(1, 0, 0)^T$ и $(0, 0, 1)^T$ являются линейно независимыми, а значит $f$ - базис в $\mathbb{R}^3$.

$$
	C_{e \rightarrow f} = f =
	\begin{pmatrix}
		0  & 1  & -1 \\
		1  & 4  & -3 \\
		-3 & -6 & 4  \\
	\end{pmatrix}\text{; }
	C_{f \rightarrow e} = f^{-1} =
	\begin{pmatrix}
		0  & 1  & -1 \\
		1  & 4  & -3 \\
		-3 & -6 & 4  \\
	\end{pmatrix}^{-1}\text{:}
$$

$$
	\left[
		\begin{array}{ccc|ccc}
			0  & 1  & -1 & 1 & 0 & 0 \\
			1  & 4  & -3 & 0 & 1 & 0 \\
			-3 & -6 & 4  & 0 & 0 & 1 \\
		\end{array}
		\right]
	\begin{matrix} \\  \\ +3II \end{matrix} \sim
	\left[
		\begin{array}{ccc|ccc}
			0 & 1 & -1 & 1 & 0 & 0 \\
			1 & 4 & -3 & 0 & 1 & 0 \\
			0 & 6 & -5 & 0 & 3 & 1 \\
		\end{array}
		\right]
	\begin{matrix} \\  \\ -6I \end{matrix} \sim
$$

$$
	\sim
	\left[
		\begin{array}{ccc|ccc}
			0 & 1 & -1 & 1  & 0 & 0 \\
			1 & 4 & -3 & 0  & 1 & 0 \\
			0 & 0 & 1  & -6 & 3 & 1 \\
		\end{array}
		\right]
	\begin{matrix} +1III \\ +3III \\\\\end{matrix} \sim
	\left[
		\begin{array}{ccc|ccc}
			0 & 1 & 0 & -5  & 3  & 1 \\
			1 & 4 & 0 & -18 & 10 & 3 \\
			0 & 0 & 1 & -6  & 3  & 1 \\
		\end{array}
		\right]
	\begin{matrix} \\ -4I \\\\\end{matrix} \sim
$$
$$
	\sim
	\left[
		\begin{array}{ccc|ccc}
			0 & 1 & 0 & -5 & 3  & 1  \\
			1 & 0 & 0 & 2  & -2 & -1 \\
			0 & 0 & 1 & -6 & 3  & 1  \\
		\end{array}
		\right]
	\sim
	\left[
		\begin{array}{ccc|ccc}
			1 & 0 & 0 & 2  & -2 & -1 \\
			0 & 1 & 0 & -5 & 3  & 1  \\
			0 & 0 & 1 & -6 & 3  & 1  \\
		\end{array}
		\right]
$$

\textbf{Ответ:} $C_{e \rightarrow f} = \begin{pmatrix} 0  & 1  & -1 \\ 1  & 4  & -3 \\ -3 & -6 & 4  \\ \end{pmatrix}$, $C_{f \rightarrow e} = \begin{pmatrix} 2 & -2 & -1 \\ -5 & 3 & 1 \\ -6 & 3 & 1 \\ \end{pmatrix}$

\subsection*{Задача 2. Координаты образа вектора в новом базисе}
Определить координаты вектора $ y = A \circ B(x) $ в базисе $ f $.

$$
	y_e = A \circ B(x) = A_e \times B_e \times x_e
$$
$$
	y_f = C_{f \rightarrow e} \times y_e = C_{f \rightarrow e} \times A_e \times B_e \times x_e
$$

$$
	A_e =
	\begin{pmatrix}
		8  & -9  & 2 \\
		10 & -11 & 2 \\
		18 & -21 & 4 \\
	\end{pmatrix}\text{; }
	B_e =
	\begin{pmatrix}
		2   & 1  & -1 \\
		2   & 1  & -1 \\
		-10 & 10 & 2  \\
	\end{pmatrix}\text{; }
	x_e =
	\begin{pmatrix}
		-4 \\ -4 \\ -7 \\
	\end{pmatrix}\text{; }
	C_{f \rightarrow e} =
	\begin{pmatrix}
		2  & -2 & -1 \\
		-5 & 3  & 1  \\
		-6 & 3  & 1  \\
	\end{pmatrix}
$$

$$
	y_f =
	\begin{pmatrix}
		2  & -2 & -1 \\
		-5 & 3  & 1  \\
		-6 & 3  & 1  \\
	\end{pmatrix}
	\times
	\begin{pmatrix}
		8  & -9  & 2 \\
		10 & -11 & 2 \\
		18 & -21 & 4 \\
	\end{pmatrix}
	\times
	\begin{pmatrix}
		2   & 1  & -1 \\
		2   & 1  & -1 \\
		-10 & 10 & 2
	\end{pmatrix}
	\times
	\begin{pmatrix}
		-4 \\ -4 \\ -7 \\
	\end{pmatrix}
	=
$$
$$
	=
	\begin{pmatrix}
		2  & -2 & -1 \\
		-5 & 3  & 1  \\
		-6 & 3  & 1  \\
	\end{pmatrix}
	\times
	\begin{pmatrix}
		8  & -9  & 2 \\
		10 & -11 & 2 \\
		18 & -21 & 4
	\end{pmatrix}
	\times
	\begin{pmatrix}
		-5 \\ -5 \\ -14 \\
	\end{pmatrix}
	=
$$
$$
	=
	\begin{pmatrix}
		2  & -2 & -1 \\
		-5 & 3  & 1  \\
		-6 & 3  & 1  \\
	\end{pmatrix}
	\times
	\begin{pmatrix}
		-23 \\ -4 \\ -41 \\
	\end{pmatrix}
	=
	\begin{pmatrix}
		41 \\ 5 \\ 28 \\
	\end{pmatrix}
$$

\textbf{Ответ:} $y_f = (41, 5, 28)^T$

\subsection*{Задача 3. Обратимость операторов}
Будет ли каждый из операторов $ A $ и $ B $ обратимым?

Матрица обратима тогда и только тогда, когда её определитель не равен 0.

$$
	\det(A) =
	\left|
	\begin{matrix}
		8  & -9  & 2 \\
		10 & -11 & 2 \\
		18 & -21 & 4 \\
	\end{matrix}
	\right|
	= -8 * 11 * 4
	- 9 * 2 * 18
	- 2 * 10 * 21
	+ 2 * 11 * 18
	+ 8 * 2 * 21
	+ 9 * 10 * 4
	=
$$
$$
	= - 352 - 324 - 420 + 396 + 336 + 360 = -4 \neq 0
$$

$\det(B) = 0$ т.к. в матрице $B$ есть две повторяющиеся строки.

\textbf{Ответ:} оператор матрицы $A$ обратим, оператор матрицы $B$ не обратим.

\subsection*{Задача 4. Матрица обратного оператора}
Найдите матрицы оператора $ A^{-1} $ в базисах $ e $ и $ f $.

$$
	A_e =
	\begin{pmatrix}
		8  & -9  & 2 \\
		10 & -11 & 2 \\
		18 & -21 & 4 \\
	\end{pmatrix}\text{; }
	A_f^{-1} = C_{f \rightarrow e} \times A_e^{-1}\text{; }
	C_{f \rightarrow e} =
	\begin{pmatrix}
		2  & -2 & -1 \\
		-5 & 3  & 1  \\
		-6 & 3  & 1  \\
	\end{pmatrix}\text{; }
	A_e^{-1}\text{:}
$$
$$
	\left[
		\begin{array}{ccc|ccc}
			8  & -9  & 2 & 1 & 0 & 0 \\
			10 & -11 & 2 & 0 & 1 & 0 \\
			18 & -21 & 4 & 0 & 0 & 1 \\
		\end{array}
		\right]
	\begin{matrix} \\ -1I \\ -2I \\ \end{matrix} \sim
	\left[
		\begin{array}{ccc|ccc}
			8 & -9 & 2 & 1  & 0 & 0 \\
			2 & -2 & 0 & -1 & 1 & 0 \\
			2 & -3 & 0 & -2 & 0 & 1 \\
		\end{array}
		\right]
	\begin{matrix} \\ -1III\\ \\ \end{matrix} \sim
$$
$$
	\sim
	\left[
		\begin{array}{ccc|ccc}
			8 & -9 & 2 & 1  & 0 & 0  \\
			0 & 1  & 0 & 1  & 1 & -1 \\
			2 & -3 & 0 & -2 & 0 & 1  \\
		\end{array}
		\right]
	\begin{matrix} +9II \\ \\ +3II \\ \end{matrix} \sim
	\left[
		\begin{array}{ccc|ccc}
			8 & 0 & 2 & 10 & 9 & -9 \\
			0 & 1 & 0 & 1  & 1 & -1 \\
			2 & 0 & 0 & 1  & 3 & -2 \\
		\end{array}
		\right]
	\begin{matrix} -4III \\ *2 \\ \\ \end{matrix} \sim
$$
$$
	\sim
	\left[
		\begin{array}{ccc|ccc}
			0 & 0 & 2 & 6 & -12 & -1 \\
			0 & 2 & 0 & 2 & 2   & -2 \\
			2 & 0 & 0 & 1 & 3   & -2 \\
		\end{array}
		\right]
	\begin{matrix} \\ \\ \\ \end{matrix} \sim
	\frac{1}{2}
	\left[
		\begin{array}{ccc|ccc}
			2 & 0 & 0 & 1 & 3  & -2 \\
			0 & 2 & 0 & 2 & 2  & -2 \\
			0 & 0 & 2 & 6 & -3 & -1 \\
		\end{array}
		\right]
$$
$$
	A_e^{-1} = \frac{1}{2}
	\begin{pmatrix}
		1 & 3  & -2 \\
		2 & 2  & -2 \\
		6 & -3 & -1 \\
	\end{pmatrix}
$$
$$
	A_f^{-1} =
	C_{f \rightarrow e} \times A_e^{-1} = \frac{1}{2}
	\begin{pmatrix}
		2  & -2 & -1 \\
		-5 & 3  & 1  \\
		-6 & 3  & 1  \\
	\end{pmatrix}
	\times
	\begin{pmatrix}
		1 & 3  & -2 \\
		2 & 2  & -2 \\
		6 & -3 & -1 \\
	\end{pmatrix}
	= \frac{1}{2}
	\begin{pmatrix}
		-22 & 25 & -4 \\
		8   & -9 & 0  \\
		6   & -6 & 0  \\
	\end{pmatrix}
$$

\textbf{Ответ:} $A_e^{-1} = \frac{1}{2} \begin{pmatrix} 1 & 3  & -2 \\ 2 & 2  & -2 \\ 6 & -3 & -1 \\ \end{pmatrix}$, $A_f^{-1} = \frac{1}{2} \begin{pmatrix} -22 & 25 & -4 \\ 8   & -9 & 0  \\ 6   & -6 & 0  \\ \end{pmatrix}$.

\subsection*{Задача 5. Ядро и образ оператора}
Найдите размерность ядра и образа оператора $ B $
\[
	B_e=
	\begin{pmatrix}
		2   & 1  & -1 \\
		2   & 1  & -1 \\
		-10 & 10 & 2
	\end{pmatrix}.
\]

Рассмотрим однородную систему $B_e x=0$:
\[
	\begin{cases}
		2x_1 + x_2 - x_3 = 0, \\
		2x_1 + x_2 - x_3 = 0, \\
		-10x_1 + 10x_2 + 2x_3 = 0.
	\end{cases}
\]

Из первого уравнения получаем
\[
	x_3 = 2x_1 + x_2.
\]

Подставим это в третье:
\[
	-10x_1 + 10x_2 + 2\,(2x_1 + x_2) = 0
	\;\Longrightarrow\;
	-10x_1 + 10x_2 + 4x_1 + 2x_2 = 0
	\;\Longrightarrow\;
	-6x_1 + 12x_2 = 0
	\;\Longrightarrow\;
	x_2 = \tfrac12 x_1.
\]

Тогда
\[
	x_3 = 2x_1 + \tfrac12 x_1 = \tfrac52 x_1.
\]

Пусть $x_1 = t$. Тогда общий вид решения:
\[
	x = t
	\begin{pmatrix}
		1        \\
		\tfrac12 \\
		\tfrac52
	\end{pmatrix}
	= t\;\frac12
	\begin{pmatrix}
		2 \\1\\5
	\end{pmatrix}.
\]

Следовательно
\[
	\ker B = \mathrm{span}\{(2,1,5)^T\},
	\quad
	\dim\ker B = 1.
\]

По теореме о ранге и дефекте:
\[
	\dim\ker B + \dim \mathrm{im} B = 3
	\;\Longrightarrow\;
	\dim \mathrm{im} B = 3 - 1 = 2.
\]
.

\textbf{Ответ:} \(\dim\ker B = 1\), \(\dim\mathrm{Im}\,B = 2\).

\subsection*{Задача 6. Ортонормированный базис ядра и образа}
Постройте ортонормированный базис ядра и образа оператора $ B $.

\paragraph*{Ядро.}
\[
	\ker B = \mathrm{span}\{(2,1,5)^T\}.
\]

Положим
\[
	v = (2,1,5)^T,
	\qquad
	\|v\| = \sqrt{2^2 + 1^2 + 5^2} = \sqrt{30}.
\]

Тогда
\[
	e_1 = \frac{v}{\sqrt{30}}
	= \frac{1}{\sqrt{30}}
	\begin{pmatrix}
		2 \\1\\5
	\end{pmatrix}
\]
— ортонормированный базис \(\ker B\).

\paragraph*{Образ.}
Столбцы матрицы \(B_e\) лежат в образе:
\[
	u_1 = \begin{pmatrix}2\\2\\-10\end{pmatrix},
	\quad
	u_2 = \begin{pmatrix}1\\1\\10\end{pmatrix}.
\]

\[
	\|u_1\| = \sqrt{2^2 + 2^2 + (-10)^2} = \sqrt{108},
	\qquad
	e_1' = \frac{u_1}{\sqrt{108}}
	= \frac{1}{\sqrt{108}}
	\begin{pmatrix}
		2 \\2\\-10
	\end{pmatrix}.
\]

Проекция \(u_2\) на \(e_1'\):
\[
	\langle u_2,e_1'\rangle
	= u_2^T e_1'
	= \frac{1\cdot2 + 1\cdot2 + 10\cdot(-10)}{\sqrt{108}}
	= \frac{-96}{\sqrt{108}},
\]
\[
	\mathrm{proj}_{e_1'}u_2
	= \langle u_2,e_1'\rangle\,e_1'
	= -\frac{96}{108}\,u_1
	= -\frac{8}{9}\,u_1.
\]

Ортогональная составляющая:
\[
	v_2 = u_2 - \mathrm{proj}_{e_1'}u_2
	= u_2 + \frac{8}{9}\,u_1
	= \begin{pmatrix}5\\5\\2\end{pmatrix},
\]
\[
	\|v_2\| = \sqrt{5^2 + 5^2 + 2^2} = \sqrt{54},
	\qquad
	e_2' = \frac{v_2}{\sqrt{54}}
	= \frac{1}{\sqrt{54}}
	\begin{pmatrix}
		5 \\5\\2
	\end{pmatrix}.
\]

Итак, ортонормированный базис \(\mathrm{Im}\,B\):
\[
	\bigl\{\,e_1',\,e_2'\bigr\}
	= \left\{
	\frac{1}{\sqrt{108}}
	\begin{pmatrix}2\\2\\-10\end{pmatrix},
	\;
	\frac{1}{\sqrt{54}}
	\begin{pmatrix}5\\5\\2\end{pmatrix}
	\right\}.
\]

\textbf{Ответ:} ортонормированный базис:
\[
	\ker B:\quad \left\{\frac{1}{\sqrt{30}}
	\begin{pmatrix}2\\1\\5\end{pmatrix}\right\},
	\qquad
	\mathrm{Im}\,B:\quad
	\left\{\frac{1}{\sqrt{108}}
	\begin{pmatrix}2\\2\\-10\end{pmatrix},\;
	\frac{1}{\sqrt{54}}
	\begin{pmatrix}5\\5\\2\end{pmatrix}\right\}.
\]

\subsection*{Задача 7. Спектральные характеристики операторов}
Найдите собственные числа и собственные вектора операторов \(A\) и \(B\).

\paragraph*{Оператор \(A\).}
Характеристический многочлен:
\[
	\chi_A(\lambda)
	=\det\bigl(A_e-\lambda I\bigr)
	=\det
	\begin{pmatrix}
		8-\lambda & -9          & 2         \\
		10        & -11-\lambda & 2         \\
		18        & -21         & 4-\lambda
	\end{pmatrix}
	= -(\lambda+2)(\lambda-1)(\lambda-2).
\]

Отсюда
\[
	\mathrm{Spec}(A)=\{-2,1,2\}.
\]

\paragraph*{Случай \(\lambda=-2\).}
Тогда \(A_e+2I\) равна
\[
	A_e+2I=
	\begin{pmatrix}
		10 & -9  & 2 \\
		10 & -9  & 2 \\
		18 & -21 & 6
	\end{pmatrix}.
\]

Система \((A_e+2I)x=0\) в координатах:
\[
	\begin{cases}
		10x_1 - 9x_2 + 2x_3 = 0, \\
		10x_1 - 9x_2 + 2x_3 = 0, \\
		18x_1 -21x_2 + 6x_3 = 0.
	\end{cases}
\]

Упрощаем строками:
\[
	R_2 \to R_2 - R_1,\quad
	R_3 \to R_3 - \tfrac{9}{5}R_1
	\;\Longrightarrow\;
	\begin{cases}
		0 = 0, \\
		-\tfrac{24}{5}x_2 + \tfrac{12}{5}x_3 = 0.
	\end{cases}
\]

Из второго уравнения \(-2x_2 + x_3 = 0\), значит \(x_3 = 2x_2\).

Подставляем в первое:
\[
	10x_1 -9x_2 +2\cdot(2x_2)=10x_1 -5x_2=0
	\;\Longrightarrow\;
	x_1 = \tfrac12 x_2.
\]

Пусть \(x_2=t\). Тогда
\[
	x = t\begin{pmatrix}\tfrac12\\1\\2\end{pmatrix}
	= t\tfrac12\begin{pmatrix}1\\2\\4\end{pmatrix},
\]
и
\[
	\ker(A_e+2I)=\mathrm{span}\{(1,2,4)^T\}.
\]

\paragraph*{Случай \(\lambda=1\).}
Теперь \(A_e - I\):
\[
	A_e - I=
	\begin{pmatrix}
		7  & -9  & 2 \\
		10 & -12 & 2 \\
		18 & -21 & 3
	\end{pmatrix}.
\]

Система \((A_e - I)x=0\):
\[
	\begin{aligned}
		7x_1 - 9x_2 + 2x_3  & = 0, \\
		10x_1 -12x_2 + 2x_3 & = 0, \\
		18x_1 -21x_2 + 3x_3 & = 0.
	\end{aligned}
\]

Выполним
\[
	R_2 \to R_2 - \tfrac{10}{7}R_1,\quad
	R_3 \to R_3 - \tfrac{18}{7}R_1
\]
и получаем две одинаковые строки вида
\[
	\frac{6}{7}x_2 - \frac{6}{7}x_3 = 0
	\;\Longrightarrow\;
	x_2 = x_3.
\]

Подставляем в \(7x_1 -9x_2 +2x_2=7x_1 -7x_2=0\), отсюда \(x_1 = x_2\).

Пусть \(x_2=t\). Тогда
\[
	x = t\,(1,1,1)^T,
	\qquad
	\ker(A_e - I) = \mathrm{span}\{(1,1,1)^T\}.
\]

\paragraph*{Случай \(\lambda=2\).}
Имеем
\[
	A_e - 2I =
	\begin{pmatrix}
		6  & -9  & 2 \\
		10 & -13 & 2 \\
		18 & -21 & 2
	\end{pmatrix}.
\]

Система \((A_e - 2I)x=0\):
\[
	\begin{aligned}
		6x_1 - 9x_2 + 2x_3  & = 0, \\
		10x_1 -13x_2 + 2x_3 & = 0, \\
		18x_1 -21x_2 + 2x_3 & = 0.
	\end{aligned}
\]

Далее
\[
	R_2 \to R_2 - \tfrac{5}{3}R_1,\quad
	R_3 \to R_3 - 3R_1
	\;\Longrightarrow\;
	\begin{cases}
		3x_2 - 2x_3 = 0, \\
		3x_2 - 2x_3 = 0,
	\end{cases}
\]
то есть \(x_3 = \tfrac{3}{2}x_2\).

Подставляем в первое уравнение:
\[
	6x_1 -9x_2 +2\cdot\tfrac{3}{2}x_2
	=6x_1 -6x_2=0
	\;\Longrightarrow\;
	x_1 = x_2.
\]

Пусть \(x_2=t\). Тогда
\[
	x = t\,(1,1,\tfrac32)^T
	= t\tfrac12(2,2,3)^T,
\]
и
\[
	\ker(A_e - 2I)=\mathrm{span}\{(2,2,3)^T\}.
\]

\paragraph*{Оператор \(B\).}

Характеристический многочлен:
\[
	\chi_B(\lambda)
	=\det\bigl(B_e-\lambda I\bigr)
	=\det
	\begin{pmatrix}
		2-\lambda & 1         & -1        \\
		2         & 1-\lambda & -1        \\
		-10       & 10        & 2-\lambda
	\end{pmatrix}
	= -\lambda(\lambda-2)(\lambda-3).
\]

Отсюда
\[
	\mathrm{Spec}(B)=\{0,2,3\}.
\]

\paragraph*{Случай \(\lambda=0\).}
Тогда \(B_e - 0I = B_e\):
\[
	B_e =
	\begin{pmatrix}
		2   & 1  & -1 \\
		2   & 1  & -1 \\
		-10 & 10 & 2
	\end{pmatrix}.
\]

Система \((B_e - 0I)x=0\):
\[
	\begin{cases}
		2x_1 + x_2 - x_3 = 0, \\
		2x_1 + x_2 - x_3 = 0, \\
		-10x_1 + 10x_2 + 2x_3 = 0.
	\end{cases}
\]

Выполним
\[
	R_2 \to R_2 - R_1,\quad
	R_3 \to R_3 + 5R_1
	\;\Longrightarrow\;
	\begin{cases}
		0 = 0, \\
		15x_2 - 3x_3 = 0.
	\end{cases}
\]

Отсюда \(5x_2 - x_3 = 0\), то есть \(x_3 = 5x_2\).

Подставляем в первое уравнение:
\[
	2x_1 + x_2 - 5x_2 = 2x_1 - 4x_2 = 0
	\;\Longrightarrow\;
	x_1 = 2x_2.
\]

Пусть \(x_2 = t\). Тогда
\[
	x = t
	\begin{pmatrix}
		2 \\1\\5
	\end{pmatrix},
\]
и
\[
	\ker(B_e) = \mathrm{span}\{(2,1,5)^T\}.
\]

\paragraph*{Случай \(\lambda=2\).}

Тогда
\[
	B_e - 2I =
	\begin{pmatrix}
		0   & 1  & -1 \\
		2   & -1 & -1 \\
		-10 & 10 & 0
	\end{pmatrix}.
\]

Система \((B_e - 2I)x=0\):
\[
	\begin{aligned}
		x_2 - x_3        & = 0, \\
		2x_1 - x_2 - x_3 & = 0, \\
		-10x_1 + 10x_2   & = 0.
	\end{aligned}
\]

Из первого уравнения \(x_2 = x_3\).

Подставляем в второе:
\[
	2x_1 - 2x_2 = 0
	\;\Longrightarrow\;
	x_1 = x_2,
\]
третье даёт то же: \(-10x_1+10x_2=0\).

Пусть \(x_2 = t\). Тогда
\[
	x = t\,(1,1,1)^T,
\]
и
\[
	\ker(B_e - 2I) = \mathrm{span}\{(1,1,1)^T\}.
\]

\paragraph*{Случай \(\lambda=3\).}
Тогда
\[
	B_e - 3I =
	\begin{pmatrix}
		-1  & 1  & -1 \\
		2   & -2 & -1 \\
		-10 & 10 & -1
	\end{pmatrix}.
\]

Система \((B_e - 3I)x=0\):
\[
	\begin{aligned}
		-x_1 + x_2 - x_3     & = 0, \\
		2x_1 - 2x_2 - x_3    & = 0, \\
		-10x_1 + 10x_2 - x_3 & = 0.
	\end{aligned}
\]

Из первого: \(x_1 = x_2 - x_3\).

Подставляем во второе:
\[
	2(x_2 - x_3) - 2x_2 - x_3 = -3x_3 = 0
	\;\Longrightarrow\;
	x_3 = 0.
\]

Тогда \(x_1 = x_2\) и третье уравнение автоматически выполняется.

Пусть \(x_2 = t\). Тогда
\[
	x = t\,(1,1,0)^T,
\]
и
\[
	\ker(B_e - 3I) = \mathrm{span}\{(1,1,0)^T\}.
\]


\textbf{Ответ:}
\[
	\begin{aligned}
		A: & \;\lambda=-2,\;v=(1,2,4)^T;\quad
		\lambda=1,\;v=(1,1,1)^T;\quad
		\lambda=2,\;v=(2,2,3)^T;              \\
		B: & \;\lambda=0,\;v=(2,1,5)^T;\quad
		\lambda=2,\;v=(1,1,1)^T;\quad
		\lambda=3,\;v=(1,1,0)^T.
	\end{aligned}
\]

\subsection*{Задача 8. Диагонализация операторов}
Выписать матрицы операторов \(A\) и \(B\) в базисах из собственных векторов.

\paragraph*{Оператор \(A\).}
Собственные числа и векторы:
\[
	\lambda_1=-2,\;v_1=(1,2,4)^T;
	\quad
	\lambda_2=1,\;v_2=(1,1,1)^T;
	\quad
	\lambda_3=2,\;v_3=(2,2,3)^T.
\]

Пусть в новый базис \(f_A\) входят векторы \(v_1,v_2,v_3\). Тогда
\[
	[A]_{f_A}
	= P_A^{-1}\,A_e\,P_A
	= \begin{pmatrix}
		-2 & 0 & 0 \\
		0  & 1 & 0 \\
		0  & 0 & 2
	\end{pmatrix},
\]
где \(P_A = \bigl[v_1\;v_2\;v_3\bigr]\).

\paragraph*{Оператор \(B\).}
Собственные числа и векторы:
\[
	\lambda_1=0,\;w_1=(2,1,5)^T;
	\quad
	\lambda_2=2,\;w_2=(1,1,1)^T;
	\quad
	\lambda_3=3,\;w_3=(1,1,0)^T.
\]

В базисе \(f_B=\{w_1,w_2,w_3\}\) матрица \(B\) станет диагональной:
\[
	[B]_{f_B}
	= P_B^{-1}\,B_e\,P_B
	= \begin{pmatrix}
		0 & 0 & 0 \\
		0 & 2 & 0 \\
		0 & 0 & 3
	\end{pmatrix},
\]
где \(P_B = \bigl[w_1\;w_2\;w_3\bigr]\).

\textbf{Ответ:}
\[
	[A]_{f_A}
	= \begin{pmatrix}
		-2 & 0 & 0 \\
		0  & 1 & 0 \\
		0  & 0 & 2
	\end{pmatrix},
	\qquad
	[B]_{f_B}
	= \begin{pmatrix}
		0 & 0 & 0 \\
		0 & 2 & 0 \\
		0 & 0 & 3
	\end{pmatrix}.
\]


\end{document}