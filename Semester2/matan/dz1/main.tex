\documentclass[12pt,a4paper]{article}
\usepackage[utf8]{inputenc}
\usepackage[T2A]{fontenc}
\usepackage[russian]{babel}
\usepackage{amsmath, amsthm, amssymb}
\usepackage[left=15mm,right=15mm,top=15mm,bottom=15mm]{geometry}
\usepackage{indentfirst}
\usepackage{hyperref}

\begin{document}

\begin{center}
  \Large \textbf{Решение ИДЗ. Вариант: Николаев.}
\end{center}

\vspace{5mm}

%%%%%%%%%%%%%%%%%%%%%%%%%%%%%%%%%%%%%%%%%%%%%%%%%%%%%%%%%%%%%%%%%%%%%%
\section*{Задача 1. Площадь фигуры, ограниченной кривыми}
\textbf{Условие:} Найти площадь фигуры, ограниченной кривыми
\[
x^2+y^2=2,\qquad y=-x^2,\quad (y > -x^2).
\]

\textbf{Анализ:} 
\begin{itemize}
  \item Окружность \(x^2+y^2=2\) имеет центр в начале координат и радиус \(R=\sqrt{2}\).
  \item Парабола \(y=-x^2\) проходит через точку \((0,0)\) и является ветвью, лежащей ниже оси \(x\) (при \(x\ne0\)), но условие \(y > -x^2\) означает, что рассматривается та часть, где точка находится выше этой кривой.
\end{itemize}

Найдём точки пересечения, подставляя \(y=-x^2\) в уравнение окружности:
\[
x^2+(-x^2)^2=x^2+x^4=2.
\]
Обозначим \(t=x^2\): тогда
\[
t^2+t-2=0.
\]
Дискриминант равен \(D=1+8=9\), откуда
\[
t=\frac{-1\pm3}{2}.
\]
Отрицательное значение \(t\) отбрасываем, получаем \(x^2=1\) и, соответственно, \(x=\pm1\). При \(x=\pm1\) имеем
\[
y=-1.
\]

Таким образом, пересечения: \((-1,-1)\) и \((1,-1)\).

\textbf{Описание области:}  
Для фиксированного \(x\) внутри круга (то есть для \(x\) от \(-\sqrt{2}\) до \(\sqrt{2}\)), вертикальный отрезок внутри окружности имеет границы \(y=-\sqrt{2-x^2}\) и \(y=\sqrt{2-x^2}\).  
При условии \(y> -x^2\) нижняя граница становится следующей:
\begin{itemize}
  \item Если \(|x|\le1\), то \(-x^2 > -\sqrt{2-x^2}\) (так как, например, при \(x=0\): \(0 > -\sqrt{2}\)). Здесь нижней границей выбираем \(y=-x^2\).
  \item Если \(|x|>1\), то нижняя граница определяется условием круга, т.е. \(y=-\sqrt{2-x^2}\) (так как \(-\sqrt{2-x^2}>-x^2\)).
\end{itemize}

\textbf{Выразим площадь через интегралы:}

Область разбивается по \(x\):
\[
A = \underbrace{\int_{-\sqrt{2}}^{-1} \left[\sqrt{2-x^2}-\bigl(-\sqrt{2-x^2}\bigr)\right]dx + \int_{-1}^{1} \left[\sqrt{2-x^2} - \bigl(-x^2\bigr)\right]dx}_{\text{для } x\in[-1,1]} +
\]
\[
+ \int_{1}^{\sqrt{2}} \left[\sqrt{2-x^2}-\bigl(-\sqrt{2-x^2}\bigr)\right]dx.
\]
Используя чётность, можно записать:
\[
A = 2\int_{1}^{\sqrt{2}} 2\sqrt{2-x^2}\,dx + 2\int_{0}^{1} \left[\sqrt{2-x^2} + x^2\right]dx.
\]

Обозначим:
\[
I_1 = \int_{1}^{\sqrt{2}} 2\sqrt{2-x^2}\,dx,\qquad I_2 = \int_{0}^{1} \left[\sqrt{2-x^2}+x^2\right]dx.
\]

Для \(I_1\) воспользуемся стандартной формулой:
\[
\int \sqrt{a^2-x^2}\,dx = \frac{x}{2}\sqrt{a^2-x^2}+\frac{a^2}{2}\arcsin\frac{x}{a}+C,\quad a=\sqrt{2}.
\]
Вычисляем:
\[
F(x)= \frac{x}{2}\sqrt{2-x^2} + \arcsin\frac{x}{\sqrt{2}}.
\]
При \(x=\sqrt{2}\): \(\sqrt{2-\;2}=0,\quad \arcsin(1)=\frac{\pi}{2}\).  
При \(x=1\): \(\sqrt{2-1}=1,\quad \arcsin\frac{1}{\sqrt{2}}=\frac{\pi}{4}\).  

Тогда:
\[
I_1=2\Bigl[F(\sqrt{2})-F(1)\Bigr]=2\left[\frac{\pi}{2}-\left(\frac{1}{2}+\frac{\pi}{4}\right)\right]
=2\left(\frac{\pi}{4}-\frac{1}{2}\right)
=\frac{\pi}{2}-1.
\]

Для \(I_2\):
\[
\int_{0}^{1}\sqrt{2-x^2}\,dx = \frac{1}{2}+\frac{\pi}{4},\qquad \int_{0}^{1}x^2\,dx = \frac{1}{3}.
\]
Следовательно,
\[
I_2 = \left(\frac{1}{2}+\frac{\pi}{4}\right)+\frac{1}{3} = \frac{5}{6}+\frac{\pi}{4}.
\]

Таким образом, общая площадь:
\[
A=2I_1+2I_2 = 2\left(\frac{\pi}{2}-1\right)+2\left(\frac{5}{6}+\frac{\pi}{4}\right)
=\left(\pi-2\right)+\left(\frac{5}{3}+\frac{\pi}{2}\right)
=\frac{3\pi}{2}-2+\frac{5}{3}.
\]
Приведя к общему знаменателю (6):
\[
A=\frac{9\pi}{6}-\frac{12}{6}+\frac{10}{6}=\frac{9\pi-2}{6}.
\]

\textbf{Ответ к задаче 1:} 
\[
A = \frac{9\pi-2}{6}.
\]

%%%%%%%%%%%%%%%%%%%%%%%%%%%%%%%%%%%%%%%%%%%%%%%%%%%%%%%%%%%%%%%%%%%%%%
\section*{Задача 2. Длина дуги параболы}
\textbf{Условие:} Найти длину дуги параболы
\[
y=4x^2-4x+2,
\]
от \(x_1=5\) до \(x_2=6\).

\textbf{Решение:} Для функции \(y=f(x)\) дуга определяется по формуле
\[
L=\int_{x_1}^{x_2}\sqrt{1+\left(f'(x)\right)^2}\,dx.
\]
Вычисляем производную:
\[
f'(x)=\frac{d}{dx}(4x^2-4x+2)=8x-4.
\]
Следовательно,
\[
L=\int_{5}^{6}\sqrt{1+(8x-4)^2}\,dx
=\int_{5}^{6}\sqrt{1+64(x-0.5)^2}\,dx.
\]

Для удобства сделаем подстановку:
\[
u=8(x-0.5),\quad du=8dx,\quad dx=\frac{du}{8}.
\]
При \(x=5\): \(u=8(4.5)=36\); при \(x=6\): \(u=8(5.5)=44\).

Получаем:
\[
L=\frac{1}{8}\int_{36}^{44}\sqrt{1+u^2}\,du.
\]
Антидифференциал:
\[
\int \sqrt{1+u^2}\,du=\frac{1}{2}\Bigl(u\sqrt{1+u^2}+\operatorname{arcsinh}(u)\Bigr)+C.
\]
Таким образом,
\[
L=\frac{1}{16}\left[\;u\sqrt{1+u^2}+\operatorname{arcsinh}(u)\;\right]_{u=36}^{u=44}.
\]
То есть,
\[
L=\frac{1}{16}\Bigl[44\sqrt{1+44^2}+\operatorname{arcsinh}(44)-36\sqrt{1+36^2}-\operatorname{arcsinh}(36)\Bigr].
\]
Заметим, что \(44^2=1936\) и \(36^2=1296\).

\textbf{Ответ к задаче 2:}
\[
L=\frac{1}{16}\Bigl[44\sqrt{1937}-36\sqrt{1297}+\operatorname{arcsinh}(44)-\operatorname{arcsinh}(36)\Bigr].
\]

%%%%%%%%%%%%%%%%%%%%%%%%%%%%%%%%%%%%%%%%%%%%%%%%%%%%%%%%%%%%%%%%%%%%%%
\section*{Задача 3. Абсцисса центра масс фигуры, ограниченной параболой и касательными}
\textbf{Условие:} Пусть дана парабола 
\[
y=4x^2-4x+2,
\]
а \(x_1=5\) и \(x_2=6\). Построим касательные к параболе в точках \(x=5\) и \(x=6\).

Найдем уравнения касательных.

\textbf{Для \(x=5\):}
\[
y(5)=4\cdot5^2-4\cdot5+2=100-20+2=82,\quad f'(5)=8\cdot5-4=36.
\]
Уравнение касательной:
\[
y-82=36(x-5)\quad\Rightarrow\quad y=36(x-5)+82.
\]

\textbf{Для \(x=6\):}
\[
y(6)=4\cdot6^2-4\cdot6+2=144-24+2=122,\quad f'(6)=8\cdot6-4=44.
\]
Уравнение касательной:
\[
y-122=44(x-6)\quad\Rightarrow\quad y=44(x-6)+122.
\]

Найдём точку пересечения касательных:
\[
36(x-5)+82=44(x-6)+122.
\]
Приводим к общему виду:
\[
36x-180+82=44x-264+122\quad\Longrightarrow\quad 36x-98=44x-142.
\]
Вынесем \(x\):
\[
-98+142=44x-36x\quad\Rightarrow\quad 44=8x,\quad x=5.5.
\]
Подставляя \(x=5.5\) в одно из уравнений, получаем:
\[
y=36(0.5)+82=18+82=100.
\]
Таким образом, касательные пересекаются в точке \((5.5,100)\).

\textbf{Описание фигуры:} Фигура ограничена:
\begin{enumerate}
  \item Дугой параболы от точки \((5,82)\) до \((6,122)\).
  \item Отрезками касательных: от \((5,82)\) до \((5.5,100)\) (касательная в \(x=5\)) и от \((5.5,100)\) до \((6,122)\) (касательная в \(x=6\)).
\end{enumerate}

Чтобы найти абсциссу центра масс, выразим фигуру в виде вертикальных сечений по \(x\). Заметим, что нижняя граница задаётся касательными, а верхняя --- параболой.

Для \(x\in[5,6]\) нижняя граница определяется по частям:
\begin{itemize}
  \item При \(x\in[5,5.5]\) нижняя кривая --- касательная в \(x=5\): 
    \[
    g_1(x)=36(x-5)+82.
    \]
  \item При \(x\in[5.5,6]\) нижняя кривая --- касательная в \(x=6\):
    \[
    g_2(x)=44(x-6)+122.
    \]
\end{itemize}
Верхняя граница задаётся параболой:
\[
f(x)=4x^2-4x+2.
\]

Найдем разности \(f(x)-g_i(x)\):

\textbf{Для \(x\in[5,5.5]\):}
\[
f(x)-g_1(x)=\bigl(4x^2-4x+2\bigr)-\bigl[36(x-5)+82\bigr]
=4x^2-4x+2-36x+180-82=4x^2-40x+100.
\]
Заметим, что
\[
4x^2-40x+100=4\bigl(x^2-10x+25\bigr)=4(x-5)^2.
\]
Область по \(x\in[5.5,6]\) аналогично:
\[
f(x)-g_2(x)=4x^2-4x+2-\bigl[44(x-6)+122\bigr]
=4x^2-48x+144=4(x-6)^2.
\]

Площадь фигуры \(R\) вычисляется как
\[
A_R=\int_{5}^{5.5}4(x-5)^2\,dx+\int_{5.5}^{6}4(x-6)^2\,dx.
\]
Сделав замену \(u=x-5\) (при первом интеграле) и \(v=6-x\) (при втором), получаем
\[
A_R=4\int_{0}^{0.5}u^2\,du+4\int_{0}^{0.5}v^2\,dv
=2\cdot \frac{(0.5)^3}{3}+2\cdot \frac{(0.5)^3}{3}
=\frac{1}{6}+\frac{1}{6}=\frac{1}{3}.
\]

Для нахождения абсциссы центра масс воспользуемся формулой:
\[
\bar{x}=\frac{1}{A_R}\int_{R} x\,dA=\frac{1}{A_R}\left[\int_{5}^{5.5}x\cdot 4(x-5)^2dx+\int_{5.5}^{6}x\cdot 4(x-6)^2dx\right].
\]

\textbf{Вычисление первого слагаемого:} Сделаем замену \(u=x-5\), тогда \(x=u+5\) и \(u\in[0,0.5]\):
\[
M_1=4\int_{0}^{0.5}(u+5)u^2du
=4\Bigl[\int_{0}^{0.5}u^3du+5\int_{0}^{0.5}u^2du\Bigr].
\]
Так как
\[
\int_{0}^{0.5}u^3du=\frac{(0.5)^4}{4}=\frac{1}{64},\qquad \int_{0}^{0.5}u^2du=\frac{(0.5)^3}{3}=\frac{1}{24},
\]
то
\[
M_1=4\left(\frac{1}{64}+5\cdot\frac{1}{24}\right)
=4\left(\frac{1}{64}+\frac{5}{24}\right)
=\frac{43}{48}\quad \text{(в точных дробях)}.
\]

\textbf{Вычисление второго слагаемого:} Сделаем замену \(v=6-x\) (при \(x\in[5.5,6]\)), тогда \(x=6-v\) и \(v\in[0,0.5]\):
\[
M_2=4\int_{0}^{0.5}(6-v)v^2dv
=4\left[6\int_{0}^{0.5}v^2dv-\int_{0}^{0.5}v^3dv\right]
=4\left(6\cdot\frac{1}{24}-\frac{1}{64}\right)
=\frac{15}{16}.
\]

Общий момент:
\[
M=\frac{43}{48}+\frac{15}{16}=\frac{43+45}{48}=\frac{88}{48}=\frac{11}{6}.
\]

Таким образом,
\[
\bar{x}=\frac{M}{A_R}=\frac{11/6}{1/3}=\frac{11}{6}\cdot3=\;5.5.
\]

\textbf{Ответ к задаче 3:} 
\[
\bar{x}=5.5.
\]

%%%%%%%%%%%%%%%%%%%%%%%%%%%%%%%%%%%%%%%%%%%%%%%%%%%%%%%%%%%%%%%%%%%%%%
\section*{Задача 4. Ордината центра масс фигуры}
Чтобы найти ординату центра масс, воспользуемся формулой для момента по \(y\):
\[
\bar{y}=\frac{1}{A_R}\int_{R}\frac{f^2(x)-g^2(x)}{2}\,dx,
\]
где \(f(x)\) --- верхняя граница (парабола), а \(g(x)\) --- нижняя (касательные). При разбиении по областям получаем:
\[
J_1=\frac{1}{2}\int_{5}^{5.5}\Bigl[f^2(x)-g_1^2(x)\Bigr]dx,\qquad
J_2=\frac{1}{2}\int_{5.5}^{6}\Bigl[f^2(x)-g_2^2(x)\Bigr]dx.
\]
Заметим, что
\[
f^2(x)-g_i^2(x)=(f(x)-g_i(x))\bigl(f(x)+g_i(x)\bigr),
\]
а мы уже доказали, что
\[
f(x)-g_1(x)=4(x-5)^2,\qquad f(x)-g_2(x)=4(x-6)^2.
\]
Путём аккуратных замен (подстановки \(u=x-5\) для \(J_1\) и \(v=6-x\) для \(J_2\)) можно показать, что
\[
J_1=\frac{479}{30},\qquad J_2=\frac{529}{30}.
\]
Таким образом, суммарный момент по \(y\) равен
\[
J=J_1+J_2=\frac{479+529}{30}=\frac{1008}{30}=\frac{168}{5}\quad (=33.6).
\]
При этом, поскольку площадь \(A_R=\frac{1}{3}\), ордината центра масс будет равна
\[
\bar{y}=\frac{J}{A_R}=\frac{168/5}{1/3}=\frac{168}{5}\cdot3=\frac{504}{5}=100.8.
\]

\textbf{Ответ к задаче 4:}
\[
\bar{y}=\frac{504}{5}\quad.
\]

\end{document}
