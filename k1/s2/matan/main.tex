\documentclass[a4paper,fleqn]{article}
\usepackage[left=1cm, right=1cm, top=1cm, bottom=1cm]{geometry} 
\setlength{\arraycolsep}{2pt} 
\usepackage[utf8]{inputenc}  
\usepackage[russian]{babel}  
\usepackage{amsmath, amssymb} 
\usepackage{pgffor}          
\usepackage{setspace}        
\usepackage{parskip}

% Отключаем отступы между абзацами и уменьшаем межстрочный интервал
\parskip=0pt
\setstretch{0.5}
\allowdisplaybreaks

\newcommand{\hm}[1]{#1\nobreak\discretionary{}{\hbox{\ensuremath{#1}}}{}}

% Убираем выравнивание по ширине и прижимаем к левому краю

\begin{document}

\title{Решения задач}
\author{Николаев Всеволод Юрьевич, группа №4395}
\date{\today}
\maketitle

% Автоматическое подключение всех .tex файлов из папки tex/
\foreach \x in {27, 28} {  
    \section*{Задача \x}
    \input{tex/\x.tex}
}

\end{document}
