% ==================
% answer.tex
% (Основной материал: Достаточные условия существования экстремума
%  (по второй производной))
% ==================

\begin{customtheorem}[Достаточные условия экстремума по второй производной]
	Пусть $f$ дифференцируема на $(a,b)$ и $x_0 \in (a,b)$ — такая точка,
	где $f'(x_0)=0$. Предположим, что у $f$ существует непрерывная в $x_0$
	вторая производная $f''(x_0)$. Тогда:

	\begin{enumerate}
		\item Если $f''(x_0)>0$, то $x_0$ — точка \textbf{локального минимума}.
		\item Если $f''(x_0)<0$, то $x_0$ — точка \textbf{локального максимума}.
		\item Если $f''(x_0)=0$, вывод о виде экстремума не делается
		      (нужен дополнительный анализ).
	\end{enumerate}
\end{customtheorem}

\begin{proofplan}
	\begin{enumerate}
		\item Исходя из Теоремы Ферма, имеем $f'(x_0)=0$.
		\item Случай $f''(x_0)>0$: показывает, что $f'(x)$ возрастает вблизи $x_0$,
		      отсюда $x_0$ становится локальным минимумом.
		\item Случай $f''(x_0)<0$: говорит, что $f'(x)$ убывает вблизи $x_0$,
		      получаем локальный максимум.
		\item Если $f''(x_0)=0$, дополнительно надо исследовать ситуацию (пример $x^3$).
	\end{enumerate}
\end{proofplan}

\begin{customproof}
	Пусть $f'(x_0)=0$ и $f''(x_0)$ существует и непрерывна в $x_0$.

	\textbf{Случай $f''(x_0)>0$.}
	Из непрерывности $f''$ около $x_0$ следует, что при $x$ достаточно близком к $x_0$,
	вторая производная $f''(x)$ остаётся положительной. Это значит, что $f'(x)$ строго возрастает вблизи $x_0$. Так как $f'(x_0)=0$,
	то при $x>x_0$ значения $f'(x)$ становятся положительными,
	а при $x<x_0$ — отрицательными. Следовательно,
	\[
		x>x_0 \;\Longrightarrow\; f'(x)>0 \implies f \text{ возрастает справа},
	\]
	\[
		x<x_0 \;\Longrightarrow\; f'(x)<0 \implies f \text{ убывает слева}.
	\]
	Значит $x_0$ — локальный минимум.

	\textbf{Случай $f''(x_0)<0$.}
	Аналогично, теперь $f'(x)$ убывает при $x$ около $x_0$. Поскольку $f'(x_0)=0$,
	при $x>x_0$ значения $f'(x)$ оказываются отрицательными,
	а при $x<x_0$ — положительными.
	Тогда
	\[
		x<x_0 \;\Longrightarrow\; f'(x)>0 \implies f \text{ возрастает слева},
	\]
	\[
		x>x_0 \;\Longrightarrow\; f'(x)<0 \implies f \text{ убывает справа}.
	\]
	Следовательно, $x_0$ — локальный максимум.

	\textbf{Случай $f''(x_0)=0$.}
	Здесь нельзя сделать однозначный вывод об экстремуме (например, $f(x)=x^3$ при $x=0$ даёт $f'(0)=0, f''(0)=0$, но это не экстремум).
	Нужны другие способы анализа (см. более высокие производные, графический анализ и т. п.).
\end{customproof}

\begin{customexample}
	\begin{itemize}
		\item \textbf{Пример:} $f(x)=x^2$. Имеем $f'(0)=0$, $f''(0)=2>0$, значит
		      в $x=0$ локальный минимум.
		\item \textbf{Пример:} $f(x)=x^3$. Имеем $f'(0)=0$, но $f''(0)=0$,
		      что не даёт никакого вывода о минимуме/максимуме.
		      На практике $x=0$ — это точка перегиба без экстремума.
		\item \textbf{Пример:} $f(x)=-x^2$. Имеем $f'(0)=0$, $f''(0)=-2<0$,
		      стало быть $x=0$ — локальный максимум.
	\end{itemize}
\end{customexample}
