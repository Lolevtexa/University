% ==================
% auxiliary.tex
% (Вспомогательные определения для темы
% "Вывод рядов Тейлора для e^x, sin x, cos x и Формула Эйлера")
% ==================

\paragraph{Ряд Маклорена.}
Пусть функция $f$ имеет все производные в некоторой окрестности точки $0$. Тогда
\emph{рядом Маклорена} для $f$ называют
\[
	f(0) \;+\; \frac{f'(0)}{1!}\,x \;+\; \frac{f''(0)}{2!}\,x^2 \;+\; \dots \;+\;
	\frac{f^{(n)}(0)}{n!}\,x^n \;+\; \dots
\]
Если этот ряд сходится к $f(x)$ при соответствующих значениях $x$, то мы получаем разложение $f(x)$ в степенной ряд около $0$.

\bigskip

\paragraph{Остаточный член в форме Лагранжа.}
В случае, когда у функции $f$ есть $(n+1)$-я производная в окрестности $0$, можно записать
\[
	R_n(x)
	\;=\;
	\frac{f^{(n+1)}(\xi)}{(n+1)!}\,x^{n+1},
\]
где $\xi$ — некоторая точка между $0$ и $x$. Это называют \emph{остаточным членом} (или недостающим звеном) в формуле Тейлора (Маклорена).

\bigskip

\paragraph{Идея применения Теоремы Лагранжа.}
Для доказательства формулы Тейлора с остатком в форме Лагранжа часто используют теорему Лагранжа о среднем значении для производных:
если $P_n(x)$ — многочлен Тейлора степени $n$, то на отрезке $[0,x]$ применяют теорему Лагранжа к функции $f(x)-P_n(x)$, чтобы получить вид «\,$f^{(n+1)}(\xi)\,x^{n+1}/(n+1)!$».

\bigskip

\paragraph{Комплексная экспонента.}
Функция $e^{ix}$, где $i$ — мнимая единица, можно рассматривать как обобщение экспоненты на комплексную плоскость:
\[
	e^{ix} \;=\; \sum_{n=0}^{\infty} \frac{(i x)^n}{n!}.
\]
Из этого получается \emph{Формула Эйлера} при разбиении на действительную и мнимую часть.
