% ===== 2.tex =====
% "Дифференциал функции. Теорема Ферма. Теорема Ролля. Примеры."

% 1. Определения

\textbf{Дифференциал функции.}
Пусть функция $f$ определена в окрестности точки $x_0$ и дифференцируема в $x_0$. Тогда \emph{дифференциалом} $df(x_0)$ функции $f$ в точке $x_0$ называется величина
\[
df(x_0) \;=\; f'(x_0)\,(x - x_0),
\]
где $dx = x - x_0$ считается «малым» приращением аргумента. В более общем смысле при малом $dx$ пишут
\[
df(x_0) \;=\; f'(x_0)\,dx.
\]
Это выражает, что приращение функции раскладывается на линейную часть $df(x_0)$ и малую «остаточную» часть $o(dx)$:
\[
f(x_0 + dx) - f(x_0) \;=\; df(x_0) + o(dx).
\]

\medskip

% 2. Теоремы и ключевые утверждения

\textbf{Теорема Ферма (о локальном экстремуме).}
Если функция $f$ дифференцируема в точке $x_0$ и имеет там локальный минимум или максимум, то
\[
f'(x_0) \;=\; 0.
\]

\textbf{Теорема Ролля.}
Пусть $f$ удовлетворяет трём условиям:
\begin{enumerate}
  \item Непрерывна на отрезке $[a,b]$;
  \item Дифференцируема на интервале $(a,b)$;
  \item При этом $f(a) = f(b)$.
\end{enumerate}
Тогда существует хотя бы одна точка $c \in (a,b)$ такая, что
\[
f'(c) = 0.
\]

\medskip

% 3. Основные идеи доказательств (коротко)

\textbf{Идея доказательства Теоремы Ферма.}
\begin{itemize}
  \item Предположим, $x_0$ — точка локального минимума. Тогда для $x$ вблизи $x_0$ имеем $f(x)\ge f(x_0)$.  
  \item Рассматриваем разность $\frac{f(x)-f(x_0)}{x - x_0}$ при $x > x_0$ и при $x < x_0$.  
  \item Переходя к пределу $x\to x_0^+$ и $x\to x_0^-$, получаем $f'(x_0)\ge0$ и $f'(x_0)\le0$ соответственно. Значит $f'(x_0)=0$.  
  \item Для локального максимума аналогично.
\end{itemize}

\textbf{Идея доказательства Теоремы Ролля.}
\begin{itemize}
  \item Если $f$ постоянна на $[a,b]$, то $f'(x)=0$ на $(a,b)$ и теорема доказана.  
  \item Если $f$ не постоянна, по \textbf{теореме о достижении экстремума} (см. \texttt{sub\_2.tex}) она достигает минимума и максимума в некотором $x_{\min}, x_{\max}\in[a,b]$.  
  \item Поскольку $f(a)=f(b)$, хотя бы один из экстремумов не может быть только на концах. Тогда внутри $(a,b)$ есть точка локального экстремума $c$.  
  \item По Теореме Ферма, $f'(c)=0$.
\end{itemize}

\medskip

% 4. Полное доказательство (шаги, ссылаясь на определения)

\textbf{Доказательство Теоремы Ферма (пошагово):}
\begin{enumerate}
  \item Пусть $x_0$ — точка локального минимума, то есть существует $\delta>0$ такое, что при $|x-x_0|<\delta$, $f(x)\ge f(x_0)$.  
  \item Для $x>x_0$ рассмотрим $\frac{f(x)-f(x_0)}{x - x_0}\ge0$. При переходе $x\to x_0^+$, этот предел есть $f'(x_0)\ge0$.  
  \item Для $x<x_0$ аналогично, но тогда $x - x_0<0$, и из неравенства $f(x)\ge f(x_0)$ получаем $f'(x_0)\le0$.  
  \item Значит $f'(x_0)\ge0$ и $f'(x_0)\le0$, откуда $f'(x_0)=0$.  
  \item Случай локального максимума разбирается аналогично (только знак меняется).
\end{enumerate}

\textbf{Доказательство Теоремы Ролля (пошагово):}
\begin{enumerate}
  \item Если $f$ постоянна на $[a,b]$, то $f'(x)=0$ на всём $(a,b)$, и точка $c$ может быть любая.  
  \item Иначе $f$ непостоянна и, благодаря непрерывности на $[a,b]$, достигает минимума и максимума (Теорема Вейерштрасса, см. \texttt{sub\_2.tex}).  
  \item Пусть $x_{\min}$ и $x_{\max}$ — точки, где достигаются минимум и максимум. Поскольку $f(a)=f(b)$, как минимум одно из этих значений будет «внутренним» для отрезка, иначе функция была бы постоянно равна этому значению. Значит есть $c\in(a,b)$ — точка локального экстремума.  
  \item По Теореме Ферма, в точке локального экстремума $c$ имеем $f'(c)=0$.  
  \item Следовательно, нашли искомую точку $c\in(a,b)$, что доказывает Теорему Ролля.
\end{enumerate}

\medskip

% 5. Примеры

\textbf{Пример (дифференциал).}
Для $f(x)=x^2$, в точке $x_0=2$ имеем $f'(2)=4$. Тогда малому приращению $dx$ соответствует $df(2)=4\,dx$. Если $dx=0.1$, то $df(2)=0.4$, а реальное $f(2.1)-f(2)=4.41-4=0.41$, что близко к $0.4$.

\textbf{Пример (Теорема Ферма).}
Функция $f(x)=x^2$ имеет локальный минимум в $x=0$, причём $f'(0)=0$.

\textbf{Пример (Теорема Ролля).}
На отрезке $[0,4]$ возьмём $f(x)=x^2 - 4x$. Тогда $f(0)=f(4)=0$, $f$ непрерывна на $[0,4]$ и дифференцируема на $(0,4)$. По Теореме Ролля существует $c\in(0,4)$ с $f'(c)=0$. Действительно, $f'(x)=2x-4$, отсюда $c=2$.

\medskip

% -- Логическая связность --
% (Все определения локального экстремума, непрерывности,
% теоремы Вейерштрасса - во вспомогательном sub_2.tex)

