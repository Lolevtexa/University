% ==================
% answer.tex
% (Теоремы, планы доказательств, примеры по теме
% "Вывод рядов Тейлора для e^x, sin x, cos x через Т. Лагранжа. Формула Эйлера.")
% ==================

\begin{customtheorem}[Формула Тейлора (с остатком в форме Лагранжа)]
	Пусть функция $f$ имеет $(n+1)$-ю производную в окрестности $0$. Тогда для $x$
	из этой окрестности справедливо разложение:
	\[
		f(x)
		\;=\;
		f(0)
		+ \frac{f'(0)}{1!}\,x
		+ \frac{f''(0)}{2!}\,x^2
		+ \dots
		+ \frac{f^{(n)}(0)}{n!}\,x^n
		\;+\;
		\underbrace{
		\frac{f^{(n+1)}(\xi)}{(n+1)!}\,x^{n+1}
		}_{R_{n}(x)},
	\]
	где $\xi$ — некоторая точка между $0$ и $x$.
	Это и называют \emph{рядом Тейлора (Маклорена)} с остаточным членом в форме Лагранжа.
\end{customtheorem}

\begin{proofplan}
	\begin{enumerate}
		\item Рассмотреть многочлен Тейлора $P_n(x)$, равный сумме первых $n+1$ членов (то есть до $x^n$).
		\item Применить теорему Лагранжа к функции $f(x)-P_n(x)$ на отрезке $[0,x]$.
		\item Показать, что «остаток» $R_n(x)$ принимает вид
		      $\tfrac{f^{(n+1)}(\xi)}{(n+1)!}\,x^{n+1}$.
		\item Подставляя в эту схему конкретные $f(x)$ (как $e^x$, $\sin x$, $\cos x$), получаем соответствующие ряды.
	\end{enumerate}
\end{proofplan}

\begin{customproof}
	Пусть $f$ удовлетворяет условиям теоремы (все производные до порядка $n+1$
	определены и непрерывны в окрестности $0$). Определим
	\[
		P_n(x) \;=\; f(0)
		+ \frac{f'(0)}{1!}\,x
		+ \dots
		+ \frac{f^{(n)}(0)}{n!}\,x^n.
	\]
	Тогда рассмотрим на промежутке $[0,x]$ (предполагая $x>0$ для определённости) функцию
	\[
		g(t) \;=\; f(t) - P_n(t).
	\]
	Заметим, что $g(0)=0$. По построению, $g$ непрерывна и дифференцируема.
	Применим теорему Лагранжа (о среднем значении): существует $\xi\in(0,x)$ такое, что
	\[
		g(x) - g(0) \;=\; g'(\xi)\,\bigl(x - 0\bigr).
	\]
	Так как $g(0)=0$, получаем
	\[
		g(x) \;=\; g'(\xi)\,x.
	\]
	Но
	\[
		g'(t)
		\;=\; f'(t)
		- \bigl[\tfrac{d}{dt}P_n(t)\bigr]
		\;=\; f'(t)
		- \Bigl[
		f'(0)
		+ \dots
		+ \tfrac{f^{(n)}(0)}{(n-1)!}\,t^{n-1}
		\Bigr].
	\]
	Ещё раз применяя теорему Лагранжа к разности $f'(t)$ и многочлена производных,
	доказывают, что это выражение сводится к $\tfrac{f^{(n+1)}(\eta)}{n!}\,t^n$ (или детально, если нужно).
	Таким образом, окончательно получается
	\[
		f(x) - P_n(x)
		\;=\;
		\frac{f^{(n+1)}(\xi)}{(n+1)!}\,x^{n+1}.
	\]
	Это и есть остаток $R_n(x)$.
	В итоге,
	\[
		f(x)
		\;=\;
		P_n(x) + R_n(x),
		\quad
		R_n(x) = \frac{f^{(n+1)}(\xi)}{(n+1)!}\,x^{n+1}.
	\]
\end{customproof}

\begin{customexample}
	\textbf{Применение к $e^x$, $\sin x$, $\cos x$.}
	\begin{itemize}
		\item \textbf{Функция $f(x)=e^x$.}
		      Все её производные равны $e^x$, значит в точке 0 они все равны 1.
		      Отсюда ряд:
		      \[
			      e^x
			      \;=\;
			      1
			      + x
			      + \frac{x^2}{2!}
			      + \dots
			      + \frac{x^n}{n!}
			      + R_n(x),
		      \]
		      где $R_n(x)=\frac{e^\xi}{(n+1)!}\,x^{n+1}$ с некоторой $\xi\in(0,x)$.
		\item \textbf{Функция $f(x)=\sin x$.}
		      Производные цикличны, на $x=0$ берут значения $0$, $1$, $0$, $-1$, ... .
		      Получаем
		      \[
			      \sin x
			      = x - \frac{x^3}{3!} + \frac{x^5}{5!} - \dots + R_n(x).
		      \]
		\item \textbf{Функция $f(x)=\cos x$.}
		      Аналогично,
		      \[
			      \cos x
			      = 1 - \frac{x^2}{2!} + \frac{x^4}{4!} - \dots + R_n(x).
		      \]
	\end{itemize}
\end{customexample}

\begin{customexample}
	\textbf{Формула Эйлера.}
	\[
		e^{i x} = \cos x + i\,\sin x.
	\]
	Если в ряде для $e^x$ подставить $x\mapsto i x$, раскрыв степенные множители $i^n$ (где $i^2=-1$, $i^3=-i$, и т.д.), происходит естественное разбиение на действительную часть (совпадающую с рядом $\cos x$) и мнимую часть (совпадающую с рядом $\sin x$).
	Это даёт знаменитую тождественность Эйлера.
\end{customexample}
