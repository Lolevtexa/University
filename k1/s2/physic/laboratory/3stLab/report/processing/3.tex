\subsection{Оценка погрешности \(\bar a_i\)}

По каждому цвету \(i\) (красный, зелёный, фиолетовый) собрана выборка из \(N=6\) значений параметра
\[
	a_{i,m} = \frac{\sin\varphi_{i,m}}{m}\,, \quad m=\pm1,\pm2,\pm3.
\]

Для оценки результата косвенного измерения \(a_i\) и его погрешности при доверительной
вероятности \(P=95\%\) используем выборочный метод:
\[
	\bar a_i = \frac{1}{N}\sum_{k=1}^N a_{i,k},
	\qquad
	S_i = \sqrt{\frac{1}{N-1}\sum_{k=1}^N\bigl(a_{i,k}-\bar a_i\bigr)^2},
\]
\[
	\Delta\bar a_i = t_{0.95,\,N-1}\,\frac{S_i}{\sqrt{N}}.
\]

Затем результат оформляется в виде
\(\displaystyle a_i = \bar a_i \pm \Delta\bar a_i\)
и заносится в Таблицу 5.
