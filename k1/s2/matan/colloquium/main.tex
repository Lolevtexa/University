\documentclass[12pt,a4paper]{article}
\usepackage[utf8]{inputenc}
\usepackage[T2A]{fontenc}
\usepackage[russian]{babel}
\usepackage{amsmath, amsthm, amssymb}
\usepackage[left=2cm,right=2cm,top=2cm,bottom=2cm]{geometry}
\usepackage{indentfirst}
\usepackage{hyperref}

% --- Здесь объявляем все окружения ---
\newtheorem{definition}{Определение}[section]

\makeatletter
\newenvironment{customtheorem}[1][]{%
  \par\noindent\textbf{Теорема}%
  % Если пользователь написал \begin{customtheorem}[НАЗВАНИЕ],
  % то #1 НЕ пустая; если пустая -- не выводим ничего.
  \@ifempty{#1}{}{\,(\textit{#1})}%
  .\par % точка, перенос строки + пол-эм отступ
}{%
  \par\bigskip % небольшой вертикальный отступ в конце
}
\makeatother

\newenvironment{customexample}{%
  \par\noindent\textbf{Пример.}\par%
}{%
  \par\bigskip
}

\newenvironment{proofplan}{%
  \par\noindent\textbf{План доказательства.}\par%
}{%
  \hfill$\square$\par\bigskip
}

\newenvironment{customproof}{%
  \par\noindent\textbf{Доказательство.}\par%
}{%
  \hfill$\blacksquare$\par\bigskip
}
% -------------------------------------

\begin{document}

\tableofcontents
\newpage

\section{Равномерная непрерывность. Примеры. Теорема Кантора}

\subsection*{Вспомогательные понятия}
% ==================
% auxiliary.tex
% (Вспомогательные определения для темы:
%  "Достаточные условия существования экстремума (по второй производной).")
% ==================

\paragraph{Локальный минимум и максимум.}
Точка $x_0$ внутри промежутка $(a,b)$ называется \emph{локальным минимумом} функции $f$,
если существует $\delta>0$ такое, что при $|x - x_0| < \delta$
выполняется $f(x) \ge f(x_0)$.
Аналогично, $x_0$ называется \emph{локальным максимумом},
если в некоторой окрестности $x_0$ верно $f(x) \le f(x_0)$.

\bigskip

\paragraph{Вторая производная.}
Пусть $f$ дифференцируема на интервале $(a,b)$, и $f'(x)$ тоже дифференцируема на $(a,b)$.
Тогда в точках, где это возможно, определена \emph{вторая производная} $f''(x)=\bigl(f'(x)\bigr)'$.

\bigskip

\paragraph{Теорема Ферма (напоминание).}
Если функция $f$ дифференцируема в точке $x_0$ и имеет там локальный минимум или максимум,
то $f'(x_0)=0$.
Это — необходимое условие экстремума (без второй производной).


\subsection*{Ответ на вопрос}
% ==================
% answer.tex
% (Основной материал:
%  "Формула Стирлинга (с эквивалентностью).")
% ==================

\begin{customtheorem}[Формула Стирлинга (эквивалентность)]
	При $n\to\infty$ справедливо
	\[
		n! \;\sim\;\sqrt{2\pi\,n}
		\,\Bigl(\tfrac{n}{e}\Bigr)^n
		\quad\Longleftrightarrow\quad
		\lim_{n\to\infty} \frac{n!}{\sqrt{2\pi\,n}\,\bigl(\tfrac{n}{e}\bigr)^n}=1.
	\]
\end{customtheorem}

\begin{proofplan}
	\begin{enumerate}
		\item Рассмотреть логарифм факториала: $\ln(n!)=\sum_{k=1}^n \ln k$.
		\item Сравнить $\sum_{k=1}^n \ln k$ с интегралом $\int_{1}^{n}\ln x\,dx$,
		      получить приближение $n\ln n - n + 1$ (плюс поправка).
		\item Использовать более точный учёт (например, формулу Эйлера–Маклорена) для
		      уточнения поправки: $\tfrac12\ln n + O(1)$.
		\item Экспоненцировать результат, получая
		      $n! = \sqrt{2\pi n}\,\bigl(\frac{n}{e}\bigr)^n \,\bigl(1+o(1)\bigr).$
	\end{enumerate}
\end{proofplan}

\begin{customproof}
	\textbf{Шаг 1: Логарифмы.}
	Пусть $L_n=\ln(n!)=\sum_{k=1}^n \ln k.$

	\smallskip

	\textbf{Шаг 2: Сравнение с интегралом.}
	Замечаем, что
	\[
		\sum_{k=1}^n \ln k
		\;\approx\;
		\int_{1}^{n} \ln x \,dx
		\;=\; n\ln n - n + 1.
	\]
	Разница между суммой и интегралом даёт эффект порядка $\ln(n)$.

	\smallskip

	\textbf{Шаг 3: Уточнение (Эйлера–Маклорена).}
	Более детальный анализ (или полная формула Эйлера–Маклорена) показывает:
	\[
		\ln(n!)
		\;=\;
		n\ln n \;-\; n
		\;+\;\tfrac12\ln(n)
		\;+\; O(1).
	\]
	Иными словами,
	\[
		L_n
		= n\ln n - n + \tfrac12\ln n + O(1).
	\]

	\smallskip

	\textbf{Шаг 4: Экспоненцирование.}
	Тогда
	\[
		n!
		=
		\exp(L_n)
		=
		\exp\bigl(n\ln n - n + \tfrac12\ln n + O(1)\bigr)
		=
		\sqrt{n}\,\Bigl(\tfrac{n}{e}\Bigr)^n \exp(O(1)).
	\]
	Поскольку $\exp(O(1))$ означает некий постоянный множитель в пределе,
	тщательный учёт показывает, что этот множитель есть $\sqrt{2\pi}$, т. е.
	\[
		n! \;\sim\;\sqrt{2\pi\,n}\,\Bigl(\tfrac{n}{e}\Bigr)^n.
	\]
	Следовательно, при $n\to\infty$ факториал $n!$ эквивалентен $\sqrt{2\pi\,n}\,\bigl(n/e\bigr)^n$.
\end{customproof}

\begin{customexample}
	\textbf{Сравнение значений.}
	Уже при $n=10$, $10! = 3\,628\,800$, а по формуле Стирлинга
	$\sqrt{2\pi\cdot 10}\,\bigl(\tfrac{10}{e}\bigr)^{10} \approx 3\,598\,695$,
	что даёт небольшое расхождение. С ростом $n$ относительная ошибка убывает очень быстро.
\end{customexample}



\newpage
\section{Дифференциал функции. Теорема Ферма. Теорема Ролля. Примеры.}

\subsection*{Вспомогательные понятия}
% ==================
% auxiliary.tex
% (Вспомогательные определения для темы:
%  "Достаточные условия существования экстремума (по второй производной).")
% ==================

\paragraph{Локальный минимум и максимум.}
Точка $x_0$ внутри промежутка $(a,b)$ называется \emph{локальным минимумом} функции $f$,
если существует $\delta>0$ такое, что при $|x - x_0| < \delta$
выполняется $f(x) \ge f(x_0)$.
Аналогично, $x_0$ называется \emph{локальным максимумом},
если в некоторой окрестности $x_0$ верно $f(x) \le f(x_0)$.

\bigskip

\paragraph{Вторая производная.}
Пусть $f$ дифференцируема на интервале $(a,b)$, и $f'(x)$ тоже дифференцируема на $(a,b)$.
Тогда в точках, где это возможно, определена \emph{вторая производная} $f''(x)=\bigl(f'(x)\bigr)'$.

\bigskip

\paragraph{Теорема Ферма (напоминание).}
Если функция $f$ дифференцируема в точке $x_0$ и имеет там локальный минимум или максимум,
то $f'(x_0)=0$.
Это — необходимое условие экстремума (без второй производной).


\subsection*{Ответ на вопрос}
% ==================
% answer.tex
% (Основной материал:
%  "Формула Стирлинга (с эквивалентностью).")
% ==================

\begin{customtheorem}[Формула Стирлинга (эквивалентность)]
	При $n\to\infty$ справедливо
	\[
		n! \;\sim\;\sqrt{2\pi\,n}
		\,\Bigl(\tfrac{n}{e}\Bigr)^n
		\quad\Longleftrightarrow\quad
		\lim_{n\to\infty} \frac{n!}{\sqrt{2\pi\,n}\,\bigl(\tfrac{n}{e}\bigr)^n}=1.
	\]
\end{customtheorem}

\begin{proofplan}
	\begin{enumerate}
		\item Рассмотреть логарифм факториала: $\ln(n!)=\sum_{k=1}^n \ln k$.
		\item Сравнить $\sum_{k=1}^n \ln k$ с интегралом $\int_{1}^{n}\ln x\,dx$,
		      получить приближение $n\ln n - n + 1$ (плюс поправка).
		\item Использовать более точный учёт (например, формулу Эйлера–Маклорена) для
		      уточнения поправки: $\tfrac12\ln n + O(1)$.
		\item Экспоненцировать результат, получая
		      $n! = \sqrt{2\pi n}\,\bigl(\frac{n}{e}\bigr)^n \,\bigl(1+o(1)\bigr).$
	\end{enumerate}
\end{proofplan}

\begin{customproof}
	\textbf{Шаг 1: Логарифмы.}
	Пусть $L_n=\ln(n!)=\sum_{k=1}^n \ln k.$

	\smallskip

	\textbf{Шаг 2: Сравнение с интегралом.}
	Замечаем, что
	\[
		\sum_{k=1}^n \ln k
		\;\approx\;
		\int_{1}^{n} \ln x \,dx
		\;=\; n\ln n - n + 1.
	\]
	Разница между суммой и интегралом даёт эффект порядка $\ln(n)$.

	\smallskip

	\textbf{Шаг 3: Уточнение (Эйлера–Маклорена).}
	Более детальный анализ (или полная формула Эйлера–Маклорена) показывает:
	\[
		\ln(n!)
		\;=\;
		n\ln n \;-\; n
		\;+\;\tfrac12\ln(n)
		\;+\; O(1).
	\]
	Иными словами,
	\[
		L_n
		= n\ln n - n + \tfrac12\ln n + O(1).
	\]

	\smallskip

	\textbf{Шаг 4: Экспоненцирование.}
	Тогда
	\[
		n!
		=
		\exp(L_n)
		=
		\exp\bigl(n\ln n - n + \tfrac12\ln n + O(1)\bigr)
		=
		\sqrt{n}\,\Bigl(\tfrac{n}{e}\Bigr)^n \exp(O(1)).
	\]
	Поскольку $\exp(O(1))$ означает некий постоянный множитель в пределе,
	тщательный учёт показывает, что этот множитель есть $\sqrt{2\pi}$, т. е.
	\[
		n! \;\sim\;\sqrt{2\pi\,n}\,\Bigl(\tfrac{n}{e}\Bigr)^n.
	\]
	Следовательно, при $n\to\infty$ факториал $n!$ эквивалентен $\sqrt{2\pi\,n}\,\bigl(n/e\bigr)^n$.
\end{customproof}

\begin{customexample}
	\textbf{Сравнение значений.}
	Уже при $n=10$, $10! = 3\,628\,800$, а по формуле Стирлинга
	$\sqrt{2\pi\cdot 10}\,\bigl(\tfrac{10}{e}\bigr)^{10} \approx 3\,598\,695$,
	что даёт небольшое расхождение. С ростом $n$ относительная ошибка убывает очень быстро.
\end{customexample}



\newpage
\section{Теорема Лагранжа. Необходимoе и достаточное условие постоянства дифференцируемой функции на промежутке. Необходимое и достаточное условие монотонности дифференцируемой функции на промежутке.}

\subsection*{Вспомогательные понятия}
% ==================
% auxiliary.tex
% (Вспомогательные определения для темы:
%  "Достаточные условия существования экстремума (по второй производной).")
% ==================

\paragraph{Локальный минимум и максимум.}
Точка $x_0$ внутри промежутка $(a,b)$ называется \emph{локальным минимумом} функции $f$,
если существует $\delta>0$ такое, что при $|x - x_0| < \delta$
выполняется $f(x) \ge f(x_0)$.
Аналогично, $x_0$ называется \emph{локальным максимумом},
если в некоторой окрестности $x_0$ верно $f(x) \le f(x_0)$.

\bigskip

\paragraph{Вторая производная.}
Пусть $f$ дифференцируема на интервале $(a,b)$, и $f'(x)$ тоже дифференцируема на $(a,b)$.
Тогда в точках, где это возможно, определена \emph{вторая производная} $f''(x)=\bigl(f'(x)\bigr)'$.

\bigskip

\paragraph{Теорема Ферма (напоминание).}
Если функция $f$ дифференцируема в точке $x_0$ и имеет там локальный минимум или максимум,
то $f'(x_0)=0$.
Это — необходимое условие экстремума (без второй производной).


\subsection*{Ответ на вопрос}
% ==================
% answer.tex
% (Основной материал:
%  "Формула Стирлинга (с эквивалентностью).")
% ==================

\begin{customtheorem}[Формула Стирлинга (эквивалентность)]
	При $n\to\infty$ справедливо
	\[
		n! \;\sim\;\sqrt{2\pi\,n}
		\,\Bigl(\tfrac{n}{e}\Bigr)^n
		\quad\Longleftrightarrow\quad
		\lim_{n\to\infty} \frac{n!}{\sqrt{2\pi\,n}\,\bigl(\tfrac{n}{e}\bigr)^n}=1.
	\]
\end{customtheorem}

\begin{proofplan}
	\begin{enumerate}
		\item Рассмотреть логарифм факториала: $\ln(n!)=\sum_{k=1}^n \ln k$.
		\item Сравнить $\sum_{k=1}^n \ln k$ с интегралом $\int_{1}^{n}\ln x\,dx$,
		      получить приближение $n\ln n - n + 1$ (плюс поправка).
		\item Использовать более точный учёт (например, формулу Эйлера–Маклорена) для
		      уточнения поправки: $\tfrac12\ln n + O(1)$.
		\item Экспоненцировать результат, получая
		      $n! = \sqrt{2\pi n}\,\bigl(\frac{n}{e}\bigr)^n \,\bigl(1+o(1)\bigr).$
	\end{enumerate}
\end{proofplan}

\begin{customproof}
	\textbf{Шаг 1: Логарифмы.}
	Пусть $L_n=\ln(n!)=\sum_{k=1}^n \ln k.$

	\smallskip

	\textbf{Шаг 2: Сравнение с интегралом.}
	Замечаем, что
	\[
		\sum_{k=1}^n \ln k
		\;\approx\;
		\int_{1}^{n} \ln x \,dx
		\;=\; n\ln n - n + 1.
	\]
	Разница между суммой и интегралом даёт эффект порядка $\ln(n)$.

	\smallskip

	\textbf{Шаг 3: Уточнение (Эйлера–Маклорена).}
	Более детальный анализ (или полная формула Эйлера–Маклорена) показывает:
	\[
		\ln(n!)
		\;=\;
		n\ln n \;-\; n
		\;+\;\tfrac12\ln(n)
		\;+\; O(1).
	\]
	Иными словами,
	\[
		L_n
		= n\ln n - n + \tfrac12\ln n + O(1).
	\]

	\smallskip

	\textbf{Шаг 4: Экспоненцирование.}
	Тогда
	\[
		n!
		=
		\exp(L_n)
		=
		\exp\bigl(n\ln n - n + \tfrac12\ln n + O(1)\bigr)
		=
		\sqrt{n}\,\Bigl(\tfrac{n}{e}\Bigr)^n \exp(O(1)).
	\]
	Поскольку $\exp(O(1))$ означает некий постоянный множитель в пределе,
	тщательный учёт показывает, что этот множитель есть $\sqrt{2\pi}$, т. е.
	\[
		n! \;\sim\;\sqrt{2\pi\,n}\,\Bigl(\tfrac{n}{e}\Bigr)^n.
	\]
	Следовательно, при $n\to\infty$ факториал $n!$ эквивалентен $\sqrt{2\pi\,n}\,\bigl(n/e\bigr)^n$.
\end{customproof}

\begin{customexample}
	\textbf{Сравнение значений.}
	Уже при $n=10$, $10! = 3\,628\,800$, а по формуле Стирлинга
	$\sqrt{2\pi\cdot 10}\,\bigl(\tfrac{10}{e}\bigr)^{10} \approx 3\,598\,695$,
	что даёт небольшое расхождение. С ростом $n$ относительная ошибка убывает очень быстро.
\end{customexample}


\newpage
\section{Равномерная непрерывность. Примеры. Теорема Кантора о равномерной непрерывности.}

\subsection*{Вспомогательные понятия}
% ==================
% auxiliary.tex
% (Вспомогательные определения для темы:
%  "Достаточные условия существования экстремума (по второй производной).")
% ==================

\paragraph{Локальный минимум и максимум.}
Точка $x_0$ внутри промежутка $(a,b)$ называется \emph{локальным минимумом} функции $f$,
если существует $\delta>0$ такое, что при $|x - x_0| < \delta$
выполняется $f(x) \ge f(x_0)$.
Аналогично, $x_0$ называется \emph{локальным максимумом},
если в некоторой окрестности $x_0$ верно $f(x) \le f(x_0)$.

\bigskip

\paragraph{Вторая производная.}
Пусть $f$ дифференцируема на интервале $(a,b)$, и $f'(x)$ тоже дифференцируема на $(a,b)$.
Тогда в точках, где это возможно, определена \emph{вторая производная} $f''(x)=\bigl(f'(x)\bigr)'$.

\bigskip

\paragraph{Теорема Ферма (напоминание).}
Если функция $f$ дифференцируема в точке $x_0$ и имеет там локальный минимум или максимум,
то $f'(x_0)=0$.
Это — необходимое условие экстремума (без второй производной).


\subsection*{Ответ на вопрос}
% ==================
% answer.tex
% (Основной материал:
%  "Формула Стирлинга (с эквивалентностью).")
% ==================

\begin{customtheorem}[Формула Стирлинга (эквивалентность)]
	При $n\to\infty$ справедливо
	\[
		n! \;\sim\;\sqrt{2\pi\,n}
		\,\Bigl(\tfrac{n}{e}\Bigr)^n
		\quad\Longleftrightarrow\quad
		\lim_{n\to\infty} \frac{n!}{\sqrt{2\pi\,n}\,\bigl(\tfrac{n}{e}\bigr)^n}=1.
	\]
\end{customtheorem}

\begin{proofplan}
	\begin{enumerate}
		\item Рассмотреть логарифм факториала: $\ln(n!)=\sum_{k=1}^n \ln k$.
		\item Сравнить $\sum_{k=1}^n \ln k$ с интегралом $\int_{1}^{n}\ln x\,dx$,
		      получить приближение $n\ln n - n + 1$ (плюс поправка).
		\item Использовать более точный учёт (например, формулу Эйлера–Маклорена) для
		      уточнения поправки: $\tfrac12\ln n + O(1)$.
		\item Экспоненцировать результат, получая
		      $n! = \sqrt{2\pi n}\,\bigl(\frac{n}{e}\bigr)^n \,\bigl(1+o(1)\bigr).$
	\end{enumerate}
\end{proofplan}

\begin{customproof}
	\textbf{Шаг 1: Логарифмы.}
	Пусть $L_n=\ln(n!)=\sum_{k=1}^n \ln k.$

	\smallskip

	\textbf{Шаг 2: Сравнение с интегралом.}
	Замечаем, что
	\[
		\sum_{k=1}^n \ln k
		\;\approx\;
		\int_{1}^{n} \ln x \,dx
		\;=\; n\ln n - n + 1.
	\]
	Разница между суммой и интегралом даёт эффект порядка $\ln(n)$.

	\smallskip

	\textbf{Шаг 3: Уточнение (Эйлера–Маклорена).}
	Более детальный анализ (или полная формула Эйлера–Маклорена) показывает:
	\[
		\ln(n!)
		\;=\;
		n\ln n \;-\; n
		\;+\;\tfrac12\ln(n)
		\;+\; O(1).
	\]
	Иными словами,
	\[
		L_n
		= n\ln n - n + \tfrac12\ln n + O(1).
	\]

	\smallskip

	\textbf{Шаг 4: Экспоненцирование.}
	Тогда
	\[
		n!
		=
		\exp(L_n)
		=
		\exp\bigl(n\ln n - n + \tfrac12\ln n + O(1)\bigr)
		=
		\sqrt{n}\,\Bigl(\tfrac{n}{e}\Bigr)^n \exp(O(1)).
	\]
	Поскольку $\exp(O(1))$ означает некий постоянный множитель в пределе,
	тщательный учёт показывает, что этот множитель есть $\sqrt{2\pi}$, т. е.
	\[
		n! \;\sim\;\sqrt{2\pi\,n}\,\Bigl(\tfrac{n}{e}\Bigr)^n.
	\]
	Следовательно, при $n\to\infty$ факториал $n!$ эквивалентен $\sqrt{2\pi\,n}\,\bigl(n/e\bigr)^n$.
\end{customproof}

\begin{customexample}
	\textbf{Сравнение значений.}
	Уже при $n=10$, $10! = 3\,628\,800$, а по формуле Стирлинга
	$\sqrt{2\pi\cdot 10}\,\bigl(\tfrac{10}{e}\bigr)^{10} \approx 3\,598\,695$,
	что даёт небольшое расхождение. С ростом $n$ относительная ошибка убывает очень быстро.
\end{customexample}


\newpage
\section{Вывод рядов Тейлора для функций y=exp(x), y=sinx, y=cosx через следствие из теоремы Лагранжа. Формула Эйлера.}

\subsection*{Вспомогательные понятия}
% ==================
% auxiliary.tex
% (Вспомогательные определения для темы:
%  "Достаточные условия существования экстремума (по второй производной).")
% ==================

\paragraph{Локальный минимум и максимум.}
Точка $x_0$ внутри промежутка $(a,b)$ называется \emph{локальным минимумом} функции $f$,
если существует $\delta>0$ такое, что при $|x - x_0| < \delta$
выполняется $f(x) \ge f(x_0)$.
Аналогично, $x_0$ называется \emph{локальным максимумом},
если в некоторой окрестности $x_0$ верно $f(x) \le f(x_0)$.

\bigskip

\paragraph{Вторая производная.}
Пусть $f$ дифференцируема на интервале $(a,b)$, и $f'(x)$ тоже дифференцируема на $(a,b)$.
Тогда в точках, где это возможно, определена \emph{вторая производная} $f''(x)=\bigl(f'(x)\bigr)'$.

\bigskip

\paragraph{Теорема Ферма (напоминание).}
Если функция $f$ дифференцируема в точке $x_0$ и имеет там локальный минимум или максимум,
то $f'(x_0)=0$.
Это — необходимое условие экстремума (без второй производной).


\subsection*{Ответ на вопрос}
% ==================
% answer.tex
% (Основной материал:
%  "Формула Стирлинга (с эквивалентностью).")
% ==================

\begin{customtheorem}[Формула Стирлинга (эквивалентность)]
	При $n\to\infty$ справедливо
	\[
		n! \;\sim\;\sqrt{2\pi\,n}
		\,\Bigl(\tfrac{n}{e}\Bigr)^n
		\quad\Longleftrightarrow\quad
		\lim_{n\to\infty} \frac{n!}{\sqrt{2\pi\,n}\,\bigl(\tfrac{n}{e}\bigr)^n}=1.
	\]
\end{customtheorem}

\begin{proofplan}
	\begin{enumerate}
		\item Рассмотреть логарифм факториала: $\ln(n!)=\sum_{k=1}^n \ln k$.
		\item Сравнить $\sum_{k=1}^n \ln k$ с интегралом $\int_{1}^{n}\ln x\,dx$,
		      получить приближение $n\ln n - n + 1$ (плюс поправка).
		\item Использовать более точный учёт (например, формулу Эйлера–Маклорена) для
		      уточнения поправки: $\tfrac12\ln n + O(1)$.
		\item Экспоненцировать результат, получая
		      $n! = \sqrt{2\pi n}\,\bigl(\frac{n}{e}\bigr)^n \,\bigl(1+o(1)\bigr).$
	\end{enumerate}
\end{proofplan}

\begin{customproof}
	\textbf{Шаг 1: Логарифмы.}
	Пусть $L_n=\ln(n!)=\sum_{k=1}^n \ln k.$

	\smallskip

	\textbf{Шаг 2: Сравнение с интегралом.}
	Замечаем, что
	\[
		\sum_{k=1}^n \ln k
		\;\approx\;
		\int_{1}^{n} \ln x \,dx
		\;=\; n\ln n - n + 1.
	\]
	Разница между суммой и интегралом даёт эффект порядка $\ln(n)$.

	\smallskip

	\textbf{Шаг 3: Уточнение (Эйлера–Маклорена).}
	Более детальный анализ (или полная формула Эйлера–Маклорена) показывает:
	\[
		\ln(n!)
		\;=\;
		n\ln n \;-\; n
		\;+\;\tfrac12\ln(n)
		\;+\; O(1).
	\]
	Иными словами,
	\[
		L_n
		= n\ln n - n + \tfrac12\ln n + O(1).
	\]

	\smallskip

	\textbf{Шаг 4: Экспоненцирование.}
	Тогда
	\[
		n!
		=
		\exp(L_n)
		=
		\exp\bigl(n\ln n - n + \tfrac12\ln n + O(1)\bigr)
		=
		\sqrt{n}\,\Bigl(\tfrac{n}{e}\Bigr)^n \exp(O(1)).
	\]
	Поскольку $\exp(O(1))$ означает некий постоянный множитель в пределе,
	тщательный учёт показывает, что этот множитель есть $\sqrt{2\pi}$, т. е.
	\[
		n! \;\sim\;\sqrt{2\pi\,n}\,\Bigl(\tfrac{n}{e}\Bigr)^n.
	\]
	Следовательно, при $n\to\infty$ факториал $n!$ эквивалентен $\sqrt{2\pi\,n}\,\bigl(n/e\bigr)^n$.
\end{customproof}

\begin{customexample}
	\textbf{Сравнение значений.}
	Уже при $n=10$, $10! = 3\,628\,800$, а по формуле Стирлинга
	$\sqrt{2\pi\cdot 10}\,\bigl(\tfrac{10}{e}\bigr)^{10} \approx 3\,598\,695$,
	что даёт небольшое расхождение. С ростом $n$ относительная ошибка убывает очень быстро.
\end{customexample}


\newpage
\section{Теорема Коши. Правило Лопиталя (доказательство – только для случая 0/0). Примеры, когда правило неприменимо.}

\subsection*{Вспомогательные понятия}
% ==================
% auxiliary.tex
% (Вспомогательные определения для темы:
%  "Достаточные условия существования экстремума (по второй производной).")
% ==================

\paragraph{Локальный минимум и максимум.}
Точка $x_0$ внутри промежутка $(a,b)$ называется \emph{локальным минимумом} функции $f$,
если существует $\delta>0$ такое, что при $|x - x_0| < \delta$
выполняется $f(x) \ge f(x_0)$.
Аналогично, $x_0$ называется \emph{локальным максимумом},
если в некоторой окрестности $x_0$ верно $f(x) \le f(x_0)$.

\bigskip

\paragraph{Вторая производная.}
Пусть $f$ дифференцируема на интервале $(a,b)$, и $f'(x)$ тоже дифференцируема на $(a,b)$.
Тогда в точках, где это возможно, определена \emph{вторая производная} $f''(x)=\bigl(f'(x)\bigr)'$.

\bigskip

\paragraph{Теорема Ферма (напоминание).}
Если функция $f$ дифференцируема в точке $x_0$ и имеет там локальный минимум или максимум,
то $f'(x_0)=0$.
Это — необходимое условие экстремума (без второй производной).


\subsection*{Ответ на вопрос}
% ==================
% answer.tex
% (Основной материал:
%  "Формула Стирлинга (с эквивалентностью).")
% ==================

\begin{customtheorem}[Формула Стирлинга (эквивалентность)]
	При $n\to\infty$ справедливо
	\[
		n! \;\sim\;\sqrt{2\pi\,n}
		\,\Bigl(\tfrac{n}{e}\Bigr)^n
		\quad\Longleftrightarrow\quad
		\lim_{n\to\infty} \frac{n!}{\sqrt{2\pi\,n}\,\bigl(\tfrac{n}{e}\bigr)^n}=1.
	\]
\end{customtheorem}

\begin{proofplan}
	\begin{enumerate}
		\item Рассмотреть логарифм факториала: $\ln(n!)=\sum_{k=1}^n \ln k$.
		\item Сравнить $\sum_{k=1}^n \ln k$ с интегралом $\int_{1}^{n}\ln x\,dx$,
		      получить приближение $n\ln n - n + 1$ (плюс поправка).
		\item Использовать более точный учёт (например, формулу Эйлера–Маклорена) для
		      уточнения поправки: $\tfrac12\ln n + O(1)$.
		\item Экспоненцировать результат, получая
		      $n! = \sqrt{2\pi n}\,\bigl(\frac{n}{e}\bigr)^n \,\bigl(1+o(1)\bigr).$
	\end{enumerate}
\end{proofplan}

\begin{customproof}
	\textbf{Шаг 1: Логарифмы.}
	Пусть $L_n=\ln(n!)=\sum_{k=1}^n \ln k.$

	\smallskip

	\textbf{Шаг 2: Сравнение с интегралом.}
	Замечаем, что
	\[
		\sum_{k=1}^n \ln k
		\;\approx\;
		\int_{1}^{n} \ln x \,dx
		\;=\; n\ln n - n + 1.
	\]
	Разница между суммой и интегралом даёт эффект порядка $\ln(n)$.

	\smallskip

	\textbf{Шаг 3: Уточнение (Эйлера–Маклорена).}
	Более детальный анализ (или полная формула Эйлера–Маклорена) показывает:
	\[
		\ln(n!)
		\;=\;
		n\ln n \;-\; n
		\;+\;\tfrac12\ln(n)
		\;+\; O(1).
	\]
	Иными словами,
	\[
		L_n
		= n\ln n - n + \tfrac12\ln n + O(1).
	\]

	\smallskip

	\textbf{Шаг 4: Экспоненцирование.}
	Тогда
	\[
		n!
		=
		\exp(L_n)
		=
		\exp\bigl(n\ln n - n + \tfrac12\ln n + O(1)\bigr)
		=
		\sqrt{n}\,\Bigl(\tfrac{n}{e}\Bigr)^n \exp(O(1)).
	\]
	Поскольку $\exp(O(1))$ означает некий постоянный множитель в пределе,
	тщательный учёт показывает, что этот множитель есть $\sqrt{2\pi}$, т. е.
	\[
		n! \;\sim\;\sqrt{2\pi\,n}\,\Bigl(\tfrac{n}{e}\Bigr)^n.
	\]
	Следовательно, при $n\to\infty$ факториал $n!$ эквивалентен $\sqrt{2\pi\,n}\,\bigl(n/e\bigr)^n$.
\end{customproof}

\begin{customexample}
	\textbf{Сравнение значений.}
	Уже при $n=10$, $10! = 3\,628\,800$, а по формуле Стирлинга
	$\sqrt{2\pi\cdot 10}\,\bigl(\tfrac{10}{e}\bigr)^{10} \approx 3\,598\,695$,
	что даёт небольшое расхождение. С ростом $n$ относительная ошибка убывает очень быстро.
\end{customexample}


\newpage
\section{Формула Тейлора для многочлена. Формула Тейлора с остатком в форме Пеано.}

\subsection*{Вспомогательные понятия}
% ==================
% auxiliary.tex
% (Вспомогательные определения для темы:
%  "Достаточные условия существования экстремума (по второй производной).")
% ==================

\paragraph{Локальный минимум и максимум.}
Точка $x_0$ внутри промежутка $(a,b)$ называется \emph{локальным минимумом} функции $f$,
если существует $\delta>0$ такое, что при $|x - x_0| < \delta$
выполняется $f(x) \ge f(x_0)$.
Аналогично, $x_0$ называется \emph{локальным максимумом},
если в некоторой окрестности $x_0$ верно $f(x) \le f(x_0)$.

\bigskip

\paragraph{Вторая производная.}
Пусть $f$ дифференцируема на интервале $(a,b)$, и $f'(x)$ тоже дифференцируема на $(a,b)$.
Тогда в точках, где это возможно, определена \emph{вторая производная} $f''(x)=\bigl(f'(x)\bigr)'$.

\bigskip

\paragraph{Теорема Ферма (напоминание).}
Если функция $f$ дифференцируема в точке $x_0$ и имеет там локальный минимум или максимум,
то $f'(x_0)=0$.
Это — необходимое условие экстремума (без второй производной).


\subsection*{Ответ на вопрос}
% ==================
% answer.tex
% (Основной материал:
%  "Формула Стирлинга (с эквивалентностью).")
% ==================

\begin{customtheorem}[Формула Стирлинга (эквивалентность)]
	При $n\to\infty$ справедливо
	\[
		n! \;\sim\;\sqrt{2\pi\,n}
		\,\Bigl(\tfrac{n}{e}\Bigr)^n
		\quad\Longleftrightarrow\quad
		\lim_{n\to\infty} \frac{n!}{\sqrt{2\pi\,n}\,\bigl(\tfrac{n}{e}\bigr)^n}=1.
	\]
\end{customtheorem}

\begin{proofplan}
	\begin{enumerate}
		\item Рассмотреть логарифм факториала: $\ln(n!)=\sum_{k=1}^n \ln k$.
		\item Сравнить $\sum_{k=1}^n \ln k$ с интегралом $\int_{1}^{n}\ln x\,dx$,
		      получить приближение $n\ln n - n + 1$ (плюс поправка).
		\item Использовать более точный учёт (например, формулу Эйлера–Маклорена) для
		      уточнения поправки: $\tfrac12\ln n + O(1)$.
		\item Экспоненцировать результат, получая
		      $n! = \sqrt{2\pi n}\,\bigl(\frac{n}{e}\bigr)^n \,\bigl(1+o(1)\bigr).$
	\end{enumerate}
\end{proofplan}

\begin{customproof}
	\textbf{Шаг 1: Логарифмы.}
	Пусть $L_n=\ln(n!)=\sum_{k=1}^n \ln k.$

	\smallskip

	\textbf{Шаг 2: Сравнение с интегралом.}
	Замечаем, что
	\[
		\sum_{k=1}^n \ln k
		\;\approx\;
		\int_{1}^{n} \ln x \,dx
		\;=\; n\ln n - n + 1.
	\]
	Разница между суммой и интегралом даёт эффект порядка $\ln(n)$.

	\smallskip

	\textbf{Шаг 3: Уточнение (Эйлера–Маклорена).}
	Более детальный анализ (или полная формула Эйлера–Маклорена) показывает:
	\[
		\ln(n!)
		\;=\;
		n\ln n \;-\; n
		\;+\;\tfrac12\ln(n)
		\;+\; O(1).
	\]
	Иными словами,
	\[
		L_n
		= n\ln n - n + \tfrac12\ln n + O(1).
	\]

	\smallskip

	\textbf{Шаг 4: Экспоненцирование.}
	Тогда
	\[
		n!
		=
		\exp(L_n)
		=
		\exp\bigl(n\ln n - n + \tfrac12\ln n + O(1)\bigr)
		=
		\sqrt{n}\,\Bigl(\tfrac{n}{e}\Bigr)^n \exp(O(1)).
	\]
	Поскольку $\exp(O(1))$ означает некий постоянный множитель в пределе,
	тщательный учёт показывает, что этот множитель есть $\sqrt{2\pi}$, т. е.
	\[
		n! \;\sim\;\sqrt{2\pi\,n}\,\Bigl(\tfrac{n}{e}\Bigr)^n.
	\]
	Следовательно, при $n\to\infty$ факториал $n!$ эквивалентен $\sqrt{2\pi\,n}\,\bigl(n/e\bigr)^n$.
\end{customproof}

\begin{customexample}
	\textbf{Сравнение значений.}
	Уже при $n=10$, $10! = 3\,628\,800$, а по формуле Стирлинга
	$\sqrt{2\pi\cdot 10}\,\bigl(\tfrac{10}{e}\bigr)^{10} \approx 3\,598\,695$,
	что даёт небольшое расхождение. С ростом $n$ относительная ошибка убывает очень быстро.
\end{customexample}


\newpage
\section{Достаточные условия существования экстремума (по второй производной).}

\subsection*{Вспомогательные понятия}
% ==================
% auxiliary.tex
% (Вспомогательные определения для темы:
%  "Достаточные условия существования экстремума (по второй производной).")
% ==================

\paragraph{Локальный минимум и максимум.}
Точка $x_0$ внутри промежутка $(a,b)$ называется \emph{локальным минимумом} функции $f$,
если существует $\delta>0$ такое, что при $|x - x_0| < \delta$
выполняется $f(x) \ge f(x_0)$.
Аналогично, $x_0$ называется \emph{локальным максимумом},
если в некоторой окрестности $x_0$ верно $f(x) \le f(x_0)$.

\bigskip

\paragraph{Вторая производная.}
Пусть $f$ дифференцируема на интервале $(a,b)$, и $f'(x)$ тоже дифференцируема на $(a,b)$.
Тогда в точках, где это возможно, определена \emph{вторая производная} $f''(x)=\bigl(f'(x)\bigr)'$.

\bigskip

\paragraph{Теорема Ферма (напоминание).}
Если функция $f$ дифференцируема в точке $x_0$ и имеет там локальный минимум или максимум,
то $f'(x_0)=0$.
Это — необходимое условие экстремума (без второй производной).


\subsection*{Ответ на вопрос}
% ==================
% answer.tex
% (Основной материал:
%  "Формула Стирлинга (с эквивалентностью).")
% ==================

\begin{customtheorem}[Формула Стирлинга (эквивалентность)]
	При $n\to\infty$ справедливо
	\[
		n! \;\sim\;\sqrt{2\pi\,n}
		\,\Bigl(\tfrac{n}{e}\Bigr)^n
		\quad\Longleftrightarrow\quad
		\lim_{n\to\infty} \frac{n!}{\sqrt{2\pi\,n}\,\bigl(\tfrac{n}{e}\bigr)^n}=1.
	\]
\end{customtheorem}

\begin{proofplan}
	\begin{enumerate}
		\item Рассмотреть логарифм факториала: $\ln(n!)=\sum_{k=1}^n \ln k$.
		\item Сравнить $\sum_{k=1}^n \ln k$ с интегралом $\int_{1}^{n}\ln x\,dx$,
		      получить приближение $n\ln n - n + 1$ (плюс поправка).
		\item Использовать более точный учёт (например, формулу Эйлера–Маклорена) для
		      уточнения поправки: $\tfrac12\ln n + O(1)$.
		\item Экспоненцировать результат, получая
		      $n! = \sqrt{2\pi n}\,\bigl(\frac{n}{e}\bigr)^n \,\bigl(1+o(1)\bigr).$
	\end{enumerate}
\end{proofplan}

\begin{customproof}
	\textbf{Шаг 1: Логарифмы.}
	Пусть $L_n=\ln(n!)=\sum_{k=1}^n \ln k.$

	\smallskip

	\textbf{Шаг 2: Сравнение с интегралом.}
	Замечаем, что
	\[
		\sum_{k=1}^n \ln k
		\;\approx\;
		\int_{1}^{n} \ln x \,dx
		\;=\; n\ln n - n + 1.
	\]
	Разница между суммой и интегралом даёт эффект порядка $\ln(n)$.

	\smallskip

	\textbf{Шаг 3: Уточнение (Эйлера–Маклорена).}
	Более детальный анализ (или полная формула Эйлера–Маклорена) показывает:
	\[
		\ln(n!)
		\;=\;
		n\ln n \;-\; n
		\;+\;\tfrac12\ln(n)
		\;+\; O(1).
	\]
	Иными словами,
	\[
		L_n
		= n\ln n - n + \tfrac12\ln n + O(1).
	\]

	\smallskip

	\textbf{Шаг 4: Экспоненцирование.}
	Тогда
	\[
		n!
		=
		\exp(L_n)
		=
		\exp\bigl(n\ln n - n + \tfrac12\ln n + O(1)\bigr)
		=
		\sqrt{n}\,\Bigl(\tfrac{n}{e}\Bigr)^n \exp(O(1)).
	\]
	Поскольку $\exp(O(1))$ означает некий постоянный множитель в пределе,
	тщательный учёт показывает, что этот множитель есть $\sqrt{2\pi}$, т. е.
	\[
		n! \;\sim\;\sqrt{2\pi\,n}\,\Bigl(\tfrac{n}{e}\Bigr)^n.
	\]
	Следовательно, при $n\to\infty$ факториал $n!$ эквивалентен $\sqrt{2\pi\,n}\,\bigl(n/e\bigr)^n$.
\end{customproof}

\begin{customexample}
	\textbf{Сравнение значений.}
	Уже при $n=10$, $10! = 3\,628\,800$, а по формуле Стирлинга
	$\sqrt{2\pi\cdot 10}\,\bigl(\tfrac{10}{e}\bigr)^{10} \approx 3\,598\,695$,
	что даёт небольшое расхождение. С ростом $n$ относительная ошибка убывает очень быстро.
\end{customexample}


\newpage
\section{Теорема Лиувилля. Пример трансцендентного числа.}

\subsection*{Вспомогательные понятия}
% ==================
% auxiliary.tex
% (Вспомогательные определения для темы:
%  "Достаточные условия существования экстремума (по второй производной).")
% ==================

\paragraph{Локальный минимум и максимум.}
Точка $x_0$ внутри промежутка $(a,b)$ называется \emph{локальным минимумом} функции $f$,
если существует $\delta>0$ такое, что при $|x - x_0| < \delta$
выполняется $f(x) \ge f(x_0)$.
Аналогично, $x_0$ называется \emph{локальным максимумом},
если в некоторой окрестности $x_0$ верно $f(x) \le f(x_0)$.

\bigskip

\paragraph{Вторая производная.}
Пусть $f$ дифференцируема на интервале $(a,b)$, и $f'(x)$ тоже дифференцируема на $(a,b)$.
Тогда в точках, где это возможно, определена \emph{вторая производная} $f''(x)=\bigl(f'(x)\bigr)'$.

\bigskip

\paragraph{Теорема Ферма (напоминание).}
Если функция $f$ дифференцируема в точке $x_0$ и имеет там локальный минимум или максимум,
то $f'(x_0)=0$.
Это — необходимое условие экстремума (без второй производной).


\subsection*{Ответ на вопрос}
% ==================
% answer.tex
% (Основной материал:
%  "Формула Стирлинга (с эквивалентностью).")
% ==================

\begin{customtheorem}[Формула Стирлинга (эквивалентность)]
	При $n\to\infty$ справедливо
	\[
		n! \;\sim\;\sqrt{2\pi\,n}
		\,\Bigl(\tfrac{n}{e}\Bigr)^n
		\quad\Longleftrightarrow\quad
		\lim_{n\to\infty} \frac{n!}{\sqrt{2\pi\,n}\,\bigl(\tfrac{n}{e}\bigr)^n}=1.
	\]
\end{customtheorem}

\begin{proofplan}
	\begin{enumerate}
		\item Рассмотреть логарифм факториала: $\ln(n!)=\sum_{k=1}^n \ln k$.
		\item Сравнить $\sum_{k=1}^n \ln k$ с интегралом $\int_{1}^{n}\ln x\,dx$,
		      получить приближение $n\ln n - n + 1$ (плюс поправка).
		\item Использовать более точный учёт (например, формулу Эйлера–Маклорена) для
		      уточнения поправки: $\tfrac12\ln n + O(1)$.
		\item Экспоненцировать результат, получая
		      $n! = \sqrt{2\pi n}\,\bigl(\frac{n}{e}\bigr)^n \,\bigl(1+o(1)\bigr).$
	\end{enumerate}
\end{proofplan}

\begin{customproof}
	\textbf{Шаг 1: Логарифмы.}
	Пусть $L_n=\ln(n!)=\sum_{k=1}^n \ln k.$

	\smallskip

	\textbf{Шаг 2: Сравнение с интегралом.}
	Замечаем, что
	\[
		\sum_{k=1}^n \ln k
		\;\approx\;
		\int_{1}^{n} \ln x \,dx
		\;=\; n\ln n - n + 1.
	\]
	Разница между суммой и интегралом даёт эффект порядка $\ln(n)$.

	\smallskip

	\textbf{Шаг 3: Уточнение (Эйлера–Маклорена).}
	Более детальный анализ (или полная формула Эйлера–Маклорена) показывает:
	\[
		\ln(n!)
		\;=\;
		n\ln n \;-\; n
		\;+\;\tfrac12\ln(n)
		\;+\; O(1).
	\]
	Иными словами,
	\[
		L_n
		= n\ln n - n + \tfrac12\ln n + O(1).
	\]

	\smallskip

	\textbf{Шаг 4: Экспоненцирование.}
	Тогда
	\[
		n!
		=
		\exp(L_n)
		=
		\exp\bigl(n\ln n - n + \tfrac12\ln n + O(1)\bigr)
		=
		\sqrt{n}\,\Bigl(\tfrac{n}{e}\Bigr)^n \exp(O(1)).
	\]
	Поскольку $\exp(O(1))$ означает некий постоянный множитель в пределе,
	тщательный учёт показывает, что этот множитель есть $\sqrt{2\pi}$, т. е.
	\[
		n! \;\sim\;\sqrt{2\pi\,n}\,\Bigl(\tfrac{n}{e}\Bigr)^n.
	\]
	Следовательно, при $n\to\infty$ факториал $n!$ эквивалентен $\sqrt{2\pi\,n}\,\bigl(n/e\bigr)^n$.
\end{customproof}

\begin{customexample}
	\textbf{Сравнение значений.}
	Уже при $n=10$, $10! = 3\,628\,800$, а по формуле Стирлинга
	$\sqrt{2\pi\cdot 10}\,\bigl(\tfrac{10}{e}\bigr)^{10} \approx 3\,598\,695$,
	что даёт небольшое расхождение. С ростом $n$ относительная ошибка убывает очень быстро.
\end{customexample}


\newpage
\section{Формулы Маклорена для функций y=exp(x), y=sinx, y=cosx, y=ln(1+x), y=pow((1+x),a).}

\subsection*{Вспомогательные понятия}
% ==================
% auxiliary.tex
% (Вспомогательные определения для темы:
%  "Достаточные условия существования экстремума (по второй производной).")
% ==================

\paragraph{Локальный минимум и максимум.}
Точка $x_0$ внутри промежутка $(a,b)$ называется \emph{локальным минимумом} функции $f$,
если существует $\delta>0$ такое, что при $|x - x_0| < \delta$
выполняется $f(x) \ge f(x_0)$.
Аналогично, $x_0$ называется \emph{локальным максимумом},
если в некоторой окрестности $x_0$ верно $f(x) \le f(x_0)$.

\bigskip

\paragraph{Вторая производная.}
Пусть $f$ дифференцируема на интервале $(a,b)$, и $f'(x)$ тоже дифференцируема на $(a,b)$.
Тогда в точках, где это возможно, определена \emph{вторая производная} $f''(x)=\bigl(f'(x)\bigr)'$.

\bigskip

\paragraph{Теорема Ферма (напоминание).}
Если функция $f$ дифференцируема в точке $x_0$ и имеет там локальный минимум или максимум,
то $f'(x_0)=0$.
Это — необходимое условие экстремума (без второй производной).


\subsection*{Ответ на вопрос}
% ==================
% answer.tex
% (Основной материал:
%  "Формула Стирлинга (с эквивалентностью).")
% ==================

\begin{customtheorem}[Формула Стирлинга (эквивалентность)]
	При $n\to\infty$ справедливо
	\[
		n! \;\sim\;\sqrt{2\pi\,n}
		\,\Bigl(\tfrac{n}{e}\Bigr)^n
		\quad\Longleftrightarrow\quad
		\lim_{n\to\infty} \frac{n!}{\sqrt{2\pi\,n}\,\bigl(\tfrac{n}{e}\bigr)^n}=1.
	\]
\end{customtheorem}

\begin{proofplan}
	\begin{enumerate}
		\item Рассмотреть логарифм факториала: $\ln(n!)=\sum_{k=1}^n \ln k$.
		\item Сравнить $\sum_{k=1}^n \ln k$ с интегралом $\int_{1}^{n}\ln x\,dx$,
		      получить приближение $n\ln n - n + 1$ (плюс поправка).
		\item Использовать более точный учёт (например, формулу Эйлера–Маклорена) для
		      уточнения поправки: $\tfrac12\ln n + O(1)$.
		\item Экспоненцировать результат, получая
		      $n! = \sqrt{2\pi n}\,\bigl(\frac{n}{e}\bigr)^n \,\bigl(1+o(1)\bigr).$
	\end{enumerate}
\end{proofplan}

\begin{customproof}
	\textbf{Шаг 1: Логарифмы.}
	Пусть $L_n=\ln(n!)=\sum_{k=1}^n \ln k.$

	\smallskip

	\textbf{Шаг 2: Сравнение с интегралом.}
	Замечаем, что
	\[
		\sum_{k=1}^n \ln k
		\;\approx\;
		\int_{1}^{n} \ln x \,dx
		\;=\; n\ln n - n + 1.
	\]
	Разница между суммой и интегралом даёт эффект порядка $\ln(n)$.

	\smallskip

	\textbf{Шаг 3: Уточнение (Эйлера–Маклорена).}
	Более детальный анализ (или полная формула Эйлера–Маклорена) показывает:
	\[
		\ln(n!)
		\;=\;
		n\ln n \;-\; n
		\;+\;\tfrac12\ln(n)
		\;+\; O(1).
	\]
	Иными словами,
	\[
		L_n
		= n\ln n - n + \tfrac12\ln n + O(1).
	\]

	\smallskip

	\textbf{Шаг 4: Экспоненцирование.}
	Тогда
	\[
		n!
		=
		\exp(L_n)
		=
		\exp\bigl(n\ln n - n + \tfrac12\ln n + O(1)\bigr)
		=
		\sqrt{n}\,\Bigl(\tfrac{n}{e}\Bigr)^n \exp(O(1)).
	\]
	Поскольку $\exp(O(1))$ означает некий постоянный множитель в пределе,
	тщательный учёт показывает, что этот множитель есть $\sqrt{2\pi}$, т. е.
	\[
		n! \;\sim\;\sqrt{2\pi\,n}\,\Bigl(\tfrac{n}{e}\Bigr)^n.
	\]
	Следовательно, при $n\to\infty$ факториал $n!$ эквивалентен $\sqrt{2\pi\,n}\,\bigl(n/e\bigr)^n$.
\end{customproof}

\begin{customexample}
	\textbf{Сравнение значений.}
	Уже при $n=10$, $10! = 3\,628\,800$, а по формуле Стирлинга
	$\sqrt{2\pi\cdot 10}\,\bigl(\tfrac{10}{e}\bigr)^{10} \approx 3\,598\,695$,
	что даёт небольшое расхождение. С ростом $n$ относительная ошибка убывает очень быстро.
\end{customexample}


\newpage
\section{Формула Тейлора с остатком в форме Лагранжа. Приближенные вычисления по формуле Тейлора.}

\subsection*{Вспомогательные понятия}
% ==================
% auxiliary.tex
% (Вспомогательные определения для темы:
%  "Достаточные условия существования экстремума (по второй производной).")
% ==================

\paragraph{Локальный минимум и максимум.}
Точка $x_0$ внутри промежутка $(a,b)$ называется \emph{локальным минимумом} функции $f$,
если существует $\delta>0$ такое, что при $|x - x_0| < \delta$
выполняется $f(x) \ge f(x_0)$.
Аналогично, $x_0$ называется \emph{локальным максимумом},
если в некоторой окрестности $x_0$ верно $f(x) \le f(x_0)$.

\bigskip

\paragraph{Вторая производная.}
Пусть $f$ дифференцируема на интервале $(a,b)$, и $f'(x)$ тоже дифференцируема на $(a,b)$.
Тогда в точках, где это возможно, определена \emph{вторая производная} $f''(x)=\bigl(f'(x)\bigr)'$.

\bigskip

\paragraph{Теорема Ферма (напоминание).}
Если функция $f$ дифференцируема в точке $x_0$ и имеет там локальный минимум или максимум,
то $f'(x_0)=0$.
Это — необходимое условие экстремума (без второй производной).


\subsection*{Ответ на вопрос}
% ==================
% answer.tex
% (Основной материал:
%  "Формула Стирлинга (с эквивалентностью).")
% ==================

\begin{customtheorem}[Формула Стирлинга (эквивалентность)]
	При $n\to\infty$ справедливо
	\[
		n! \;\sim\;\sqrt{2\pi\,n}
		\,\Bigl(\tfrac{n}{e}\Bigr)^n
		\quad\Longleftrightarrow\quad
		\lim_{n\to\infty} \frac{n!}{\sqrt{2\pi\,n}\,\bigl(\tfrac{n}{e}\bigr)^n}=1.
	\]
\end{customtheorem}

\begin{proofplan}
	\begin{enumerate}
		\item Рассмотреть логарифм факториала: $\ln(n!)=\sum_{k=1}^n \ln k$.
		\item Сравнить $\sum_{k=1}^n \ln k$ с интегралом $\int_{1}^{n}\ln x\,dx$,
		      получить приближение $n\ln n - n + 1$ (плюс поправка).
		\item Использовать более точный учёт (например, формулу Эйлера–Маклорена) для
		      уточнения поправки: $\tfrac12\ln n + O(1)$.
		\item Экспоненцировать результат, получая
		      $n! = \sqrt{2\pi n}\,\bigl(\frac{n}{e}\bigr)^n \,\bigl(1+o(1)\bigr).$
	\end{enumerate}
\end{proofplan}

\begin{customproof}
	\textbf{Шаг 1: Логарифмы.}
	Пусть $L_n=\ln(n!)=\sum_{k=1}^n \ln k.$

	\smallskip

	\textbf{Шаг 2: Сравнение с интегралом.}
	Замечаем, что
	\[
		\sum_{k=1}^n \ln k
		\;\approx\;
		\int_{1}^{n} \ln x \,dx
		\;=\; n\ln n - n + 1.
	\]
	Разница между суммой и интегралом даёт эффект порядка $\ln(n)$.

	\smallskip

	\textbf{Шаг 3: Уточнение (Эйлера–Маклорена).}
	Более детальный анализ (или полная формула Эйлера–Маклорена) показывает:
	\[
		\ln(n!)
		\;=\;
		n\ln n \;-\; n
		\;+\;\tfrac12\ln(n)
		\;+\; O(1).
	\]
	Иными словами,
	\[
		L_n
		= n\ln n - n + \tfrac12\ln n + O(1).
	\]

	\smallskip

	\textbf{Шаг 4: Экспоненцирование.}
	Тогда
	\[
		n!
		=
		\exp(L_n)
		=
		\exp\bigl(n\ln n - n + \tfrac12\ln n + O(1)\bigr)
		=
		\sqrt{n}\,\Bigl(\tfrac{n}{e}\Bigr)^n \exp(O(1)).
	\]
	Поскольку $\exp(O(1))$ означает некий постоянный множитель в пределе,
	тщательный учёт показывает, что этот множитель есть $\sqrt{2\pi}$, т. е.
	\[
		n! \;\sim\;\sqrt{2\pi\,n}\,\Bigl(\tfrac{n}{e}\Bigr)^n.
	\]
	Следовательно, при $n\to\infty$ факториал $n!$ эквивалентен $\sqrt{2\pi\,n}\,\bigl(n/e\bigr)^n$.
\end{customproof}

\begin{customexample}
	\textbf{Сравнение значений.}
	Уже при $n=10$, $10! = 3\,628\,800$, а по формуле Стирлинга
	$\sqrt{2\pi\cdot 10}\,\bigl(\tfrac{10}{e}\bigr)^{10} \approx 3\,598\,695$,
	что даёт небольшое расхождение. С ростом $n$ относительная ошибка убывает очень быстро.
\end{customexample}


\newpage
\section{Формула Стирлинга (с эквивалентностью).}

\subsection*{Вспомогательные понятия}
% ==================
% auxiliary.tex
% (Вспомогательные определения для темы:
%  "Достаточные условия существования экстремума (по второй производной).")
% ==================

\paragraph{Локальный минимум и максимум.}
Точка $x_0$ внутри промежутка $(a,b)$ называется \emph{локальным минимумом} функции $f$,
если существует $\delta>0$ такое, что при $|x - x_0| < \delta$
выполняется $f(x) \ge f(x_0)$.
Аналогично, $x_0$ называется \emph{локальным максимумом},
если в некоторой окрестности $x_0$ верно $f(x) \le f(x_0)$.

\bigskip

\paragraph{Вторая производная.}
Пусть $f$ дифференцируема на интервале $(a,b)$, и $f'(x)$ тоже дифференцируема на $(a,b)$.
Тогда в точках, где это возможно, определена \emph{вторая производная} $f''(x)=\bigl(f'(x)\bigr)'$.

\bigskip

\paragraph{Теорема Ферма (напоминание).}
Если функция $f$ дифференцируема в точке $x_0$ и имеет там локальный минимум или максимум,
то $f'(x_0)=0$.
Это — необходимое условие экстремума (без второй производной).


\subsection*{Ответ на вопрос}
% ==================
% answer.tex
% (Основной материал:
%  "Формула Стирлинга (с эквивалентностью).")
% ==================

\begin{customtheorem}[Формула Стирлинга (эквивалентность)]
	При $n\to\infty$ справедливо
	\[
		n! \;\sim\;\sqrt{2\pi\,n}
		\,\Bigl(\tfrac{n}{e}\Bigr)^n
		\quad\Longleftrightarrow\quad
		\lim_{n\to\infty} \frac{n!}{\sqrt{2\pi\,n}\,\bigl(\tfrac{n}{e}\bigr)^n}=1.
	\]
\end{customtheorem}

\begin{proofplan}
	\begin{enumerate}
		\item Рассмотреть логарифм факториала: $\ln(n!)=\sum_{k=1}^n \ln k$.
		\item Сравнить $\sum_{k=1}^n \ln k$ с интегралом $\int_{1}^{n}\ln x\,dx$,
		      получить приближение $n\ln n - n + 1$ (плюс поправка).
		\item Использовать более точный учёт (например, формулу Эйлера–Маклорена) для
		      уточнения поправки: $\tfrac12\ln n + O(1)$.
		\item Экспоненцировать результат, получая
		      $n! = \sqrt{2\pi n}\,\bigl(\frac{n}{e}\bigr)^n \,\bigl(1+o(1)\bigr).$
	\end{enumerate}
\end{proofplan}

\begin{customproof}
	\textbf{Шаг 1: Логарифмы.}
	Пусть $L_n=\ln(n!)=\sum_{k=1}^n \ln k.$

	\smallskip

	\textbf{Шаг 2: Сравнение с интегралом.}
	Замечаем, что
	\[
		\sum_{k=1}^n \ln k
		\;\approx\;
		\int_{1}^{n} \ln x \,dx
		\;=\; n\ln n - n + 1.
	\]
	Разница между суммой и интегралом даёт эффект порядка $\ln(n)$.

	\smallskip

	\textbf{Шаг 3: Уточнение (Эйлера–Маклорена).}
	Более детальный анализ (или полная формула Эйлера–Маклорена) показывает:
	\[
		\ln(n!)
		\;=\;
		n\ln n \;-\; n
		\;+\;\tfrac12\ln(n)
		\;+\; O(1).
	\]
	Иными словами,
	\[
		L_n
		= n\ln n - n + \tfrac12\ln n + O(1).
	\]

	\smallskip

	\textbf{Шаг 4: Экспоненцирование.}
	Тогда
	\[
		n!
		=
		\exp(L_n)
		=
		\exp\bigl(n\ln n - n + \tfrac12\ln n + O(1)\bigr)
		=
		\sqrt{n}\,\Bigl(\tfrac{n}{e}\Bigr)^n \exp(O(1)).
	\]
	Поскольку $\exp(O(1))$ означает некий постоянный множитель в пределе,
	тщательный учёт показывает, что этот множитель есть $\sqrt{2\pi}$, т. е.
	\[
		n! \;\sim\;\sqrt{2\pi\,n}\,\Bigl(\tfrac{n}{e}\Bigr)^n.
	\]
	Следовательно, при $n\to\infty$ факториал $n!$ эквивалентен $\sqrt{2\pi\,n}\,\bigl(n/e\bigr)^n$.
\end{customproof}

\begin{customexample}
	\textbf{Сравнение значений.}
	Уже при $n=10$, $10! = 3\,628\,800$, а по формуле Стирлинга
	$\sqrt{2\pi\cdot 10}\,\bigl(\tfrac{10}{e}\bigr)^{10} \approx 3\,598\,695$,
	что даёт небольшое расхождение. С ростом $n$ относительная ошибка убывает очень быстро.
\end{customexample}


\newpage
\section{Формула Стирлинга (с равенством).}

\subsection*{Вспомогательные понятия}
% ==================
% auxiliary.tex
% (Вспомогательные определения для темы:
%  "Достаточные условия существования экстремума (по второй производной).")
% ==================

\paragraph{Локальный минимум и максимум.}
Точка $x_0$ внутри промежутка $(a,b)$ называется \emph{локальным минимумом} функции $f$,
если существует $\delta>0$ такое, что при $|x - x_0| < \delta$
выполняется $f(x) \ge f(x_0)$.
Аналогично, $x_0$ называется \emph{локальным максимумом},
если в некоторой окрестности $x_0$ верно $f(x) \le f(x_0)$.

\bigskip

\paragraph{Вторая производная.}
Пусть $f$ дифференцируема на интервале $(a,b)$, и $f'(x)$ тоже дифференцируема на $(a,b)$.
Тогда в точках, где это возможно, определена \emph{вторая производная} $f''(x)=\bigl(f'(x)\bigr)'$.

\bigskip

\paragraph{Теорема Ферма (напоминание).}
Если функция $f$ дифференцируема в точке $x_0$ и имеет там локальный минимум или максимум,
то $f'(x_0)=0$.
Это — необходимое условие экстремума (без второй производной).


\subsection*{Ответ на вопрос}
% ==================
% answer.tex
% (Основной материал:
%  "Формула Стирлинга (с эквивалентностью).")
% ==================

\begin{customtheorem}[Формула Стирлинга (эквивалентность)]
	При $n\to\infty$ справедливо
	\[
		n! \;\sim\;\sqrt{2\pi\,n}
		\,\Bigl(\tfrac{n}{e}\Bigr)^n
		\quad\Longleftrightarrow\quad
		\lim_{n\to\infty} \frac{n!}{\sqrt{2\pi\,n}\,\bigl(\tfrac{n}{e}\bigr)^n}=1.
	\]
\end{customtheorem}

\begin{proofplan}
	\begin{enumerate}
		\item Рассмотреть логарифм факториала: $\ln(n!)=\sum_{k=1}^n \ln k$.
		\item Сравнить $\sum_{k=1}^n \ln k$ с интегралом $\int_{1}^{n}\ln x\,dx$,
		      получить приближение $n\ln n - n + 1$ (плюс поправка).
		\item Использовать более точный учёт (например, формулу Эйлера–Маклорена) для
		      уточнения поправки: $\tfrac12\ln n + O(1)$.
		\item Экспоненцировать результат, получая
		      $n! = \sqrt{2\pi n}\,\bigl(\frac{n}{e}\bigr)^n \,\bigl(1+o(1)\bigr).$
	\end{enumerate}
\end{proofplan}

\begin{customproof}
	\textbf{Шаг 1: Логарифмы.}
	Пусть $L_n=\ln(n!)=\sum_{k=1}^n \ln k.$

	\smallskip

	\textbf{Шаг 2: Сравнение с интегралом.}
	Замечаем, что
	\[
		\sum_{k=1}^n \ln k
		\;\approx\;
		\int_{1}^{n} \ln x \,dx
		\;=\; n\ln n - n + 1.
	\]
	Разница между суммой и интегралом даёт эффект порядка $\ln(n)$.

	\smallskip

	\textbf{Шаг 3: Уточнение (Эйлера–Маклорена).}
	Более детальный анализ (или полная формула Эйлера–Маклорена) показывает:
	\[
		\ln(n!)
		\;=\;
		n\ln n \;-\; n
		\;+\;\tfrac12\ln(n)
		\;+\; O(1).
	\]
	Иными словами,
	\[
		L_n
		= n\ln n - n + \tfrac12\ln n + O(1).
	\]

	\smallskip

	\textbf{Шаг 4: Экспоненцирование.}
	Тогда
	\[
		n!
		=
		\exp(L_n)
		=
		\exp\bigl(n\ln n - n + \tfrac12\ln n + O(1)\bigr)
		=
		\sqrt{n}\,\Bigl(\tfrac{n}{e}\Bigr)^n \exp(O(1)).
	\]
	Поскольку $\exp(O(1))$ означает некий постоянный множитель в пределе,
	тщательный учёт показывает, что этот множитель есть $\sqrt{2\pi}$, т. е.
	\[
		n! \;\sim\;\sqrt{2\pi\,n}\,\Bigl(\tfrac{n}{e}\Bigr)^n.
	\]
	Следовательно, при $n\to\infty$ факториал $n!$ эквивалентен $\sqrt{2\pi\,n}\,\bigl(n/e\bigr)^n$.
\end{customproof}

\begin{customexample}
	\textbf{Сравнение значений.}
	Уже при $n=10$, $10! = 3\,628\,800$, а по формуле Стирлинга
	$\sqrt{2\pi\cdot 10}\,\bigl(\tfrac{10}{e}\bigr)^{10} \approx 3\,598\,695$,
	что даёт небольшое расхождение. С ростом $n$ относительная ошибка убывает очень быстро.
\end{customexample}


\newpage
\section{Определение интеграла Римана. Отличие от «обычного» предела.}

\subsection*{Вспомогательные понятия}
% ==================
% auxiliary.tex
% (Вспомогательные определения для темы:
%  "Достаточные условия существования экстремума (по второй производной).")
% ==================

\paragraph{Локальный минимум и максимум.}
Точка $x_0$ внутри промежутка $(a,b)$ называется \emph{локальным минимумом} функции $f$,
если существует $\delta>0$ такое, что при $|x - x_0| < \delta$
выполняется $f(x) \ge f(x_0)$.
Аналогично, $x_0$ называется \emph{локальным максимумом},
если в некоторой окрестности $x_0$ верно $f(x) \le f(x_0)$.

\bigskip

\paragraph{Вторая производная.}
Пусть $f$ дифференцируема на интервале $(a,b)$, и $f'(x)$ тоже дифференцируема на $(a,b)$.
Тогда в точках, где это возможно, определена \emph{вторая производная} $f''(x)=\bigl(f'(x)\bigr)'$.

\bigskip

\paragraph{Теорема Ферма (напоминание).}
Если функция $f$ дифференцируема в точке $x_0$ и имеет там локальный минимум или максимум,
то $f'(x_0)=0$.
Это — необходимое условие экстремума (без второй производной).


\subsection*{Ответ на вопрос}
% ==================
% answer.tex
% (Основной материал:
%  "Формула Стирлинга (с эквивалентностью).")
% ==================

\begin{customtheorem}[Формула Стирлинга (эквивалентность)]
	При $n\to\infty$ справедливо
	\[
		n! \;\sim\;\sqrt{2\pi\,n}
		\,\Bigl(\tfrac{n}{e}\Bigr)^n
		\quad\Longleftrightarrow\quad
		\lim_{n\to\infty} \frac{n!}{\sqrt{2\pi\,n}\,\bigl(\tfrac{n}{e}\bigr)^n}=1.
	\]
\end{customtheorem}

\begin{proofplan}
	\begin{enumerate}
		\item Рассмотреть логарифм факториала: $\ln(n!)=\sum_{k=1}^n \ln k$.
		\item Сравнить $\sum_{k=1}^n \ln k$ с интегралом $\int_{1}^{n}\ln x\,dx$,
		      получить приближение $n\ln n - n + 1$ (плюс поправка).
		\item Использовать более точный учёт (например, формулу Эйлера–Маклорена) для
		      уточнения поправки: $\tfrac12\ln n + O(1)$.
		\item Экспоненцировать результат, получая
		      $n! = \sqrt{2\pi n}\,\bigl(\frac{n}{e}\bigr)^n \,\bigl(1+o(1)\bigr).$
	\end{enumerate}
\end{proofplan}

\begin{customproof}
	\textbf{Шаг 1: Логарифмы.}
	Пусть $L_n=\ln(n!)=\sum_{k=1}^n \ln k.$

	\smallskip

	\textbf{Шаг 2: Сравнение с интегралом.}
	Замечаем, что
	\[
		\sum_{k=1}^n \ln k
		\;\approx\;
		\int_{1}^{n} \ln x \,dx
		\;=\; n\ln n - n + 1.
	\]
	Разница между суммой и интегралом даёт эффект порядка $\ln(n)$.

	\smallskip

	\textbf{Шаг 3: Уточнение (Эйлера–Маклорена).}
	Более детальный анализ (или полная формула Эйлера–Маклорена) показывает:
	\[
		\ln(n!)
		\;=\;
		n\ln n \;-\; n
		\;+\;\tfrac12\ln(n)
		\;+\; O(1).
	\]
	Иными словами,
	\[
		L_n
		= n\ln n - n + \tfrac12\ln n + O(1).
	\]

	\smallskip

	\textbf{Шаг 4: Экспоненцирование.}
	Тогда
	\[
		n!
		=
		\exp(L_n)
		=
		\exp\bigl(n\ln n - n + \tfrac12\ln n + O(1)\bigr)
		=
		\sqrt{n}\,\Bigl(\tfrac{n}{e}\Bigr)^n \exp(O(1)).
	\]
	Поскольку $\exp(O(1))$ означает некий постоянный множитель в пределе,
	тщательный учёт показывает, что этот множитель есть $\sqrt{2\pi}$, т. е.
	\[
		n! \;\sim\;\sqrt{2\pi\,n}\,\Bigl(\tfrac{n}{e}\Bigr)^n.
	\]
	Следовательно, при $n\to\infty$ факториал $n!$ эквивалентен $\sqrt{2\pi\,n}\,\bigl(n/e\bigr)^n$.
\end{customproof}

\begin{customexample}
	\textbf{Сравнение значений.}
	Уже при $n=10$, $10! = 3\,628\,800$, а по формуле Стирлинга
	$\sqrt{2\pi\cdot 10}\,\bigl(\tfrac{10}{e}\bigr)^{10} \approx 3\,598\,695$,
	что даёт небольшое расхождение. С ростом $n$ относительная ошибка убывает очень быстро.
\end{customexample}


\newpage
\section{Формула Ньютона-Лейбница.}

\subsection*{Вспомогательные понятия}
% ==================
% auxiliary.tex
% (Вспомогательные определения для темы:
%  "Достаточные условия существования экстремума (по второй производной).")
% ==================

\paragraph{Локальный минимум и максимум.}
Точка $x_0$ внутри промежутка $(a,b)$ называется \emph{локальным минимумом} функции $f$,
если существует $\delta>0$ такое, что при $|x - x_0| < \delta$
выполняется $f(x) \ge f(x_0)$.
Аналогично, $x_0$ называется \emph{локальным максимумом},
если в некоторой окрестности $x_0$ верно $f(x) \le f(x_0)$.

\bigskip

\paragraph{Вторая производная.}
Пусть $f$ дифференцируема на интервале $(a,b)$, и $f'(x)$ тоже дифференцируема на $(a,b)$.
Тогда в точках, где это возможно, определена \emph{вторая производная} $f''(x)=\bigl(f'(x)\bigr)'$.

\bigskip

\paragraph{Теорема Ферма (напоминание).}
Если функция $f$ дифференцируема в точке $x_0$ и имеет там локальный минимум или максимум,
то $f'(x_0)=0$.
Это — необходимое условие экстремума (без второй производной).


\subsection*{Ответ на вопрос}
% ==================
% answer.tex
% (Основной материал:
%  "Формула Стирлинга (с эквивалентностью).")
% ==================

\begin{customtheorem}[Формула Стирлинга (эквивалентность)]
	При $n\to\infty$ справедливо
	\[
		n! \;\sim\;\sqrt{2\pi\,n}
		\,\Bigl(\tfrac{n}{e}\Bigr)^n
		\quad\Longleftrightarrow\quad
		\lim_{n\to\infty} \frac{n!}{\sqrt{2\pi\,n}\,\bigl(\tfrac{n}{e}\bigr)^n}=1.
	\]
\end{customtheorem}

\begin{proofplan}
	\begin{enumerate}
		\item Рассмотреть логарифм факториала: $\ln(n!)=\sum_{k=1}^n \ln k$.
		\item Сравнить $\sum_{k=1}^n \ln k$ с интегралом $\int_{1}^{n}\ln x\,dx$,
		      получить приближение $n\ln n - n + 1$ (плюс поправка).
		\item Использовать более точный учёт (например, формулу Эйлера–Маклорена) для
		      уточнения поправки: $\tfrac12\ln n + O(1)$.
		\item Экспоненцировать результат, получая
		      $n! = \sqrt{2\pi n}\,\bigl(\frac{n}{e}\bigr)^n \,\bigl(1+o(1)\bigr).$
	\end{enumerate}
\end{proofplan}

\begin{customproof}
	\textbf{Шаг 1: Логарифмы.}
	Пусть $L_n=\ln(n!)=\sum_{k=1}^n \ln k.$

	\smallskip

	\textbf{Шаг 2: Сравнение с интегралом.}
	Замечаем, что
	\[
		\sum_{k=1}^n \ln k
		\;\approx\;
		\int_{1}^{n} \ln x \,dx
		\;=\; n\ln n - n + 1.
	\]
	Разница между суммой и интегралом даёт эффект порядка $\ln(n)$.

	\smallskip

	\textbf{Шаг 3: Уточнение (Эйлера–Маклорена).}
	Более детальный анализ (или полная формула Эйлера–Маклорена) показывает:
	\[
		\ln(n!)
		\;=\;
		n\ln n \;-\; n
		\;+\;\tfrac12\ln(n)
		\;+\; O(1).
	\]
	Иными словами,
	\[
		L_n
		= n\ln n - n + \tfrac12\ln n + O(1).
	\]

	\smallskip

	\textbf{Шаг 4: Экспоненцирование.}
	Тогда
	\[
		n!
		=
		\exp(L_n)
		=
		\exp\bigl(n\ln n - n + \tfrac12\ln n + O(1)\bigr)
		=
		\sqrt{n}\,\Bigl(\tfrac{n}{e}\Bigr)^n \exp(O(1)).
	\]
	Поскольку $\exp(O(1))$ означает некий постоянный множитель в пределе,
	тщательный учёт показывает, что этот множитель есть $\sqrt{2\pi}$, т. е.
	\[
		n! \;\sim\;\sqrt{2\pi\,n}\,\Bigl(\tfrac{n}{e}\Bigr)^n.
	\]
	Следовательно, при $n\to\infty$ факториал $n!$ эквивалентен $\sqrt{2\pi\,n}\,\bigl(n/e\bigr)^n$.
\end{customproof}

\begin{customexample}
	\textbf{Сравнение значений.}
	Уже при $n=10$, $10! = 3\,628\,800$, а по формуле Стирлинга
	$\sqrt{2\pi\cdot 10}\,\bigl(\tfrac{10}{e}\bigr)^{10} \approx 3\,598\,695$,
	что даёт небольшое расхождение. С ростом $n$ относительная ошибка убывает очень быстро.
\end{customexample}


\end{document}
