% ==================
% answer.tex
% (Основной материал:
%  "Формула Стирлинга (с эквивалентностью).")
% ==================

\begin{customtheorem}[Формула Стирлинга (эквивалентность)]
	При $n\to\infty$ справедливо
	\[
		n! \;\sim\;\sqrt{2\pi\,n}
		\,\Bigl(\tfrac{n}{e}\Bigr)^n
		\quad\Longleftrightarrow\quad
		\lim_{n\to\infty} \frac{n!}{\sqrt{2\pi\,n}\,\bigl(\tfrac{n}{e}\bigr)^n}=1.
	\]
\end{customtheorem}

\begin{proofplan}
	\begin{enumerate}
		\item Рассмотреть логарифм факториала: $\ln(n!)=\sum_{k=1}^n \ln k$.
		\item Сравнить $\sum_{k=1}^n \ln k$ с интегралом $\int_{1}^{n}\ln x\,dx$,
		      получить приближение $n\ln n - n + 1$ (плюс поправка).
		\item Использовать более точный учёт (например, формулу Эйлера–Маклорена) для
		      уточнения поправки: $\tfrac12\ln n + O(1)$.
		\item Экспоненцировать результат, получая
		      $n! = \sqrt{2\pi n}\,\bigl(\frac{n}{e}\bigr)^n \,\bigl(1+o(1)\bigr).$
	\end{enumerate}
\end{proofplan}

\begin{customproof}
	\textbf{Шаг 1: Логарифмы.}
	Пусть $L_n=\ln(n!)=\sum_{k=1}^n \ln k.$

	\smallskip

	\textbf{Шаг 2: Сравнение с интегралом.}
	Замечаем, что
	\[
		\sum_{k=1}^n \ln k
		\;\approx\;
		\int_{1}^{n} \ln x \,dx
		\;=\; n\ln n - n + 1.
	\]
	Разница между суммой и интегралом даёт эффект порядка $\ln(n)$.

	\smallskip

	\textbf{Шаг 3: Уточнение (Эйлера–Маклорена).}
	Более детальный анализ (или полная формула Эйлера–Маклорена) показывает:
	\[
		\ln(n!)
		\;=\;
		n\ln n \;-\; n
		\;+\;\tfrac12\ln(n)
		\;+\; O(1).
	\]
	Иными словами,
	\[
		L_n
		= n\ln n - n + \tfrac12\ln n + O(1).
	\]

	\smallskip

	\textbf{Шаг 4: Экспоненцирование.}
	Тогда
	\[
		n!
		=
		\exp(L_n)
		=
		\exp\bigl(n\ln n - n + \tfrac12\ln n + O(1)\bigr)
		=
		\sqrt{n}\,\Bigl(\tfrac{n}{e}\Bigr)^n \exp(O(1)).
	\]
	Поскольку $\exp(O(1))$ означает некий постоянный множитель в пределе,
	тщательный учёт показывает, что этот множитель есть $\sqrt{2\pi}$, т. е.
	\[
		n! \;\sim\;\sqrt{2\pi\,n}\,\Bigl(\tfrac{n}{e}\Bigr)^n.
	\]
	Следовательно, при $n\to\infty$ факториал $n!$ эквивалентен $\sqrt{2\pi\,n}\,\bigl(n/e\bigr)^n$.
\end{customproof}

\begin{customexample}
	\textbf{Сравнение значений.}
	Уже при $n=10$, $10! = 3\,628\,800$, а по формуле Стирлинга
	$\sqrt{2\pi\cdot 10}\,\bigl(\tfrac{10}{e}\bigr)^{10} \approx 3\,598\,695$,
	что даёт небольшое расхождение. С ростом $n$ относительная ошибка убывает очень быстро.
\end{customexample}
