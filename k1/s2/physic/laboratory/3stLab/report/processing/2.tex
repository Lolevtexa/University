\subsection*{Расчёт средних значений углов и параметра \(a\)}

В таблицах 1–3 приведены по три независимых измерения углов \(\alpha_{m,i}\) для каждого порядка
\[
	m = 0, \pm1, \pm2, \pm3
\]
и для трёх спектральных линий (\(i=1\) — красная, \(i=2\) — зелёная, \(i=3\) — фиолетовая).

Для каждого цвета \(i\) и каждого порядка \(m\) выполняются следующие шаги:
\begin{enumerate}
	\item Перевод каждого измерения \(\alpha_{m,i}\) из формата \(\deg'\) в десятичные градусы.
	\item Вычисление среднего угла
	      \[
		      \bar\alpha_{m,i}
		      = \frac{1}{3}\sum_{k=1}^{3}\alpha_{m,i}^{(k)}.
	      \]
	\item Определение углового смещения относительно центрального максимума (\(m=0\)):
	      \[
		      \varphi_{m,i}
		      = \bar\alpha_{m,i} \;-\; \bar\alpha_{0,i}.
	      \]
	\item Расчёт параметра
	      \[
		      a_{m,i}
		      = \frac{\sin\varphi_{m,i}}{m},
		      \qquad m\neq 0.
	      \]
\end{enumerate}
