\textbf{Условие:}
Является ли линейным пространством множество матриц \( 2 \times 2 \) со следом \( 1 \)?

\textbf{Решение:}
Рассмотрим множество всех матриц \( 2 \times 2 \) со следом 1:

\[
V = \{ A \in M_{2 \times 2} \mid \text{tr}(A) = 1 \}
\]

где \( \text{tr}(A) \) обозначает след матрицы \( A \), т.е. сумму элементов на главной диагонали.

Чтобы проверить, является ли \( V \) линейным пространством, проверим основные свойства линейного пространства: замкнутость относительно сложения и умножения на скаляр.

\textbf{Проверка замкнутости относительно сложения:}  
Если \( A, B \in V \), то их сумма \( A + B \) должна также принадлежать \( V \), то есть иметь след, равный 1.

\[
\text{tr}(A + B) = \text{tr}(A) + \text{tr}(B).
\]

Так как по условию \( \text{tr}(A) = 1 \) и \( \text{tr}(B) = 1 \), то:

\[
\text{tr}(A + B) = 1 + 1 = 2.
\]

Но \( 2 \neq 1 \), следовательно, \( A + B \notin V \), и множество \( V \) не замкнуто относительно сложения.

\textbf{Проверка замкнутости относительно умножения на скаляр:}  
Пусть \( A \in V \) и \( \lambda \) — произвольное число. Тогда проверим, принадлежит ли \( \lambda A \) множеству \( V \):

\[
\text{tr}(\lambda A) = \lambda \text{tr}(A) = \lambda \cdot 1 = \lambda.
\]

Так как при произвольном \( \lambda \) значение \( \text{tr}(\lambda A) \) может быть любым, оно не обязательно равно 1. Следовательно, \( \lambda A \notin V \), и множество \( V \) не замкнуто относительно умножения на скаляр.

\textbf{Вывод:}  
Так как множество \( V \) не замкнуто ни относительно сложения, ни относительно умножения на скаляр, оно \textbf{не является линейным пространством}.
