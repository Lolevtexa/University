\section{Реализация}
\subsection{C++}

Код проекта разбит на модули:
\begin{itemize}
  \item \texttt{include/calc.h}, \texttt{src/calc.cpp} — вычисление волнового числа, длины до источников, электрического поля и интенсивности;
  \item \texttt{src/utils.cpp} — считывание параметров из \texttt{params.txt};
  \item \texttt{src/main.cpp} — поиск точки максимума интенсивности и запись результата в \texttt{IDZ2.txt}.
\end{itemize}

Алгоритм поиска $x_{\max}$ реализован в два этапа:
\begin{enumerate}
  \item Грубый перебор с шагом $0.01$ по отрезку $[-H, H]$ позволяет определить приближённое положение максимума.
  \item Далее запускается уточнение методом дихотомии. На каждом шаге сравниваются значения $I(x)$ немного левее и правее текущей середины отрезка. Интервал сужается в сторону большего значения. После 60 итераций достигается точность около $10^{-8}$.
\end{enumerate}

Сборка выполняется через CMake:
\begin{verbatim}
mkdir build && cd build
cmake ..
make
\end{verbatim}

\subsection{Python}
Скрипт \texttt{scripts/plot.py} принимает три аргумента:
\texttt{data.txt}, \texttt{params.txt}, \texttt{output\_image.png}.
Строит график $I(x)$, отмечает:
\begin{itemize}
  \item положения источников $(x_i,y_i)$,
  \item значения $I(x_i)$,
  \item точку максимума $(x_{\max},I_{\max})$.
\end{itemize}
Результат сохраняется в \texttt{figures/intensity\_with\_sources.png}.
