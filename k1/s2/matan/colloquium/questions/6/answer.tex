% ==================
% answer.tex
% (Теоремы, доказательства, примеры по теме:
%  "Теорема Коши. Правило Лопиталя (случай 0/0). Примеры неприменимости.")
% ==================

\begin{customtheorem}[Теорема Коши (обобщённая теорема Лагранжа)]
	Пусть функции $f(x)$ и $g(x)$ удовлетворяют условиям:
	\begin{itemize}
		\item непрерывны на $[a,b]$,
		\item дифференцируемы на $(a,b)$,
		\item $g'(x)\neq 0$ для всех $x\in(a,b)$.
	\end{itemize}
	Тогда существует точка $c\in(a,b)$ такая, что
	\[
		\frac{f(b)-f(a)}{g(b)-g(a)}
		\;=\;
		\frac{f'(c)}{g'(c)}.
	\]
\end{customtheorem}

\begin{proofplan}
	\begin{enumerate}
		\item Сконструировать вспомогательную функцию
		      \(\Phi(t)=f(t)-f(a)\;-\;\tfrac{f(b)-f(a)}{\,g(b)-g(a)\,}[\,g(t)-g(a)\,]\).
		\item Показать, что \(\Phi(a)=\Phi(b)=0\).
		\item Применить Теорему Ролля, найти $c\in(a,b)$ с \(\Phi'(c)=0\).
		\item Вывести оттуда \( f'(c)=\tfrac{f(b)-f(a)}{g(b)-g(a)}\,g'(c)\).
		\item Получить требуемое равенство \(\tfrac{f(b)-f(a)}{g(b)-g(a)}=\tfrac{f'(c)}{g'(c)}.\)
	\end{enumerate}
\end{proofplan}

\begin{customproof}
	Предположим $f,g$ удовлетворяют условиям: непрерывны на $[a,b]$,
	дифференцируемы на $(a,b)$, причём $g'(x)\neq0$ на $(a,b)$.
	Определим
	\[
		\Phi(t)
		\;=\;
		f(t)-f(a)
		\;-\;
		\frac{\,f(b)-f(a)\,}{\,g(b)-g(a)\,}
		\,\bigl[g(t)-g(a)\bigr].
	\]
	Нетрудно проверить, что \(\Phi(a)=0\) и
	\[
		\Phi(b)
		\;=\;
		f(b)-f(a)
		\;-\;
		\frac{\,f(b)-f(a)\,}{\,g(b)-g(a)\,}\bigl[g(b)-g(a)\bigr]
		\;=\;0.
	\]
	Поскольку $\Phi$ непрерывна на $[a,b]$ и дифференцируема на $(a,b)$,
	по Теореме Ролля существует $c\in(a,b)$, где $\Phi'(c)=0$. Но
	\[
		\Phi'(t)
		\;=\; f'(t)
		\;-\;
		\frac{\,f(b)-f(a)\,}{\,g(b)-g(a)\,}\,g'(t).
	\]
	Тогда \(\Phi'(c)=0\implies f'(c)=\tfrac{f(b)-f(a)}{\,g(b)-g(a)\,}\,g'(c)\).
	Поделив обе части на $g'(c)$ (отлично от 0), получаем
	\[
		\frac{f'(c)}{g'(c)}
		\;=\;
		\frac{f(b)-f(a)}{\,g(b)-g(a)\,}.
	\]
	Это и доказывает обобщённую Теорему Лагранжа (Коши).
\end{customproof}

\begin{customtheorem}[Правило Лопиталя (случай $0/0$)]
	Пусть $f(x)$ и $g(x)$ дифференцируемы в проколотой окрестности точки $a$.
	Предположим, что
	\[
		\lim_{x\to a} f(x) = 0,\quad
		\lim_{x\to a} g(x)=0,
	\]
	и $g'(x)\neq 0$ в этой окрестности. Если существует конечный или бесконечный предел
	\[
		\lim_{x\to a} \frac{f'(x)}{g'(x)} \;=\; L,
	\]
	то существует и $\lim_{x\to a}\tfrac{f(x)}{g(x)}$, причём
	\[
		\lim_{x\to a}\frac{f(x)}{g(x)}=L.
	\]
\end{customtheorem}

\begin{proofplan}
	\begin{enumerate}
		\item Рассмотреть отношение $\frac{f(x)}{g(x)}$ при $x\to a$ (обе функции стремятся к 0).
		\item Применить Теорему Коши к $f$ и $g$ на отрезке $[a,x]$, используя $f(a)=g(a)=0$.
		\item Утверждается, что $\frac{f(x)}{g(x)}=\frac{f'(c_x)}{g'(c_x)}$ для некоторого $c_x\in(a,x)$.
		\item Переходя к пределу $x\to a$, если $\lim_{x\to a}\frac{f'(x)}{g'(x)}=L$, то получаем $\lim_{x\to a}\frac{f(x)}{g(x)}=L$.
	\end{enumerate}
\end{proofplan}

\begin{customproof}
	По условию $\lim_{x\to a} f(x)=0$, $\lim_{x\to a} g(x)=0$, и $g'(x)\neq 0$ в проколотой окрестности. Предположим, что
	\(\lim_{x\to a}\tfrac{f'(x)}{g'(x)}=L\).

	Для $x>a$ (или $x<a$, в зависимости от ситуации) рассмотрим отрезок $[a,x]$.
	Тогда $f(a)=g(a)=0$. По Теореме Коши (см. выше) существует $c_x\in(a,x)$,
	где
	\[
		\frac{f(x)-f(a)}{\,g(x)-g(a)\,}
		\;=\;
		\frac{f'(c_x)}{g'(c_x)}.
	\]
	Но $f(a)=g(a)=0$, значит
	\[
		\frac{f(x)}{g(x)}
		\;=\;
		\frac{f'(c_x)}{g'(c_x)}.
	\]
	При $x\to a$, точка $c_x\to a$. Если \(\lim_{x\to a}\tfrac{f'(x)}{g'(x)}=L\),
	то \(\tfrac{f'(c_x)}{g'(c_x)}\to L\). Следовательно,
	\[
		\lim_{x\to a} \frac{f(x)}{g(x)}
		\;=\;
		L.
	\]
\end{customproof}

\begin{customexample}
	\textbf{Примеры, когда правило Лопиталя неприменимо:}
	\begin{itemize}
		\item \textbf{Нет неопределённости $0/0$}:
		      $\lim_{x\to 0}\tfrac{\sin x}{x+1}=\tfrac{0}{1}=0$ — здесь всё очевидно,
		      правило Лопиталя не нужно.
		\item \textbf{Предел $\frac{f'(x)}{g'(x)}$ не существует}:
		      $\lim\limits_{x\to 0}\frac{\sin(1/x)}{1/x}$ – поведение непредсказуемо;
		      производные $f'(x)$ и $g'(x)$ «скачут».
		\item \textbf{$f$ или $g$ не дифференцируемы (хотя бы в проколотой окрестности)}:
		      $f(x)=|x|$, $g(x)=x$ при $x\to0$: $f$ не дифференцируема в $0$.
	\end{itemize}
\end{customexample}
