% ===== 10.tex =====
% "Формулы Маклорена для e^x, sin x, cos x, ln(1+x), (1+x)^a."

% 1. Определения

\textbf{Ряд Маклорена (частный случай ряда Тейлора).}
Пусть функция $f$ бесконечно дифференцируема в окрестности $x=0$. Тогда её \emph{ряд Маклорена} — это разложение:
\[
f(x) \;=\; \sum_{n=0}^{\infty} \frac{f^{(n)}(0)}{n!}\,x^n,
\]
если этот ряд сходится к $f(x)$ при соответствующих $x$ (см. \texttt{sub\_10.tex} о радиусе сходимости).

\medskip

% 2. Теоремы и ключевые утверждения (Формулы Маклорена)

\textbf{1) Функция $f(x)=e^x$.}
Все производные $f^{(n)}(x)=e^x$, значит $f^{(n)}(0)=1$. Итоговая формула:
\[
e^x = \sum_{n=0}^{\infty} \frac{x^n}{n!},\quad x \in \mathbb{R}.
\]
Ряд сходится абсолютно для всех $x$.

\textbf{2) Функция $f(x)=\sin x$.}
Набор производных цикличен:
\[
f'(x)=\cos x,\;f''(x)=-\sin x,\;f^{(3)}(x)=-\cos x,\;f^{(4)}(x)=\sin x,\dots
\]
Значения в 0: $f(0)=0,\; f'(0)=1,\; f''(0)=0,\; f^{(3)}(0)=-1,\dots$  
\[
\sin x = \sum_{n=0}^{\infty}(-1)^n \frac{x^{2n+1}}{(2n+1)!},\quad x \in \mathbb{R}.
\]

\textbf{3) Функция $f(x)=\cos x$.}
Аналогично, производные:
\[
\cos x = \sum_{n=0}^{\infty} (-1)^n \frac{x^{2n}}{(2n)!},\quad x\in\mathbb{R}.
\]

\textbf{4) Функция $f(x)=\ln(1+x)$.}
Все производные в точке 0 (для $|x|<1$) дают:
\[
\ln(1+x) = x - \frac{x^2}{2} + \frac{x^3}{3} - \frac{x^4}{4} + \dots,\quad |x|<1.
\]
При $x=1$ ряд сходится условно (это знаменитый «\(\ln 2\)» ряд).

\textbf{5) Функция $f(x)=(1+x)^a$.}
Для \(|x|<1\) и произвольного действительного $a$ (бином Ньютона в обобщённом смысле):
\[
(1+x)^a = 1 + a\,x + \frac{a(a-1)}{2!}\,x^2 + \frac{a(a-1)(a-2)}{3!}\,x^3 + \dots .
\]
Ряд сходится при $|x|<1$.

\medskip

% 3. Основные идеи доказательств (коротко)

\begin{itemize}
  \item \textbf{$e^x$, sin x, cos x}:  
    Производные в 0 легко вычислить, дающие конкретные формулы для $f^{(n)}(0)$.
  \item \textbf{ln(1+x)}:  
    Рассматриваем $f(x)=\ln(1+x)$, находим $f^{(n)}(x)$ и подставляем $x=0$, выписываем коэффициенты.  
  \item \textbf{$(1+x)^a$}:  
    Используем обобщённую биномную формулу (либо вывод через производные в 0).
\end{itemize}

\medskip

% 4. Полное доказательство (по шагам, ссылаясь на sub_10.tex где нужно)

\textbf{Примерный план доказательства (для $e^x$).}
\begin{enumerate}
  \item \textbf{Показываем, что все производные $f^{(n)}(0)=1$.}
  \item \textbf{Формула Маклорена}:
    \[
    e^x = \sum_{n=0}^{\infty} \frac{x^n}{n!}.
    \]
  \item \textbf{Радиус сходимости — неограничен.}  
    По признаку д’Аламбера или частному признаку Бернулли, ряд сходится для всех $x\in \mathbb{R}$.
\end{enumerate}

\textbf{Для \(\sin x\), \(\cos x\), \(\ln(1+x)\), \((1+x)^a\)} аналогично расписываются производные и их значения в 0, а сходимость анализируется по радиусу, связанному с разложением и возможными особенностями (см. \texttt{sub\_10.tex}).

\medskip

% 5. Примеры (частные подстановки)

\begin{itemize}
  \item \(\sin x\) при $x=\pi$: \(\sin \pi=0\). Ряд даёт $0-\tfrac{\pi^3}{3!}+\tfrac{\pi^5}{5!}-\dots$ (проверка сходимости).  
  \item \(\ln(1+x)\) при $x=\frac12$: \(\ln\left(\tfrac32\right)\approx 0.40536\). Ряд: $\tfrac12-\tfrac{1}{8}+\tfrac{1}{24}-\dots$.  
  \item \((1+x)^a\) при $a=\tfrac12$: \(\sqrt{1+x}=\dots\) бином Ньютона.
\end{itemize}

\medskip

% -- Логическая связность --
% (Подробные определения производной порядка n, радиуса сходимости,
%  биномной теоремы см. sub_10.tex)
