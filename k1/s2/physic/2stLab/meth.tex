\documentclass[12pt,a4paper]{article}
\usepackage[T2A]{fontenc}      % Поддержка кириллицы
\usepackage[utf8]{inputenc}    % Кодировка UTF-8
\usepackage[english,russian]{babel}
\usepackage{amsmath,amssymb}   % Математические пакеты
\usepackage{graphicx}          % Для вставки рисунков
\usepackage{geometry}          % Управление полями страницы
\geometry{left=2cm,right=2cm,top=2cm,bottom=2cm}

\begin{document}

\section*{Методика выполнения работы (подробное описание)}

Ниже приведён максимально подробный порядок действий для проведения лабораторной работы «Исследование эффекта Холла в полупроводнике». Предполагается, что вся измерительная аппаратура уже смонтирована на соответствующем стенде и содержит следующие элементы:
\begin{itemize}
    \item Источник питания для датчика Холла (\textbf{E1});
    \item Источник питания для электромагнита (\textbf{E2});
    \item Потенциометр \textbf{R1} (на панели может быть обозначен как «Ток~ДХ»), отвечающий за регулировку тока \textbf{I1} через датчик Холла;
    \item Потенциометр \textbf{R2} (на панели может быть обозначен как «Ток~ЭМ»), отвечающий за регулировку тока \textbf{I2} в катушках электромагнита;
    \item Миллиамперметр (\textbf{mA}) для контроля тока \textbf{I1};
    \item Вольтметр \textbf{V2} для контроля тока \textbf{I2} (через шунтирующий резистор $R=1\,\Omega$);
    \item Операционный усилитель (\textbf{ОУ}) с коэффициентом усиления $k$, к которому подключён датчик Холла;
    \item Вольтметр \textbf{V1} для измерения выходного напряжения усилителя (\textbf{U1}), связанного с напряжением Холла \textbf{Ux}.
\end{itemize}

\subsection*{1. Подготовка измерительной аппаратуры}

\begin{enumerate}
    \item \textbf{Проверка включения и нулевых установок потенциометров:}
    \begin{itemize}
        \item Убедитесь, что потенциометр \textbf{R1} (регулировка тока датчика Холла) выведен в крайнее левое положение; в этом положении ток \textbf{I1} будет близок к нулю.
        \item Аналогично убедитесь, что потенциометр \textbf{R2} (регулировка тока электромагнита) также выведен в крайнее левое положение; в этом положении ток \textbf{I2} практически равен нулю.
    \end{itemize}
    \item \textbf{Установка пределов измерения на приборах:}
    \begin{itemize}
        \item Миллиамперметр (\textbf{mA}), измеряющий \textbf{I1}, установите на предел, позволяющий надёжно регистрировать токи до 10--20\,мА (например, «200\,mA»).
        \item Вольтметр \textbf{V2}, измеряющий падение напряжения на шунте $R=1\,\Omega$, обычно ставят на предел «20\,VDC». Поскольку $U=I_2 \times 1\,\Omega$, это соответствует току \textbf{I2} до 20\,A, хотя реально работа будет идти в пределах $\approx0{,}1\!-\!1\,A$.
        \item Вольтметр \textbf{V1}, контролирующий выход усилителя \textbf{U1}, также переводят на предел «20\,VDC», что даёт возможность измерять типичные значения выходного сигнала до 20\,В.
    \end{itemize}
    \item \textbf{Включение источников питания:}
    \begin{itemize}
        \item Убедившись в минимальных установках \textbf{R1} и \textbf{R2}, включите питание стенда: \textbf{E1} (питание датчика Холла) и \textbf{E2} (питание электромагнита).
        \item Проверьте, что индикаторы и дисплеи приборов активны, а рабочие показатели (напряжения и токи) пока равны нулю или близки к нулю.
    \end{itemize}
\end{enumerate}

\subsection*{2. Установление тока в датчике Холла (I1)}

\begin{enumerate}
    \item \textbf{Плавно поверните потенциометр R1} по часовой стрелке, чтобы выставить первое нужное значение тока \textbf{I1}. 
    \item \textbf{Считайте показания} миллиамперметра (\textbf{mA}) и добейтесь, например, 2\,мА (или другого запланированного значения), слегка подстраивая \textbf{R1}.
    \item Запомните или запишите это значение \textbf{I1}, чтобы при необходимости вернуть установку к нему после изменений.
    \item \textbf{Убедитесь}, что вольтметр \textbf{V1} (выход ОУ) пока показывает малое напряжение, так как ток \textbf{I2} в электромагните ещё не подан или минимален. Это нормально и не указывает на неисправность.
\end{enumerate}

\subsection*{3. Съём зависимости Ux от тока электромагнита (I2) при фиксированном I1}

\begin{enumerate}
    \item Установите \textbf{I1} на одном из выбранных значений (например, 2\,мА, как в пункте выше). Не изменяйте \textbf{R1} до окончания съёма зависимости.
    \item \textbf{Постепенно увеличивайте I2}, вращая потенциометр \textbf{R2}:
          \begin{itemize}
              \item Начните с 0\,A (практически при крайнем левом положении \textbf{R2}).
              \item Увеличивайте \textbf{I2} небольшими шагами (например, по 0,1\,A) до максимума, доступного установке (обычно $\approx 1\,A$).
          \end{itemize}
    \item \textbf{На каждом шаге} фиксируйте:
          \begin{itemize}
              \item Ток электромагнита \textbf{I2} (показания вольтметра \textbf{V2}, численно равные силе тока в амперах),
              \item Напряжение на выходе операционного усилителя \textbf{U1} (вольтметр \textbf{V1}).
          \end{itemize}
    \item \textbf{Дайте установке стабилизироваться} 5--10 секунд (особенно если измерения чувствительны или сигнал может колебаться), после чего перепишите точные значения \textbf{I2} и \textbf{U1} в таблицу измерений.
    \item Продолжайте последовательность изменений \textbf{I2} и соответствующих записей \textbf{U1}, пока не получите 7--10 точек (включая 0\,A и максимально доступное значение).
    \item \textbf{Завершив серию}, переведите \textbf{I2} обратно к минимуму (или оставьте на последней точке), но не выключайте питание, чтобы не сбросить настройки датчика Холла \textbf{I1}.
\end{enumerate}

\subsection*{4. Повторение эксперимента для других значений I1}

\begin{enumerate}
    \item \textbf{Измените ток I1} (поворотом \textbf{R1}), чтобы установить следующее запланированное значение, например, 4\,мА (или другое значение по методическим рекомендациям).
    \item \textbf{Повторите все те же действия}, описанные в пункте (3):
          \begin{itemize}
              \item Медленно увеличивайте \textbf{I2}, 
              \item На каждом шаге фиксируйте \textbf{I2} и \textbf{U1}.
          \end{itemize}
    \item При желании можно установить порядка 4--5 различных значений \textbf{I1} (2\,мА, 4\,мА, 6\,мА, 8\,мА, 10\,мА) и для каждого выполнить измерения зависимости \textbf{U1} от \textbf{I2} (или, точнее, \textbf{Ux} от \textbf{B}, когда Вы далее учтёте коэффициент усиления и формулу для $B = B_n + a\,I_2$).
    \item Для удобства делайте в тетради или в электронной таблице отдельные колонки:
    \[
        \begin{array}{c|c|c}
            \hline
            I_1 (\text{мА}) & I_2 (\text{А}) & U_1 (\text{В}) \\ \hline
            2\,\text{мА} & 0.0\,\text{А} &  \dots \\
            2\,\text{мА} & 0.1\,\text{А} &  \dots \\
            \vdots       & \vdots        &  \dots \\
            \hline
        \end{array}
    \]
\end{enumerate}

\subsection*{5. Завершение эксперимента}

\begin{enumerate}
    \item \textbf{Уменьшите ток I2} потенциометром \textbf{R2} обратно до минимального значения.
    \item \textbf{Убавьте ток I1} потенциометром \textbf{R1} до минимума.
    \item \textbf{Отключите} источники питания \textbf{E1} и \textbf{E2}.
    \item Убедитесь, что вольтметры и миллиамперметр вернулись к нулевым показаниям или близким к ним.
    \item Заполните все необходимые графы журналов/протоколов лабораторной работы, проверьте полноту записей (чтобы не запутаться при последующей обработке).
\end{enumerate}

\subsection*{6. Дополнительные замечания по надёжности и точности}

\begin{itemize}
    \item \textbf{Время стабилизации:} Если заметны колебания показаний вольтметра \textbf{V1}, дождитесь, пока сигнал более-менее стабилизируется (2--5\,с). Слабые колебания (порядка нескольких милливольт) обычно неизбежны.
    \item \textbf{Контроль перегрева:} При больших токах (\textbf{I2} до 1\,A) катушки электромагнита могут нагреваться. Перерывы между сериями измерений (30--60\,с) могут потребоваться, чтобы получить стабильное магнитное поле.
    \item \textbf{Последовательность t:} Как правило, надёжнее снимать точки последовательно от меньшего к большему току \textbf{I2}, так как при обратной прогонке (от большего к меньшему) могут возникать задержки из-за остаточного намагничивания сердечника.
    \item \textbf{Число точек:} Желательно получить не менее 7--10 значений \textbf{I2} для каждой фиксированной \textbf{I1}, чтобы по результатам можно было строить графики и увереннее выявлять линейную зависимость.
\end{itemize}

\bigskip

\noindent
\textbf{Таким образом, данная методика позволяет}\\
(1) Установить различные токи через датчик Холла (\textbf{I1}), \\
(2) Изменять магнитное поле электромагнита за счёт изменения \textbf{I2}, \\
(3) Измерять выходное напряжение усилителя (\textbf{U1}), связанное с реальным напряжением Холла \textbf{Ux}, \\
(4) Впоследствии, при обработке, использовать формулы для перевода \textbf{U1} в \textbf{Ux} (с учётом коэффициента усиления $k$), а также для расчёта индукции $B$ (через известные $B_{\text{н}}$ и $a$).

\end{document}
