% ==================
% answer.tex
% (Основной материал:
%  "Определение интеграла Римана. Отличие от обычного предела.")
% ==================

\begin{customtheorem}[Интеграл Римана]
	Пусть функция $f$ задана на $[a,b]$. Если при всех возможных способах разбиения $[a,b]$ на малые отрезки,
	и выборе точек $\xi_i$ внутри этих отрезков, интегральные суммы
	\[
		S = \sum_{i=1}^n f(\xi_i)\,\Delta x_i
	\]
	стремятся к одному и тому же числу $I$ по мере $\max_i \Delta x_i \to 0$,
	то $f$ \textbf{интегрируема по Риману}, а $I$ называется \emph{интегралом Римана}:
	\[
		I = \int_{a}^{b} f(x)\,dx.
	\]
\end{customtheorem}

\begin{proofplan}
	\begin{enumerate}
		\item Рассмотреть любые два разбиения $D$ и $D'$ на отрезке $[a,b]$ с мелкостью $\|D\|\to0$, $\|D'\|\to0$.
		\item Построить общее уточнённое разбиение $D''$, включающее все точки из $D$ и $D'$.
		\item Оценить разницу сумм $S(f,D)$ и $S(f,D')$ через равномерную непрерывность (или ограниченность) $f$ на $[a,b]$.
		\item Показать, что эта разница становится сколь угодно малой при $\|D\|\to0$ и $\|D'\|\to0$.
		\item Вывод: предел един, определение интеграла однозначно.
	\end{enumerate}
\end{proofplan}

\begin{customproof}
	\textbf{Шаг 1: Два разбиения.}
	Пусть $D=\{a=x_0<x_1<\dots<x_n=b\}$ и $D'=\{a=y_0<y_1<\dots<y_m=b\}$ — любые разбиения отрезка $[a,b]$.
	Предположим, что $\|D\|=\max_i(x_i-x_{i-1})$ и $\|D'\|=\max_j(y_j-y_{j-1})$ оба стремятся к нулю.

	\smallskip

	\textbf{Шаг 2: Уточнение.}
	Построим «общее» разбиение $D''$, содержащее все точки из $D$ и $D'$.
	То есть объединим набор $\{x_i\}$ с $\{y_j\}$ в одну возрастающую последовательность.
	Теперь можно рассмотреть интегральные суммы относительно $D''$.

	\smallskip

	\textbf{Шаг 3: Оценка разницы.}
	На каждом элементе разбиения $[z_{k-1}, z_k]$ из $D''$ значения $f(\xi_i)$ меняются незначительно, если $f$ равномерно непрерывна (или ограничена).
	Тогда можно показать, что разность сумм $S(f,D)$ и $S(f,D')$ не превосходит
	некоторой малой величины, зависящей от $\|D''\|$, которая стремится к нулю,
	когда и $\|D\|\to0$, $\|D'\|\to0$.

	\smallskip

	\textbf{Шаг 4: Вывод.}
	Таким образом, любая интегральная сумма при мелкости разбиения стремится к одной и той же границе.
	Значит определение интеграла Римана корректно, и этот предел называется $\int_a^b f(x)\,dx$.
\end{customproof}

\begin{customexample}
	\begin{itemize}
		\item Если $f(x) \equiv C$ — константа, любая интегральная сумма $=C\cdot(b-a)$.
		      При любом разбиении ответ один: $\int_a^b C\,dx = C\,(b-a)$.
		\item Сравнение с обычным пределом:
		      \(\lim_{x\to x_0}f(x)\) — локальный анализ окрестности $x_0$.
		      \(\int_a^b f(x)\,dx\) — «суммарный» (глобальный) взгляд на отрезок $[a,b]$.
	\end{itemize}
\end{customexample}
