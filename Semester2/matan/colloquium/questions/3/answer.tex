% =========================
% answer.tex
% (Основной материал вопроса)
% =========================

\begin{customtheorem}[Теорема Лагранжа (о среднем значении)]
	Пусть функция $f$ непрерывна на отрезке $[a,b]$ и дифференцируема на интервале $(a,b)$.
	Тогда существует точка $c \in (a,b)$ такая, что
	\[
		f(b) - f(a) \;=\; f'(c)\,\bigl(b - a\bigr).
	\]
\end{customtheorem}

\begin{proofplan}
	\begin{enumerate}
		\item Эта теорема является обобщением Теоремы Ролля (см. вспомогательные понятия).
		\item Рассмотрим функцию $F(x) = f(x) - \alpha x$, где
		      \(\displaystyle \alpha = \frac{f(b) - f(a)}{\,b - a\,}\).
		\item Заметим, что $F(a) = F(b)$, откуда по Теореме Ролля существует $c$ с $F'(c)=0$.
		\item Тогда $F'(c)=f'(c)-\alpha=0 \implies f'(c) = \alpha$, и $\alpha$ равна
		      $\tfrac{f(b)-f(a)}{\,b-a\,}$.
	\end{enumerate}
\end{proofplan}

\begin{customproof}
	Обозначим $\displaystyle \alpha = \frac{f(b)-f(a)}{\,b - a\,}$.
	Рассмотрим $F(x) = f(x) - \alpha x$. Тогда
	\[
		F(b) - F(a) \;=\; \bigl[f(b)-\alpha b\bigr] \;-\; \bigl[f(a)-\alpha a\bigr]
		\;=\; \bigl[f(b)-f(a)\bigr] - \alpha\,(b-a) = 0.
	\]
	По условию, $F$ непрерывна на $[a,b]$ и дифференцируема на $(a,b)$
	(как разность таких же функций).
	Из $F(a)=F(b)$ следует, что по Теореме Ролля существует $c\in(a,b)$
	с $F'(c)=0$. Но $F'(x)=f'(x)-\alpha$, значит
	\[
		f'(c) - \alpha \;=\; 0
		\quad\Longrightarrow\quad
		f'(c) \;=\; \alpha
		\;=\;
		\frac{f(b)-f(a)}{\,b - a\,}.
	\]
	Это и требовалось доказать.
\end{customproof}

\begin{customtheorem}[Условие постоянства дифференцируемой функции]
	Пусть $f$ дифференцируема на промежутке $(a,b)$.
	Тогда $f$ постоянна на $(a,b)$ \textbf{тогда и только тогда},
	когда
	\[
		f'(x) \;=\; 0 \quad\text{для всех}\; x \in (a,b).
	\]
\end{customtheorem}

\begin{proofplan}
	\begin{enumerate}
		\item Если $f'(x)=0$ всюду, по Теореме Лагранжа (или Ролля) разность $f(x_2)-f(x_1)$
		      оказывается равной нулю, значит $f$ постоянна.
		\item Если $f$ постоянна, то очевидно $f'(x)=0$.
	\end{enumerate}
\end{proofplan}

\begin{customproof}
	\textbf{(Необходимость)} Если $f$ константа, тогда для любых $x_1,x_2$ выполняется
	$f(x_2)=f(x_1)$, откуда $f'(x)=0$ в любой точке, где существует производная.

	\smallskip

	\textbf{(Достаточность)} Пусть $f'(x)=0$ для всех $x\in(a,b)$. Возьмём любые
	$x_1<x_2$ в $(a,b)$. Применим Теорему Лагранжа на отрезке $[x_1,x_2]$: найдётся
	$c\in(x_1,x_2)$, что
	\[
		f(x_2)-f(x_1) \;=\; f'(c)\,\bigl(x_2-x_1\bigr).
	\]
	Но $f'(c)=0$, значит $f(x_2)=f(x_1)$. Следовательно, $f$ одно и то же число
	на всём $(a,b)$.
\end{customproof}

\begin{customtheorem}[Условие монотонности дифференцируемой функции]
	Пусть $f$ дифференцируема на промежутке $(a,b)$.
	\begin{itemize}
		\item $f$ \textbf{возрастает} на $(a,b)$
		      $\;\Longleftrightarrow\;$
		      $f'(x)\ge 0$ для всех $x\in(a,b)$
		      (причём множество нулей $f'(x)=0$ не содержит интервалов).
		\item $f$ \textbf{убывает} на $(a,b)$
		      $\;\Longleftrightarrow\;$
		      $f'(x)\le 0$ для всех $x\in(a,b)$
		      (и множество нулей не содержит интервалов).
	\end{itemize}
\end{customtheorem}

\begin{proofplan}
	\begin{enumerate}
		\item Если $f'(x)\ge 0$ на $(a,b)$, то для $x_2>x_1$ по Теореме Лагранжа
		      $f(x_2)-f(x_1)=f'(c)\,(x_2-x_1)\ge 0$.
		\item Если $f$ возрастает, то
		      \(\tfrac{f(x_2)-f(x_1)}{\,x_2-x_1\,}\ge 0\). Переходя к пределу, получаем $f'(x)\ge0$.
		\item Уточнение про то, что при равенстве нулю на целом подинтервале, функция
		      фактически становится постоянной там, что «ломает» строгое возрастание.
	\end{enumerate}
\end{proofplan}

\begin{customproof}
	\textbf{(Случай возрастания)}
	Пусть $x_1 < x_2$.
	Если $f'(x)\ge 0$, по Теореме Лагранжа найдётся $c\in(x_1,x_2)$:
	\[
		f(x_2)-f(x_1) \;=\; f'(c)\,(x_2-x_1)\,\ge\,0,
	\]
	значит $f(x_2)\ge f(x_1)$ — неубывание.

	Обратное: если $f$ возрастает,
	\(\tfrac{f(x_2)-f(x_1)}{x_2-x_1}\ge0\). Переходя к пределу при $x_2\to x_1$,
	получаем $f'(x_1)\ge0$. Аналогичные аргументы для убывания (меняются знаки).

	Если $f'(x)=0$ на целом подинтервале, там $f$ постоянна, нарушая «строгое» возрастание.

\end{customproof}

\begin{customexample}
	\begin{itemize}
		\item \textbf{Функция}, у которой $f'(x)\ge0$, но есть точки с $f'(x)=0$,
		      будет возрастать (не строго), однако если такие нули идут целым отрезком,
		      то там $f$ постоянна.
		\item \textbf{Пример}: $f(x)=x^3$ на $\mathbb{R}$: $f'(x)=3x^2\ge0$,
		      значит $f$ возрастает на всей оси. При этом $f'(0)=0$, но это всего одна точка.
	\end{itemize}
\end{customexample}
