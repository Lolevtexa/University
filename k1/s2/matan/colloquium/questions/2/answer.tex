% ============================
% Ответ на вопрос №2
% "Дифференциал функции. Теорема Ферма. Теорема Ролля"
% ============================

\begin{customtheorem}[Теорема Ферма]
	Если функция $f$ дифференцируема в точке $x_0$ и имеет там локальный минимум
	или максимум, то $f'(x_0) = 0$.
\end{customtheorem}

\begin{proofplan}
	\begin{enumerate}
		\item Пусть $x_0$ — точка локального минимума, тогда для $x$ рядом с $x_0$ выполняется
		      $f(x)\ge f(x_0)$.
		\item Рассмотреть разность $\frac{f(x)-f(x_0)}{x-x_0}$ при $x>x_0$ и при $x<x_0$
		      и перейти к пределу.
		\item Получить, что $f'(x_0)\ge0$ и $f'(x_0)\le0$, откуда $f'(x_0)=0$.
		\item Случай локального максимума аналогичен.
	\end{enumerate}
\end{proofplan}

\begin{customproof}
	Пусть $x_0$ — точка локального минимума. Тогда существует $\delta>0$, что при
	$|x - x_0| < \delta$ верно $f(x)\ge f(x_0)$. Для $x>x_0$ имеем
	\[
		\frac{f(x) - f(x_0)}{x - x_0}\,\ge\,0.
	\]
	Переходя к пределу при $x \to x_0^+$, получаем $f'(x_0)\ge0$.
	Аналогично, если $x<x_0$, то $x - x_0 < 0$, и разность $f(x) - f(x_0)$ остаётся неотрицательной,
	что даёт $f'(x_0)\le 0$.
	Значит $f'(x_0)\ge0$ и $f'(x_0)\le0$, откуда $f'(x_0)=0$.
	В случае локального максимума знак меняется, но рассуждение то же.
	Таким образом, если у $f$ есть локальный экстремум в точке $x_0$, то $f'(x_0)=0$.
\end{customproof}

\begin{customtheorem}[Теорема Ролля]
	Пусть $f$ непрерывна на отрезке $[a,b]$, дифференцируема на интервале $(a,b)$
	и при этом $f(a)=f(b)$. Тогда существует хотя бы одна точка $c\in(a,b)$ такая, что
	\[
		f'(c)\;=\;0.
	\]
\end{customtheorem}

\begin{proofplan}
	\begin{enumerate}
		\item Если $f$ постоянна на $[a,b]$, то $f'(x)=0$ на $(a,b)$, и нужная точка $c$ может быть любая.
		\item Если $f$ не постоянна, по Теореме Вейерштрасса достигаются минимум и максимум
		      на $[a,b]$ в точках $x_{\min}, x_{\max}$.
		\item Поскольку $f(a)=f(b)$, хотя бы один из экстремумов
		      не может «жить» только на концах, значит есть локальный экстремум внутри $(a,b)$.
		\item По Теореме Ферма в точке локального экстремума $c$ имеем $f'(c)=0$.
	\end{enumerate}
\end{proofplan}

\begin{customproof}
	Предположим, что $f$ не постоянна (иначе всё очевидно). По непрерывности и
	Теореме Вейерштрасса, функция $f$ достигает своего минимума и максимума
	на отрезке $[a,b]$ (в точках $x_{\min}$ и $x_{\max}$).
	Поскольку $f(a)=f(b)$, по крайней мере один из этих экстремумов
	не может приходиться только на границы; значит существует $c\in(a,b)$,
	где $f$ имеет локальный экстремум. По Теореме Ферма это даёт $f'(c)=0$.
\end{customproof}

\begin{customexample}
	\begin{itemize}
		\item \textbf{Дифференциал:}
		      Для $f(x)=x^2$ в точке $x_0=2$ получаем $f'(2)=4$. Тогда
		      при малом $dx$, $df(2) = 4\,dx$. Если $dx=0.1$, то $df(2)=0.4$, а
		      реальное $f(2.1)-f(2)$ будет $4.41-4=0.41$, что близко к $0.4$.
		\item \textbf{Пример (Теорема Ферма):}
		      $f(x)=x^2$ имеет локальный минимум в $x=0$, причём $f'(0)=0$.
		\item \textbf{Пример (Теорема Ролля):}
		      На $[0,4]$ возьмём $f(x)=x^2 - 4x$. Тогда $f(0)=f(4)=0$.
		      Применяем Теорему Ролля: найдётся $c\in(0,4)$ с $f'(c)=0$.
		      И вправду, $f'(x)=2x-4$, значит $c=2$.
	\end{itemize}
\end{customexample}
