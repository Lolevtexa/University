% ===== 4.tex =====
% "Равномерная непрерывность. Примеры. Теорема Кантора о равномерной непрерывности."

% 1. Определения

\textbf{Непрерывность (точечная).}
Функция $f$ называется \emph{непрерывной} в точке $x_0$, если
\[
\forall \varepsilon>0\;\;\exists \delta(\varepsilon, x_0)>0:\;
|x - x_0|<\delta \;\Longrightarrow\; |f(x)-f(x_0)|<\varepsilon.
\]
Если это выполняется для всех $x_0$ из множества $X$, то говорят, что $f$ непрерывна на $X$.

\medskip

\textbf{Равномерная непрерывность.}
Пусть $f$ задана на $X\subseteq \mathbb{R}$. Функция $f$ называется \emph{равномерно непрерывной} на $X$, если
\[
\forall \varepsilon>0\;\;\exists \delta(\varepsilon)>0:\;
\forall x_1, x_2 \in X,\quad |x_1 - x_2| < \delta
\;\Longrightarrow\; |f(x_1)-f(x_2)| < \varepsilon.
\]
Существенное отличие от обычной непрерывности — в том, что \(\delta\) выбирается \emph{только} в зависимости от \(\varepsilon\) и \textbf{не} зависит от точки $x_0$.

\medskip

% 2. Теоремы и ключевые утверждения

\textbf{Теорема Кантора.}
Если $f$ непрерывна на \emph{компактном} множестве $X\subset \mathbb{R}$, то $f$ \textbf{равномерно непрерывна} на $X$.

\medskip

% 3. Основные идеи доказательства (коротко)

\begin{itemize}
  \item Доказывать будем \emph{от противного}: предположим, что $f$ непрерывна на компакте, но не равномерно непрерывна.
  \item Это означает существование $\varepsilon_0>0$ такого, что нельзя подобрать «глобальное» $\delta$, выполняющее условие равномерной непрерывности.
  \item Для каждой $n$, выбрав $\delta=1/n$, находятся точки $(x_n,y_n)$ с $|x_n-y_n|<1/n$, но $|f(x_n)-f(y_n)|\ge \varepsilon_0$.
  \item Компактность $X$ обеспечивает возможность извлечения сходящейся подпоследовательности $(x_{n_k}) \to c$. Поскольку $|x_{n_k}-y_{n_k}|<1/{n_k}\to0$, то $y_{n_k}\to c$ тоже.
  \item По непрерывности $f$ получаем $f(x_{n_k})\to f(c)$ и $f(y_{n_k})\to f(c)$, что даёт $|f(x_{n_k})-f(y_{n_k})|\to 0$, противореча условию \(\ge \varepsilon_0\).
  \item Противоречие доказывает равномерную непрерывность.
\end{itemize}

\medskip

% 4. Полное доказательство (шаги)

\subsection*{Доказательство Теоремы Кантора}

\textbf{Шаг 1: Предположение от противного.}  
Пусть $f$ непрерывна на компактном $X$, но \textbf{не} равномерно непрерывна. Тогда
\[
\exists\,\varepsilon_0>0 :\quad
\forall \delta>0,\;\exists x,y\in X:\;
|x-y|<\delta,\;\;
|f(x)-f(y)| \ge \varepsilon_0.
\]

\textbf{Шаг 2: Построение последовательностей.}  
Выбираем $\delta=1/n$. Появляются пары $(x_n, y_n)\in X$ со свойствами:
\[
|x_n - y_n| < \frac{1}{n},\quad
|f(x_n) - f(y_n)| \ge \varepsilon_0.
\]

\textbf{Шаг 3: Извлечение сходящейся подпоследовательности (компактность).}  
Последовательность $(x_n)$, являясь «лежачей» в $X$, по теореме Болльцано–Вейерштрасса (см. \texttt{sub\_4.tex}) имеет сходящуюся подпоследовательность $(x_{n_k})\to c\in X$. Тогда $|x_{n_k}-y_{n_k}| < 1/{n_k}\to0 \implies y_{n_k}\to c$.

\textbf{Шаг 4: Применение непрерывности к подпоследовательностям.}  
Из непрерывности $f$ в точке $c$: 
\[
f(x_{n_k}) \to f(c),
\quad
f(y_{n_k}) \to f(c),
\]
значит \(|f(x_{n_k}) - f(y_{n_k})|\to 0\).

\textbf{Шаг 5: Противоречие.}  
По построению же \(|f(x_{n_k}) - f(y_{n_k})|\ge \varepsilon_0>0\). Получаем противоречие «\(\to 0\)» против «\(\ge \varepsilon_0\)». Значит исходное предположение ложно.

\textbf{Шаг 6: Вывод.}  
Таким образом, функция $f$ \textbf{является} равномерно непрерывной на $X$.

\medskip

% 5. Примеры

\textbf{Пример (равномерная непрерывность на $\mathbb{R}$).}
\begin{itemize}
  \item Линейная функция: $f(x)=kx+b$. Для любой пары $x_1, x_2$ справедливо $|f(x_1)-f(x_2)|=|k|\,|x_1-x_2|$, здесь очевидна равномерная непрерывность (достаточно взять $\delta=\tfrac{\varepsilon}{|k|}$).
  \item Тригонометрические функции $\sin x$, $\cos x$ на всей $\mathbb{R}$ тоже равномерно непрерывны (ограничены и периодичны, их изменения «контролируются»).
\end{itemize}

\textbf{Пример (не равномерно непрерывные на $\mathbb{R}$).}
\begin{itemize}
  \item $f(x)=x^2$: хотя непрерывна на $\mathbb{R}$, она не равномерно непрерывна на всей оси (с ростом $x$ глобальная «\(\delta\)» становится недостаточной).
  \item $e^x$ тоже не равномерно непрерывна на всей $\mathbb{R}$ (аналогичная причина).
\end{itemize}

\medskip

% -- Логическая связность --
% (Определения "компактности" и "непрерывности" в полной мере см. sub_4.tex)
