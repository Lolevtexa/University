% ==================
% auxiliary.tex
% (Вспомогательные определения к теме:
%  "Формулы Маклорена для e^x, sin x, cos x, ln(1+x), (1+x)^a.")
% ==================

\paragraph{Ряд Маклорена (частный случай ряда Тейлора).}
Пусть функция $f$ бесконечно дифференцируема в окрестности $x=0$.
Тогда её \emph{ряд Маклорена} --- это разложение вида
\[
	f(x) \;=\; \sum_{n=0}^\infty \frac{f^{(n)}(0)}{n!}\,x^n,
\]
если данный ряд сходится к $f(x)$ для соответствующих значений $x$.
Радиус и область сходимости могут быть разными в зависимости от особенностей $f$.

\bigskip

\paragraph{Производные порядка $n$ и значения в точке 0.}
Если $f$ имеет все производные (бесконечно дифференцируема) у $x=0$, тогда
\[
	f^{(n)}(0)
	\;\text{ --- } n\text{-я производная в точке }0.
\]
Эти значения формируют коэффициенты при $x^n$ в ряде Маклорена.

\bigskip

\paragraph{Обобщённая биномиальная формула.}
Для произвольного действительного $a$ и $|x|<1$:
\[
	(1+x)^a
	\;=\;
	1
	+ a\,x
	+ \frac{a(a-1)}{2!}\,x^2
	+ \frac{a(a-1)(a-2)}{3!}\,x^3 + \dots.
\]
Этот ряд сходится при $|x|<1$ и является расширением классического бинома Ньютона (в котором $a$ -- целое неотрицательное число).
