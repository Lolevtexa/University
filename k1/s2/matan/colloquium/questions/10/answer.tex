% ==================
% answer.tex
% (Основной материал:
%  "Формулы Маклорена для e^x, sin x, cos x, ln(1+x), (1+x)^a.")
% ==================

\begin{customtheorem}[Формулы Маклорена]
	Пусть $f$ бесконечно дифференцируема в некоторой окрестности $0$. Тогда для
	каждой из нижеуказанных функций верны следующие ряды Маклорена (при своих радиусах сходимости):

	\begin{enumerate}
		\item $e^x = \displaystyle \sum_{n=0}^{\infty} \frac{x^n}{n!}, \quad x\in\mathbb{R}.$
		\item $\sin x = \displaystyle \sum_{n=0}^{\infty} (-1)^n \frac{x^{2n+1}}{(2n+1)!}, \quad x\in\mathbb{R}.$
		\item $\cos x = \displaystyle \sum_{n=0}^{\infty} (-1)^n \frac{x^{2n}}{(2n)!}, \quad x\in\mathbb{R}.$
		\item $\ln(1+x) = \displaystyle x - \frac{x^2}{2} + \frac{x^3}{3} - \frac{x^4}{4} + \dots,\quad |x|<1.$
		\item $(1+x)^a = \displaystyle 1 + a\,x + \frac{a(a-1)}{2!}\,x^2 + \dots,\quad |x|<1.$
	\end{enumerate}
\end{customtheorem}

\begin{proofplan}
	\begin{enumerate}
		\item Для $e^x,\;\sin x,\;\cos x$ вычислить все производные в точке 0, получить $f^{(n)}(0)$.
		\item Подставить в общий вид \(\sum_{n=0}^\infty \tfrac{f^{(n)}(0)}{n!}\,x^n\).
		\item Показать (или сослаться на известные результаты) о радиусе сходимости:
		      $e^x, \sin x, \cos x$ сходятся на всей $\mathbb{R}$.
		\item Для $\ln(1+x)$ разложить в степенной ряд при $|x|<1$, найти формулы производных,
		      увидеть знакочередующиеся коэффициенты.
		\item Для $(1+x)^a$ --- использовать бином Ньютона (обобщённый) или вывести через производные.
	\end{enumerate}
\end{proofplan}

\begin{customproof}
	\textbf{(1) Функция $e^x$.}
	Все производные $f^{(n)}(x)=e^x$, значит в точке 0 они равны 1. По определению:
	\[
		e^x
		= \sum_{n=0}^\infty \frac{f^{(n)}(0)}{n!}\,x^n
		= \sum_{n=0}^\infty \frac{x^n}{n!}.
	\]
	Радиус сходимости --- неограничен (по признаку д'Аламбера).

	\smallskip

	\textbf{(2) Функция $\sin x$.}
	Производные идут по циклу: $f'(x)=\cos x$, $f''(x)=-\sin x$, \dots .
	В точке 0 они чередуются: 0, 1, 0, -1, \dots . Поэтому
	\[
		\sin x
		= \sum_{n=0}^\infty \frac{f^{(n)}(0)}{n!}\,x^n
		= \sum_{n=0}^\infty (-1)^n \frac{x^{2n+1}}{(2n+1)!}.
	\]
	Ряд сходится для всех $x\in\mathbb{R}$.

	\smallskip

	\textbf{(3) Функция $\cos x$.}
	Аналогично, если в $\sin x$ заменить фазы производных, получаем
	\[
		\cos x
		= \sum_{n=0}^\infty (-1)^n \frac{x^{2n}}{(2n)!}.
	\]
	Тоже сходится на всей $\mathbb{R}$.

	\smallskip

	\textbf{(4) Функция $\ln(1+x)$.}
	Для $|x|<1$, последовательно вычисляются $f^{(n)}(0)$, давая чередующиеся коэффициенты:
	\[
		\ln(1+x)
		= x - \frac{x^2}{2} + \frac{x^3}{3} - \frac{x^4}{4} + \dots .
	\]
	При $x=1$ получается «ln(2)»-ряд, который сходится условно.

	\smallskip

	\textbf{(5) Функция $(1+x)^a$.}
	Применяем обобщённую биномную формулу (или дифференцируемость порядка $n$) ---
	получаем:
	\[
		(1+x)^a
		= 1 + a\,x + \frac{a(a-1)}{2!}\,x^2
		+ \frac{a(a-1)(a-2)}{3!}\,x^3 + \dots ,
		\quad |x|<1.
	\]
	Это ряд сходящийся в круге $|x|<1$.
	Таким образом, все 5 функций имеют свой ряд Маклорена, рассчитываемый
	из $f^{(n)}(0)$, и каждый сходится в определённой области (свой радиус сходимости).
\end{customproof}

\begin{customexample}
	\textbf{Примеры использования рядов:}
	\begin{itemize}
		\item Подстановка $x=\pi$ в \(\sin x\) даёт \(\sin\pi=0\), а ряд:
		      \(0 - \frac{\pi^3}{3!} + \frac{\pi^5}{5!} - \dots = 0\) (знакочередующаяся сумма).
		\item Для \(\ln(1+\tfrac12) = \ln\!\bigl(\tfrac32\bigr)\approx 0.40536\) можно использовать
		      разложение \( \tfrac12 - \tfrac{(\tfrac12)^2}{2} + \tfrac{(\tfrac12)^3}{3} - \dots \).
		\item \((1+x)^{\tfrac12} = \sqrt{1+x}\) --- частный случай биномной формулы.
	\end{itemize}
\end{customexample}
