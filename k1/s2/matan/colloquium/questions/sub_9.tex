% ===== sub_9.tex =====
% Вспомогательные определения и факты,
% не являющиеся "центральными" в вопросе, но используемые в доказательстве

\textbf{Минимальный многочлен алгебраического числа.}
Пусть $\alpha$ — алгебраическое (корень целого ненулевого многочлена). Его \emph{минимальным многочленом} называется многочлен $P(x)\in \mathbb{Z}[x]$, у которого $\alpha$ — корень, степень $P$ — наименьшая возможная, и старший коэффициент положителен, а все общие делители коэффициентов равны 1.

\medskip

\textbf{Оценка "разности корней".}
Из теории алгебраических уравнений известно, что если $P(x)$ — многочлен степени $d$ с целыми коэффициентами, то расстояния между его корнями не могут быть «слишком маленькими» относительно высоты коэффициентов. Точнее, если $r_1,\dots,r_d$ — корни, то существуют нижние границы $|r_i-r_j|$ в зависимости от коэффициентов $P$ (см. теорему о разложении в произведение линейных множителей).

\medskip

\textbf{Рациональное приближение.}
Если $\alpha$ действительно алгебраична степени $d$, то для больших $q$ рациональные приближения $\frac{p}{q}$ не могут удовлетворять $|\alpha - \frac{p}{q}| < \tfrac{1}{q^n}$ при $n>d$, иначе возникнет противоречие (Теорема Лиувилля).
