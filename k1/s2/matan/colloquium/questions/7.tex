% ===== 7.tex =====
% "Формула Тейлора для многочлена. Формула Тейлора с остатком в форме Пеано."

% 1. Определения

\textbf{Многочлен и его производные.}
Пусть $P(x)$ — многочлен степени $n$, т. е.
\[
P(x) \;=\; a_0 + a_1 x + a_2 x^2 + \dots + a_n x^n.
\]
Он бесконечно дифференцируем на всей $\mathbb{R}$, но начиная с порядка выше $n$, все производные равны нулю.

\medskip

\textbf{Формула Тейлора для многочлена.}
Если $f(x)=P(x)$ — многочлен степени $n$, то для точки $x_0\in\mathbb{R}$ имеем
\[
P(x) \;=\; P(x_0) + \frac{P'(x_0)}{1!}\,(x - x_0) + \dots + \frac{P^{(n)}(x_0)}{n!}\,(x - x_0)^n.
\]
Поскольку производные порядка выше $n$ у многочлена равны $0$, \textit{остаточный член} отсутствует (или равен нулю).

\medskip

\textbf{Формула Тейлора с остатком в форме Пеано.}
Пусть $f$ имеет $(n)$-ю производную в окрестности $x_0$. Тогда
\[
f(x) \;=\; f(x_0) + \frac{f'(x_0)}{1!}\,(x - x_0) + \dots + \frac{f^{(n)}(x_0)}{n!}\,(x - x_0)^n + o\bigl((x - x_0)^n\bigr).
\]
Здесь используется \emph{малое «о»}: 
\[
r_n(x) \;=\; o\bigl((x - x_0)^n\bigr),\quad x \to x_0,
\]
значит
\[
\lim_{x \to x_0} \frac{r_n(x)}{(x - x_0)^n} = 0.
\]

\medskip

% 2. Теоремы и ключевые утверждения

\textbf{Теорема (Формула Тейлора для многочлена).}
Если $P(x)$ — многочлен степени $n$, то его разложение в окрестности $x_0$ \emph{совпадает} со стандартным полиномом Тейлора степени $n$, а \emph{остаток} (производные порядка выше $n$) равен $0$.

\textbf{Теорема (Формула Тейлора с остатком в форме Пеано).}
Пусть $f$ $n$ раз дифференцируема в точке $x_0$. Тогда
\[
f(x) = \sum_{k=0}^{n}\frac{f^{(k)}(x_0)}{k!}\,(x - x_0)^k \;+\; o\bigl((x - x_0)^n\bigr).
\]

\medskip

% 3. Основные идеи доказательства (коротко)

\textbf{Для многочлена:}
\begin{itemize}
  \item Все производные порядка выше $n$ у многочлена равны нулю.  
  \item Поэтому «ряд Тейлора» фактически совпадает с самим многочленом. Остаточный член $R_n(x)$ тождественно $0$.
\end{itemize}

\textbf{Для формы Пеано:}
\begin{itemize}
  \item Малая функция $o\bigl((x - x_0)^n\bigr)$ означает, что остаток \emph{«уходит в ноль»} быстрее, чем $(x - x_0)^n$.  
  \item Используется определение непрерывности производных и соответствующих пределов, где высокие приращения в $(x - x_0)^n$ доминируют над остаточным членом.
\end{itemize}

\medskip

% 4. Полное доказательство (ссылаясь на sub_7 при необходимости)

\subsection*{Доказательство Формулы Тейлора для многочлена}
\begin{enumerate}
  \item \textbf{Расширенный полином.}  
  Пусть $P(x)$ — многочлен степени $n$. Рассмотрим многочлен Тейлора (степени $n$) вокруг $x_0$:
  \[
  T_n(x) \;=\; \sum_{k=0}^{n}\frac{P^{(k)}(x_0)}{k!}\,(x - x_0)^k.
  \]
  \item \textbf{Равенство старших производных нулю.}  
  Поскольку $P^{(m)}(x)\equiv0$ при $m>n$, никакой остаточный член не возникает.
  \item \textbf{Сравнение $P(x)$ и $T_n(x)$.}  
  Уже само определение производных наглядно показывает: $P(x)$ и $T_n(x)$ — один и тот же полином (коэффициенты совпадают).  
  \item \textbf{Вывод.}  
  Таким образом, $P(x) = T_n(x)$ \(\implies\) формула Тейлора для многочлена полностью совпадает с самим многочленом.
\end{enumerate}

\subsection*{Доказательство Формулы Тейлора с остатком в форме Пеано}
\begin{enumerate}
  \item \textbf{Постановка задачи.}  
  Требуется показать, что
  \[
  f(x) = f(x_0) + \frac{f'(x_0)}{1!}(x - x_0) + \dots + \frac{f^{(n)}(x_0)}{n!}(x - x_0)^n \;+\; r_n(x),
  \]
  где $r_n(x)=o((x-x_0)^n)$ при $x\to x_0$.
  \item \textbf{Индикатор быстрого убывания $r_n(x)$.}  
  То есть 
  \(\lim\limits_{x\to x_0}\frac{r_n(x)}{(x - x_0)^n}=0.\)
  \item \textbf{Использование дифференцируемости порядка $n$.}  
  По определению (см. \texttt{sub\_7.tex}), если $f$ имеет непрерывные производные до порядка $n$, то мы можем разложить $f$ по (классической) формуле Тейлора (с формой Лагранжа для остатка) и затем показать, что этот остаток $\frac{f^{(n+1)}(\xi)}{(n+1)!}(x-x_0)^{n+1}$ ведёт себя как $o((x-x_0)^n)$, если $f^{(n+1)}$ непрерывна в $x_0$ и, например, равна нулю там \emph{или} $f^{(n+1)}$ ограничена в малой окрестности.
  \item \textbf{Точнее.}  
  Если $f^{(n)}$ непрерывна в $x_0$, то $f^{(n+1)}(\xi)\to f^{(n+1)}(x_0)$ при $\xi\to x_0$ (если $(n+1)$-я производная существует в окрестности). Даже если $f^{(n+1)}(x_0)$ сама равна нулю — остаток $(x-x_0)^{n+1}$ «уходит» быстрее, чем $(x-x_0)^n$.  
  \item \textbf{Вывод о малом «о»}.  
  Отсюда следует требуемое $r_n(x)=o((x - x_0)^n)$. 
\end{enumerate}

\medskip

% 5. Короткие примеры

\textbf{Пример 1:} $f(x)=x^3$. Разложение вокруг $x_0=1$:
\[
f'(x)=3x^2,\quad f''(x)=6x,\quad f^{(3)}(x)=6,\quad f^{(4)}(x)=0,\dots
\]
По формуле Тейлора (степени $3$), никакого «остатка» не остаётся, так как это многочлен. Итог:
\[
x^3 = 1 + 3(x-1) + 3(x-1)^2 + (x-1)^3.
\]

\textbf{Пример 2:} $f(x)=e^x$. Разложение до $n$-го члена в точке $0$ даёт:
\[
e^x = 1 + x + \frac{x^2}{2!} + \dots + \frac{x^n}{n!} + o(x^n),\quad x\to0.
\]
\emph{Здесь} уже работает «форма Пеано», так как $e^x$ не полином, но $e^{(k)}(0)=1$.

\medskip

% -- Логическая связность --
% (Все факты о существовании производных, определении "o(...)" см. в sub_7.tex).
