\begin{enumerate}
	\item Найдём матрицы перехода $C_{e\to f}$ и $C_{f\to e}$:

	      Запишем матрицы \(E\) и \(F\):
	      \[
		      E = \begin{pmatrix}
			      1  & -1 & 0  \\
			      -3 & 4  & -2 \\
			      -1 & 1  & 1  \\
		      \end{pmatrix},
		      F = \begin{pmatrix}
			      1  & -1 & 1  \\
			      -3 & 3  & -2 \\
			      -2 & 3  & -1 \\
		      \end{pmatrix}.
	      \]

	      Матрица перехода \(C_{e \to f}\) это такая матрица, что:
	      \[
		      F = E C_{e \to f} \Rightarrow C_{e \to f} = E^{-1} F.
	      \]

	      Вычислим \(E^{-1}\):
	      \[
		      \left[
			      \begin{array}{ccc|ccc}
				      1  & -1 & 0  & 1 & 0 & 0 \\
				      -3 & 4  & -2 & 0 & 1 & 0 \\
				      -1 & 1  & 1  & 0 & 0 & 1 \\
			      \end{array}
			      \right]
		      \begin{array}{c}
			      \\
			      +3 \text{I} \\
			      +\text{I}   \\
		      \end{array}
		      \sim
		      \left[
			      \begin{array}{ccc|ccc}
				      1 & -1 & 0  & 1 & 0 & 0 \\
				      0 & 1  & -2 & 3 & 1 & 0 \\
				      0 & 0  & 1  & 1 & 0 & 1 \\
			      \end{array}
			      \right]
		      \begin{array}{c}
			      \\
			      +2 \text{III} \\
			      \\
		      \end{array}
		      \sim
	      \]
	      \[
		      \sim
		      \left[
			      \begin{array}{ccc|ccc}
				      1 & -1 & 0 & 1 & 0 & 0 \\
				      0 & 1  & 0 & 5 & 1 & 2 \\
				      0 & 0  & 1 & 1 & 0 & 1 \\
			      \end{array}
			      \right]
		      \begin{array}{c}
			      +\text{II} \\
			      \\
			      \\
		      \end{array}
		      \sim
		      \left[
			      \begin{array}{ccc|ccc}
				      1 & 0 & 0 & 6 & 1 & 2 \\
				      0 & 1 & 0 & 5 & 1 & 2 \\
				      0 & 0 & 1 & 1 & 0 & 1 \\
			      \end{array}
			      \right]
	      \]
	      \[
		      E^{-1} = \begin{pmatrix}
			      6 & 1 & 2 \\
			      5 & 1 & 2 \\
			      1 & 0 & 1 \\
		      \end{pmatrix}.
	      \]

	      \[
		      C_{e \to f} = E^{-1} F =
		      \begin{pmatrix}
			      6 & 1 & 2 \\
			      5 & 1 & 2 \\
			      1 & 0 & 1 \\
		      \end{pmatrix}
		      \begin{pmatrix}
			      1  & -1 & 1  \\
			      -3 & 3  & -2 \\
			      -2 & 3  & -1 \\
		      \end{pmatrix} =
		      \begin{pmatrix}
			      -1 & 3 & 2 \\
			      -2 & 4 & 1 \\
			      -1 & 2 & 0 \\
		      \end{pmatrix}.
	      \]

	      Матрица перехода $C_{f \to e}$ — обратная к $C_{e \to f}$:
	      \[
		      \left[
			      \begin{array}{ccc|ccc}
				      -1 & 3 & 2 & 1 & 0 & 0 \\
				      -2 & 4 & 1 & 0 & 1 & 0 \\
				      -1 & 2 & 0 & 0 & 0 & 1 \\
			      \end{array}
			      \right]
		      \begin{array}{c}
			      *(-1)       \\
			      -2 \text{I} \\
			      -\text{I}   \\
		      \end{array}
		      \sim
		      \left[
			      \begin{array}{ccc|ccc}
				      1 & -3 & -2 & -1 & 0 & 0 \\
				      0 & -2 & -3 & -2 & 1 & 0 \\
				      0 & -1 & -2 & -1 & 0 & 1 \\
			      \end{array}
			      \right]
		      \begin{array}{c}
			      \\
			      -2 \text{III} \\
			      *(-1)         \\
		      \end{array}
		      \sim
	      \]
	      \[
		      \sim
		      \left[
			      \begin{array}{ccc|ccc}
				      1 & -3 & -2 & -1 & 0 & 0  \\
				      0 & 0  & 1  & 0  & 1 & -2 \\
				      0 & 1  & 2  & 1  & 0 & -1 \\
			      \end{array}
			      \right]
		      \begin{array}{c}
			      +2 \text{II} \\
			      \\
			      -2 \text{II} \\
		      \end{array}
		      \sim
		      \left[
			      \begin{array}{ccc|ccc}
				      1 & -3 & 0 & -1 & 2  & -4 \\
				      0 & 0  & 1 & 0  & 1  & -2 \\
				      0 & 1  & 0 & 1  & -2 & 3  \\
			      \end{array}
			      \right]
		      \begin{array}{c}
			      +3 \text{III} \\
			      \\
			      \\
		      \end{array}
		      \sim
	      \]
	      \[
		      \sim
		      \left[
			      \begin{array}{ccc|ccc}
				      1 & 0 & 0 & 2 & -4 & 5  \\
				      0 & 0 & 1 & 0 & 1  & -2 \\
				      0 & 1 & 0 & 1 & -2 & 3  \\
			      \end{array}
			      \right]
		      \begin{array}{c}
			      \\
			      = \text{III} \\
			      = \text{II}  \\
		      \end{array}
		      \sim
		      \left[
			      \begin{array}{ccc|ccc}
				      1 & 0 & 0 & 2 & -4 & 5  \\
				      0 & 1 & 0 & 1 & -2 & 3  \\
				      0 & 0 & 1 & 0 & 1  & -2 \\
			      \end{array}
			      \right].
	      \]

	\item Найдём координаты $x$ в базисе $e$:
	      \[
		      E x_e = x \Rightarrow x_e = E^{-1} x =
		      \begin{pmatrix}
			      6 & 1 & 2 \\
			      5 & 1 & 2 \\
			      1 & 0 & 1 \\
		      \end{pmatrix}
		      \begin{pmatrix}
			      3  \\
			      -3 \\
			      -2 \\
		      \end{pmatrix} =
		      \begin{pmatrix}
			      11 \\
			      8  \\
			      1  \\
		      \end{pmatrix}
	      \]
\end{enumerate}