\subsection{Число штрихов \(N\)}

Число штрихов определяется по формуле
\[
	N = \frac{L}{d}\,,
\]
где \(L = 1{,}50\ \text{см}\) — длина активной части решётки,
и \(d = 447{,}786\ \text{мкм}\) — постоянная решётки.
Переводим в одни и те же единицы (метры):
\[
	L = 1{,}50\,\text{см} = 0{,}0150\,\text{м},
	\quad
	d = 447{,}786\,\mu\text{м} = 447{,}786 \times 10^{-6}\,\text{м}.
\]

Получаем
\[
	N = \frac{0{,}0150}{447{,}786\times10^{-6}}
	\approx 33{,}52.
\]

Округляем до целого: \(N = 34\).
Результат занесён в Таблицу 2.
