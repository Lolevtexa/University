\documentclass[12pt]{article}
\usepackage[left=1cm, right=1cm, top=1cm, bottom=1cm]{geometry} 
\usepackage[T2A]{fontenc}
\usepackage[utf8]{inputenc}
\usepackage[russian,english]{babel}
\usepackage{multicol}
\setlength{\parindent}{0pt}

\newcommand{\readunder}[3]{%
  \begin{tabular}[t]{@{}c@{}}#1\\ \textit{\footnotesize #2}\\ {\tiny #3} \end{tabular}%
}

\begin{document}

\section*{Albert Einstein — текст}

\noindent
\readunder{Albert}{Альберт}{Альберт} \readunder{Einstein}{Айнштайн}{Эйнштейн} \readunder{was}{уоз}{был} \readunder{born}{борн}{рождён} \readunder{in}{ин}{в} \readunder{Germany}{Джёрмани}{Германии} \readunder{in}{ин}{в} \readunder{1879}{эйтин севенти найн}{1879}.\\
\readunder{As}{эз}{Когда} \readunder{a}{э}{} \readunder{boy}{бой}{мальчиком}, \readunder{he}{хи}{он} \readunder{thought}{сот}{думал} \readunder{school}{скул}{школу} \readunder{was}{уоз}{была} \readunder{boring}{боринг}{скучной}.\\
\readunder{At}{эт}{В} \readunder{the}{зе}{} \readunder{age}{эйдж}{возрасте} \readunder{of}{ов}{в} \readunder{fifteen}{фифтин}{пятнадцать}, \readunder{he}{хи}{он} \readunder{left}{лефт}{покинул} \readunder{school}{скул}{школу} \readunder{without}{визаут}{без} \readunder{any}{эни}{каких-либо} \readunder{qualifications}{квалификейшнз}{аттестатов}.\\

\readunder{However}{хауэвэ}{Однако}, \readunder{a}{э}{} \readunder{few}{фью}{несколько} \readunder{years}{йиаз}{лет} \readunder{later}{лейтэ}{спустя} \readunder{he}{хи}{он} \readunder{continued}{кэнтиньюд}{продолжил} \readunder{with}{виз}{с} \readunder{his}{хиз}{своими} \readunder{studies}{стадиз}{учёбой} \readunder{in}{ин}{в} \readunder{Switzerland}{Суицэрлэнд}{Швейцарии} \readunder{and}{энд}{и} \readunder{graduated}{грэджуэйтид}{закончил} \readunder{in}{ин}{в} \readunder{1900}{найнтин хандред}{1900}.\\

\readunder{Over}{оувэ}{В течение} \readunder{the}{зе}{} \readunder{next}{некст}{следующих} \readunder{few}{фью}{нескольких} \readunder{years}{йиаз}{лет}, \readunder{he}{хи}{он} \readunder{did}{дид}{выполнил} \readunder{a}{э}{} \readunder{lot}{лот}{много} \readunder{of}{ов}{из} \readunder{research}{рисёрч}{исследований} \readunder{in}{ин}{в} \readunder{mathematics}{мэсэмэтикс}{математике} \readunder{and}{энд}{и} \readunder{physics}{физикс}{физике}.\\

\readunder{He}{хи}{Он} \readunder{wrote}{роут}{написал} \readunder{articles}{артиклз}{статьи} \readunder{for}{фо}{для} \readunder{scientific}{сайэнтифик}{научных} \readunder{magazines}{мэгэзинз}{журналов} \readunder{about}{эбаут}{о} \readunder{his}{хиз}{своих} \readunder{discoveries}{дискавериз}{открытиях}, \readunder{which}{вич}{которые} \readunder{changed}{чейнджд}{изменили} \readunder{man's}{мэнз}{человеческое} \readunder{view}{вью}{представление} \readunder{of}{ов}{о} \readunder{the}{зе}{} \readunder{universe}{юнивёс}{вселенной}.\\

\readunder{In}{ин}{В} \readunder{1921}{найнтин твэнти уан}{1921} \readunder{he}{хи}{он} \readunder{won}{вон}{выиграл} \readunder{the}{зе}{} \readunder{Nobel}{нобел}{Нобелевскую} \readunder{Prize}{прайз}{премию} \readunder{for}{фо}{по} \readunder{Physics}{физикс}{физике} \readunder{and}{энд}{и} \readunder{became}{бикейм}{стал} \readunder{one}{уан}{одним} \readunder{of}{ов}{из} \readunder{the}{зе}{} \readunder{most}{моуст}{самых} \readunder{respected}{риспэктэд}{уважаемых} \readunder{physicists}{физисистс}{физиков} \readunder{in}{ин}{в} \readunder{the}{зе}{} \readunder{world}{вёулд}{мире}.\\

\readunder{In}{ин}{В} \readunder{1939}{найнтин сёрти найн}{1939}, \readunder{he}{хи}{он} \readunder{left}{лефт}{покинул} \readunder{Germany}{Джёрмани}{Германию} \readunder{and}{энд}{и} \readunder{settled}{сэтлд}{поселился} \readunder{in}{ин}{в} \readunder{America}{Америкэ}{Америке}, \readunder{where}{вэа}{где} \readunder{he}{хи}{он} \readunder{did}{дид}{проводил} \readunder{research}{рисёрч}{исследования}.\\

\readunder{His}{хиз}{Его} \readunder{research}{рисёрч}{исследования} \readunder{and}{энд}{и} \readunder{theories}{сиориз}{теории} \readunder{were}{вё}{были} \readunder{later}{лейтэ}{позже} \readunder{used}{юзд}{использованы} \readunder{to}{ту}{чтобы} \readunder{develop}{девелоп}{разработать} \readunder{the}{зе}{} \readunder{atomic}{атомик}{атомную} \readunder{and}{энд}{и} \readunder{hydrogen}{хайдроджен}{водородную} \readunder{bombs}{бомз}{бомбы}.\\

\readunder{Einstein}{Айнштайн}{Эйнштейн} \readunder{died}{дайд}{умер} \readunder{in}{ин}{в} \readunder{1955}{найнтин фифти файв}{1955} \readunder{in}{ин}{в} \readunder{the}{зе}{} \readunder{United}{Юнайтед}{Соединённых} \readunder{States}{Стэйтс}{Штатах}.

\section*{Полный перевод на русский}
Альберт Эйнштейн родился в Германии в 1879 году. В детстве он считал школу скучной. В пятнадцать лет он бросил школу, не получив никаких квалификаций.

Однако спустя несколько лет он продолжил учёбу в Швейцарии и окончил её в 1900 году. В последующие несколько лет он провёл много исследований в области математики и физики. Он писал статьи для научных журналов о своих открытиях, которые изменили представление человека о Вселенной.

В 1921 году он получил Нобелевскую премию по физике и стал одним из самых уважаемых физиков в мире. В 1939 году он покинул Германию и поселился в Америке, где продолжил исследования.

Его исследования и теории впоследствии были использованы для разработки атомной и водородной бомб.

Эйнштейн умер в 1955 году в Соединённых Штатах.

\end{document}
