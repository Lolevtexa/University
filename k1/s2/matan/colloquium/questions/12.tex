% ===== 12.tex =====
% "Формула Стирлинга (с эквивалентностью)."

% 1. Определения

\textbf{Факториал $n!$.}
Для натурального $n$ вводится произведение:
\[
n! \;=\;1\cdot2\cdot\dots\cdot n.
\]

\medskip

\textbf{Формула Стирлинга (эквивалентность).}
При $n\to\infty$
\[
n! \;\sim\;\sqrt{2\pi\,n}\,\Bigl(\tfrac{n}{e}\Bigr)^n,
\]
то есть
\[
\lim_{n\to\infty}\frac{n!}{\sqrt{2\pi\,n}\,\bigl(\tfrac{n}{e}\bigr)^n}=1.
\]

\medskip

% 2. Теоремы и ключевые утверждения

\textbf{Основная идея.}
Используется:
\[
\ln(n!)=\sum_{k=1}^n\ln k
\quad\approx\quad
\int_{1}^{n}\ln x\,dx=n\ln n - n +1.
\]
Разность «сумма – интеграл» даёт поправку, которая приводит к множителю \(\sqrt{2\pi n}\).

\medskip

% 3. Полное доказательство (шаги)

\begin{enumerate}
  \item \textbf{Переход к логарифмам.}
    \[
      \ln(n!)=\sum_{k=1}^n\ln k.
    \]
  \item \textbf{Сравнение с интегралом.}
    \[
      \sum_{k=1}^n\ln k
      \;\approx\;\int_{1}^{n}\ln x\,dx
      = n\ln n - n + 1.
    \]
  \item \textbf{Тонкая оценка (через формулу Эйлера–Маклорена).}
    \[
      \ln(n!)=n\ln n - n +\tfrac12\ln(n)+O(1).
    \]
  \item \textbf{Экспоненцирование.}
    \[
      n!=\exp(n\ln n - n +\tfrac12\ln n +O(1))
      =\sqrt{n}\,\Bigl(\tfrac{n}{e}\Bigr)^n\,\exp\bigl(O(1)\bigr).
    \]
    При более точном разборе получается желаемый множитель \(\sqrt{2\pi}\), то есть
    \[
      n!\;\sim\;\sqrt{2\pi n}\,\Bigl(\tfrac{n}{e}\Bigr)^n.
    \]
\end{enumerate}

\medskip

% 4. Сравнение

Уже при умеренных $n$, например $n=10$, точность формулы достаточно хороша; отклонение от \(\sqrt{2\pi n}\,\bigl(\tfrac{n}{e}\bigr)^n\) в долях процента относительно $n!$.
