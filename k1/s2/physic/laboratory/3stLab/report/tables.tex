\section{Таблицы расчётов}

\begin{table}[H]
	\centering
	\caption{Измерение углов дифракции для линий красного цвета}

	\scalebox{0.9}{
		\begin{tblr}{
				vlines = {},
				hlines = {},
				cells = {c, mode = dmath},
				cell{2}{1} = {r = 3}{},
				cell{13}{1} = {r = 3}{},
			}
			\left\lvert m \right\rvert                        &
			\qquad 0 \qquad                                   &
			\qquad 1 \qquad                                   &
			\qquad 2 \qquad                                   &
			\qquad 3 \qquad                                                                                                       \\
			\alpha_{+m}                                       & 352\degree 25' & 351\degree 26' & 350\degree 17' & 349\degree 10' \\
			                                                  & 352\degree 27' & 351\degree 28' & 350\degree 18' & 349\degree 6'  \\
			                                                  & 352\degree 31' & 351\degree 28' & 350\degree 21' & 349\degree 11' \\
			\bar{\alpha}_{+m}                                 & 352\degree 28' & 351\degree 27' & 350\degree 19' & 349\degree 9'  \\
			\varphi_{+m} = \bar{\alpha}_{+m} - \bar{\alpha}_0 & 0\degree 0'    & -1\degree 1'   & -2\degree 9'   & -3\degree 19'  \\
			a = \frac{\sin(\varphi_{+m})}{m}                  &                & -0.017646      & -0.018806      & -0.019285      \\
			\theta_a = \frac{\cos(\varphi_{+m})}{m}           &                & 0.999844       & 0.499646       & 0.332775       \\
			\theta_a = \frac{\cos(\varphi_{-m})}{m}           &                & 0.999749       & 0.499547       & 0.332669       \\
			a = \frac{\sin(\varphi_{-m})}{m}                  &                & 0.022397       & 0.021277       & 0.021027       \\
			\varphi_{-m} = \bar{\alpha}_{-m} - \bar{\alpha}_0 & 0\degree 0'    & 1\degree 17'   & 2\degree 26'   & 3\degree 37'   \\
			\bar{\alpha}_{-m}                                 & 352\degree 28' & 353\degree 45' & 354\degree 54' & 356\degree 5'  \\
			\alpha_{-m}                                       & 352\degree 27' & 353\degree 33' & 354\degree 49' & 356\degree 5'  \\
			                                                  & 352\degree 31' & 353\degree 48' & 354\degree 56' & 356\degree 4'  \\
			                                                  & 352\degree 30' & 353\degree 54' & 354\degree 58' & 356\degree 6'  \\
		\end{tblr}}
\end{table}

\begin{table}[H]
	\centering
	\caption{Измерение углов дифракции для линий зелёного цвета}
	\scalebox{0.9}{
		\begin{tblr}{
				vlines = {},
				hlines = {},
				cells = {c, mode = dmath},
				cell{2}{1} = {r = 3}{},
				cell{13}{1} = {r = 3}{},
			}
			\left\lvert m \right\rvert                        &
			\qquad 0 \qquad                                   &
			\qquad 1 \qquad                                   &
			\qquad 2 \qquad                                   &
			\qquad 3 \qquad                                                                                                       \\
			\alpha_{+m}                                       & 352\degree 27' & 351\degree 30' & 350\degree 32' & 349\degree 28' \\
			                                                  & 352\degree 28' & 351\degree 35' & 350\degree 39' & 349\degree 33' \\
			                                                  & 352\degree 26' & 351\degree 34' & 350\degree 40' & 349\degree 30' \\
			\bar{\alpha}_{+m}                                 & 352\degree 28' & 351\degree 33' & 350\degree 37' & 349\degree 30' \\
			\varphi_{+m} = \bar{\alpha}_{+m} - \bar{\alpha}_0 & 0\degree 0'    & -0\degree 55'  & -1\degree 51'  & -2\degree 58'  \\
			a = \frac{\sin(\varphi_{+m})}{m}                  &                & -0.015998      & -0.016141      & -0.017219      \\
			\theta_a = \frac{\cos(\varphi_{+m})}{m}           &                & 0.999872       & 0.499739       & 0.332888       \\
			\theta_a = \frac{\cos(\varphi_{-m})}{m}           &                & 0.999843       & 0.499597       & 0.332799       \\
			a = \frac{\sin(\varphi_{-m})}{m}                  &                & 0.017743       & 0.020066       & 0.018865       \\
			\varphi_{-m} = \bar{\alpha}_{-m} - \bar{\alpha}_0 & 0\degree 0'    & 1\degree 1'    & 2\degree 18'   & 3\degree 15'   \\
			\bar{\alpha}_{-m}                                 & 352\degree 28' & 353\degree 29' & 354\degree 46' & 355\degree 43' \\
			\alpha_{-m}                                       & 352\degree 27' & 353\degree 26' & 354\degree 52' & 355\degree 43' \\
			                                                  & 352\degree 31' & 353\degree 32' & 354\degree 41' & 355\degree 42' \\
			                                                  & 352\degree 30' & 353\degree 29' & 354\degree 45' & 355\degree 43' \\
		\end{tblr}}
\end{table}

\begin{table}[H]
	\centering
	\caption{Измерение углов дифракции для линий фиолетового цвета}
	\scalebox{0.9}{
		\begin{tblr}{
				vlines = {},
				hlines = {},
				cells = {c, mode = dmath},
				cell{2}{1} = {r = 3}{},
				cell{13}{1} = {r = 3}{},
			}
			\left\lvert m \right\rvert                        &
			\qquad 0 \qquad                                   &
			\qquad 1 \qquad                                   &
			\qquad 2 \qquad                                   &
			\qquad 3 \qquad                                                                                                       \\
			\alpha_{+m}                                       & 352\degree 30' & 351\degree 49' & 350\degree 45' & 350\degree 10' \\
			                                                  & 352\degree 31' & 351\degree 52' & 350\degree 44' & 350\degree 5'  \\
			                                                  & 352\degree 29' & 351\degree 42' & 350\degree 37' & 350\degree 4'  \\
			\bar{\alpha}_{+m}                                 & 352\degree 28' & 351\degree 48' & 350\degree 42' & 350\degree 6'  \\
			\varphi_{+m} = \bar{\alpha}_{+m} - \bar{\alpha}_0 & 0\degree 0'    & -0\degree 40'  & -1\degree 46'  & -2\degree 22'  \\
			a = \frac{\sin(\varphi_{+m})}{m}                  &                & -0.011732      & -0.015415      & -0.013733      \\
			\theta_a = \frac{\cos(\varphi_{+m})}{m}           &                & 0.999931       & 0.499762       & 0.333050       \\
			\theta_a = \frac{\cos(\varphi_{-m})}{m}           &                & 0.999869       & 0.499746       & 0.333013       \\
			a = \frac{\sin(\varphi_{-m})}{m}                  &                & 0.016192       & 0.015948       & 0.014604       \\
			\varphi_{-m} = \bar{\alpha}_{-m} - \bar{\alpha}_0 & 0\degree 0'    & 0\degree 56'   & 1\degree 50'   & 2\degree 31'   \\
			\bar{\alpha}_{-m}                                 & 352\degree 28' & 353\degree 24' & 354\degree 18' & 354\degree 59' \\
			\alpha_{-m}                                       & 352\degree 26' & 353\degree 22' & 354\degree 18' & 354\degree 58' \\
			                                                  & 352\degree 27' & 353\degree 23' & 354\degree 19' & 354\degree 58' \\
			                                                  & 352\degree 28' & 353\degree 26' & 354\degree 16' & 355\degree 0'  \\
		\end{tblr}}
\end{table}

\begin{table}[H]
	\centering
	\caption{Константы эксперимента}
	\scalebox{0.9}{
		\begin{tblr}{
			colspec = {Q[wd=3cm]Q[wd=4.2cm]Q[wd=4cm]Q[wd=3cm]},
			vlines = {},
			hlines = {},
			cells = {valign = m, halign = c},
			}
			Длина волны залёного цвета, $\lambda = \overline{\lambda} \pm \Delta \overline{\lambda} \text{, нм}$ &
			Постоянная решётки, $d = \overline{d} \pm \Delta \overline{d} \text{, мкм}$                          &
			Длина решётки, $L$, см                                                                               &
			Число штрихов $N=\dfrac{L}{\overline{d}}$                                                                                      \\
			$\lambda = 546 \pm 5 \text{, нм} $ $ \text{с } P = 95 \%$                                            & $447.786\pm 7467.851$ &
			1,5 см                                                                                               & 34                      \\
		\end{tblr}}
\end{table}

\begin{table}[H]
	\centering
	\caption{Определение длины волны и характеристик дифракционной решётки}
	\label{tab:lambda-params}
	\scalebox{0.9}{
		\begin{tblr}{
			colspec = {Q[wd=2.1cm]Q[wd=2.4cm]Q[wd=2.6cm]Q[wd=1.6cm]Q[wd=2.8cm]Q[wd=1.7cm]Q[wd=1.6cm]},
			vlines = {},
			hlines = {},
			row{1} = {valign = m, halign = c},
			cell{2,4,6}{1,2,3} = {r = 2}{},
				}
			Цвет спектральной линии                                            &
			Угловой коэффициент
			$a = \overline{a} \pm \Delta \overline{a}$                         &
			Длина волны
			$\lambda = \overline{\lambda} \pm \Delta \overline{\lambda},\,$ нм &
			Порядок
			спектра, $m$                                                       &
			$D_{\varphi} = \dfrac{m}{\,\overline{d} \cos \varphi_m}$ мин/нм    &
			$R = mN$                                                           &
			$\Delta \lambda = \dfrac{\overline{\lambda}}{R}$ нм                                                                                    \\
			Красная                                                            & $0.001494\pm0.023088$ & $669.0\pm15210.6$ & 1 & 0.01 & 34  & 19.7 \\
			                                                                   & $0.001494\pm0.023088$ & $669.0\pm15210.6$ & 3 & 0.02 & 102 & 6.6  \\
			Зелёная                                                            & $0.001219\pm0.020335$ & $545.9\pm12875.7$ & 1 & 0.01 & 34  & 16.1 \\
			                                                                   & $0.001219\pm0.020335$ & $545.9\pm12875.7$ & 3 & 0.02 & 34  & 5.4  \\
			Фиолетовая                                                         & $0.000977\pm0.016843$ & $437.5\pm10493.6$ & 1 & 0.01 & 102 & 12.9 \\
			                                                                   & $0.000977\pm0.016843$ & $437.5\pm10493.6$ & 3 & 0.02 & 102 & 4.3  \\
		\end{tblr}}
\end{table}