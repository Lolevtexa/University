% ==================
% auxiliary.tex
% (Вспомогательные определения для темы
% "Теорема Коши. Правило Лопиталя (случай 0/0). Примеры неприменимости.")
% ==================

\paragraph{Проколотая окрестность.}
Окрестность точки $a$ называют \emph{проколотой}, если в ней
учитывают все точки, кроме, возможно, самой $a$. Формально,
это множество
\[
	\{\,x : 0<|x-a|<\delta\},
\]
где $\delta>0$. Функции $f$ и $g$ говорят «дифференцируемы в проколотой окрестности $a$», если они имеют производные для всех $x$ этого множества (кроме, возможно, самой точки $a$).

\bigskip

\paragraph{Теорема Ролля (напоминание).}
Пусть $h$ непрерывна на $[p,q]$, дифференцируема на $(p,q)$
и $h(p)=h(q)$. Тогда существует $c\in(p,q)$ с $h'(c)=0$.
Часто используется для построения доказательств Лагранж‐типа.

\bigskip

\paragraph{Неопределённости вида $0/0$ и $\infty/\infty$.}
Правило Лопиталя применимо только к случаям, когда
\(\lim\limits_{x\to a} f(x)\) и \(\lim\limits_{x\to a} g(x)\)
одновременно равны нулю, либо одновременно стремятся к $\pm\infty$.
Во всех остальных случаях правило не даёт результата.

\bigskip

\paragraph{Условия дифференцируемости.}
Если функции $f,g$ дифференцируемы в некоторой проколотой окрестности $a$,
то мы можем говорить о $f'(x)$ и $g'(x)$ при $x\to a$,
даже если $f(a)$ или $g(a)$ не определены (или не дифференцируемы) строго в точке $a$.
