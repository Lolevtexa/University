% ===== 6.tex =====
% "Теорема Коши. Правило Лопиталя (случай 0/0). Примеры, когда правило неприменимо."

% 1. Определения

\textbf{Теорема Коши (обобщённая теорема Лагранжа).}
Пусть функции $f(x)$ и $g(x)$ удовлетворяют условиям:
\begin{itemize}
  \item непрерывны на $[a,b]$,
  \item дифференцируемы на $(a,b)$,
  \item $g'(x)\neq 0$ для всех $x\in(a,b)$.
\end{itemize}
Тогда существует точка $c\in(a,b)$ такая, что
\[
\frac{f(b)-f(a)}{g(b)-g(a)} \;=\; \frac{f'(c)}{g'(c)}.
\]

\medskip

\textbf{Правило Лопиталя (только для случая \(\tfrac{0}{0}\)).}
Пусть функции $f(x)$ и $g(x)$ дифференцируемы в \emph{проколотой} окрестности точки $a$ (то есть всюду, кроме, возможно, самой точки $a$). Предположим:
\[
\lim_{x\to a} f(x) = 0, \quad \lim_{x\to a} g(x)=0,
\]
и $g'(x)\neq 0$ в этой окрестности. Если существует (конечный или бесконечный) предел
\[
\lim_{x\to a} \frac{f'(x)}{g'(x)} = L,
\]
то существует и предел \(\lim\limits_{x\to a}\frac{f(x)}{g(x)}\), причём
\[
\lim_{x\to a}\frac{f(x)}{g(x)} = L.
\]

\medskip

% 2. Теоремы и ключевые утверждения

\textbf{Основная идея Теоремы Коши:}
\begin{itemize}
  \item Является обобщением Теоремы Лагранжи (достаточно взять $g(x)=x$).  
  \item Используется для доказательства Правила Лопиталя: рассматриваем $f(x), g(x)$ и применяем теорему Коши к $[a,x]$.
\end{itemize}

\textbf{Основная идея Правила Лопиталя (0/0):}
\begin{itemize}
  \item Рассматриваем отношение $\frac{f(x)}{g(x)}$ при $x\to a$.  
  \item Применяем теорему Коши к функциям $F(t)=f(t)-f(a)$, $G(t)=g(t)-g(a)$ на отрезке $[a,x]$.  
  \item Переходя к пределу, получаем $\lim_{x\to a}\frac{f(x)}{g(x)}=\lim_{x\to a}\frac{f'(x)}{g'(x)}$, если последний существует.
\end{itemize}

\medskip

% 3. Полное доказательство (детально, отсылая к sub_6 при необходимости)

\subsection*{Доказательство Теоремы Коши (обобщённая Лагранжа)}

\begin{enumerate}
  \item \textbf{Условие:} $f, g$ непрерывны на $[a,b]$, дифференцируемы на $(a,b)$, причём $g'(x)\neq0$.  
  \item \textbf{Построение.} Рассмотрим функцию
    \[
      \Phi(t) \;=\; f(t) - f(a) \;-\; \frac{f(b)-f(a)}{g(b)-g(a)}\,\bigl[g(t)-g(a)\bigr].
    \]
    Тогда $\Phi(a)=0$ и \(\Phi(b)=f(b)-f(a)-\tfrac{f(b)-f(a)}{g(b)-g(a)}[\,g(b)-g(a)\,]=0\).  
  \item \textbf{Применение Теоремы Ролля (см. \texttt{sub\_6.tex}):}  
    Поскольку $\Phi$ непрерывна на $[a,b]$ и дифференцируема на $(a,b)$ (как разность таковых), и $\Phi(a)=\Phi(b)=0$, существует $c\in(a,b)$ с \(\Phi'(c)=0\).  
  \item \textbf{Производная:} 
    \[
    \Phi'(t)=f'(t) - \frac{f(b)-f(a)}{g(b)-g(a)}\,g'(t).
    \]
    Тогда $\Phi'(c)=0 \implies f'(c)=\dfrac{f(b)-f(a)}{g(b)-g(a)}\,g'(c)$.  
  \item \textbf{Следствие:} 
    \[
      \frac{f(b)-f(a)}{g(b)-g(a)} = \frac{f'(c)}{g'(c)}.
    \]
\end{enumerate}

\medskip

\subsection*{Доказательство Правила Лопиталя (случай $0/0$)}

\begin{enumerate}
  \item \textbf{Условие:} $\lim_{x\to a}f(x)=0,\; \lim_{x\to a}g(x)=0,\; g'(x)\neq0$. Предположим, что существует $\lim_{x\to a}\frac{f'(x)}{g'(x)}=L$.  
  \item \textbf{Рассмотрим $f$ и $g$ на отрезке $[a,x]$ (при $x>a$).}  
    По условию $f(a)=g(a)=0$. Применяем Теорему Коши к $f$ и $g$ на $[a,x]$:
    \[
      \exists\,c_x\in(a,x): \quad \frac{f(x)-f(a)}{g(x)-g(a)} = \frac{f'(c_x)}{g'(c_x)}.
    \]
  \item \textbf{Переход к пределу:}  
    Поскольку $f(a)=0$, $g(a)=0$, имеем
    \[
      \frac{f(x)}{g(x)} = \frac{f'(c_x)}{g'(c_x)}.
    \]
    Когда $x\to a$, точка $c_x$ тоже $\to a$ (так как $c_x$ лежит между $a$ и $x$). Если \(\lim_{x\to a}\frac{f'(x)}{g'(x)}=L\), то \(\frac{f'(c_x)}{g'(c_x)}\to L\).  
    Значит
    \[
      \lim_{x\to a}\frac{f(x)}{g(x)} \;=\; L.
    \]
\end{enumerate}

\medskip

% 4. Примеры, когда правило Лопиталя НЕприменимо

\begin{itemize}
  \item \textbf{Если нет неопределённости 0/0 или $\infty/\infty$.}  
    Пример: $\lim_{x\to 0}\frac{\sin x}{x+1}=\frac{0}{1}=0$ — тут нет неопределённости, применять Лопиталя смысла нет.
  \item \textbf{Если $\lim_{x\to a}\frac{f'(x)}{g'(x)}$ не существует.}  
    Пример: $\lim_{x\to 0}\frac{\sin(1/x)}{1/x}$:  
    - $f(x)=\sin(1/x)$, $g(x)=1/x$. При $x\to0$, $f'(x)$ и $g'(x)$ ведут себя крайне нестабильно. Предел $\frac{f'(x)}{g'(x)}$ не существует.  
  \item \textbf{Если $f$ или $g$ не дифференцируемы в проколотой окрестности точки.}  
    Пример: $f(x)=|x|$, $g(x)=x$ при $x\to 0$ не дифференцируемо в $0$.  
\end{itemize}

\medskip

% -- Логическая связность --
% (Определение "проколотой окрестности", "теоремы Ролля" и т.д. в sub_6.tex).
