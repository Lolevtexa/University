% ==================
% auxiliary.tex
% (Вспомогательные определения для темы:
%  "Теорема Лиувилля. Пример трансцендентного числа.")
% ==================

\paragraph{Алгебраическое и трансцендентное число.}
\begin{itemize}
	\item \emph{Алгебраическое число} — корень некоторого ненулевого многочлена
	      с рациональными (или целыми) коэффициентами. Например, $\sqrt{2}$, $\sqrt[3]{7}$.
	\item \emph{Трансцендентное число} — не является алгебраическим. Примеры: $e$, $\pi$,
	      а также специальные конструкции (числа Лиувилля).
\end{itemize}

\paragraph{Приближение чисел рациональными дробями.}
Говорят, что действительное число $\alpha$ \emph{допускает «слишком хорошие» рациональные приближения},
если существуют бесконечные наборы дробей $\frac{p}{q}$, для которых
\[
	\left|\alpha - \frac{p}{q}\right| \;<\; \frac{1}{\,q^n\,}
\]
при некоторых больших $q$ и $n>1$ (как правило $n$ существенно больше 1).

\bigskip

\paragraph{Минимальный многочлен (напоминание).}
Для алгебраического числа $\alpha$ его \emph{минимальным многочленом} называют многочлен
$P(x)$ наименьшей степени $d$ (со старшим коэффициентом 1, без общих делителей), у которого
$P(\alpha)=0$.

Если $\alpha$ было бы таким, что у него «слишком хорошие» приближения, то анализ $P\bigl(\tfrac{p}{q}\bigr)$ приводит к противоречию, используемому в доказательстве теоремы Лиувилля.
