% ============================
% Вспомогательная информация
% ============================

\paragraph{Дифференциал функции.}
Пусть функция $f$ задана в окрестности точки $x_0$ и дифференцируема в $x_0$.
\emph{Дифференциалом} $df(x_0)$ называют
\[
	df(x_0) \;=\; f'(x_0)\,(x - x_0).
\]
При малом приращении $dx = x - x_0$ это выражение можно записать как
\[
	df(x_0) = f'(x_0)\,dx,
\]
где остаток $f(x_0 + dx) - f(x_0) - df(x_0)$ мал по сравнению с $dx$ (обозначают $o(dx)$).

\bigskip

\paragraph{Локальный экстремум.}
Точка $x_0$ называется \emph{локальным минимумом} (соответственно, максимумом) функции $f$,
если существует $\delta>0$ такое, что при $|x - x_0| < \delta$ выполняется
\[
	f(x) \ge f(x_0)\quad (\text{или } f(x) \le f(x_0) \text{ для макс.}).
\]

\bigskip

\paragraph{Теорема Вейерштрасса (о достижении экстремума).}
Если функция $f$ непрерывна на отрезке $[a,b]$, то
\[
	\exists\, x_{\min}, x_{\max}\in [a,b] :
	\quad f(x_{\min}) \le f(x) \le f(x_{\max}) \ \ \forall\,x\in [a,b].
\]

\bigskip

\paragraph{Производная.}
Напомним, что $f'(x_0)$ есть предел
\[
	\lim_{x \to x_0}\frac{f(x) - f(x_0)}{\,x - x_0\,},
\]
если этот предел конечен. Если $f'(x_0)$ существует для всех $x_0$ из некоторого промежутка,
говорят, что $f$ дифференцируема на этом промежутке.
