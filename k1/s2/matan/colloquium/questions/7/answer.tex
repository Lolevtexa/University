% ==================
% answer.tex
% (Основной материал: Теорема о Тейлоре для многочлена,
%  Формула Тейлора с остатком в форме Пеано и доказательства.)
% ==================

\begin{customtheorem}[Формула Тейлора для многочлена]
	Пусть $P(x)$ — многочлен степени $n$. Тогда его разложение в окрестности $x_0$
	совпадает со стандартным полиномом Тейлора степени $n$, а
	\emph{остаток} (производные порядка выше $n$) равен 0.
\end{customtheorem}

\begin{proofplan}
	\begin{enumerate}
		\item Заметим, что для $m>n$, $P^{(m)}(x)\equiv0$ (у многочлена).
		\item Построить формулу Тейлора $T_n(x)$ до степени $n$, сослаться на нулевые старшие производные.
		\item Показать, что фактически $P(x)=T_n(x)$, поскольку коэффициенты полностью совпадают.
	\end{enumerate}
\end{proofplan}

\begin{customproof}
	Пусть $P(x)$ — многочлен степени $n$. Рассмотрим «полином Тейлора» порядка $n$ вокруг $x_0$:
	\[
		T_n(x) \;=\; \sum_{k=0}^{n} \frac{P^{(k)}(x_0)}{k!}\,(x - x_0)^k.
	\]
	Так как производные $P^{(k)}(x)$ для $k>n$ тождественно равны нулю, в формуле не возникает никаких членов выше $n$‐го порядка, и «остаток» $R_n(x)$ отсутствует.

	Кроме того, само определение производной многочлена показывает, что $P^{(k)}(x_0)$ являются соответствующими коэффициентами, и $T_n(x)$ на самом деле совпадает с исходным многочленом $P(x)$ (коэффициенты совпадают). Значит
	\[
		P(x) \;=\; T_n(x),
	\]
	и никакого дополнительного остатка нет.
\end{customproof}

\begin{customtheorem}[Формула Тейлора с остатком в форме Пеано]
	Пусть функция $f$ $n$ раз дифференцируема в точке $x_0$. Тогда
	\[
		f(x)
		\;=\;
		\sum_{k=0}^{n}
		\frac{f^{(k)}(x_0)}{k!}\,(x - x_0)^k
		\;+\;
		o\bigl((x - x_0)^n\bigr).
	\]
\end{customtheorem}

\begin{proofplan}
	\begin{enumerate}
		\item Написать классическую формулу Тейлора (с формой Лагранжа для остатка).
		\item Показать, что если $f^{(n)}$ непрерывна, то этот остаток становится $o((x - x_0)^n)$.
		\item Использовать аргумент, что $(x - x_0)^{n+1}$ «уходит» быстрее, чем $(x - x_0)^n$ при $x\to x_0$.
	\end{enumerate}
\end{proofplan}

\begin{customproof}
	Предположим, $f$ имеет непрерывные производные вплоть до порядка $n$. По классической Формуле Тейлора с остатком в форме Лагранжа,
	\[
		f(x)
		\;=\;
		\sum_{k=0}^{n}
		\frac{f^{(k)}(x_0)}{k!}\,(x - x_0)^k
		\;+\;
		\frac{f^{(n+1)}(\xi)}{(n+1)!}\,(x - x_0)^{n+1}
	\]
	для некоторой $\xi$ между $x_0$ и $x$.
	Поскольку $f^{(n+1)}$ непрерывна в $x_0$, при $x\to x_0$ значение $f^{(n+1)}(\xi)$ остаётся ограниченным, а $(x - x_0)^{n+1}$ «уходит» быстрее, чем $(x - x_0)^n$. Таким образом
	\[
		\frac{f^{(n+1)}(\xi)}{(n+1)!}\,(x - x_0)^{n+1}
		\;=\;
		o\bigl((x - x_0)^n\bigr).
	\]
	Значит вся формула переписывается в виде
	\[
		f(x)
		\;=\;
		\sum_{k=0}^{n}
		\frac{f^{(k)}(x_0)}{k!}\,(x - x_0)^k
		\;+\;
		o\bigl((x - x_0)^n\bigr),
	\]
	что и требовалось доказать.
\end{customproof}

\begin{customexample}
	\textbf{Примеры:}
	\begin{itemize}
		\item \textbf{Многочлен $P(x)$.}
		      Для $P(x)$ степени $n$ справедлива формула Тейлора, где \emph{нет} остатка,
		      потому что $P^{(m)}(x)\equiv0$ при $m>n$.
		\item \textbf{$f(x)=e^x$.}
		      Не является многочленом, но при разложении вокруг $0$:
		      \[
			      e^x \;=\; 1 + x + \frac{x^2}{2!} + \dots + \frac{x^n}{n!} \;+\; o(x^n),
			      \quad x\to0.
		      \]
		      Остаточный член именно «форма Пеано», показывающая, что остаток делится на $x^n$ с показателем $n$ и уходит в ноль.
	\end{itemize}
\end{customexample}
