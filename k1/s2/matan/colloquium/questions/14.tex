% ===== 14.tex =====
% "Определение интеграла Римана. Отличие от обычного предела."

% 1. Определения

\textbf{Интеграл Римана.}
Пусть $f$ задана на $[a,b]$. Разобьём отрезок:
\[
a=x_0<x_1<\dots<x_n=b,
\quad
\Delta x_i=x_i - x_{i-1},
\quad
\xi_i\in[x_{i-1},x_i].
\]
Тогда \emph{интегральная сумма}:
\[
S=\sum_{i=1}^n f(\xi_i)\,\Delta x_i.
\]
Если при \(\max_i \Delta x_i\to0\) все такие суммы $S$ \emph{стремятся к одному и тому же} числу $I$, независимо от выбора точек $\xi_i$, то $f$ называется \textbf{интегрируемой по Риману}, а $I$ — её интегралом:
\[
I=\int_{a}^{b}f(x)\,dx.
\]

\medskip

% 2. Теоремы и ключевые утверждения

\begin{itemize}
  \item При \emph{непрерывности} $f$ на $[a,b]$ интеграл Римана существует.
  \item Критерий Дарбу: если верхние и нижние суммы сближаются, интеграл существует.
\end{itemize}

\medskip

% 3. Основные идеи доказательств

\begin{itemize}
  \item Рассматривается равномерная непрерывность на $[a,b]$ и «мелкие» разбиения, чтобы функция не успевала сильно меняться в каждом отрезке.
  \item Используется факт, что всякая пара разбиений «уточняется» до одного общего, и разность сумм делается малой.
\end{itemize}

\medskip

% 4. Полное доказательство (принцип)

\begin{enumerate}
  \item \textbf{Два разбиения.}
    Пусть $D$ и $D'$ — любые разбиения, $\|D\|\to0$ и $\|D'\|\to0$.
  \item \textbf{Уточнение.}
    Построить «общее» разбиение $D''$ содержащее все точки $D$ и $D'$. Сопоставить интегральные суммы. 
  \item \textbf{Оценка разницы.}
    При малых $\Delta x_i$, из равномерной непрерывности (или ограниченности) $f$ следует, что интегральные суммы $S(f,D)$ и $S(f,D')$ близки. 
  \item \textbf{Вывод.}
    Предел един, определение интеграла однозначно.
\end{enumerate}

\medskip

% 5. Отличие от «обычного» предела

\begin{itemize}
  \item Обычный предел: $\lim\limits_{x\to x_0} f(x)$ — \emph{точечный} анализ, рассматриваем поведение функции в одной точке.
  \item Интеграл Римана: \emph{глобальный} (берёт во внимание всё множество $[a,b]$), есть \textit{предел} \emph{сумм} при возрастании числа разбиений.
\end{itemize}

\medskip

% 6. Пример

Если $f(x)=\text{const}$, то любая интегральная сумма есть const\((b-a)\), предел одинаков независимо от разбиения.
